\subsection{\emph{Pushout} de espaços topológicos} %afirmação aqui significa teorema/proposição/colorário/lema
\label{pushout-de-espacos-topologicos-def}
\begin{titlemize}{Lista de dependências}
	\item \hyperref[topologia-quociente-def]{Espaços Quociente};\\ %'dependencia1' é o label onde o conceito Dependência 1 aparece (--à arrumar um padrão para referencias e labels--) 
    \item \hyperref[funcao-continua-em-topologia-quociente-prop]{Função contínua em topologia quociente}.
% quantas dependências forem necessárias.
\end{titlemize}

\begin{defi}
    Sejam $X,Y,Z$ espaços topológicos, e sejam $f:Z\rightarrow X$ e $g:Z\rightarrow Y$ funções contínuas. O \textbf{\emph{pushout} de $f$ e $g$} é o espaço quociente $X\sqcup_Z Y=X\sqcup Y/\sim$, onde $\sim$ é a menor relação de equivalência que contém $\{(f(z),g(z))\in X\times Y:z\in Z\}$. 
\end{defi}

\begin{prop}
    Sejam $X,Y,Z$ espaços topológicos. Além disso, sejam $f:Z\rightarrow X$ e $g:Z\rightarrow Y$ funções contínuas. Seja também $\pi:X\sqcup Y\rightarrow X\sqcup_Z Y$ a função projeção associada ao quociente. Então, o espaço $X\sqcup_Z Y$, juntamente com as funções contínuas definidas por:
    \begin{align*}
        i_X:X &\longrightarrow X\sqcup Y/\sim & i_Y:Y&\longrightarrow X\sqcup Y/\sim\\
        x&\longmapsto \pi(x) & y &\longmapsto \pi(y)
    \end{align*}
    forma um diagrama de \emph{pushout}. Ou seja, dadas quaisquer duas funções contínuas $h_X:X\rightarrow W$ e $h_Y:Y\rightarrow W$ que satisfaçam $h_X\circ f=h_Y\circ g$, existe uma única função contínua 
    $\phi:X\sqcup_Z Y\rightarrow W$ tal que 
    $$h_X=\phi\circ i_X \;\;\;\text{ e }\;\;\; h_Y=\phi\circ i_Y.$$ 
    Isso é ilustrado no diagrama seguinte:
% https://q.uiver.app/#q=WzAsNSxbMCwwLCJaIl0sWzAsMiwiWCJdLFsyLDAsIlkiXSxbMiwyLCJYXFxzcWN1cF9aIFkiXSxbMywzLCJXIl0sWzAsMSwiZiIsMl0sWzAsMiwiZyJdLFsxLDMsImlfWCJdLFsyLDMsImlfWSIsMl0sWzEsNCwiaF9YIiwyXSxbMiw0LCJoX1kiXSxbMyw0LCJcXGV4aXN0cyEgXFxwaGkiLDEseyJzdHlsZSI6eyJib2R5Ijp7Im5hbWUiOiJkYXNoZWQifX19XV0=
\[\begin{tikzcd}
	Z && Y \\
	\\
	X && {X\sqcup_Z Y} \\
	&&& W.
	\arrow["g", from=1-1, to=1-3]
	\arrow["f"', from=1-1, to=3-1]
	\arrow["{i_Y}"', from=1-3, to=3-3]
	\arrow["{h_Y}", from=1-3, to=4-4]
	\arrow["{i_X}", from=3-1, to=3-3]
	\arrow["{h_X}"', from=3-1, to=4-4]
	\arrow["{\exists! \phi}"{description}, dashed, from=3-3, to=4-4]
\end{tikzcd}\]
\end{prop}

\begin{dem}
    De acordo com a construção da topologia quociente e da topologia de união disjunta, as funções $i_X,i_Y$ são contínuas. Além disso, a função dada por 
    \begin{align*}
        h_X\sqcup h_Y:X\sqcup Y&\longrightarrow W\\
        a&\longmapsto h_X\sqcup h_Y(a)=\begin{cases}
         h_X(a) & \text{ if }a\in X\\
         h_Y(a) & \text{ if }a\in Y.
        \end{cases}
    \end{align*}
    também é contínua. Como $h_X\circ f=h_Y\circ g$, pela definição de \emph{pushout} de $f$ e $g$, a função $\phi:=(h_X\sqcup h_Y)\circ \pi^{-1}$ é bem-definida. Além disso, a função $\phi$ é contínua, pois a função $\phi\circ\pi=h_X\sqcup h_Y$ é contínua (pela proposição \ref{funcao-continua-em-topologia-quociente-prop}). Pela construção de $\phi$, temos $h_X=\phi\circ i_X$ e $h_Y=\phi\circ i_Y$, o que prova a existência de tal função.

    Finalmente, provamos que esta função é única: suponha que $\phi'$ seja outra função contínua que satisfaça $h_X=\phi'\circ i_X$ e $h_Y=\phi'\circ i_Y$. Então, temos $\phi'|_{i_X(X)}=\phi|_{i_X(X)}$ e $\phi'|_{i_Y(Y)}=\phi|_{i_Y(Y)}$. Como $i_X(X)\cup i_Y(Y)=X\sqcup_ZY$, concluímos que $\phi=\phi'$.
\end{dem}

%\begin{titlemize}{Lista de consequências}
	%\item %\hyperref[consequencia1]{Consequência 1};\\ %'consequencia1' é o label onde o conceito Consequência 1 aparece
%\end{titlemize}
