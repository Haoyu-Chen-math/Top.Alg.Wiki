%-------------------------------------------------------------------------------------------------------------!Draft!-------------------------------------------------------------------------------------------------------------------------
\section{Superfícies}
\label{superficies}

% \begin{titlemize}{Lista de Dependências}
% 	\item \hyperref[assunto1]{Assunto 1};\\ %assunto1 é o label onde o Assunto 1 aparece
% 	\item \hyperref[]{};
% \end{titlemize}

Nesta seção, introduziremos variedades, superfícies, simplexos, complexos simpliciais, triangulação, complexos celulares e provaremos o teorema de classificação de superfícies.

%---------------------------------------------------------------------------------------------------------------------!Draft!-----------------------------------------------------------------------------------------------------------------
\subsection{Variedades Topológicas}
\label{variedade-def}
%\begin{titlemize}{Lista de dependências}
	%\item \hyperref[dependecia1]{Dependência 1};\\ %'dependencia1' é o label onde o conceito Dependência 1 aparece (--à arrumar um padrão para referencias e labels--) 
	%\item \hyperref[]{};\\
% quantas dependências forem necessárias.
%\end{titlemize}
\begin{defi}[Variedade Topológica]
	Fixemos $m\geq 0$. Uma $m$-\textbf{variedade topológica} é um espaço topológico $M$ Hausdorff, $2^o$ enumerável munido de um \textbf{atlas} $\{(\phi_i, U_i) : i \in I\}$. Isto é, $\{U_i : i \in I\}$ é uma cobertura aberta de $M$ e $\phi_i: U_i \to \phi_i(U_i) \subset \mathbb{R}^m$ é um homeomorfismo, para todo $i \in I$.
    
    Denotemos por $\mathbb{H}^m$ o semiespaço $\mathbb{R}^{m-1}\times \left[0,\infty\right[$. Uma $m$-\textbf{variedade topológica com bordo} é um espaço topológico $M$ Hausdorff, $2^o$ enumerável munido de $\{(\phi_i, U_i) : i \in I\}$ (ao que também nos referimos como atlas), em que $\{U_i : i \in I\}$ é uma cobertura aberta de $M$ e $\phi_i: U_i \to \phi_i(U_i) \subset \mathbb{H}^m$ é um homeomorfismo, para todo $i \in I$.

Uma \textbf{superfície (com bordo)} é uma 2-variedade topológica (com bordo).
\end{defi}

\begin{titlemize}{Lista de consequências}
	\item \hyperref[triangulacao-def]{Triangulação};\\
    \item \hyperref[soma-conexa-def]{Soma conexa}
\end{titlemize}

%[Bianca]: é mais fácil criar a lista de dependências do que a de consequências.
%---------------------------------------------------------------------------------------------------------------------!Draft!-----------------------------------------------------------------------------------------------------------------
\subsection{Combinações afins e convexas}
\label{comb-afim-convexa-def}
% \begin{titlemize}{Lista de dependências}
% 	\item \hyperref[dependecia1]{Dependência 1};\\ %'dependencia1' é o label onde o conceito Dependência 1 aparece (--à arrumar um padrão para referencias e labels--) 
% 	\item \hyperref[]{};\\
% % quantas dependências forem necessárias.
% \end{titlemize}
\begin{defi}[Combinações lineares, afins e convexas]
	Seja $V$ um espaço vetorial sobre $\mathbb{R}$ e seja $S \subset V$ um subconjunto. Recordamos que uma \textbf{combinação linear} de elementos de $S$ é um vetor da forma $y = \sum_{j=0}^n \lambda_j x_j$, onde $\lambda_j \in \mathbb{R}$ e $x_j \in S$ para todo $0 \leq j \leq n$. Se $\sum_{j=0}^n \lambda_j = 1$, dizemos que $y$ é \textbf{combinação afim} de elementos de $S$. Por fim, se $\sum_{j=0}^n \lambda_j = 1$ e $\lambda_j \geq 0$ para todo $0 \leq j \leq n$, dizemos que $y$ é \textbf{combinação convexa} de elementos de $S$.

    O conjunto de combinações lineares de elementos de $S$ é o \textbf{espaço gerado} por $S$ e é denotado como $\text{span}(S)$. Do mesmo modo, o conjunto de combinações afins de elementos de $S$ é chamado de \textbf{espaço afim gerado} por $S$, e denotado por $\text{aff}(S)$. Por fim, $\text{conv}(S)$ é o conjunto de combinações convexas de $S$ e é chamado de \textbf{envoltória convexa} de $S$.
\end{defi}

É simples provar que se $x_0 \in S$, então $\text{aff}(S) = \text{span}(S-x_0) + x_0$. De fato, suponha que $y = \sum_{j=0}^n \lambda_j x_j$, onde $\sum_{j=0}^n \lambda_j = 1$ e $x_j \in S$ para todo $1 \leq j \leq n$ (podemos supor que $x_0$ é usado na representação de $y$, tomando $\lambda_0 = 0$ se necessário). Então
\begin{align*}
    y &= \sum_{j=0}^n \lambda_j x_j - x_0 + x_0
    = \sum_{j=0}^n \lambda_j x_j - \sum_{j=0}^n \lambda_j x_0 + x_0\\
    &= \sum_{j=1}^n \lambda_j(x_j - x_0) + x_0
    \in \text{span}(S-x_0) + x_0
\end{align*}

\begin{defi}[Independência afim]
    Dizemos que um subconjunto $S \neq \varnothing$ de um espaço vetorial $V$ é \textbf{``affine independent'' (a.i.)} se $S - x_0$ é linearmente independente, onde $x_0 \in S$. Ou seja, se não existe $S' \subsetneq S$ tal que $\text{aff}(S') = \text{aff}(S)$.
\end{defi}

A equivalência das duas definições segue do raciocínio anterior, pois caso exista $S' \subsetneq S$ tal que $\text{aff}(S') = \text{aff}(S)$ e $x_0 \in S$, então $\text{span}(S'-x_0) = \text{span}(S-x_0)$. Assim, $S - x_0$ não é linearmente independente. E reciprocamente, caso $S - x_0$ não seja linearmente independente, então existe $S' \subsetneq S$ tal que $\text{span}(S' - x_0) = \text{span}(S - x_0)$, logo $\text{aff}(S') = \text{aff}(S)$.

\begin{titlemize}{Lista de consequências}
	\item \hyperref[simplexo-def]{Simplexos};\\ %'consequencia1' é o label onde o conceito Consequência 1 aparece
	%\item \hyperref[]{}
\end{titlemize}
%---------------------------------------------------------------------------------------------------------------------!Draft!-----------------------------------------------------------------------------------------------------------------
\subsection{Simplexos}
\label{simplexo-def}
\begin{titlemize}{Lista de dependências}
	\item \hyperref[comb-afim-convexa-def]{Combinações afins e convexas};\\ %'dependencia1' é o label onde o conceito Dependência 1 aparece (--à arrumar um padrão para referencias e labels--) 
	%\item \hyperref[]{};\\
% quantas dependências forem necessárias.
\end{titlemize}

A seguir, introduzimos os conceitos de simplexos e de faces.

\begin{defi}[Simplexos]
    Se $\{x_0,\ldots,x_n\}$ são a.i. ($n \geq 0$), dizemos que $\sigma = \text{conv}\{x_0,\ldots,x_n\}$ é um $n$-\textbf{simplexo}, e o denotamos por $[x_0,\ldots, x_n]$. Dizemos que $n$ é a dimensão do simplexo $\sigma$, e escrevemos $n = \text{dim}(\sigma)$. O $n$-\textbf{simplexo padrão} é $\Delta^n = [0, e_1, \ldots, e_n]$ onde $\{e_1,\ldots, e_n\}$ é a base canônica de $\mathbb{R}^n$.

    Um $k$-simplexo $\tau$ é dito uma $k$-\textbf{face} de $\sigma$ caso existam $0 \leq i_0 < \ldots < i_k \leq n$ tais que $\tau = [x_{i_0}, \ldots, x_{i_k}]$. $\tau$ é uma face \textbf{própria} se $\tau \neq \sigma$. Notação: $\tau \leq \sigma$ se $\tau$ é face de $\sigma$, e $\tau < \sigma$ se $\tau$ for face própria de $\sigma$.

    As $0$-faces, $1$-faces e $2$-faces de $\sigma$ também são chamadas, respectivamente, de \textbf{vértices}, \textbf{arestas} e \textbf{triângulos} de $\sigma$.
\end{defi}

\begin{titlemize}{Lista de consequências}
	\item \hyperref[complexo-simplicial-def]{Complexos Simpliciais};\\ %'consequencia1' é o label onde o conceito Consequência 1 aparece
	%\item \hyperref[]{}
\end{titlemize}

%[Bianca]: é mais fácil criar a lista de dependências do que a de consequências.
%---------------------------------------------------------------------------------------------------------------------!Draft!-----------------------------------------------------------------------------------------------------------------
\subsection{Complexos Simpliciais}
\label{complexo-simplicial-def}
\begin{titlemize}{Lista de dependências}
	\item \hyperref[simplexo-def]{Simplexos};\\ %'dependencia1' é o label onde o conceito Dependência 1 aparece (--à arrumar um padrão para referencias e labels--) 
	%\item \hyperref[]{};\\
% quantas dependências forem necessárias.
\end{titlemize}

\begin{defi}[Complexos Simpliciais]
    Um \textbf{complexo simplicial} é um conjunto $K$ de simplexos em $\mathbb{R}^m$ tais que:
    \begin{enumerate}
        \item $\tau \leq \sigma, \sigma \in K \Rightarrow \tau \in K$;
        \item $\tau,\sigma \in K, \tau \cap \sigma \neq \varnothing \Rightarrow \tau \cap \sigma \leq \sigma, \tau \cap \sigma \leq \tau$.
    \end{enumerate}

    A \textbf{dimensão} de $K$ é $\text{dim}(K) = \sup\{\text{dim}(\tau): \tau \in K\}$.

    A \textbf{realização geométrica} de $K$ é
    \[|K| = \bigcup_{\sigma \in K} \sigma.\]
\end{defi}

\begin{defi}
    O \textbf{bordo} e o \textbf{interior} de um simplexo $\sigma$ são definidos, respectivamente, como:
    \[\partial \sigma = \bigcup_{\tau < \sigma}\tau \qquad\text{e}\qquad \text{int}(\sigma) = \sigma \setminus \partial \sigma.\]
    Note que $\partial \sigma$ é a realização geométrica do complexo simplicial $K = \{\tau : \tau < \sigma\}$.
\end{defi}

%\begin{titlemize}{Lista de consequências}
	%\item \hyperref[complexo-simplicial-def]{Complexos Simpliciais};\\ %'consequencia1' é o label onde o conceito Consequência 1 aparece
	%\item \hyperref[]{}
%\end{titlemize}

%[Bianca]: é mais fácil criar a lista de dependências do que a de consequências.
%---------------------------------------------------------------------------------------------------------------------!Draft!-----------------------------------------------------------------------------------------------------------------
\subsection{Triangulação}
\label{triangulacao-def}
\begin{titlemize}{Lista de dependências}
	\item \hyperref[complexo-simplicial-def]{Complexos simpliciais};\\ %'dependencia1' é o label onde o conceito Dependência 1 aparece (--à arrumar um padrão para referencias e labels--) 
	\item \hyperref[variedade-def]{Variedades Topológicas};\\
% quantas dependências forem necessárias.
\end{titlemize}

\begin{defi}[Triangulação de uma Variedade]
    Uma \textbf{triangulação} de uma superfície $M$ é um par ordenado $(K,\gamma)$, onde $K$ é um complexo simplicial e $\gamma:K\to 2^M$ é uma função, onde:
    \begin{enumerate}
        \item $\gamma(\sigma \cap \tau) = \gamma(\sigma) \cap \gamma(\tau)$, para todos $\sigma, \tau \in K$;
        \item existe um homeomorfismo $\varphi_{\sigma}:|\sigma| \to \gamma(\sigma)$, para todo $\sigma \in K$;
        \item se $\sigma \in K$ e $\tau \leq \sigma$, então $\varphi_{\sigma}|_{\tau}: |\tau| \to \gamma(\tau)$ é homeomorfismo;
        \item $\{\gamma(\sigma): \sigma \in K\}$ é uma cobertura fechada localmente finita de $M$.
    \end{enumerate}
\end{defi}
Neste caso, em especial, vale que $\text{dim}(K) = 2$.

O conceito de triangulação está intimamente relacionado ao de complexos celulares.

\begin{prop}
    Se $(K,\gamma)$ é uma triangulação de $M$, então existe um homeomorfismo $\varphi: |K| \to M$.
\end{prop}

Além disso, se $M$ e $N$ são duas superfícies, $\phi:M\to N$ é um homeomorfismo e $(K,\gamma)$ é uma triangulação de $M$, então $(K,\phi_* \circ \gamma)$ é triangulação de $N$, onde $\phi_*: 2^M \to 2^N$ é a função imagem direta de $\phi$. Assim, é natural se perguntar em que casos $(K,|.|)$ é uma triangulação de $|K|$. A proposição a seguir fornece uma caracterização desta propriedade. Para a demonstração ou mais informações, consulte a Proposição 3.5 de \textit{Jean Gallier, Dianna Xu, A Guide to the Classification Theorem for Compact Surfaces, Springer Berlin, Heidelberg, 2013.} 

\begin{prop}
    Seja $K$ um $2$-complexo simplicial. Então $(K,|.|)$ é uma triangulação de $|K|$, onde $|.|$ mapeia $\sigma \in K$ em $|\sigma|$, se, e somente se, valem as condições:
    \begin{enumerate}
        \item Toda aresta pertence a exatamente 2 triângulos;
        \item para todo vértice $v$, existe um inteiro $k\geq 0$ tal que $v$ pertence a exatamente $k$ triângulos e $k$ arestas; além disso, é possível ordenar as arestas e os triângulos contendo $v$ em sequências, $(a_1,\ldots,a_k)$ e $(A_1,\ldots,A_k)$, respectivamente, de como que $a_i$ e $a_{i+1}$ são arestas de $A_i$ para todo $1\leq i\leq k-1$, bem como $a_k$ e $a_1$ são arestas de $A_k$;
        \item $|K|$ é conexo.
    \end{enumerate}
\end{prop}

Para a demonstração ou mais informações, consulte a Proposição 3.6 de \textit{Jean Gallier, Dianna Xu, A Guide to the Classification Theorem for Compact Surfaces, Springer Berlin, Heidelberg, 2013.} 

% \begin{titlemize}{Lista de consequências}
% 	\item \hyperref[triangulacao-def]{Triangulação};
% \end{titlemize}

%[Bianca]: é mais fácil criar a lista de dependências do que a de consequências.
%---------------------------------------------------------------------------------------------------------------------!Draft!-----------------------------------------------------------------------------------------------------------------
\subsection{Regiões Poligonais}
\label{regiao-poligonal-def}
\begin{titlemize}{Lista de dependências}
	\item \hyperref[simplexo-def]{Simplexos}%;\\ %'dependencia1' é o label onde o conceito Dependência 1 aparece (--à arrumar um padrão para referencias e labels--) 
	%\item \hyperref[variedade-def]{Variedades Topológicas};\\
% quantas dependências forem necessárias.
\end{titlemize}
%%%%%%%%%%%% Versão antiga %%%%%%%%%%%%%%%%%%
% Antes de definirmos complexos celulares, introduzimos alguns conceitos auxiliares.
% \begin{defi}[Orientação formal]
%     Dado um conjunto $X$, definimos sua \textbf{orientação formal} como $\{-1,1\} \times X$, em que denotamos $(1,x)$ simplesmente como $x$ e $(-1,x)$ como $x^{-1}$, para todo $x\in X$. Também escrevemos $(x^{-1})^{-1} = x$. Dessa forma, a orientação formal de $X$ é denotada como $X \cup X^{-1}$. Definimos também o operador inversão, que mapeia $x \in X\cup X^{-1}$ em $x^{-1}$.
%      Seja $Y$ um conjunto. Denotamos por $Y^\#$ o quociente do conjunto de sequências finitas sobre $Y$, pela relação de equivalência dada por permutações cíclicas dos elementos da sequência. Denotamos a classe de equivalência de $(y_1,\ldots,y_n)$ por $y_1 \ldots y_n$. Desse modo, $y_1 ~y_2 \ldots y_{n-1}~y_n = y_n~y_1 \ldots y_{n-2}~y_{n-1}$.
% \end{defi}
% \begin{defi}[Complexo Celular]
%     Um \textbf{complexo celular} é uma tripla $K = (F,E,\mathcal{B})$, onde os elementos de $F$ são chamados de \textbf{faces} de $K$, os elementos de $E$ são chamados \textbf{arestas (``edges'')} de $K$, e $\mathcal{B}: F\cup F^{-1} \to (E \cup E^{-1})^\#$ é chamada de função bordo, satisfazendo:
%     \begin{enumerate}
%         \item Se $\mathcal{B}(A)= a_1 \ldots a_n$ com $a_1,\ldots, a_n \in E\cup E^{-1}$, então $\mathcal{B}(A^{-1})= a_n^{-1} \ldots a_1^{-1}$, para todo $A \in F$;
%         \item todo $a \in E\cup E^{-1}$ é elemento do bordo de no máximo duas faces.
%     \end{enumerate}
% As vezes também nos referimos aos elementos de $E^{-1}$ de arestas, e aos elementos de $F^{-1}$ de faces, respectivamente.
% Dizemos que um complexo celular $K$ é \textbf{conexo} se para todo par de arestas $a, b \in E$, existem arestas $a = a_1,a_2,\ldots,a_{k-1}, a_k=b\in E$ e faces $A_1,\ldots, A_{k-1} \in F$ de modo que $a$ é elemento do bordo de $A_1$ ou $A_1^{-1}$, $a_i$ é elemento do bordo de $A_{i-1}$ ou $A_{i-1}^{-1}$ bem como de $A_{i}$ ou $A_i^{-1}$, e $b$ é elemento do bordo de $A_{k-1}$ ou $A_{k-1}^{-1}$.
% \end{defi}
% O conceito de complexos celulares está intimamente relacionado ao de \hyperref[triangulacao-def]{triangulação} e também \hyperref[complexo-simplicial-def]{complexos simpliciais}.
% Diversos complexos celulares representam a mesma intuição geométrica. Isso motiva considerarmos a seguinte definição.
% \begin{defi}
%     Dado um complexo celular $K=(F,E,\mathcal{B})$, considere as seguintes construções:
%     \begin{enumerate}
%         \item fixe uma aresta $a \in E$, tome $b,c \notin E$ e defina $K'=(F,E\cup\{b,c\}\setminus\{a\},\mathcal{B}')$; a função $\mathcal{B}'$ é definida substituindo cada ocorrência de $a$ em um bordo por $b~c$, cada ocorrência de $a^{-1}$ por $c^{-1}~b^{-1}$ e mantendo o restante inalterado;
%         \item  fixe uma face $A\in F$ cujo bordo contenha ao menos 4 arestas, ou seja, $\mathcal{B}(A) = a_1\ldots a_k$ para $a_1,\ldots,a_k \in E\cap E^{-1}$ e $k\geq 4$; tome $x\notin E$, $X,Y \notin F$ e defina $K''=(F\cup\{X,Y\}\setminus\{A\},E\cup\{x\},\mathcal{B}'')$, em que $\mathcal{B}''(X) = a_1~x^{-1}~a_4$, $\mathcal{B}''(Y)= a_2~a_3~x$ e $\mathcal{B}''$ coincide com $\mathcal{B}$ nas demais faces.
%     \end{enumerate}
% Assim, definimos uma relação de equivalência de complexos celulares gerada pelas relações $K\sim K'$ e $K\sim K''$ para $K$ um complexo celular qualquer e $K'$ e $K''$ obtidos como nas construções acima.
% \end{defi}
% Todo complexo celular conexa $K=(F,E,\mathcal{B})$ define uma 2-variedade com bordo da seguinte forma: escreva $F= \bigcup_{n \geq 2} F_n$, onde os elementos de $F_n$ são faces com $n$ arestas no bordo. Considere a união disjunta de $|E|$ intervalos $[0,1]$ e $|F_n|$ polígonos de $n$ lados, para todo $n\geq 3$ (polígonos regulares em $\mathbb{R}^2$ com centro na origem e apótema 1, por exemplo). Então é simples ver que $\mathcal{B}$ descreve uma relação de equivalência $R$ que relaciona cada aresta de um polígono com arestas de $K$. Tomemos o espaço quociente $X=Y/R$. A condição de que cada aresta de $K$ pertença ao bordo de no máximo 2 faces garante que $X$ seja de Hausdorff e localmente homeomorfo a um aberto do semiplano $\mathbb{H}^2$. Se cada aresta pertencer a exatamente 2 faces, garantimos ainda que $X$ seja localmente homeomorfo a um aberto do plano.
%%%%%%%%%%%%%%%%%%%%%%%%%%%%%%%%%%%%%%%%%%%%

\begin{defi}[Região poligonal, orientação e etiquetagem]
    Uma \textbf{região poligonal} com $n$ lados é um simplexo $P = [v_1,\ldots,v_n]$, onde $v_1,\ldots, v_n \in \mathbb{S}^2$. Por convenção, ordenamos os pontos $v_1,\ldots, v_n$ em ordem anti-horária.

    Sejam $P_1,\ldots, P_k$ regiões poligonais dadas. Denotemos por $\partial_j P_i$ o conjunto de $j$-faces de $P_i$, $1\leq i\leq k$. Então, uma \textbf{orientação} nas arestas da união de regiões poligonais $\bigsqcup_{i=1}^k P_i$ é uma função $\mathcal{O}_i: \bigsqcup_{i=1}^k\partial_1 P_i\to \bigsqcup_{i=1}^k\partial_0 P_i$ tal que $\mathcal{O}(a) \in \partial a$ para toda aresta $a$ de $P_i$, $1\leq i\leq k$. Ou seja, é a escolha de um ``ponto inicial'' para cada aresta de cada região poligonal $P_i$.

    Já uma \textbf{etiquetagem} na união de regiões poligonais $\bigsqcup_{i=1}^k P_i$ é uma função $L: \bigsqcup_{i=1}^k\partial_1 P_i\to \Lambda$, onde $\Lambda \neq \varnothing$ é dito o conjunto de \textbf{etiquetas}.
\end{defi}

\begin{defi}[Transformação linear positiva e espaço obtido por colagem de arestas]
    Dadas duas arestas $A = [v_i, v_{i+1}]$ e $B = [v_j, v_{j+1}]$ com orientação $\mathcal{O}$ fixada, seja $\overline{\mathcal{O}}$ a orientação inversa. Isto é, $\mathcal{O}(A) \neq \overline{\mathcal{O}}(A)$ e $\mathcal{O}(B) \neq \overline{\mathcal{O}}(B)$. Definimos a \textbf{transformação linear positiva} de $A$ sobre $B$ como a função $h: A\to B$ que mapeia $(1-t) \mathcal{O}(A) + t \overline{\mathcal{O}}(A)$ em $(1-t) \mathcal{O}(B) + t \overline{\mathcal{O}}(B)$ para todo $t \in [0,1]$.
    
    Considere regiões poligonais $P_1,\ldots, P_k$, uma orientação $\mathcal{O}$ e uma etiquetagem $L$ de $\bigsqcup_{i=1}^k P_i$. Defina o espaço
    \[X = \bigsqcup_{i=1}^k P_i/\sim\]
    em que $x \sim y$ se, e somente se, $x = y$ ou então $x \in A$, $y \in B$ e $h(x) = y$, onde $A$ e $B$ são arestas com a mesma etiqueta e $h$ é a transformação linear positiva de $A$ sobre $B$. Isto é,
    \begin{align*}
        x \sim y \;\Longleftrightarrow \;
        &x=y\text{ ou }\exists a \in \Lambda, \exists A,B \in L^{-1}(a), \exists t\in [0,1]:\\ 
        &x = (1-t) \mathcal{O}(A) + t \overline{\mathcal{O}}(A), 
        y = (1-t) \mathcal{O}(B) + t \overline{\mathcal{O}}(B)
    \end{align*}
    
    Então, dizemos que o espaço quociente $X$ (bem como qualquer espaço topológico homeomorfo) é \textbf{obtido das regiões poligonais $P_1,\ldots, P_k$ por colagem de arestas} de acordo com a orientação $\mathcal{O}$ e a etiquetagem $L$.
\end{defi}

Note que quaisquer duas regiões poligonais com $n$ lados são homeomorfas. Mais do que isso, o espaço quociente $X$ obtido da região poligonal $P = [v_1,\ldots,v_n]$ por colagem de arestas é totalmente determinado, a menos de homeomorfismo, pelo símbolo
\[w = a_{i_1}^{\varepsilon_1} \ldots a_{i_n}^{\varepsilon_n},\]
onde $a_{i_1}$ é a etiqueta de $[v_1, v_2]$, $a_{i_2}$ é a etiqueta de $[v_2, v_3]$, e assim por diante ($a_{i_n}$ é a etiqueta de $[v_n, v_1]$), e $\varepsilon_i = \pm 1$, a depender se a orientação fixada em $P$ coincide com a orientação na ordenação dos vértices. Por exemplo, se o ponto inicial em $[v_1, v_2]$ é $v_1$, então $\varepsilon_1 = +1$. O símbolo $w$ é dito o \textbf{esquema de etiquetagem} para $P$ (com respeito à orientação e etiquetagem fixadas).

Para um espaço $X$ obtido pela colagem de arestas das regiões poligonais $P_1,\ldots, P_k$, o esquema de etiquetagem é dado por $w_1,\ldots, w_k$, onde $w_i$ é o esquema de etiquetagem de $P_i$ para cada $1\leq i \leq k$.

\begin{titlemize}{Lista de consequências}
    \item \hyperref[construcoes-regiao-poligonal-prop]{Construções com Regiões Poligonais}%;\\ %'consequencia1' é o label onde o conceito Consequência 1 aparece
    %\item \hyperref[]{}
\end{titlemize}
%---------------------------------------------------------------------------------------------------------------------!Draft!-----------------------------------------------------------------------------------------------------------------
\subsection{Construções com Regiões Poligonais}
\label{construcoes-regiao-poligonal-prop}
\begin{titlemize}{Lista de dependências}
	\item \hyperref[regiao-poligonal-def]{Regiões Poligonais}%;\\ %'dependencia1' é o label onde o conceito Dependência 1 aparece (--à arrumar um padrão para referencias e labels--) 
	%\item \hyperref[variedade-def]{Variedades Topológicas};\\
% quantas dependências forem necessárias.
\end{titlemize}

\begin{lemma}\label{varias-etiquetagens-lemma}
    Sejam $P_1,\ldots, P_k$ regiões poligonais, e seja $w_1,\ldots, w_k$ um esquema de etiquetagem. Para cada $1\leq i\leq n$, considere o espaço quociente $X_i = P_i/\sim_i$ (com respeito ao esquema de etiquetagem $w_i$) e, depois, realize a colagem dos espaços resultantes da seguinte forma:
    \[\bigsqcup_{i=1}^k X_i/\approx\]
    onde
    % \begin{align*}
    %     [x]\approx [y] ~\Longleftrightarrow ~ &[x]=[y]\text{ ou }\exists a\in \Lambda, \exists i,j\leq k, \exists A \in \partial_1 P_i, \exists B \in \partial_1 P_j, \exists t\in [0,1]:\\
    %     &L(A) = L(B) = a,\\
    %     &x = (1-t) \mathcal{O}(A) + t \overline{\mathcal{O}}(A), 
    %     y = (1-t) \mathcal{O}(B) + t \overline{\mathcal{O}}(B).
    % \end{align*}
    \[[x]_i \approx [y]_j ~\Longleftrightarrow x \sim y.\]
    Então, o espaço resultante é homeomorfo ao espaço $X = \bigsqcup_{i=1}^k P_i/\sim$ obtido pelo esquema de etiquetagem $w_1,\ldots, w_k$.
    \begin{dem}
        Note que, se $\Tilde{x} \in [x]_i \in X_i$, então o único $t \in [0,1]$ tal que $x = (1-t) \mathcal{O}(A) + t \overline{\mathcal{O}}(A)$ também satisfaz $\Tilde{x} = (1-t) \mathcal{O}(\Tilde{A}) + t \overline{\mathcal{O}}(\Tilde{A})$ para alguma aresta $\Tilde{A}$ com mesma etiqueta que $A$. Aplicando o mesmo raciocínio para $y$, concluímos que $\sim$ está bem definido.
        
        Pelo $1^o$ Teorema do Homomorfismo, o espaço quociente é unicamente determinado por sua propriedade universal. Fixemos um espaço topológico $Y$ e uma função contínua $\phi: \bigsqcup_{i=1}^k X_i \to Y$ constante nas classes de equivalência de $\approx$. Então, $\phi$ é induzida por uma função $\phi_0: \bigsqcup_{i=1}^k P_i \to Y$ constante nas classes de equivalência de $\sim$, e, desse modo, induz uma função $\overline{\phi}: \bigsqcup_{i=1}^k P_i/\sim \to Y$. Sendo $\pi: \bigsqcup_{i=1}^k P_i \to \bigsqcup_{i=1}^k X_i/\approx$ a projeção canônica, vale que $\overline{\phi}\circ \pi = \phi$, e concluímos.
    \end{dem}
\end{lemma}

Vamos definir algumas construções possíveis para alterar o esquema de etiquetagem de modo que o espaço quociente obtido seja homeomorfo (como pode ser facilmente verificado, de maneira análoga à demonstração do lema anterior). Vamos utilizar estas construções para classificar as superfícies a menos de homeomorfismo.

\begin{prop}
    Os seguintes procedimentos sobre os esquemas de etiquetagem resultam em espaços quociente homeomorfos:
    \begin{enumerate}
        \item \textbf{Recorte:} dados um esquema de etiquetagem $w = a_{i_1}^{\varepsilon_1} \ldots a_{i_n}^{\varepsilon_n}$, $1\leq j\leq n$ uma etiqueta $b$ não utilizada anteriormente e $\varepsilon = \pm 1$, substituímos $w$ por $w_1, w_2$, onde $w_1 = a_{i_1}^{\varepsilon_1} \ldots a_{i_j}^{\varepsilon_j} b^{\varepsilon}$ e $w_2 = b^{-\varepsilon} a_{i_{j+1}}^{\varepsilon_{j+1}} \ldots a_{i_n}^{\varepsilon_n}$.
        
        Geometricamente, estamos dividindo uma região poligonal em duas, adicionando uma aresta no interior da região poligonal original. Como utilizamos a mesma etiqueta nas arestas ``novas'' das regiões poligonais resultantes, o espaço quociente não se altera, pois tais arestas serão identificadas.
    
        \item \textbf{Colagem:} o processo contrário ao de recorte. Dado um esquema de etiquetagem $w_1, w_2$, onde $w_1 = a_{i_1}^{\varepsilon_1} \ldots a_{i_j}^{\varepsilon_j} b$ e $w_2 = b^{-1} a_{i_{j+1}}^{\varepsilon_{j+1}} \ldots a_{i_n}^{\varepsilon_n}$, suponha que a etiqueta $b$ só tenha uma ocorrência em $w_1$ e uma ocorrência em $w_2$. Então, podemos substituir $w_1, w_2$ por $w = a_{i_1}^{\varepsilon_1} \ldots a_{i_n}^{\varepsilon_n}$.
    
        \item \textbf{Endireitar de arestas:} dado um esquema de etiquetagem $w$, suponha que exista uma sequência $y = c_1^{\delta_1} \ldots c_k^{\delta_k}$ tal que $c_i \neq c_j$ para todos $i\neq j$, e tal que as únicas ocorrências das etiquetas $c_i$ são em uma sequência $y$ ``contida'' em $w$. Então, podemos substituir todas as ocorrências da sequência $y$ por $b^{\varepsilon}$, onde $b$ é uma etiqueta não utilizada anteriormente e $\varepsilon = \pm 1$. Uma sequência $y$ nessas condições é dita \textbf{removível}.
    
        Geometricamente, estamos substituindo uma sequência de lados da região poligonal (que sempre aparecem juntos no esquema de etiquetagem) por apenas um lado.
    
        \item \textbf{Dobradura de arestas:} o processo contrário ao de endireitar de arestas. Dado um esquema de etiquetagem $w$, uma etiqueta $b$ cujas ocorrências em $w$ sempre possuem mesma orientação $\varepsilon$ e uma sequência $y$ com etiquetas não utilizadas em $w$, substituímos todas as ocorrências de $b^{\varepsilon}$ por $y$.

        \item \textbf{Troca de etiquetas:} podemos substituir todas as ocorrências de uma etiqueta $a$ por outra etiqueta $c$ não utilizada. Disso é imediato que podemos trocar as ocorrências de quaisquer duas etiquetas dadas $a$ e $b$ (substituindo $a$ por $c$, depois $b$ por $a$ e, por fim, $c$ por $b$).
        
        \item \textbf{Troca de orientação:} podemos inverter o sinal da orientação de todas as ocorrências de uma etiqueta $b$ fixada. Isso segue de que o espaço quociente é definido por meio de transformações lineares positivas entre estes lados (que não se alteram, caso a orientação de todos estes lados seja invertida).
        
        \item \textbf{Permutação cíclica:} um esquema de etiquetagem $w$ representa o mesmo espaço quociente, caso comecemos a ordenar os pontos a partir de pontos distintos. Desse modo, podemos substituir $w = a_{i_1}^{\varepsilon_1} \ldots a_{i_n}^{\varepsilon_n}$ por\break $w' = a_{i_n}^{\varepsilon_n} a_{i_1}^{\varepsilon_1} \ldots a_{i_{n-1}}^{\varepsilon_{n-1}}$ (bem como qualquer permutação cíclica da sequência).
    
        \item \textbf{Inversão formal:} dado um esquema de etiquetagem $w$, caso usássemos a ordenação dos pontos em sentido horário, o espaço resultante seria o mesmo, a menos de uma reflexão (em especial, seriam homeomorfos). Assim, podemos substituir $w = a_{i_1}^{\varepsilon_1} \ldots a_{i_n}^{\varepsilon_n}$ por $w = a_{i_n}^{-\varepsilon_n} \ldots a_{i_1}^{-\varepsilon_1}$.

        \item \textbf{Cancelamento:} dado um esquema de etiquetagem $w$, suponha que existam duas sequências $y_0, y_1$ com comprimento maior ou igual a 2 e uma etiqueta $c$ sem ocorrências em $y_0$ e em $y_1$ tais que $w = [y_0] cc^{-1} [y_1]$. Então, podemos substituir $w$ por $w' = [y_0 y_1]$.

        \item \textbf{Adjunção:} dado um esquema de etiquetagem $w$, suponha que existam duas sequências $y_0, y_1$ com comprimento maior ou igual a 2 tal que $w = [y_0 y_1]$, e seja $c$ uma etiqueta sem ocorrências em $y_0$ e em $y_1$. Então, podemos substituir $w$ por $w' = [y_0] cc^{-1} [y_1]$.
    \end{enumerate}
\end{prop}

\begin{defi}
    Um esquema de etiquetagem é \textbf{próprio} se cada etiqueta possui exatamente 2 ocorrências. Já este é dito \textbf{irredutível} se não há ocorrência de uma sequência da forma $c c^{-1}$ ou $c^{-1} c$ para alguma etiqueta $c$.
    
    Dizemos que dois esquemas de etiquetagem próprios $w_1,\ldots,w_k$ e $\Tilde{w}_1,\ldots,\Tilde{w}_l$ são \textbf{equivalentes} se é possível obter um a partir do outro por meio das construções da proposição anterior.
\end{defi}

Como tais construções são reversíveis, isto define uma relação de equivalência entre os esquemas de etiquetagem próprios. Além disso, por conta da proposição anterior, dois esquemas de etiquetagem equivalentes definem espaços topológicos homeomorfos.

É interessante nos restringirmos a analisar esquemas de etiquetagem próprios pois, se realizamos a colagem de $k\geq 1$ arestas, o espaço quociente não é uma superfície para $k\neq 2$. Para ver isso, note que, nesse caso, qualquer ponto em tal aresta possui uma vizinhança homeomorfa à colagem de $k$ hemisférios de um disco $D^2$ (identificando o equador de todos os hemisférios), o que não é localmente euclidiano para $k\neq 2$ (no caso em que $k=1$, teríamos uma variedade com bordo).

Também vamos nos restringirmos a analisar esquemas de etiquetagem de comprimento $4$ ou maior.

\begin{defi}
    Seja $w$ um esquema de etiquetagem próprio (de uma única região poligonal). Se cada etiqueta aparece uma vez com a orientação $+1$ e uma vez com a orientação $-1$, dizemos que $w$ é do \textbf{tipo toro}. Caso contrário, $w$ é dito do \textbf{tipo projetivo}.
\end{defi}

\begin{titlemize}{Lista de consequências}
	\item \hyperref[forma-normal-thm]{Teorema de Forma Normal}
\end{titlemize}
%---------------------------------------------------------------------------------------------------------------------!Draft!-----------------------------------------------------------------------------------------------------------------
\subsection{Caso A do Teorema de Forma Normal}
\label{forma-normal-caso-a-thm}
\begin{titlemize}{Lista de dependências}
	\item \hyperref[regiao-poligonal-def]{Regiões Poligonais};\\
	\item \hyperref[construcoes-regiao-poligonal-prop]{Construções com Regiões Poligonais};\\
% quantas dependências forem necessárias.
\end{titlemize}

Seja $w_1,\ldots,w_k$ um esquema de etiquetagem próprio, e suponha que toda etiqueta aparece uma vez com a orientação $+1$ e uma vez com a orientação $-1$.

% \begin{titlemize}{Lista de consequências}
% 	\item \hyperref[forma-normal-caso-a-thm]{Caso A do Teorema de Forma Normal};\\
% 	\item \hyperref[forma-normal-caso-b-thm]{Caso B do Teorema de Forma Normal}
% \end{titlemize}
\input{conteudo/forma-normal-caso-b-thm}

% Cada novo assunto deve ser adcionado no corpo do texto, como explicado no arquivo Alg.Top-Wiki.tex.
