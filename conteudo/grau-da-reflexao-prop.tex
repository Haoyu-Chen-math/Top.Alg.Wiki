\subsection{Grau da reflexão} %afirmação aqui significa teorema/proposição/colorário/lema
\label{grau-da-reflexao-prop}
\begin{titlemize}{Lista de dependências}
    \item \hyperref[homologia-singular-def]{Homologia singular};\\
    \item \hyperref[homomorfismo-de-homologias-singulares-induzido-prop]{Homomorfismo de homologias singulares induzido};\\
    \item \hyperref[homologia-singular-de-um-espaco-contratil-prop]{Homologia singular de um espaço contrátil};\\
    \item \hyperref[sequencia-de-mayer-vietoris-prop]{Sequência de Mayer-Vietoris};\\
    \item \hyperref[homologia-singular-de-S1-prop]{Homologia singular da circunferência};\\
    \item \hyperref[grupo-de-homologia-singular-de-n-esfera-prop]{Grupo de homologia singular de n-esfera};\\
    \item \hyperref[grau-de-funcoes-em-esferas-def]{Grau de funções em esferas};\\
    \item \hyperref[propriedades-de-grau-de-funções-prop]{Propriedade de grau de funções em esferas}
\end{titlemize}

\begin{defi}
    Uma \textbf{reflexão} é uma função de $r:\mathbb{S}^n\rightarrow \mathbb{S}^n$ que troca o sinal de uma coordenada. Quando a troca de sinal ocorrer na i-ésima coordenada, diremos que $r$ é uma \textbf{reflexão da i-ésima coordenada}. 
\end{defi}

\begin{lemma}
    A reflexão da primeira coordenada na esfera $\mathbb{S}^n$ tem grau $-1$.
\end{lemma}

\begin{dem}
    Provamos o lema por indução sobre a dimensão n.

    Base de indução: consideramos a reflexão $r:\mathbb{S}^1\rightarrow \mathbb{S}^1$ dada por $r(x,y)=(-x,y)$. Pelo exemplo \ref{homologia-singular-de-S1-prop}, $H_1(\mathbb{S}^1)$ é gerado pela classe de homologia do 1-ciclo $z_1=c_1+c_2$, onde $c_1$ e $c_2$ são os 1-simplexos singulares correspondentes aos arcos hemisféricos da circunferência, orientados no sentido anti-horário. A função $r$ mapeia $c_1$ a $-c_1$ e $c_2$ a $-c_2$. Isso mostra que $r\circ (z_1)=-z_1$ e, por conseguinte, o homomorfismo $r_*:H_1(\mathbb{S}^1)\rightarrow H_1(\mathbb{S}^1)$ tem grau $-1$.

    Passo de indução: Suponhamos que o resultado valha para a dimensão $n-1\ge 1$ e seja $r:\mathbb{S}^n\rightarrow \mathbb{S}^n$ a reflexão da primeira coordenada. Considerando a inclusão $e:\mathbb{S}^{n-1}\hookrightarrow \mathbb{S}^n$ dada por $(x_1,...,x_n)\mapsto (x_1,...,x_n,0)$ e considerando os abertos $U=\mathbb{S}^n\setminus\{(0,...,0,-1)\}$ e $V=\mathbb{S}^n\setminus \{(0,...,0,1)\}$. Note que a inclusão $i:\mathbb{S}^{n-1}\hookrightarrow U\cap V$ é uma equivalência de homotopia. Como o equador $\mathbb{S}^{n-1}$ é invariante por $r$ (i.e., $r(\mathbb{S}^{n-1})\subseteq\mathbb{S}^{n-1}$), a função restrição $r|:\mathbb{S}^{n-1}\rightarrow \mathbb{S}^{n-1}$ é uma reflexão da primeira coordenada bem definida. Pela hipótese de indução, $r|$ tem grau $-1$. Como $n\ge 2$ e $U$ e $V$ são contráteis, o seguinte trecho da sequência de Mayer-Vietoris 
    \[0=H_{n}(U)\oplus H_n(V)\rightarrow H_n(\mathbb{S}^n)\xrightarrow{\delta}H_{n-1}(U\cap V)\rightarrow H_{n-1}(U)\oplus H_{n-1}(V)=0\]
    é exata, o que mostra que o homomorfismo $\delta$ é um isomorfismo. Com isso, obtemos o diagrama comutativo seguinte 
    % https://q.uiver.app/#q=WzAsNixbMCwwLCJIX3tufShcXG1hdGhiYntTfV5uKSJdLFsxLDAsIkhfe24tMX0oVVxcY2FwIFYpIl0sWzIsMCwiSF9uKFxcbWF0aGJie1N9XntuLTF9KSJdLFswLDEsIkhfe259KFxcbWF0aGJie1N9Xm4pIl0sWzEsMSwiSF97bi0xfShVXFxjYXAgVikiXSxbMiwxLCJIX24oXFxtYXRoYmJ7U31ee24tMX0pIl0sWzAsMSwiXFxkZWx0YSIsMCx7InN0eWxlIjp7InRhaWwiOnsibmFtZSI6ImFycm93aGVhZCJ9fX1dLFswLDMsInJfKiIsMl0sWzEsNCwicnxfKiIsMl0sWzMsNCwiXFxkZWx0YSIsMix7InN0eWxlIjp7InRhaWwiOnsibmFtZSI6ImFycm93aGVhZCJ9fX1dLFsyLDUsInJ8XyoiXSxbMiwxLCJpXyoiLDIseyJzdHlsZSI6eyJ0YWlsIjp7Im5hbWUiOiJhcnJvd2hlYWQifX19XSxbNSw0LCJpXyoiLDAseyJzdHlsZSI6eyJ0YWlsIjp7Im5hbWUiOiJhcnJvd2hlYWQifX19XV0=
\[\begin{tikzcd}
	{H_{n}(\mathbb{S}^n)} & {H_{n-1}(U\cap V)} & {H_n(\mathbb{S}^{n-1})} \\
	{H_{n}(\mathbb{S}^n)} & {H_{n-1}(U\cap V)} & {H_n(\mathbb{S}^{n-1})}
	\arrow["\delta", tail reversed, from=1-1, to=1-2]
	\arrow["{r_*}"', from=1-1, to=2-1]
	\arrow["{r|_*}"', from=1-2, to=2-2]
	\arrow["{i_*}"', tail reversed, from=1-3, to=1-2]
	\arrow["{r|_*}", from=1-3, to=2-3]
	\arrow["\delta"', tail reversed, from=2-1, to=2-2]
	\arrow["{i_*}", tail reversed, from=2-3, to=2-2]
\end{tikzcd}\]
    onde flechas horizontais são isomorfismo.
    Seja $\overline{z}$ um gerador de $H_n(\mathbb{S}^n)$. Então, 
    \begin{align*}
        r_*(\overline{z})=&\delta^{-1}\circ r|_*\circ \delta(\overline{z})=\delta^{-1}\circ i_*\circ r|_*\circ i_*^{-1}\circ \delta(\overline{z})\\
        =& -\delta^{-1}\circ i_*\circ i_*^{-1}\circ \delta(\overline{z})=-\overline{z}.
    \end{align*}
    Isso demonstra que $deg(r)=-1$. Assim concluímos o passo de indução e a prova do lema.
\end{dem}

\begin{corol}
    Qualquer reflexão na esfera $\mathbb{S}^n$ tem grau -1.
\end{corol}

\begin{dem}
    Para cada $1\le i\le n+1$, seja $r_i: \mathbb{S}^n\rightarrow \mathbb{S}^n$ a reflexão da i-ésima coordenada. Pelo lema anterior, $deg(r_1)=-1$. Para $i\ge 2$, temos $r_i=h_i\circ r_1\circ h_i$, onde $h_i:\mathbb{S}^n\rightarrow \mathbb{S}^n$ é o homeomorfismo que permuta a primeira e a i-ésima coordenadas. Pela propriedades de grau de funções, segue que 
    \[deg(r_i)=deg(h_i)deg(r_1)deg(h_i)=deg(h_i)^2deg(r_1)=deg(r_1)=-1\]
\end{dem}

\begin{titlemize}{Lista de consequências}
    \item \hyperref[grau-de-antipoda-prop]{Grau de antípoda}.\\
	%\item \hyperref[]{}
\end{titlemize}
