\documentclass{article}
\title{Topologia Algébrica}
%\usepackage{import}
\usepackage{amsmath} %propósitos gerais
\usepackage{amssymb} % comandos como \mathbb
\usepackage{amsthm} % criar novos teoremas

\newif\ifplastex
\plastexfalse
\ifplastex
\else
\usepackage{mdframed} %criar caixas em volta de ambientes
\fi
%-----------------------------------------------------------------------------------------------!!!!Adicionar pacotes apenas acima dessa linha!!!!--------------------------------------------------------------------------------------------
\usepackage{hyperref} %referencias internas e externas


%novos estilos
\ifplastex

\else
\mdfdefinestyle{MyFrame}{%
	innertopmargin=\baselineskip,
	innerbottommargin=\baselineskip,
	innerrightmargin=20pt,
	innerleftmargin=20pt}
\fi

%novos ambientes e teoremas
\theoremstyle{definition}
\newtheorem{defi}{Definição}

\theoremstyle{plain}
\newtheorem{thm}{Teorema}
\newtheorem{prop}{Proposição}
\newtheorem{lemma}{Lema}
\newtheorem{af}{Afirmação}
\newtheorem{corol}{Corolário}
\newtheorem{ex}{Exemplo}

\theoremstyle{remark}
\newtheorem{nota}{Nota}
\ifplastex


\newenvironment{titlemize}[1]{%
	\textbf{{#1}}
	\begin{itemize}}
	{\end{itemize}}

\else

\newenvironment{titlemize}[1]{%
	\begin{mdframed}[style=MyFrame]
	\textbf{{#1}}
	\begin{itemize}}
	{\end{itemize} \end{mdframed}}

\fi
\newenvironment{dem}{
	\begin{proof}[{\bf Demonstração:}]}
	{\end{proof}}
%novos comandos
\usepackage{quiver}
%comandos renovados



\begin{document}
\maketitle
%-------------------------------------------------------------------------------------------------------------!Draft!-------------------------------------------------------------------------------------------------------------------------
\section{Alguns espaços topológicos importantes}
\label{alguns-espacos-topologicos-importantes}
Nesta seção, apresentamos alguns espaços topológicos importantes para o estudo da geometria e da topologia.

\subsection{Esfera}
\label{esfera-def}

\begin{defi}
     Dado $n\geq 0$, a \textbf{$n$-esfera} (unitária com centro na origem), denotada por $\mathbb{S}^n$, é o subespaço topológico de $\mathbb{R}^{n+1}$ definido por 
     \[\mathbb{S}^n=\{(x_0,...,x_n)\in \mathbb{R}^{n+1}:x_0^2+...+x_n^2=1\}.\]
\end{defi}

\begin{prop}
    Seja $N=(0,...,0,1)\in \mathbb{R}^{n+1}$ um ponto. A projeção estereográfica definida por 
    \begin{align*}
        p_N:\mathbb{S}^n\setminus \{N\}&\longrightarrow \mathbb{R}^n\\
        (x_0,...,x_n)&\longmapsto \frac{1}{1-x_n}(x_0,...,x_{n-1})
    \end{align*}
    é um homeomorfismo.
\end{prop}
\begin{dem}
    Aqui, denotamos $x_0^2+...+x_n^2$ por $||(x_0,...,x_n)||_2^2$. Como $p_N$ é contínua em cada coordenada, então $p_N$ é contínua. Agora, definimos uma função da seguinte forma 
    \begin{align*}
        f:\mathbb{R}^n&\longrightarrow \mathbb{S}^n\\
        X=(X_1,...,X_n)&\longmapsto \frac{1}{||X||_2^2+1}(2X_1,...,2X_{n},||X||_2^2-1).
    \end{align*}
    Essa função é bem definida, pois 
    \[\frac{1}{(||X||_2^2+1)^2}(4X_1^2+...+4X_n^2+(||X||_2^2-1)^2)=\frac{4||X||_2^2+(||X||_2^2-1)^2}{(||X||_2^2+1)^2}=1.\]
    Além disso, $f$ é contínua em cada coordenada, logo $f$ é contínua. Por um lado, temos 
    \begin{align*}
        f\circ p_N(x_0,...,x_n)&=\frac{1}{1-x_n}\cdot \frac{(2x_0,...,2x_{n-1},(1-x_n)\cdot(||p_N(x_0,...,x_{n}||_2^2-1))}{||p_N(x_0,...,x_n)||_2^2+1}\\
        &=\frac{1}{1-x_n}\cdot\frac{(2x_0,...,2x_{n-1},(1-x_n)\cdot(||p_N(x_0,...,x_{n}||_2^2-1))}{(\frac{1}{(1-x_n)^2}\cdot ||(x_0,...,x_{n-1})||_2^2)+1}\\
        &=\frac{1}{1-x_n}\cdot\frac{(2x_0,...,2x_{n-1},\frac{1-x_n^2}{(1-x_n)}-(1-x_n))}{\frac{(1-x_n^2)}{(1-x_n)^2}+1}\\
        &=\frac{(1-x_n)}{(1-x_n^2+(1-x_n)^2)}\cdot(2x_0,...,2x_{n-1},\frac{1-x_n^2-(1-x_n)^2}{(1-x_n)})\\
        &=\frac{1}{2}\cdot(2x_0,...,2x_{n-1},2x_n)\\
        &=(x_0,...,x_n).
    \end{align*}
    E por outro lado, temos 
    \begin{align*}
        p_N\circ f(X)&=\frac{1}{1-\frac{||X||_2^2-1}{||X||_2^2+1}}\frac{2X}{||X||_2^2+1}\\
        &=\frac{2X}{||X||_2^2+1-||X||_2^2+1}\\
        &=\frac{2X}{2}\\
        &=X.
    \end{align*}
    Isso mostra que $p_N$ é um homeomorfismo.
\end{dem}

\subsection{Toro}
\label{toro-def}
\begin{titlemize}{Lista de dependências}
	\item \hyperref[esfera-def]{Esfera}. %'dependencia1' é o label onde o conceito Dependência 1 aparece (--à arrumar um padrão para referencias e labels--) 
% quantas dependências forem necessárias.
\end{titlemize}
\begin{defi}
     Dado $n\geq 0$, o \textbf{$n$-toro}, denotado por $\mathbb{T}^n$, é definido como o espaço topológico produto $\mathbb{S}^1\times \ldots \times \mathbb{S}^1$ ($n$ fatores).
\end{defi}
\subsection{Disco}
\label{disco-def}
\begin{titlemize}{Lista de dependências}
	\item \hyperref[esfera-def]{Esfera}.
\end{titlemize}
\begin{defi}
     Dado $n\geq 0$, o \textbf{$n$-disco} (unitário com centro na origem), denotado por $D^n$, é o subespaço topológico de $\mathbb{R}^{n}$ definido por 
     \[D^n=\{(x_1,...,x_n)\in \mathbb{R}^{n}:x_1^2+...+x_n^2\le 1\}.\]
\end{defi}
É fácil ver que o bordo do $n$-disco é a ($n-1$)-esfera: $\partial D^n = \mathbb{S}^{n-1}$, e o interior do $n$-disco é a bola aberta $\{(x_1,...,x_n)\in \mathbb{R}^n:x_1^2+...+x_n^2<1\}$.
%%% Local Variables:
%%% mode: LaTeX
%%% TeX-master: "../Alg.Top-Wiki"
%%% End:

%-------------------------------------------------------------------------------------------------------------!Draft!-------------------------------------------------------------------------------------------------------------------------
\section{Topologia quociente}
\label{topologia-quociente}
Um assunto que aparece de forma recorrente na topologia algébrica é o conceito de topologia quociente, que exploraremos a seguir. 

\subsection{Topologia Quociente}
\label{topologia-quociente-def}
% \begin{titlemize}{Lista de dependências}
% 	\item \hyperref[topologia-final]{Topologia Final}; 
% \end{titlemize}
\begin{defi}[Topologia Quociente]
	Seja \(X\) um espaço topológico e \(\sim\) uma relação de equivalência em \(X\).
	Podemos conferir ao espaço \(X/\sim\) uma estrutura de espaço topológico da seguinte maneira. Considere a função projeção
	\begin{align*}
		\pi:X&\to X/\sim;\\
		x&\mapsto [x].
	\end{align*}
	Podemos fazer com que \(\pi\) seja uma função contínua munindo \(X/\sim\) com a \emph{topologia final} com relação à \(\pi\). Isto é, um subconjunto de \(X/\sim\) é aberto se, e somente se, sua pré-imagem por $\pi$ é aberto de \(X\).
\end{defi}

Varios exemplos importantes de espaços topológicos com os quais trabalharemos no estudo de topologia algébrica podem ser construídos como espaços quocientes. Em particular, uma construção muito útil é a de tomar o quociente de um espaço por um subespaço, como explicado na seguinte definição.
\begin{defi}[Quociente por um subespaço]
	Seja \(X\) um espaço topológico e \(A \subseteq X\) um subespaço. Definimos a seguinte relação binária, \(\sim_A\):\\
    \centerline{
	\(a\sim_A b\) se e somente se \(a=b\) ou \(a,b\in A\).}\\ Essa relação é de equivalência, e assim definimos \(X/A = X/\sim_A\). 
\end{defi}

Vejamos alguns exemplos simples.

\begin{ex}
    \begin{itemize}
        \item O círculo \(\mathbb{S}^1 = \mathbb{T}^1\) pode ser construído como \(I/\{0,1\}\), onde \(I=[0,1]\).
        \item Mais geralmente, o $n$-toro $\mathbb{T}^n$ pode ser construído como $[0,1]^n/\sim$, onde $\sim$ é a relação de equivalência que identifica $x = (x_1,\ldots,x_n), y = (y_1,\ldots,y_n) \in [0,1]^n$ se existe $1\leq i\leq n$ tal que $x_j = y_j$ para todo $j \neq i$ e $\{x_i,x_j\} = \{0,1\}$, ou então se $x=y$.
    \end{itemize}
\end{ex}

\begin{titlemize}{Lista de consequências}
    \item \hyperref[funcao-continua-em-topologia-quociente-prop]{Função contínua em topologia quociente}
	\item \hyperref[topologia-quociente-hausdorff-thm]{Espaços quocientes Hausdorff}
\end{titlemize}


% onde conteudos.tex é o nome do arquivo tex que voce quer incluir nessa secção.
\subsection{Função contínua em topologia quociente}
\label{funcao-continua-em-topologia-quociente-prop}
\begin{titlemize}{Lista de dependências}
	\item \hyperref[topologia-quociente-def]{Topologia quociente}; 
\end{titlemize}

\begin{prop}
    Sejam $X,Y$ espaços topológicos. E seja $\sim$ uma relação de equivalência em $X$. Uma função $f:(X/\sim) \longrightarrow Y$ é contínua se e somente se $f\circ \pi$ é contínua, onde $\pi$ é a função projeção associada ao quociente.
\end{prop}
\begin{dem}
    Por um lado, suponhamos que $f$ seja contínua. Como a composição de funções contínuas é contínua, a função $f\circ \pi$ também seré contínua.

    Por outro lado, suponhamos que $f\circ \pi$ seja contínua. Seja $V\subseteq Y$ um aberto, pela hipóteses, $\pi^{-1}(f^{-1}(V))$ é um aberto. Agora, pela definição de topologia quociente, $f^{-1}(V)$ é um aberto em $X/\sim$, o que implica que $f$ é contínua. 
\end{dem}

Alguns exemplos importantes de espaços quociente são os seguintes.
\subsection{Espaço Projetivo}
\label{espaco-projetivo-def}
\begin{defi}
     Sejam $n\geq 0$ e $V$ um espaço vetorial sobre o corpo $\mathbb{K}$. Definimos o \textbf{espaço projetivo sobre $V$} como o espaço topológico quociente $\mathbb{P}(V) = V/\sim$, onde $x\sim y$ se, e somente, existe $\alpha \in \mathbb{K} \setminus\{0\}$ tal que $x = \alpha y$.

     O \textbf{espaço projetivo $n$-dimensional sobre $\mathbb{K}$} é definido como $\mathbb{KP}^n = \mathbb{P}(\mathbb{K}^n)$.
\end{defi}

\input{conteudo/cone-suspensao}
%---------------------------------------------------------------------------------------------------------------------!Draft!-----------------------------------------------------------------------------------------------------------------
\subsection{Espaços Quociente e a propriedade Hausdorff} %afirmação aqui significa teorema/proposição/colorário/lema
\label{topologia-quociente-hausdorff-thm}
\begin{titlemize}{Lista de dependências}
	\item \hyperref[topologia-quociente-def]{Espaços Quociente};\\ %'dependencia1' é o label onde o conceito Dependência 1 aparece (--à arrumar um padrão para referencias e labels--) 
% quantas dependências forem necessárias.
\end{titlemize}
%Comentário sobre os objetos envolvidos na afirmação.
\begin{thm}[Espaços quocientes Hausdorff]% ou af(afirmação)/prop(proposição)/corol(corolário)/lemma(lema)/outros ambientes devem ser definidos no preambulo de Alg.Top-Wiki.tex 
Sejam $X$ um espaço Hausdorff e $\sim$ uma relação de equivalência em $X$ para a qual a projeção $\pi: X \rightarrow X/\sim$ é uma aplicação aberta. Defina o conjunto $R=\{(x,x')\in X\times X| x\sim x'\}$.

Então $X/\sim$ é Hausdorff se, e somente se, $R\subset X\times X$ é fechado.

\end{thm}
\begin{dem}
    $(\Longrightarrow)$ Se $X/\sim$ é de Hausdorff, gostaríamos de mostrar que $X\times X\backslash R$ é aberto. Para qualquer ponto $(x,x')\in (X\times X)\backslash R$, $x$ e $x'$ são tais que $\pi(x)\neq \pi(x')$. Como $X/\sim$ é de Hausdorff, existem abertos $U_x$ e $U_{x'}$ disjuntos em $X/\sim$ que são vizinhanças abertas de $\pi(x)$ e de $\pi(x')$, respectivamente.% e tais que $U_x\cap U_x' = \emptyset$.

    Temos ainda que $\pi^{-1}(U_x)$ e $\pi^{-1}(U_{x'})$ são abertos, pois a topologia de $X/\sim$ é a topologia quociente, e o produto $U=\pi^{-1}(U_x)\times \pi^{-1}(U_{x'})$ é aberto de $X\times X$ na topologia produto. Além disso, $(x,x')\in U$. Afirmamos que $U\subset X\times X\backslash R$. De fato, se $U\cap R\neq \emptyset$, teríamos $(v_1,v_2)\in U\cap R$ tal que $\pi(v_1)=\pi(v_2)$, mas $v_1 \in \pi^{-1}(U_x)$ e $v_2\in \pi^{-1}(U_{x'})$, e desse modo $\pi(v_1) = \pi(v_2) \in U_x \cap U_{x'} = \varnothing$, absurdo. Portanto, para todo $(x,x')\in X\times X\backslash R$, é possível encontrar uma vizinhança aberta $U$ de $(x,x')$ contida em $X\times X\backslash R$; $R$ é fechado, como queríamos.\newline

    $(\Longleftarrow)$ Dado que $R$ é fechado, gostaríamos de encontrar vizinhanças disjuntas de $a,~b\in X/\sim$ quaisquer para concluir que $X/\sim$ é Hausdorff. Sabemos que existem $x,~y\in X$ tais que $\pi(x)=a$ e $\pi(y)=b$ pois a projeção $\pi$ é uma aplicação sobrejetora. Como $X$ é de Hausdorff, existem abertos disjuntos $U_x$ e $U_y$, vizinhanças de $x$ e de $y$, respectivamente. Além disso, uma vez que $R$ é fechado, $X\times X\backslash R$ é aberto e, portanto, $(U_x\times U_y)\cap((X\times X)\backslash R)$ é aberto na topologia produto.

    Sejam $p_1:X\times X\rightarrow X$ e $p_2:X\times X\rightarrow X$ definidos por $$p_1(x_1,x_2)=x_1,\qquad p_2(x_1,x_2)=x_2 \qquad\forall (x_1,x_2)\in X\times X.$$ Como a topologia produto em $X\times Y$ é gerada pela base dada pelos produtos de abertos $X$ e de $Y$, é possível concluir que $p_1$ e $p_2$ são aplicações abertas. Desse modo, $U_1=p_1((U_x\times U_y)\cap((X\times X)\backslash R))$ e $U_2=p_2((U_x\times U_y)\cap((X\times X)\backslash R))$ são abertos em $X$. Por fim, basta observar que os abertos $\pi(U_1)$ e $\pi(U_2)$ são tais que $\pi(U_1)\cap \pi(U_2)=\emptyset$ uma vez que se $v\in \pi(U_1)\cap\pi(U_2)$, teríamos $v=\pi(v_1)$ para algum $v_1\in U_1$ e $v=\pi(v_2)$ para algum $v_2\in U_2$, o que implicaria $v_1\sim v_2$, um absurdo pois, pela construção de $U_1$ e $U_2$, $(v_1,v_2)\not\in R$.  Também temos $a\in U_1$ e $b\in U_2$ pois, como $a\neq b$, $x\not\sim y$. Encontramos assim os dois abertos que separam $a$ e $b$, mostrando que $X/\sim$ é Hausdorff.
\end{dem}

% Comentários sobre a afirmação.
% \begin{titlemize}{Lista de consequências}
% 	\item \hyperref[consequencia1]{Consequência 1};\\ %'consequencia1' é o label onde o conceito Consequência 1 aparece
% \end{titlemize}
O espaço quociente também é essencial para realizar a colagem de espaços topológicos.
\subsection{\emph{Pushout} de espaços topológicos} %afirmação aqui significa teorema/proposição/colorário/lema
\label{pushout-de-espacos-topologicos-def}
\begin{titlemize}{Lista de dependências}
	\item \hyperref[topologia-quociente-def]{Espaços Quociente};\\ %'dependencia1' é o label onde o conceito Dependência 1 aparece (--à arrumar um padrão para referencias e labels--) 
    \item \hyperref[funcao-continua-em-topologia-quociente-prop]{Função contínua em topologia quociente}.
% quantas dependências forem necessárias.
\end{titlemize}

\begin{defi}
    Sejam $X,Y,Z$ espaços topológicos, e sejam $f:Z\rightarrow X$ e $g:Z\rightarrow Y$ funções contínuas. O \textbf{\emph{pushout} de $f$ e $g$} é o espaço quociente $X\sqcup_Z Y=X\sqcup Y/\sim$, onde $\sim$ é a menor relação de equivalência que contém $\{(f(z),g(z))\in X\times Y:z\in Z\}$. 
\end{defi}

\begin{prop}
    Sejam $X,Y,Z$ espaços topológicos. Além disso, sejam $f:Z\rightarrow X$ e $g:Z\rightarrow Y$ funções contínuas. Seja também $\pi:X\sqcup Y\rightarrow X\sqcup_Z Y$ a função projeção associada ao quociente. Então, o espaço $X\sqcup_Z Y$, juntamente com as funções contínuas definidas por:
    \begin{align*}
        i_X:X &\longrightarrow X\sqcup Y/\sim & i_Y:Y&\longrightarrow X\sqcup Y/\sim\\
        x&\longmapsto \pi(x) & y &\longmapsto \pi(y)
    \end{align*}
    forma um diagrama de \emph{pushout}. Ou seja, dadas quaisquer duas funções contínuas $h_X:X\rightarrow W$ e $h_Y:Y\rightarrow W$ que satisfaçam $h_X\circ f=h_Y\circ g$, existe uma única função contínua 
    $\phi:X\sqcup_Z Y\rightarrow W$ tal que 
    $$h_X=\phi\circ i_X \;\;\;\text{ e }\;\;\; h_Y=\phi\circ i_Y.$$ 
    Isso é ilustrado no diagrama seguinte:
% https://q.uiver.app/#q=WzAsNSxbMCwwLCJaIl0sWzAsMiwiWCJdLFsyLDAsIlkiXSxbMiwyLCJYXFxzcWN1cF9aIFkiXSxbMywzLCJXIl0sWzAsMSwiZiIsMl0sWzAsMiwiZyJdLFsxLDMsImlfWCJdLFsyLDMsImlfWSIsMl0sWzEsNCwiaF9YIiwyXSxbMiw0LCJoX1kiXSxbMyw0LCJcXGV4aXN0cyEgXFxwaGkiLDEseyJzdHlsZSI6eyJib2R5Ijp7Im5hbWUiOiJkYXNoZWQifX19XV0=
\[\begin{tikzcd}
	Z && Y \\
	\\
	X && {X\sqcup_Z Y} \\
	&&& W.
	\arrow["g", from=1-1, to=1-3]
	\arrow["f"', from=1-1, to=3-1]
	\arrow["{i_Y}"', from=1-3, to=3-3]
	\arrow["{h_Y}", from=1-3, to=4-4]
	\arrow["{i_X}", from=3-1, to=3-3]
	\arrow["{h_X}"', from=3-1, to=4-4]
	\arrow["{\exists! \phi}"{description}, dashed, from=3-3, to=4-4]
\end{tikzcd}\]
\end{prop}

\begin{dem}
    De acordo com a construção da topologia quociente e da topologia de união disjunta, as funções $i_X,i_Y$ são contínuas. Além disso, a função dada por 
    \begin{align*}
        h_X\sqcup h_Y:X\sqcup Y&\longrightarrow W\\
        a&\longmapsto h_X\sqcup h_Y(a)=\begin{cases}
         h_X(a) & \text{ if }a\in X\\
         h_Y(a) & \text{ if }a\in Y.
        \end{cases}
    \end{align*}
    também é contínua. Como $h_X\circ f=h_Y\circ g$, pela definição de \emph{pushout} de $f$ e $g$, a função $\phi:=(h_X\sqcup h_Y)\circ \pi^{-1}$ é bem-definida. Além disso, a função $\phi$ é contínua, pois a função $\phi\circ\pi=h_X\sqcup h_Y$ é contínua (pela proposição \ref{funcao-continua-em-topologia-quociente-prop}). Pela construção de $\phi$, temos $h_X=\phi\circ i_X$ e $h_Y=\phi\circ i_Y$, o que prova a existência de tal função.

    Finalmente, provamos que esta função é única: suponha que $\phi'$ seja outra função contínua que satisfaça $h_X=\phi'\circ i_X$ e $h_Y=\phi'\circ i_Y$. Então, temos $\phi'|_{i_X(X)}=\phi|_{i_X(X)}$ e $\phi'|_{i_Y(Y)}=\phi|_{i_Y(Y)}$. Como $i_X(X)\cup i_Y(Y)=X\sqcup_ZY$, concluímos que $\phi=\phi'$.
\end{dem}

%\begin{titlemize}{Lista de consequências}
	%\item %\hyperref[consequencia1]{Consequência 1};\\ %'consequencia1' é o label onde o conceito Consequência 1 aparece
%\end{titlemize}

\subsection{Colagem de n-célula} %afirmação aqui significa teorema/proposição/colorário/lema
\label{colagem-de-n-celula-def}
\begin{titlemize}{Lista de dependências}
	\item \hyperref[topologia-quociente-def]{Espaços Quociente};\\
    \item \hyperref[pushout-de-espacos-topologicos-def]{\emph{Pushout} de espaços topológicos}.%'dependencia1' é o label onde o conceito Dependência 1 aparece (--à arrumar um padrão para referencias e labels--) 
% quantas dependências forem necessárias.
\end{titlemize}

\begin{defi}
    Seja $X$ um espaço topológico, e sejam $f:\mathbb{S}^{n-1}\rightarrow X$ uma função contínua e $i:\mathbb{S}^{n-1}\hookrightarrow D^n$ uma inclusão, onde $n\ge 2$. O \textbf{espaço obtido de} $X$ \textbf{pela colagem de uma $n$-célula por meio da função} $f$ é o \emph{pushout} de $f$ e $i$, denotado por $X_f$ ou $D^n\cup_f X$.
\end{defi}

%\begin{titlemize}{Lista de consequências}
	%\item %\hyperref[consequencia1]{Consequência 1};\\ %'consequencia1' é o label onde o conceito Consequência 1 aparece
%\end{titlemize}

\subsection{Colagem de um disco com um ponto} %afirmação aqui significa teorema/proposição/colorário/lema
\label{colagem-de-um-disco-com-um-ponto-ex}
\begin{titlemize}{Lista de dependências}
	\item \hyperref[topologia-quociente-def]{Espaços Quociente};\\
    \item \hyperref[pushout-de-espacos-topologicos-def]{Pushout de espaços topológicos};\\
    \item \hyperref[colagem-de-n-celula-def]{Colagem de n-célula}%'dependencia1' é o label onde o conceito Dependência 1 aparece (--à arrumar um padrão para referencias e labels--) 
% quantas dependências forem necessárias.
\end{titlemize}

\begin{ex}
    Dado $n\ge 2$, sejam $N = (0,\ldots,0,1) \in \mathbb{S}^n$ e $S = -N$. A colagem $\{x\}_f=D^n\cup_f \{x\}$, em que $f:\mathbb{S}^{n-1}\rightarrow \{x\}$ é a função constante, é homeomorfa à esfera $\mathbb{S}^n$. 
\end{ex}

\begin{dem}
    Note que $\text{int}(D^n)\cong \mathbb{R}^n\cong \mathbb{S}^n\setminus\{N\}$. Seja $g_0:\text{int}(D^n)\rightarrow \mathbb{R}^n$ um homeomorfismo. Podemos estender $g_0$ na seguinte forma 
    \begin{align*}
        g:\{x\}_f&\longrightarrow \mathbb{S}^n\\
        p&\longmapsto g(p)=\begin{cases}
            g_0(p) &\text{ se }p\ne [x]\\
            N &\text{ se }p=[x].
        \end{cases}
    \end{align*}
    Essa função é bem-definida e bijetora. Agora, para mostrar que $g$ é um homeomorfismo, basta mostrar que $g$ é contínua e aberta.\\
    A função $g$ é contínua: A função $g$ é contínua se, e somente se, $g\circ \pi$ é contínua, onde $\pi:D^n\bigsqcup \{x\}\rightarrow \{x\}_f$ é a projeção associada ao quociente. A função $g\circ \pi$ é contínua, pois dado um aberto $U$ de $\mathbb{S}^n$, temos:
    \begin{itemize}
        \item se $N\notin U$, então $(g\circ\pi)^{-1}(U)=g_0^{-1}(U)$, que é aberto;
        \item se $N\in U$, então $(g\circ \pi)^{-1}(U)=g_0^{-1}(U\setminus\{N\})\cup \{x\}$, que é um aberto, pois $x$ é um ponto isolado.
    \end{itemize}
    Portanto, $g$ é contínua.\\
    A função $g$ é aberta: Seja $U$ um aberto de $\{x\}_f$. Teremos dois casos: 
    \begin{itemize}
        \item Se $x\notin U$, então $g(U)=g_0(U)$, que é um aberto em  $\mathbb{S}^n\setminus\{N\}$. Assim, ou $g_0(U)$ é aberto em $\mathbb{S}^n$, ou $g_0(U)\cup\{N\}$ é aberto em $\mathbb{S}^n$. Se $g_0(U)$ for aberto, então $g(U)$ será um aberto em $\mathbb{S}^n$. Se $g_0(U)\cup \{N\}$ for aberto, então $g_0(U)=(g_0(U)\cup\{N\})\setminus\{N\}$ é um aberto em $\mathbb{S}^n$. Em ambos os casos, $g(U)$ será um aberto em $\mathbb{S}^n$;
        \item Se $x\in U$, então $\pi^{-1} (U)$ é um aberto em $D^n\sqcup \{x\}$. Pela construção do quociente, temos $\partial D^n\subseteq\pi^{-1}(U)$, logo, para todo ponto $y\in \partial D^n$, existe um $r_y>0$ tal que 
        $$B_y:=\{z\in D^n: ||z-y||<r_y\}\subseteq \pi_1^{-1}(U).$$
        A coleção $\{B_y\}_{y\in \partial D^n}$ é uma cobertura aberta de $\partial D^n$. Como o bordo $\partial D^n$ é compacto, existem $y_1,...,y_k$ tal que 
        \[\partial D^n\subseteq B_{y_1}\cup...\cup B_{y_k}.\]
        Considere $r=\text{inf}\{r_{y_1},...,r_{y_k}\}$. Assim, o conjunto 
        \[B=\pi(\{y\in D^n: ||y||>(1-r)\}\cup\{x\})\subseteq U\]
        é um aberto em $\{x\}_f$, e $g(B)\subseteq g(U)$ corresponde a uma bola centrada em $N$ em $\mathbb{S}^n$. Note que $U\setminus \{x\}$ é um aberto, pois $x$ é um ponto fechado. Pelo item anterior, temos que o aberto
        \[g(U)=g(B)\cup g(U\setminus\{x\})\]
        é uma união de abertos, o que implica que $g(U)$ é um aberto em $\mathbb{S}^n$.
    \end{itemize} 
    Portanto, $g$ é aberta.
\end{dem}

%\begin{titlemize}{Lista de consequências}
	%\item %\hyperref[consequencia1]{Consequência 1};\\ %'consequencia1' é o label onde o conceito Consequência 1 aparece
%\end{titlemize}

\subsection{Produto \emph{wedge} de espaços topológicos} %afirmação aqui significa teorema/proposição/colorário/lema
\label{produto-wedge-def}
\begin{titlemize}{Lista de dependências}
	\item \hyperref[topologia-quociente-def]{Espaços Quociente};\\ %'dependencia1' é o label onde o conceito Dependência 1 aparece (--à arrumar um padrão para referencias e labels--) 
\end{titlemize}

\begin{defi}
    Sejam $(X,x_0),(Y,y_0)$ espaços topológicos pontuados. O \textbf{produto \emph{wedge}} de $(X,x_0)$ e $(Y,y_0)$, denotado por $(X,x_0)\vee (Y,y_0)$, é o espaço quociente obtido da união disjunta $X\sqcup Y$ por meio da identificação de $x_0$ e $y_0$ a um único ponto. 
\end{defi}



%\begin{titlemize}{Lista de consequências}
	%\item %\hyperref[consequencia1]{Consequência 1};\\ %'consequencia1' é o label onde o conceito Consequência 1 aparece
%\end{titlemize}
%%% Local Variables:
%%% mode: LaTeX
%%% TeX-master: "../Alg.Top-Wiki"
%%% End:


\section{Homotopia}
\label{homotopia}
Um assunto que aparece na definição de objetos importantes na topologia algébrica, como os grupos de homotopia e, em particular, o grupo fundamental.

\input{conteudo/homotopia-def}% onde conteudos.tex é o nome do arquivo tex que voce quer incluir nessa secção.
\input{conteudo/homotopia-relativa-def}
\input{conteudo/homotopia-relaçao-de-equivalencia-prop}
%---------------------------------------------------------------------------------------------------------------------!Draft!-----------------------------------------------------------------------------------------------------------------
\subsection{Equivalência de Homotopia}
\label{equiv-homotopia}
\begin{titlemize}{Lista de dependências}
	\item \hyperref[homotopia-def]{Homotopia};\\
\end{titlemize}

\begin{defi}[Equivalência de Homotopia]
	Sejam $X$ e $Y$ espaços topológicos. Uma função contínua $f:X\to Y$ é dita uma \textbf{equivalência de homotopia} se existe outra função contínua $g:Y\to X$ tal que $f\circ g \sim \text{id}_Y$ e $g\circ f \sim \text{id}_X$. Nesse caso, dizemos que $X$ e $Y$ são \textbf{homotopicamente equivalentes}, e $g$ é \textbf{inversa a menos de homotopia} de $f$.
\end{defi}

É claro que todo homeomorfismo é uma equivalência de homotopia.

\begin{titlemize}{Lista de consequências}
    \item \hyperref[espaco-contratil-def]{Espaço contrátil}
	\item \hyperref[equiv-homotopia-induz-iso]{Equivalência de homotopia e grupo fundamental}
\end{titlemize}

%[Bianca]: é mais fácil criar a lista de dependências do que a de consequências.

\subsection{Espaço contrátil}
\label{espaco-contratil-def}
\begin{titlemize}{Lista de dependências}
	\item \hyperref[homotopia-def]{Homotopia};\\
        \item \hyperref[equiv-homotopia]{Equivalência de Homotopia}.
\end{titlemize}

\begin{defi}
	Seja $X$ um espaço topológico. Diremos que o espaço $X$ é \textbf{contrátil} se $X$ é homotopicamente equivalente a um ponto.
\end{defi}

%%% Local Variables:
%%% mode: LaTeX
%%% TeX-master: "../Alg.Top-Wiki"
%%% End:

\section{Grupo Fundamental}
\label{grupo-fundamental}

\begin{titlemize}{Lista de Dependências}
	\item \hyperref[homotopia]{Homotopia}\\ %homotopia
\end{titlemize}

Considere um espaço topológico com um ponto base fixado. O seu grupo fundamental é o grupo das classes de equivalência (sob homotopia relativa aos extremos) dos laços no espaço saindo do ponto base. Tal grupo armazena certas informações sobre buracos do espaço topologico, e é invariante sobre a equivalência homotópica. Esta é uma ferramenta poderosa para verificar se dois espaços topológicos são homeomorfos. % (homotópicos). %retirei, não entendi o que querem dizer
Veremos como sua construção se dá com mais detalhes.

%---------------------------------------------------------------------------------------------------------------------!Draft!-----------------------------------------------------------------------------------------------------------------
\subsection{Espaço de Laços}
\label{espaco-lacos-def}
\begin{titlemize}{Lista de dependências}
	%\item \hyperref[dependecia1]{Dependência 1};\\ %'dependencia1' é o label onde o conceito Dependência 1 aparece (--à arrumar um padrão para referencias e labels--)
    \item \hyperref[homotopia-relativa-def]{Homotopia Relativa}
	\item \hyperref[homotopia-relaçao-de-equivalencia-prop]{Homotopia é relação de equivalência};\\
% quantas dependências forem necessárias.
\end{titlemize}
\begin{defi}[Espaço de Laços]
	Seja $X$ um espaço topológico e seja $x_0\in X$ um ponto base. O \textbf{espaço de laços} em $X$ que saem de $x_0$ é definido como
\[\Omega(X,x_0) = \left\{\gamma: I \to X ~|~ \gamma\text{ é contínua e }\gamma(0)=\gamma(1)=x_0\right\}.\]
\end{defi}

Investigaremos a fundo o conjunto $\pi_1(X,x_0) = \Omega(X,x_0)/\sim$, onde $\alpha \sim \beta$ se, e somente se, $\alpha$ e $\beta$ são homotópicas relativo a $\partial I = \{0,1\}$.


%[Bianca]: é mais fácil criar a lista de dependências do que a de consequências.
\input{conteudo/produto-concatenacao-def}
\input{conteudo/produto-bem-definido-gr-fundamental-prop}
\subsection{Grupo fundamental}
\label{grupo-fundamental-def}
\begin{titlemize}{Lista de dependências}
	\item \hyperref[espaco-lacos-def]{O espaço de laços}
	\item \hyperref[produto-bem-definido-prop]{O produto do grupo fundamental};\\ %'dependencia1' é o label onde o conceito Dependência 1 aparece (--à arrumar um padrão para referencias e labels--) 
% quantas dependências forem necessárias.
\end{titlemize}
\begin{defi}[Grupo fundamental]
    Seja $X$ um espaço topológico e seja $x_0$ um ponto de $X.$ O grupo fundamental de $X$ em $x_0$ é $(\pi_1(X,x_0),\cdot)$, onde $\pi_1(X,x_0) = \Omega(X,x_0)/\sim$, onde $\alpha \sim \beta$ se, e somente se, $\alpha$ e $\beta$ são homotópicas relativo aos extremos, e o produto $\cdot$ é dado por $[\alpha]\cdot[\beta] = [\alpha \ast \beta]$, em que $\alpha \ast \beta$ é a concatenação de $\alpha$ e $\beta$.
\end{defi}

No geral, o grupo fundamental depende da escolha do ponto base $x_0$. A seguir, apresentamos um exemplo elementar de grupo fundamental.
\begin{ex}
    Seja $X=\{x\}$ é um espaço topológico contendo apenas um ponto. Nesse caso, o único laço em $X$ é a função constante $c_x:I\rightarrow \{x\}$. Assim, a única classe de homotopia é $[c_x]$, o que implica que $\pi_1(\{x\},x)=0$.
\end{ex}

\begin{titlemize}{Lista de consequências}
	\item \hyperref[hom-grupo-fundamental]{Homomorfismo de grupos fundamentais};%'consequencia1' é o label onde o conceito Consequência 1 aparece
	%\item \hyperref[]{}
\end{titlemize}

\input{conteudo/homomorfismo-de-grupo-fundamental-prop}
%---------------------------------------------------------------------------------------------------------------------!Draft!-----------------------------------------------------------------------------------------------------------------
\subsection{Conjugação por uma Curva} %[conjugacao-por-curva-prop]{Conjugação por uma Curva}
\label{conjugacao-por-curva-prop}
\begin{titlemize}{Lista de dependências}
    \item \hyperref[espaco-lacos-def]{Espaço de Laços};\\
    \item \hyperref[produto-bem-definido-prop]{O produto do grupo fundamental};\\
	\item \hyperref[grupo-fundamental-def]{O Grupo Fundamental};
% quantas dependências forem necessárias.
\end{titlemize}
%Comentário sobre os objetos envolvidos na afirmação.
\begin{defi}[Conjugação de laços por uma curva] 
	Sejam $x_0$ e $x_1$ pontos em um espaço topológico $X$ e seja $\gamma: I \to X$ uma curva contínua ligando $x_0$ a $x_1$; isto é, $\gamma(0)=x_0$ e $\gamma(1) = x_1$. Seja também $\eta \in \Omega(X,x_0)$ um laço saindo de $x_0$. Definimos a conjugação de $\eta$ por $\gamma$ como $\overline{\gamma} * \eta * \gamma \in \Omega(X,x_1)$, laço saindo de $x_1$. Isto define uma função $A_{\gamma}: \Omega(X,x_0) \to \Omega(X,x_1)$.
\end{defi}

\begin{prop}[Isomorfismo de grupos induzido por $A_{\gamma}$]
    Sejam $x_0$ e $x_1$ pontos em um espaço topológico $X$ e seja $\gamma: I \to X$ uma curva ligando $x_0$ a $x_1$.
    
    Então $A_{\gamma}$ induz um isomorfismo de grupos \begin{align*}
        \hat{A}_{\gamma}: \pi_1(X,x_0)&\to \pi_1(X,x_1)\\
        \hat{A}_{\gamma}([\eta]) &= [A_{\gamma}(\eta)] = [\overline{\gamma} * \eta * \gamma].
    \end{align*}

    \begin{dem}
        Provemos primeiramente que $\hat{A}_{\gamma}$ está bem definida. Considere $c_{\gamma}: \gamma \Rightarrow \gamma$ e $c_{\overline{\gamma}}: \overline{\gamma} \Rightarrow \overline{\gamma}$ as homotopias constantes. Assim, se $\eta, \nu \in \Omega(X,x_0)$ e $H: \eta \Rightarrow \nu$ é uma homotopia relativa a $\partial I$ então é claro que $c_{\overline{\gamma}}*H*c_{\gamma}: A_{\gamma}(\eta) \Rightarrow A_{\gamma}(\nu)$ também é homotopia relativa a $\partial I$.

        $\hat{A}_{\gamma}$ é um homomorfismo de grupos, já que dadas $\eta, \nu \in \Omega(X,x_0)$,
        \begin{align*}
            \hat{A}_{\gamma}([\eta]\cdot[\nu]^{-1})
            &= \hat{A}_{\gamma}([\eta * \overline{\nu}])\\
            &= [\overline{\gamma} * (\eta * \overline{\nu}) * \gamma]\\
            &= [(\overline{\gamma} * \eta * \gamma)*(\overline{\gamma} * \overline{\nu} * \gamma)]\\
            &= [\overline{\gamma} * \eta * \gamma]\cdot[\overline{\overline{\gamma} * \nu * \gamma}]\\
            &= \hat{A}_{\gamma}([\eta]) \cdot \hat{A}_{\gamma}([\nu])^{-1}.
        \end{align*}
        
        Por fim, note que $\hat{A}_{\gamma}$ e $\hat{A}_{\overline{\gamma}}$ são inversas, pois \[A_{\gamma} \circ A_{\overline{\gamma}}(\eta) = (\overline{\gamma} * \gamma) * \eta * (\overline{\gamma} * \gamma) \sim \eta\text{ relativa a }\partial I\]
        para toda curva $\gamma: I \to X$. Desse modo $\hat{A}_{\gamma}$ é um isomorfismo de grupos.
    \end{dem}
\end{prop}

Um fato importante decorrente de tal proposição é o seguinte.

\begin{corol}
    Se $X$ é um espaço topológico então $\pi_1(X,x_0)$ é isomorfo a $\pi_1(X,x_1)$, para quaisquer $x_0, x_1 \in X$ na mesma componente conexa por caminhos de $X$. Em especial, o grupo fundamental independe do ponto base caso $X$ seja conexo por caminhos.
\end{corol}

Dessa forma, se $X$ é um espaço conexo por caminhos, podemos denotar o grupo fundamental de $X$ por $\pi_1(X)$, omitindo o ponto base.

\begin{nota}
    Sejam $X$ e $Y$ espaços topológicos, $x_0, x_1\in X$ e $\gamma:I\to X$ uma curva ligando $x_0$ a $x_1$. Seja também $f: X\to Y$ uma função contínua e denotemos $y_0 = f(x_0)$ e $y_1 = f(x_1)$. Então $f(\gamma):I \to Y$ liga $y_0$ a $y_1$, e vale que
    \[f_{*,x_1} \circ \hat{A}_{\gamma} = \hat{A}_{f(\gamma)} \circ f_{*,x_0}.\]
    \begin{dem}
        Para cada $\eta \in \Omega(X,x_0)$,
        \begin{align*}
            \hat{A}_{f(\gamma)} \circ f_{*,x_0}([\eta])
            &= \hat{A}_{f(\gamma)} ([f(\eta]))\\
            &= [\overline{f(\gamma)} * f(\eta) * f(\gamma)]\\
            &= [f(\overline{\gamma}) * f(\eta) * f(\gamma)]\\
            &= [f(\overline{\gamma} * \eta * \gamma)]\\
            &= [f(A_{\gamma}(\eta)]
            = f_{*,x_1}\circ \hat{A}_{\gamma}([\eta]).
        \end{align*}
    \end{dem}
\end{nota}

\begin{titlemize}{Lista de consequências}
	\item \hyperref[equiv-homotopia-induz-iso]{Equivalência de homotopia e o grupo fundamental};\\ %'consequencia1' é o label onde o conceito Consequência 1 aparece
	%\item \hyperref[]{}
\end{titlemize}

%[Bianca]: Um arquivo tex pode ter mais de uma afirmação (ou definição, ou exemplo), mas nesse caso cada afirmação deve ter seu próprio label. Dar preferência para agrupar afirmações que dependam entre sí de maneira próxima (um teorema e seu corolário, por exemplo)

\input{conteudo/equivalencia-de-homotopia-induz-iso-thm}
\subsection{Grupo fundamental de espaço contrátil}
\label{grupo-fundamental-de-contratil-prop}
\begin{titlemize}{Lista de dependências}
    \item \hyperref[equiv-homotopia]{Equivalência de Homotopia};\\
	\item \hyperref[hom-grupo-fundamental]{Homomorfismo de grupos fundamentais};\\
    \item \hyperref[equiv-homotopia-induz-iso]{Equivalência de homotopia e o grupo fundamental}.
\end{titlemize}

\begin{prop}
    Se $X$ é um espaço contrátil com $x_0\in X$, então o grupo fundamental $\pi_1(X,x_0)$ é trivial.
\end{prop}

\begin{dem}
    Sem perda de generalidade, $X$ é homotopicamente equivalente ao ponto $x_0$. Logo, existe uma equivalência de homotopia $f:X\rightarrow \{x_0\}$. Pelo Teorema \ref{equiv-homotopia-induz-iso}, o homomorfismo induzido $f_*:\pi_1(X,x_0)\rightarrow \pi_1(\{x_0\},x_0)$ é um isomorfismo. Isso implica que $\pi_1(X,x_0)=0$.
\end{dem}

%\begin{titlemize}{Lista de consequências}
	%\item \hyperref[consequencia1]{Consequência 1};\\ %'consequencia1' é o label onde o conceito Consequência 1 aparece
	%\item \hyperref[]{}
%\end{titlemize}
\subsection{O Grupo Fundamental de um Espaço Convexo}
\label{grupo-fundamental-convexo}
\begin{titlemize}{Lista de dependências}
	\item \hyperref[grupo-fundamental-def]{Grupo Fundamental};\\ %'dependencia1' é o label onde o conceito Dependência 1 aparece (--à arrumar um padrão para referencias e labels--) 
% quantas dependências forem necessárias.
\end{titlemize}

\begin{ex}[Grupo fundamental de um espaço convexo]
O grupo fundamental de um espaço convexo X é sempre trivial, i.e, $\pi_1(X, x_0) = \{ 1\}$.
\end{ex}

De fato, isso se verifica pois, se $\alpha$ é um laço em um espaço topológico $X$ começando em um ponto $x_0$, tomando a homotopia $F:I \times I \longrightarrow X$, onde $F(s,t) = (1 - t)\alpha(s) + tx_0$, tem-se que $\alpha \sim c_{x_0}$. Assim, $\pi_1(X, x_0) = \{1\}$.

%\begin{figure}[]
%	\centering
%	\includegraphics[width=0.8\textwidth]{}
%	\caption{}
%	\label{fig:}
%\end{figure}


\subsection{Grupo fundamental de espaço de produtos}
\label{grupo-fundamental-de-espaco-de-produtos-prop}
\begin{titlemize}{Lista de dependências}
    \item \hyperref[homotopia-def]{Homotopia};\\
    \item \hyperref[grupo-fundamental]{Grupo fundamental};\\
    \item \hyperref[hom-grupo-fundamental]{Homomorfismo de grupos fundamentais}.
\end{titlemize}

\begin{prop}
    Sejam $X,Y$ espaços topológicos, com $x_0\in X$ e $y_0\in Y$, e sejam $p:X\times Y\rightarrow X$ e $q:X\times Y\rightarrow Y$ projeções canônicas. Então, o homomorfismo
    \begin{align*}
        ((p_*,q_*):\pi_1(X\times Y,(x_0,y_0))&\longrightarrow \pi_1 (X,x_0)\times \pi_1(Y,y_0)\\
        [\alpha]&\longmapsto ([p\circ \alpha],[q\circ \alpha]) 
    \end{align*}
    é um isomorfismo.
\end{prop}

\begin{dem}
    Como $p_*$ e $q_*$ são homomorfismos de grupos, $(p_*,q_*)$ também é um homomorfismo de grupos. Vamos agora verificar que $(p_*,q_*)$ é bijetivo.\\
    Injetividade: Note que $([c_{x_0}],[c_{y_0}])$ é a unidade de $\pi_1 (X,x_0)\times \pi_1(Y,y_0)$. Assim, se $[\alpha]\in \text{Ker}(p_*,q_*)$, então $[p\circ \alpha]=[c_{x_0}]$ e $[q\circ \alpha]=[c_{y_0}]$. Ou seja, existem homotopias relativa a $\partial I$, $H_1:p\circ\alpha \Rightarrow c_{x_0}$ e $H_2:q\circ \alpha \Rightarrow c_{y_0}$, o que implica que a função $H:(X\times Y)\times I\rightarrow X\times Y$ definida por
    \begin{align*}
        H((x,y),t):=(H_1(x,t),H_2(y,t))
    \end{align*}
    é uma homotopia relativa a $\partial I$ entre $(p\circ \alpha,q\circ\alpha)$ e $c_{(x_0,y_0)}$. Assim, concluímos que $[\alpha]=[c_{(x_0,y_0)}]$, o que implica que $(p_*,q_*)$ é injetivo.\\
    Sobrejetividade: Sejam $(\alpha,\beta)\in \Omega(X,x_0)\times \Omega(Y,y_0)$. Basta mostrar que existe um $\gamma\in \Omega(X\times Y,(x_0,y_0))$ tal que $p\circ\gamma=\alpha$ e $q\circ \gamma=\beta$. Porém, o laço $(\alpha,\beta)$ satisfaz exatamente essa condição. Portanto, concluímos que $(p_*,q_*)$ é sobrejetivo.
\end{dem}



%%% Local Variables:
%%% mode: LaTeX
%%% TeX-master: "../Alg.Top-Wiki"
%%% End:

\section{Categorias}
\label{categorias}

% \begin{titlemize}{Lista de Dependências}
% 	\item \hyperref[]{};\\
% 	\item \hyperref[]{};
% \end{titlemize}

A teoria das categorias pode ser vista como uma ferramenta usada para os estudos das conexões das diversas áreas da matemática. Nessa Wiki, usaremos a linguagem de teoria das categorias de modo a expressar as relações entre a topologia e a álgebra que a topologia algébrica está interessada.

\subsection{Categorias}
\label{categorias-def}
\begin{defi}[Categorias]
	    Uma categoria $\mathcal{C}$ é formada pelas seguintes coisas:


\begin{itemize}
    \item Uma coleção de objetos $\text{Obj}(\mathcal{C})$, que geralmente serão denotados por letras maiúsculas $A$, $B$, $C$...
    \item Uma coleção de morfismos $\text{Mor}(\mathcal{C})$, que usualmente serão denotadas por letras minúsculas $f$, $g$, $h$...
\end{itemize}

Onde valem os seguintes axiomas:

\begin{enumerate}
    \item A cada morfismo $f$ de $\text{Mor}(\mathcal{C})$ são associados dois objetos $\text{Dom}(f)$ e $\text{Codom}(f)$ de $\text{Obj}(\mathcal{C})$. \\
    Escrevemos % https://tikzcd.yichuanshen.de/#N4Igdg9gJgpgziAXAbVABwnAlgFyxMJZABgBpiBdUkANwEMAbAVxiRAEEQBfU9TXfIRQBGclVqMWbAELdxMKAHN4RUADMAThAC2SMiBwQkokAzoAjGAwAK-PATYMYanCGr1mrRCDVyuQA
\begin{tikzcd}
A \arrow[r, "f"] & B
\end{tikzcd}
 para abreviar $f \in \text{Mor}(\mathcal{C})$, $\text{Dom}(f) = A$ e $\text{Codom}(f) = B$.
 \item A cada objeto $A$ de $\mathcal{C}$ está associado um morfismo $1_A \in \text{Mor}(\mathcal{C})$ tal que $\text{Codom}(1_A) = \text{Dom}(1_A) = A$.
 \item Para quaisquer dois morfismos $f$ e $g$, tais que $\text{Dom}(f)=\text{Codom}(g)$, há um morfismo associado $f \circ g$, onde $\text{Dom}(f \circ g) = \text{Dom}(g)$ e $\text{Codom}(f \circ g) = \text{Codom}(f)$.
 \\ Isso pode ser representado dizendo que o seguinte diagrama comuta:
% https://q.uiver.app/#q=WzAsMyxbMCwwLCJBIl0sWzEsMCwiQiJdLFsxLDEsIkMiXSxbMCwxLCJmIl0sWzEsMiwiZyJdLFswLDIsImYgXFxjaXJjIGciLDJdXQ==
\[\begin{tikzcd}[column sep=large]
	A & B \\
	& C
	\arrow["f", from=1-1, to=1-2]
	\arrow["{f \circ g}"', from=1-1, to=2-2]
	\arrow["g", from=1-2, to=2-2]
\end{tikzcd}\]

\item Para todo morfismo $f$ de $\text{Mor}(\mathcal{C})$ com $\text{Dom}(f) = A$ e $\text{Codom}(f) = B$, vale que $f \circ 1_A = f$ e $1_B \circ f = f$. Ou seja, o seguinte diagrama comuta:

% https://q.uiver.app/#q=WzAsNCxbMCwwLCJBIl0sWzEsMCwiQSJdLFsxLDEsIkIiXSxbMiwxLCJCIl0sWzAsMSwiMV9BIl0sWzEsMiwiZiJdLFswLDIsImYiLDJdLFsxLDMsImYiXSxbMiwzLCIxX0IiLDJdXQ==
\[\begin{tikzcd}[sep=large]
	A & A \\
	& B & B
	\arrow["{1_A}", from=1-1, to=1-2]
	\arrow["f"', from=1-1, to=2-2]
	\arrow["f", from=1-2, to=2-2]
	\arrow["f", from=1-2, to=2-3]
	\arrow["{1_B}"', from=2-2, to=2-3]
\end{tikzcd}\]
\item Dados os morfismos $f$, $g$, $h$ de $\text{Mor}(\mathcal{C})$, vale que 
$(f \circ g) \circ h = f \circ (g \circ h)$. Ou seja, o seguinte diagrama comuta:
% https://q.uiver.app/#q=WzAsNCxbMCwwLCJBIl0sWzEsMCwiQiJdLFsxLDEsIkMiXSxbMiwxLCJEIl0sWzAsMSwiZiJdLFsxLDIsImciXSxbMCwyLCJnIFxcY2lyYyBmIl0sWzIsMywiaCJdLFsxLDMsImggXFxjaXJjIGciXV0=
\[\begin{tikzcd}[sep=large]
	A & B \\
	& C & D
	\arrow["f", from=1-1, to=1-2]
	\arrow["{g \circ f}", from=1-1, to=2-2]
	\arrow["g", from=1-2, to=2-2]
	\arrow["{h \circ g}", from=1-2, to=2-3]
	\arrow["h", from=2-2, to=2-3]
\end{tikzcd}\]
\end{enumerate}
\end{defi}



%[Bianca]: é mais fácil criar a lista de dependências do que a de consequências.
 
%---------------------------------------------------------------------------------------------------------------------!Draft!-----------------------------------------------------------------------------------------------------------------
\subsection{Categorias}
\label{categorias-ex}
\begin{titlemize}{Lista de dependências}
	\item \hyperref[categorias-def]{Definição de Categoria};\\ %'dependencia1' é o label onde o conceito Dependência 1 aparece (--à arrumar um padrão para referencias e labels--) 

\end{titlemize}

\begin{ex}[Exemplos de Categorias]
	Alguns dos seguintes exemplos não serão tratados com detalhes. No entanto, pode-se consultá-los em quaisquer livros de teoria das categorias.
\begin{itemize}
\item \textbf{Mon} é uma categorial em que os objetos são monóides e os morfismos são homomorfismos de monóides.
\item \textbf{Grp} é a categoria dos grupos e homomorfismo de grupos (A categoria \textbf{Ab} é a categoria dos grupos abelianos.
\item A categoria \textbf{TOP} tem como objetos os espaços topológicos e como morfismos as funções contínuas (há também a categoria $\mathbf{TOP_*}$ dos espaços topológicos com um ponto selecionado, onde os morfismos $f:(X,x) \longrightarrow (Y,y)$ são funções contínuas tais que $f(x) = y$).
\item $\mathbf{Vec(\mathbb{K})}$ é a categoria dos espaços vetoriais sobre o corpo $\mathbb{K}$ e as transformações lineares dos espaços
\item A categoria \textbf{SET} tem como objetos os conjuntos e os morfismos são as funções entre os conjuntos. Ainda, pode-se definir $\mathbf{SET}_\omega$, a categoria dos conjuntos finitos e as funções entre eles.
\item A categoria \textbf{Ord} dos ordinais e das funções entre eles (funções entre esses conjuntos transitivos). Da mesma forma, $\mathbf{Ord}_\omega$ é a categoria dos ordinais finitos e as funções entre eles.

\item Uma relação $\leq$ é dita relação de ordem parcial se satisfaz:
\begin{itemize}
    \item $a \leq a$ para todo $a$.
    \item Se $a \leq b$ e $b \leq a$, então $a = b$ para todos $a$ e $b$.
    \item Se $a \leq b$ e $b \leq c$, então $a \leq c$ para todos $a$, $b$ e $c$.
\end{itemize}
A categoria $\mathbf{PO}$ (partial-order) é definida com morfismos estabelecendo a ordem entre os objetos, isto é, $A \leq B$ se, e somente se, existe $f$ em $Mor(\mathbf{PO})$, tal que % https://q.uiver.app/#q=WzAsMixbMCwwLCJBIl0sWzEsMCwiQiJdLFswLDEsImYiXV0=
\begin{tikzcd}[cramped]
	A & B
	\arrow["f", from=1-1, to=1-2]
\end{tikzcd}
\item Já a categoria \textbf{POS} tem como objetos os conjuntos parcialmente ordenados e os morfismos são funções que preservam a ordem, isto é, se $f:$ % https://q.uiver.app/#q=WzAsMixbMCwwLCJBIl0sWzEsMCwiQiJdLFswLDFdXQ==
\begin{tikzcd}[cramped]
	A & B
	\arrow[from=1-1, to=1-2]
\end{tikzcd}
, e $m \leq n$ em $A$, então $f(m) \leq f(n)$ em $B$.




\end{itemize}

\end{ex}


\begin{titlemize}{Lista de consequências}
	\item \hyperref[homotopia]{homotopia};\\ %'consequencia1' é o label onde o conceito Consequência 1 aparece
\end{titlemize}

%---------------------------------------------------------------------------------------------------------------------!Draft!-----------------------------------------------------------------------------------------------------------------
\subsection{Isomorfismo}
\label{isomorfismo-em-categorias-def}
\begin{titlemize}{Lista de dependências}
	\item \hyperref[categorias-def]{Definição de Categoria};\\ %'dependencia1' é o label onde o conceito Dependência 1 aparece (--à arrumar um padrão para referencias e labels--) 
\end{titlemize}
\begin{defi}[Isomorfismo]
	Um morfismo $f:A \longrightarrow B$ de uma categoria $\mathcal{C}$ é um isomorfismo se, e somente se, existe um morfismo $g:B \longrightarrow A$, tal que $f \circ g = 1_B$ e $g \circ f = 1_A$. Nesse caso, dizemos que $A$ e $B$ são isomorfos e escrevemos $A \cong B$.
\end{defi}


%[Bianca]: é mais fácil criar a lista de dependências do que a de consequências.

%---------------------------------------------------------------------------------------------------------------------!Draft!-----------------------------------------------------------------------------------------------------------------
\subsection{Funtor}
\label{funtor-categorias-def}
\begin{titlemize}{Lista de dependências}
	\item \hyperref[categorias-def]{Definição de Categoria};\\ %'dependencia1' é o label onde o conceito Dependência 1 aparece (--à arrumar um padrão para referencias e labels--) 
\end{titlemize}
\begin{defi}[Funtor Covariante]
	Um funtor é uma função entre categorias $F: \mathcal{C} \longrightarrow \mathcal{D}$, que associa para cada $A \in Obj(\mathcal{C})$ um único objeto $F(A) \in Obj(\mathcal{D})$ e associa cada morfismo $f \in Mor(\mathcal{C})$ um morfismo $F(f): (A) \longrightarrow F(B)$ , tal que $F(f \circ g) = F(f) \circ F(g) $ e $F(1_A) = 1_{F(A)}$.
\end{defi}

O conceito de funtor é extremamente importante, pois é ele que estabelece uma "ponte" para as diversas áreas da mátematica. Desse modo, podemos ver o grupo fundamental como um funtor da categoria dos espaços topológicos pontuados para a categoria de grupos e homomorfismo de grupos.

\begin{titlemize}{Lista de consequências}
	\item \hyperref[homotopia]{Homotopia};\\ %'consequencia1' é o label onde o conceito Consequência 1 aparece
	\item \hyperref[grupo-fundamental]{Grupo fundamental}
\end{titlemize}

%[Bianca]: é mais fácil criar a lista de dependências do que a de consequências.

%---------------------------------------------------------------------------------------------------------------------!Draft!-----------------------------------------------------------------------------------------------------------------
\subsection{Funtores-Exemplos}
\label{funtor-categorias-ex}
\begin{titlemize}{Lista de dependências}
	\item \hyperref[categorias-ex]{Categorias-Exemplos};\\
	\item \hyperref[funtor-categorias-def]{Funtor};\\ %'dependencia1' é o label onde o conceito Dependência 1 aparece (--à arrumar um padrão para referencias e labels--) 
\end{titlemize}

\begin{ex}[Funtores Covariantes]
	Os detalhes dos exemplos a seguir são deixados para os leitores.
\begin{itemize}

    \item O funtor identidade $\mathbf{1}_{\mathcal{C}}: \mathcal{C} \longrightarrow \mathcal{C}$ leva todo objeto nele mesmo e todo morfismo nele mesmo, isto é, $\mathbf{1}_{\mathcal{C}}(A) = A$ e $\mathbf{1}_{\mathcal{C}}(f) = f$
    
    \item O funtor potência (power-set functor) $\mathcal{P}:\mathbf{SET} \longrightarrow \mathbf{SET}$, que leva um conjunto $A$ no conjuntos das partes $\mathcal{P}(A)$ e uma função $f:A \longrightarrow B$ para a função $\mathcal{P}(f): \mathcal{P}(A) \longrightarrow \mathcal{P}(B)$, tal que $\mathcal{P}(f)(S) = f(S)$ para todo $S \subseteq A$.

    \item Na topologia algébrica podemos obter de cada espaço topológico pontuado um grupo, chamado de n-ésimo grupo de homotopia. Além disso, para cada função contínua $f: A \longrightarrow B$, podemos obter um homomorfismo de grupos $\pi_n(f):\pi_n(A) \longrightarrow \pi_n(B)$, onde $\pi_n(A)$ e $\pi_n(B)$ são os n-ésimos grupos de homotopia. Dessa forma, construimos um funtor $\pi_n: \mathbf{TOP}_* \longrightarrow \mathbf{Grp}$. 

\end{itemize}
\end{ex}

 
%\begin{figure}[]
%	\centering
%	\includegraphics[width=0.8\textwidth]{}
%	\caption{}
%	\label{fig:}
%\end{figure}

\begin{titlemize}{Lista de consequências}
	\item \hyperref[grupo-fundamental]{Grupo fundamental};\\ %'consequencia1' é o label onde o conceito Consequência 1 aparece
\end{titlemize}

%---------------------------------------------------------------------------------------------------------------------!Draft!-----------------------------------------------------------------------------------------------------------------
\subsection{Transformação Natural}
\label{transformação-natural-categorias-def}
\begin{titlemize}{Lista de dependências}
	\item \hyperref[funtor-categorias-def]{Funtor};\\ %'dependencia1' é o label onde o conceito Dependência 1 aparece (--à arrumar um padrão para referencias e labels--) 
% quantas dependências forem necessárias.
\end{titlemize}
\begin{defi}[Transformação Natural]
	Dados dois funtores % https://q.uiver.app/#q=WzAsMixbMCwwLCJcXG1hdGhjYWx7Q30iXSxbMSwwLCJcXG1hdGhjYWx7RH0iXSxbMCwxLCJGIiwwLHsib2Zmc2V0IjotMX1dLFswLDEsIkciLDIseyJvZmZzZXQiOjF9XV0=
\begin{tikzcd}[cramped,sep=small]
	{\mathcal{C}} & {\mathcal{D}}
	\arrow["F", shift left, from=1-1, to=1-2]
	\arrow["G"', shift right, from=1-1, to=1-2]
\end{tikzcd}, definimos a transformação natural $\eta:F \Longrightarrow G$ da seguinte forma: \\
$\eta$ é uma família de flechas $(\eta_A: F(A) \longrightarrow G(A))_{A \in Obj(\mathcal{C})}$, tal que $\eta_B \circ F(f) = G(f) \circ \eta_A$. Isso equivale a dizer que o seguinte diagrama comuta.
% https://q.uiver.app/#q=WzAsNixbMiwwLCJGKEEpIl0sWzQsMCwiRyhBKSJdLFsyLDIsIkYoQikiXSxbNCwyLCJHKEIpIl0sWzAsMCwiQSJdLFswLDIsIkIiXSxbNCw1LCJmIiwyXSxbMCwyLCJGKGYpIiwyXSxbMSwzLCJHKGYpIiwyXSxbMCwxLCJcXGV0YV9BIiwxXSxbMiwzLCJcXGV0YV9CIiwxXV0=
\[\begin{tikzcd}[sep=large]
	A && {F(A)} && {G(A)} \\
	\\
	B && {F(B)} && {G(B)}
	\arrow["f"', from=1-1, to=3-1]
	\arrow["{\eta_A}"{description}, from=1-3, to=1-5]
	\arrow["{F(f)}"', from=1-3, to=3-3]
	\arrow["{G(f)}"', from=1-5, to=3-5]
	\arrow["{\eta_B}"{description}, from=3-3, to=3-5]
\end{tikzcd}\]

    
\end{defi}

A transformação natural identidade é a transformação $1_F:F \Longrightarrow F$, tal que $(1_F)_A: F(A) \longrightarrow F(A)$ é a identidade de $F(A)$. \\
Ainda, dadas as trasformações naturais $\eta:F \Longrightarrow G$ e $\mu: G \Longrightarrow H$, onde $F, G$ e $H$ são funtores de uma categoria $\mathcal{C}$ para uma categoria $\mathcal{D}$, podemos definir a transformação $(\mu \circ \eta): F \Longrightarrow H$ como sendo a família de flechas $(\mu_A \circ \eta_A: F(A) \longrightarrow G(A))_{A \in Obj(\mathcal{C})}$.

Dessa forma, podemos definir o que é a categoria de funtores: 

$\mathbf{Fun(\mathcal{C}, \mathcal{D})}$ é a categoria em que os objetos são funtores de $\mathcal{C}$ para $\mathcal{D}$ e os morfismos são transformações naturais dos funtores de $Obj(\mathbf{Fun(\mathcal{C}, \mathcal{D})})$.

\begin{titlemize}{Lista de consequências}
	\item \hyperref[grupo-fundamental]{Grupo fundamental};\\ %'consequencia1' é o label onde o conceito Consequência 1 aparece
	\item \hyperref[homotopia]{Homotopia}
\end{titlemize}

%[Bianca]: é mais fácil criar a lista de dependências do que a de consequências.

%---------------------------------------------------------------------------------------------------------------------!Draft!-----------------------------------------------------------------------------------------------------------------
\subsection{Transformação Natural}
\label{transformação-natural-categorias-ex}
\begin{titlemize}{Lista de dependências}
	\item \hyperref[transformação-natural-categorias-def]{Transformação natural};\\ %'dependencia1' é o label onde o conceito Dependência 1 aparece (--à arrumar um padrão para referencias e labels--) 
	\item \hyperref[categorias-ex]{Categorias-Exemplos};\\
% quantas dependências forem necessárias.
\end{titlemize}

\begin{ex}[Transformações Naturais]
	Alguns exemplos de transformações naturais.
 
    \begin{itemize}
        \item $J:\mathbf{1_{\mathbf{Vec(\mathbb{K})}}} \Longrightarrow ()^{**} $ é uma transformação natural do funtor identidade no funtor bidual $()^{**}$, de tal forma que $J_X(x)(z^*) = z^*(x)$. Ainda, $f^{**}: A^{**} \longrightarrow B^{**}$ para algum morfismo $f: A \longrightarrow B$ é a transformação linear, tal que $f^{**}(z^{**})(x^*) = z^{**}(x^{*} \circ f)$. Então o seguinte diagrama comuta:
       % https://q.uiver.app/#q=WzAsNixbMiwwLCJWIl0sWzQsMCwiVl57Kip9Il0sWzIsMiwiVyJdLFs0LDIsIldeeyoqfSJdLFswLDAsIlYiXSxbMCwyLCJXIl0sWzQsNSwiTCIsMl0sWzAsMiwiTCIsMl0sWzEsMywiTF57Kip9IiwyXSxbMCwxLCJKX1YiLDEseyJzdHlsZSI6eyJ0YWlsIjp7Im5hbWUiOiJob29rIiwic2lkZSI6InRvcCJ9fX1dLFsyLDMsIkpfVyIsMSx7InN0eWxlIjp7InRhaWwiOnsibmFtZSI6Imhvb2siLCJzaWRlIjoidG9wIn19fV1d
\[\begin{tikzcd}
	V && V && {V^{**}} \\
	\\
	W && W && {W^{**}}
	\arrow["L"', from=1-1, to=3-1]
	\arrow["{J_V}"{description}, hook, from=1-3, to=1-5]
	\arrow["L"', from=1-3, to=3-3]
	\arrow["{L^{**}}"', from=1-5, to=3-5]
	\arrow["{J_W}"{description}, hook, from=3-3, to=3-5]
\end{tikzcd}\].

\item Temos o funtor $\#: \mathbf{SET_\omega} \longrightarrow \mathbf{Ord}_\omega$ que leva um conjuto finito em seu respectivo ordinal. Dessa forma, uma classe de bijeções $(\alpha_A: A \hookrightarrow \#(A))_{A \in Obj(\mathbf{SET_\omega})}$ define uma transformação natural.

% https://q.uiver.app/#q=WzAsNixbMiwwLCJBIl0sWzQsMCwiXFwjQSJdLFsyLDIsIkIiXSxbNCwyLCJcXCNCIl0sWzAsMCwiQSJdLFswLDIsIkIiXSxbNCw1LCJmIiwyXSxbMCwyLCJmIiwyXSxbMSwzLCJcXCNmIiwyXSxbMCwxLCJcXGFscGhhX0EiLDFdLFsyLDMsIlxcYWxwaGFfQiIsMV1d
\[\begin{tikzcd}
	A && A && {\#A} \\
	\\
	B && B && {\#B}
	\arrow["f"', from=1-1, to=3-1]
	\arrow["{\alpha_A}"{description}, from=1-3, to=1-5]
	\arrow["f"', from=1-3, to=3-3]
	\arrow["{\#f}"', from=1-5, to=3-5]
	\arrow["{\alpha_B}"{description}, from=3-3, to=3-5]
\end{tikzcd}\]


        
    \end{itemize}
\end{ex}


\begin{titlemize}{Lista de consequências}
	\item \hyperref[hom-grupo-fundamental]{homomorfismo-de-grupo-fundamental};\\ %'consequencia1' é o label onde o conceito Consequência 1 aparece
\end{titlemize}

%---------------------------------------------------------------------------------------------------------------------!Draft!-----------------------------------------------------------------------------------------------------------------
\subsection{Equivalência de Categorias}
\label{equivalência-de-categorias-def}
\begin{titlemize}{Lista de dependências}
	\item \hyperref[funtor-categorias-def]{Funtor};\\ %'dependencia1' é o label onde o conceito Dependência 1 aparece (--à arrumar um padrão para referencias e labels--) 
	\item \hyperref[transformação-natural-categorias-def]{Transformação natural};\\
  \item \hyperref[isomorfismo-em-categorias-def]{Isomorfismo};\\
% quantas dependências forem necessárias.
\end{titlemize}
\begin{defi}[Equivalência de Categorias]
	Uma categoria $\mathcal{C}$ é equivalente a uma categoria $\mathcal{D}$ se, e somente se, existem funtores $F: \mathcal{C} \longrightarrow \mathcal{D}$ e $G: \mathcal{D} \longrightarrow \mathcal{C}$, onde se cumpre $F \circ G \cong \mathbf{1}_\mathcal{D}$ e $G \circ F \cong \mathbf{1}_\mathcal{C}$.
 Onde $\cong$ é o isomorfismo entre os objetos da categoria dos funtores $\mathbf{Fun(\mathcal{C}, \mathcal{D})}$, visto na seção \hyperref[transformação-natural-categorias-def]{Transformação Natural}.
\end{defi}

Note a semelhança dessa definição com a noção de espaços homotopicamente equivalentes.

\begin{titlemize}{Lista de consequências}
	\item \hyperref[grupo-fundamental]{Grupo fundamental};\\ %'consequencia1' é o label onde o conceito Consequência 1 aparece
\end{titlemize}

%[Bianca]: é mais fácil criar a lista de dependências do que a de consequências.



%%% Local Variables:
%%% mode: LaTeX
%%% TeX-master: "../Alg.Top-Wiki"
%%% End:

\section{Espaço de recobrimento}
\label{espaco-de-recobrimento}

\begin{titlemize}{Lista de Dependências}
	\item \hyperref[homotopia]{Homotopia};\\ %homotopia
	\item \hyperref[grupo-fundamental]{Grupo fundamental};\\
    \item \hyperref[topologia-quociente]{Espaço quciente};
\end{titlemize}

Na topologia algébrica, espaços de recobrimento estão intimamente relacionados ao grupo fundamental: Todos os recobrimentos têm a propriedade de levantamento de curva e homotopia, portanto ao invés de acha uma homotopia em espaço original podemos verificar se existir uma homotopia num recobrimento que têm melhores propriedades topológicas. Por isso, os espaços de recobrimento são uma ferramenta importante no cálculo de grupos fundamentais.
\subsection{Espaço de recobrimento}
\label{espaco-de-recobrimento-def}
\begin{titlemize}{Lista de dependências}
	\item \hyperref[topologia-quociente]{Espaço quciente};\\ %'dependencia1' é o label onde o conceito Dependência 1 aparece (--à arrumar um padrão para referencias e labels--) 
% quantas dependências forem necessárias.
\end{titlemize}
\begin{defi}[Espaço de recobrimento]
Uma função contínua $p:E\rightarrow X$ é um \textbf{recobrimento} se para todo $x\in X,$ existe uma vizinhança aberta $U\subseteq X$ de $x$ e um conjunto de índices $\Lambda\ne \varnothing$ tal que 
$$p^{-1}(U)=\amalg_{\lambda\in \Lambda} V_\lambda,$$
onde $V_\lambda\subseteq E$ é um subconjunto aberto e $p|_{V_\lambda}:V_\lambda\rightarrow U$ é homeomorfismo.
\end{defi}

\begin{nota}
Introduzimos algumas terminologias: 
    \begin{itemize}
        \item $E$ é um espaço (total) de recobrimento.
        \item $U$ é um aberto uniformemente recoberto de $X.$
        \item $V_\lambda$ é uma placa de $U$ do recobrimento.
        \item A cardinalidade de $\Lambda$ é o número de folhas do recobrimento (veremos que $\# \Lambda$ não depende de $x$). 
    \end{itemize}
\end{nota}

\begin{ex}
A função $p:\mathbb{R}\rightarrow \mathbb{S}^1$ dada por $p(x)=e^{2\pi ix}$ é um recobrimento: dado $y_0=e^{2\pi i x_0}\in\mathbb{S}^1,$ e $U=\mathbb{S}^1\setminus \{-y_0\}$ nós temos 
$$p^{-1}(U)=\amalg_{k\in \mathbb{Z}} (x_0+\frac{2k-1}{2},x_0+\frac{2k+1}{2}),$$
denotamos intervalo aberto $(x_0+\frac{2k-1}{2},x_0+\frac{2k+1}{2})$ por $V_k.$ Logo, a função $p|_{V_k}:V_k\rightarrow \mathbb{S}^1\setminus\{-y\}$ é um homeomorfismo.
\end{ex}

\begin{ex}
    A função $p:\mathbb{S}^n\rightarrow \mathbb{RP}^n=\mathbb{S}^n/\mathbb{Z}_2$ dada por $p(x)=[x]$ é um recobrimento.
\end{ex}

\begin{ex}
    Dado um conjunto $\Lambda\ne \varnothing$ qualquer munido com a topologia discreta, a função projeção $pr_2:E=\Lambda\times X\rightarrow X$ é um recobrimento. Esse recobrimento é dito \textbf{recobrimento trivial}
\end{ex}

\begin{prop}
    Suponha que $X$ é um espaço topológico conexo e $p:E\rightarrow X$ um recobrimento, então toda fibra tem a mesma cardinalidade, i.e. $\# p^{-1}(x_0)=\# p^{-1}(x_1)$ para todo $x_0,\;x_1\in X.$ Isso mostra que $\Lambda$ não depende de $x$.
\end{prop}

\begin{dem}
    Seja $x_0\in X$ e seja $A=\{x_1\in X: \#p^{-1}(x_1)=\# p^{-1}(x_0)\}.$ O conjunto $A$ não é vazio, pois $x_0\in A.$ Agora vamos provar que $A$ é aberto. Suponha que $x\in A$ e seja $U$ uma vizinhança aberta de $x$ tal que $p^{-1}(U)=\amalg_{\lambda\in \Lambda} V_\lambda$ com $p|_{V_\lambda}:V_\lambda\rightarrow U$ hemeomorfismo. Então, $U\subseteq A,$ pois se $x'\in U,$ então 
    $$\# p^{-1}(x')=\# \Lambda=\# p^{-1}(x)=\# p^{-1}(x_0).$$
    O conjunto $A$ é fechado, pois $X\setminus A$ é aberto pelo mesmo argumento acima. Como $X$ é conexo, $X=A$ como queríamos. 
\end{dem}

\begin{nota}
    Localmente todo recobrimento $p:E\rightarrow U$ é isomorfo ao recobrimento trivial, i.e. para todo $x\in X,$ existem uma vizinhança aberta $U$ de $x$, um espaço topológico discreto $\Lambda,$ e um homeomorfismo $h: E|_U\rightarrow U\times \Lambda$ tal que $pr_1\circ h= p.$
\end{nota}

\begin{titlemize}{Lista de consequências}
	\item \hyperref[levantamento-de-caminhos-prop]{Levantamento de caminhos};\\ %'consequencia1' é o label onde o conceito Consequência 1 aparece
	\item \hyperref[levantamento-de-homotopia-prop]{Levantamento de homotopia}
\end{titlemize}

\input{conteudo/levantamento-de-caminhos-prop}
\input{conteudo/levantamento-de-homotopia-prop}
\input{conteudo/grupo-fundamental-de-espaco-projetivo-ex}
\subsection{Grupo fundamental de 1-esfera}
\label{grupo-fundamental-de-S1-prop}
\begin{titlemize}{Lista de dependências}
	\item \hyperref[levantamento-de-homotopia-prop]{Levantamento de homotopia};\\ %'dependencia1' é o label onde o conceito Dependência 1 aparece (--à arrumar um padrão para referencias e labels--) 
	\item \hyperref[espaco-de-recobrimento-def]{Espaço de recobrimento};\\
    \item \hyperref[grupo-fundamental]{Grupo fundamental}
% quantas dependências forem necessárias.
\end{titlemize}

\begin{thm}
    O grupo fundamental $\pi_1(\mathbb{S}^1,1)$ é isomorfo a $\mathbb{Z}.$ 
\end{thm}

\begin{dem}
Vimos que $p:\mathbb{R}\rightarrow \mathbb{S}^1$ com $x\mapsto e^{2\pi i x}$ é um recobrimento. Seja $deg:\pi_1(\mathbb{S}^1,1)\rightarrow \mathbb{Z}=p^{-1}(1)\subseteq \mathbb{R}$ uma função dada por $deg([\alpha])=\Tilde{\alpha}_0(1),$ O corolário \ref{cor:bijedeggene} garante que $deg$ é uma bijeção. Vamos mostrar que $deg$ é um homomorfismo: Note que se $\alpha,\;\beta\in \Omega(X,x),$ então
\begin{itemize}
    \item $\Tilde{\alpha}_e*\Tilde{\beta}_{\Tilde{\alpha}_e (1)}(0)=\Tilde{\alpha}_e(0)=e,$
    \item $p(\Tilde{\alpha}_e*\Tilde{\beta}_{\Tilde{\alpha}_e (1)})=\alpha *\beta,$
\end{itemize}
logo, pelo unicidade de levantamento, temos 
\begin{align*}
\widetilde{(\alpha*\beta)}_e=\begin{cases}
    \Tilde{\alpha}_e (2s)\qquad& 0\le s\le \frac{1}{2}\\
    \Tilde{\beta}_{\Tilde{\alpha}_e (1)}(2s-1)&\frac{1}{2}\le s\le 1
    \end{cases}=\Tilde{\alpha}_e*\Tilde{\beta}_{\Tilde{\alpha}_e (1)}.
\end{align*}
Logo $deg([\alpha*\beta])=\widetilde{(\alpha*\beta)}_0 (1)=\Tilde{\alpha}_0*\Tilde{\beta}_{\Tilde{\alpha}_0 (1)}(1)=\Tilde{\beta}_{deg(\alpha)}(1).$

Por unicidade de levantamento de novo, obtemos $\Tilde{\beta}_n=n+\Tilde{\beta}_0.$ Logo,
\[deg([\alpha*\beta])=\Tilde{\beta}_{deg(\alpha)}(1)=deg(\alpha)+\Tilde{\beta}_0 (1)=deg(\alpha)+deg(\beta).\]
Portanto $deg$ é um isomorfismo de grupo.
\end{dem}

\begin{titlemize}{Lista de consequências}
	\item \hyperref[teo-ponto-fixo-brower]{Teorema Ponto Fixo de Brower};
\end{titlemize}

\subsection{Grupo fundamental de toros}
\label{grupo-fundamental-de-toro-ex}
\begin{titlemize}{Lista de dependências}
    \item \hyperref[homotopia-def]{Homotopia};\\
    \item \hyperref[grupo-fundamental]{Grupo fundamental};\\
    \item \hyperref[hom-grupo-fundamental]{Homomorfismo de grupos fundamentais};\\
    \item \hyperref[grupo-fundamental-de-espaco-de-produtos-prop]{Grupo fundamental de espaço de produtos};\\
    \item \hyperref[grupo-fundamental-de-S1-prop]{Grupo fundamental de 1-esfera}.
    
\end{titlemize}

\begin{ex}
    Como o toro $\mathbb{T}^n$ é homeomorfo ao $\mathbb{S}^1\times \ldots \times \mathbb{S}^1$ ($n$ fatores). Pelas proposições \ref{grupo-fundamental-de-espaco-de-produtos-prop}, \ref{hom-grupo-fundamental} e \ref{grupo-fundamental-de-S1-prop}, obtemos 
    \[\pi_1(\mathbb{T}^n)\cong \pi_1(\mathbb{S}^1)\times\ldots\times\pi_1(\mathbb{S}^1)\cong\mathbb{Z}^n.\]
\end{ex}

%%% Local Variables:
%%% mode: LaTeX
%%% TeX-master: "../Alg.Top-Wiki"
%%% End:

\section{Retração}
\label{retração}
Retração é uma relação de um subespaço com o espaço todo. Podemos pensar essa relação como retraindo todo o espaço para aquele subespaço (por isso o nome). Uma consequência que podemos tirar da existência ou não existência de uma retração é o famoso teorema do ponto fixo de Brower. 

\subsection{Retração}
\label{retração-def}
\begin{defi}[Retração]
Seja $A \subseteq X$ um subespaço de $X$. Uma retração de $A$ em $X$ é uma função contínua $r:X \to A$, tal que $r\restriction_A = id_A$.	 
\end{defi}

% onde conteudos.tex é o nome do arquivo tex que voce quer incluir nessa secção.
%---------------------------------------------------------------------------------------------------------------------!Draft!-----------------------------------------------------------------------------------------------------------------
\subsection{Retrato por Deformação}
\label{retrato-por-deformação-def}
\begin{titlemize}{Lista de dependências}
	\item \hyperref[retração-def]{Retração};\\ %'dependencia1' é o label onde o conceito Dependência 1 aparece (--à arrumar um padrão para referencias e labels--) 
	\item \hyperref[homotopia]{Homotopia};\\
% quantas dependências forem necessárias.
\end{titlemize}
\begin{defi}[Retrato por deformação]
	Uma retração $r:X \rightarrow Y$ é um retrato por deformação se $(i\circ r) \sim id_X$, onde $i:Y \rightarrow X$ é a inclusão de $Y$ em $X$.
\end{defi}
    \begin{ex}
    Denotamos a esfera de raio $1/2$ por $\mathbb{S}^n_{1/2}$, e a função inclusão de $\mathbb{S}^n_{1/2}$ em $\text{int}(D^n)\setminus\{0\}$ por $i$. A função $r:\text{int}(D^n)\setminus\{0\}\longrightarrow \mathbb{S}^{n-1}_{1/2} $ dada por $x\longmapsto \frac{x}{2||x||}$ é um retrato por deformação, para todo $n\ge 2$, pois a função
    \begin{align*}
        H:\text{int}(D^n)\setminus\{0\} \times I &\longrightarrow \text{int}(D^n)\setminus\{0\}\\
        (x,t)&\longmapsto (1-t)x+t\frac{x}{2||x||}
    \end{align*}
    é uma homotopia entre $id_{\text{int}(D^n)\setminus\{0\}}$ e $i\circ r$.
\end{ex}

%---------------------------------------------------------------------------------------------------------------------!Draft!-----------------------------------------------------------------------------------------------------------------
\subsection{Lema da Retração} %afirmação aqui significa teorema/proposição/colorário/lema
\label{lema-retração}
\begin{titlemize}{Lista de dependências}
	\item \hyperref[homotopia]{Homotopia};\\ %'dependencia1' é o label onde o conceito Dependência 1 aparece (--à arrumar um padrão para referencias e labels--) 
	\item \hyperref[retração-def]{Retração};\\
% quantas dependências forem necessárias.
\end{titlemize}
O lema a seguir será importante na demonstração do Teorema do Ponto Fixo de Brower.
\begin{lemma}[Lema da Retração]% ou af(afirmação)/prop(proposição)/corol(corolário)/lemma(lema)/outros ambientes devem ser definidos no preambulo de Alg.Top-Wiki.tex 
	Não existe uma retração $r:D^2 \longrightarrow \partial D^2 = S^1$.
\end{lemma}

\begin{dem}
Suponha que $r:D^2 \longrightarrow S^1$ seja uma retração. Sendo $D^2$ um espaço contrátil, pois ele é convexo, temos que para todo laço $\alpha: I \Longrightarrow D^2$ existe uma homotopia que leva esse laço no ponto $\alpha(0) = \alpha(1) = x_0$ de $D^2$. Em particular, para um laço $\beta: I \longrightarrow S^1 = \partial D^2$ em $S^1$ existe uma homotopia relativa a $\partial I$, $H: I\times I \longrightarrow D^2$ tal que $H(t, 0) = \beta(t)$ e $H(t, 1) = \beta(0) = x \in S^1$. Se a retração $r$ existe, então $r\circ H: I\times I: \longrightarrow S^1$ é uma homotopia relativa a $\partial I$. De fato, $(r\circ H)(0, t) = r(\beta(t)) = \beta(t)$ e $(r\circ H)(s, 0) = r(x) = x$ e $(r\circ H)(1, t) = r(x) = x$. Dessa forma, teríamos que $S^1$ é contrátil, contrariando $\pi_1(S^1) = \mathbb{Z}$.
\end{dem}

\begin{titlemize}{Lista de consequências}
	\item \hyperref[teo-ponto-fixo-brower]{Teorema de ponto fixo de Brouwer};\\ %'consequencia1' é o label onde o conceito Consequência 1 aparece
\end{titlemize}

%[Bianca]: Um arquivo tex pode ter mais de uma afirmação (ou definição, ou exemplo), mas nesse caso cada afirmação deve ter seu próprio label. Dar preferência para agrupar afirmações que dependam entre sí de maneira próxima (um teorema e seu corolário, por exemplo)

\subsection{Teorema de ponto fixo de Brouwer} %afirmação aqui significa teorema/proposição/colorário/lema
\label{teo-ponto-fixo-brower}
\begin{titlemize}{Lista de dependências}
	\item \hyperref[homotopia]{Homotopia};\\ %'dependencia1' é o label onde o conceito Dependência 1 aparece (--à arrumar um padrão para referencias e labels--) 
	\item \hyperref[retração-def]{Retração};\\
    \item \hyperref[lema-retração]{Lema da Retração};\\
% quantas dependências forem necessárias.
\end{titlemize}
O teorema a seguir depende de um lema que será deixado na lista de dependências acima.
\begin{thm}[Teorema do Ponto Fixo de Brower]% ou af(afirmação)/prop(proposição)/corol(corolário)/lemma(lema)/outros ambientes devem ser definidos no preambulo de Alg.Top-Wiki.tex 
	Toda função contínua na bola possui ponto fixo, i.e, se $f:D^2 \longrightarrow D^2$, então existe $x \in D^2$, tal que $f(x) = x$.
\end{thm}

\begin{dem}
    Suponha por absurdo que exista uma função contínua $f:D^2 \longrightarrow D^2$ sem pontos fixos. Defino a função $\alpha: D^2 \longrightarrow S^1$ onde $\alpha(x)$ é o único ponto de intersecção da semirreta $\overrightarrow{f(x)x}$ com $S^1$. Essa função está bem definida, pois $t_x(\lambda) = \|f(x) + \lambda(x - f(x))\|$ é uma função real contínua tal que $t_x(0) \leq 1$ e $t_x \to \infty$ quando $x \to \infty$. Assim, pelo teorema do valor intermediário, existe $\lambda_x$ tal que $t_x(\lambda_x) = 1$. $\alpha$ é contínua, pois $\alpha(x) = f(x) + \lambda_x(x - f(x))$, onde $\lambda_x$ é expresso da seguinte forma:
    $\|f(x) + \lambda_x(x - f(x))\| = \|(x - f(x))\|^2\lambda_x^2 + 2\langle f(x), (x - f(x)) \rangle\lambda_x + \|f(x)\|^2 = 1$. Isso nos dá uma equação quadrática com 2 soluções reais, sendo a maior delas $$\lambda_x = \frac{-2\langle f(x), x - f(x)\rangle + \sqrt{(2\langle f(x), x - f(x)\rangle)^2 - 4(\|x - f(x)\|^2)(\|f(x)\|^2 - 1)}}{2(\|x - f(x)\|)}.$$ Note que não há problema com o quociente desde que assumimos por hipótese que $f(x) \ne x$ para todo $x$. Ainda, o termo dentro da raiz quadrada é sempre maior ou igual a 0, pela desigualdade de Schwarz. Além disso, se $x \in S^1$, então $\alpha(x) = x$. Portanto, $\alpha$ é uma retração, contrariando o lema mencionado anteriormente.

\end{dem}

%[Bianca]: Um arquivo tex pode ter mais de uma afirmação (ou definição, ou exemplo), mas nesse caso cada afirmação deve ter seu próprio label. Dar preferência para agrupar afirmações que dependam entre sí de maneira próxima (um teorema e seu corolário, por exemplo)

\subsection{Lema da Retração (versão geral)} %afirmação aqui significa teorema/proposição/colorário/lema
\label{lema-de-retracao-geral-prop}
\begin{titlemize}{Lista de dependências}
    \item \hyperref[homologia-singular-def]{Homologia singular};\\
    \item \hyperref[homomorfismo-de-homologias-singulares-induzido-prop]{Homomorfismo de homologias singulares induzido};\\
    \item \hyperref[homologia-singular-de-S1-prop]{Homologia singular da circunferência};\\
    \item \hyperref[grupo-de-homologia-singular-de-n-esfera-prop]{Grupo de homologia singular de n-esfera}.
\end{titlemize}

Apresentamos aqui uma versão mais geral do lema de retração.

\begin{lemma}
	Para todo $n\ge 2$, não existe uma retração $r:D^n \longrightarrow \partial D^n = \mathbb{S}^{n-1}$.
\end{lemma}

\begin{dem}
Suponha que $r:D^{n} \longrightarrow \mathbb{S}^{n-1}$ seja uma retração. Sendo $D^n$ um espaço contrátil, pois ele é convexo, temos que $H_{n-1}(D^n)=0$ e, por conseguinte, $r_*$ é uma função nula. Sejam $i:\mathbb{S}^{n-1}\hookrightarrow D^n$ a inclusão. Pela definição de retração, $r\circ i=id_{\mathbb{S}^{n-1}}$. Assim, obtemos 
\[0=r_*\circ i_*=(r\circ i)_*=id_{H_{n-1}(\mathbb{S}^{n-1})}.\]
Dessa forma, teríamos que $H_{n-1}(\mathbb{S}^{n-1})=0$, contrariando $H_{n-1}(\mathbb{S}^{n-1})\cong \mathbb{Z}$.
\end{dem}

\begin{titlemize}{Lista de consequências}
    \item \hyperref[teorema-de-ponto-fixo-de-brouwer-geral-prop]{Teorema de ponto fixo de Brouwer (versão geral)}.
	%\item \hyperref[]{}
\end{titlemize}

\subsection{Teorema de ponto fixo de Brouwer (versão geral)} %afirmação aqui significa teorema/proposição/colorário/lema
\label{teorema-de-ponto-fixo-de-brouwer-geral-prop}
\begin{titlemize}{Lista de dependências}
    \item \hyperref[teo-ponto-fixo-brower]{Teorema de ponto fixo de Brouwer};\\
    \item \hyperref[lema-de-retracao-geral-prop]{Lema de retração (versão geral)}.
\end{titlemize}

Apresentamos aqui uma versão mais geral do Teorema de ponto fixo de Brouwer. A prova 

\begin{thm}[Teorema do Ponto Fixo de Brower]% ou af(afirmação)/prop(proposição)/corol(corolário)/lemma(lema)/outros ambientes devem ser definidos no preambulo de Alg.Top-Wiki.tex 
	Se $f:D^n \longrightarrow D^n$ é contínua, então existe $x \in D^n$, tal que $f(x) = x$, ou seja, existe um ponto fixo.
\end{thm}

\begin{dem}
    Suponhamos que exista uma função contínua $f:D^n\longrightarrow D^n$ sem ponto fixo. Usando a mesma construção apresentada no Teorema \ref{teo-ponto-fixo-brower}, podemos obter uma retração $r: D^n\rightarrow\mathbb
    {S}^{n-1}$, o que entra em contradição com o Lema \ref{lema-de-retracao-geral-prop}.

\end{dem}

%\begin{titlemize}{Lista de consequências}
    %\item %\hyperref[homomorfismo-de-homologias-singulares-induzido-prop]{Homomorfismo de homologias singulares induzido}.\\
	%\item \hyperref[]{}
%\end{titlemize}

%%% Local Variables:
%%% mode: LaTeX
%%% TeX-master: "../Alg.Top-Wiki"
%%% End:

\end{document}


%novos assuntos/secções devem ser adicionados através do comando \import{conteudo/}{assunto} para adicionar o arquivo conteudo/assunto.tex

%%% Local Variables:
%%% mode: LaTeX
%%% TeX-master: t
%%% End:
