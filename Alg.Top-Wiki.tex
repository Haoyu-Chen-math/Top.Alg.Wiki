\documentclass{article}
\title{Topologia Algébrica}
%\usepackage{import}
\usepackage{amsmath} %propósitos gerais
\usepackage{amssymb} % comandos como \mathbb
\usepackage{amsthm} % criar novos teoremas

\newif\ifplastex
\plastexfalse
\ifplastex
\else
\usepackage{mdframed} %criar caixas em volta de ambientes
\fi
%-----------------------------------------------------------------------------------------------!!!!Adicionar pacotes apenas acima dessa linha!!!!--------------------------------------------------------------------------------------------
\usepackage{hyperref} %referencias internas e externas


%novos estilos
\ifplastex

\else
\mdfdefinestyle{MyFrame}{%
	innertopmargin=\baselineskip,
	innerbottommargin=\baselineskip,
	innerrightmargin=20pt,
	innerleftmargin=20pt}
\fi

%novos ambientes e teoremas
\theoremstyle{definition}
\newtheorem{defi}{Definição}

\theoremstyle{plain}
\newtheorem{thm}{Teorema}
\newtheorem{prop}{Proposição}
\newtheorem{lemma}{Lema}
\newtheorem{af}{Afirmação}
\newtheorem{corol}{Corolário}
\newtheorem{ex}{Exemplo}

\theoremstyle{remark}
\newtheorem{nota}{Nota}
\ifplastex


\newenvironment{titlemize}[1]{%
	\textbf{{#1}}
	\begin{itemize}}
	{\end{itemize}}

\else

\newenvironment{titlemize}[1]{%
	\begin{mdframed}[style=MyFrame]
	\textbf{{#1}}
	\begin{itemize}}
	{\end{itemize} \end{mdframed}}

\fi
\newenvironment{dem}{
	\begin{proof}[{\bf Demonstração:}]}
	{\end{proof}}
%novos comandos
\usepackage{quiver}
%comandos renovados



\begin{document}
\maketitle
%-------------------------------------------------------------------------------------------------------------!Draft!-------------------------------------------------------------------------------------------------------------------------
\section{Alguns espaços topológicos importantes}
\label{alguns-espacos-topologicos-importantes}
Nesta seção, apresentamos alguns espaços topológicos importantes para o estudo da geometria e da topologia.

\subsection{Esfera}
\label{esfera-def}

\begin{defi}
     Dado $n\geq 0$, a \textbf{$n$-esfera} (unitária com centro na origem), denotada por $\mathbb{S}^n$, é o subespaço topológico de $\mathbb{R}^{n+1}$ definido por 
     \[\mathbb{S}^n=\{(x_0,...,x_n)\in \mathbb{R}^{n+1}:x_0^2+...+x_n^2=1\}.\]
\end{defi}

\begin{prop}
    Seja $N=(0,...,0,1)\in \mathbb{R}^{n+1}$ um ponto. A projeção estereográfica definida por 
    \begin{align*}
        p_N:\mathbb{S}^n\setminus \{N\}&\longrightarrow \mathbb{R}^n\\
        (x_0,...,x_n)&\longmapsto \frac{1}{1-x_n}(x_0,...,x_{n-1})
    \end{align*}
    é um homeomorfismo.
\end{prop}
\begin{dem}
    Aqui, denotamos $x_0^2+...+x_n^2$ por $||(x_0,...,x_n)||_2^2$. Como $p_N$ é contínua em cada coordenada, então $p_N$ é contínua. Agora, definimos uma função da seguinte forma 
    \begin{align*}
        f:\mathbb{R}^n&\longrightarrow \mathbb{S}^n\\
        X=(X_1,...,X_n)&\longmapsto \frac{1}{||X||_2^2+1}(2X_1,...,2X_{n},||X||_2^2-1).
    \end{align*}
    Essa função é bem definida, pois 
    \[\frac{1}{(||X||_2^2+1)^2}(4X_1^2+...+4X_n^2+(||X||_2^2-1)^2)=\frac{4||X||_2^2+(||X||_2^2-1)^2}{(||X||_2^2+1)^2}=1.\]
    Além disso, $f$ é contínua em cada coordenada, logo $f$ é contínua. Por um lado, temos 
    \begin{align*}
        f\circ p_N(x_0,...,x_n)&=\frac{1}{1-x_n}\cdot \frac{(2x_0,...,2x_{n-1},(1-x_n)\cdot(||p_N(x_0,...,x_{n}||_2^2-1))}{||p_N(x_0,...,x_n)||_2^2+1}\\
        &=\frac{1}{1-x_n}\cdot\frac{(2x_0,...,2x_{n-1},(1-x_n)\cdot(||p_N(x_0,...,x_{n}||_2^2-1))}{(\frac{1}{(1-x_n)^2}\cdot ||(x_0,...,x_{n-1})||_2^2)+1}\\
        &=\frac{1}{1-x_n}\cdot\frac{(2x_0,...,2x_{n-1},\frac{1-x_n^2}{(1-x_n)}-(1-x_n))}{\frac{(1-x_n^2)}{(1-x_n)^2}+1}\\
        &=\frac{(1-x_n)}{(1-x_n^2+(1-x_n)^2)}\cdot(2x_0,...,2x_{n-1},\frac{1-x_n^2-(1-x_n)^2}{(1-x_n)})\\
        &=\frac{1}{2}\cdot(2x_0,...,2x_{n-1},2x_n)\\
        &=(x_0,...,x_n).
    \end{align*}
    E por outro lado, temos 
    \begin{align*}
        p_N\circ f(X)&=\frac{1}{1-\frac{||X||_2^2-1}{||X||_2^2+1}}\frac{2X}{||X||_2^2+1}\\
        &=\frac{2X}{||X||_2^2+1-||X||_2^2+1}\\
        &=\frac{2X}{2}\\
        &=X.
    \end{align*}
    Isso mostra que $p_N$ é um homeomorfismo.
\end{dem}

\subsection{Toro}
\label{toro-def}
\begin{titlemize}{Lista de dependências}
	\item \hyperref[esfera-def]{Esfera}. %'dependencia1' é o label onde o conceito Dependência 1 aparece (--à arrumar um padrão para referencias e labels--) 
% quantas dependências forem necessárias.
\end{titlemize}
\begin{defi}
     Dado $n\geq 0$, o \textbf{$n$-toro}, denotado por $\mathbb{T}^n$, é definido como o espaço topológico produto $\mathbb{S}^1\times \ldots \times \mathbb{S}^1$ ($n$ fatores).
\end{defi}
\subsection{Disco}
\label{disco-def}
\begin{titlemize}{Lista de dependências}
	\item \hyperref[esfera-def]{Esfera}.
\end{titlemize}
\begin{defi}
     Dado $n\geq 0$, o \textbf{$n$-disco} (unitário com centro na origem), denotado por $D^n$, é o subespaço topológico de $\mathbb{R}^{n}$ definido por 
     \[D^n=\{(x_1,...,x_n)\in \mathbb{R}^{n}:x_1^2+...+x_n^2\le 1\}.\]
\end{defi}
É fácil ver que o bordo do $n$-disco é a ($n-1$)-esfera: $\partial D^n = \mathbb{S}^{n-1}$, e o interior do $n$-disco é a bola aberta $\{(x_1,...,x_n)\in \mathbb{R}^n:x_1^2+...+x_n^2<1\}$.
%%% Local Variables:
%%% mode: LaTeX
%%% TeX-master: "../Alg.Top-Wiki"
%%% End:

%-------------------------------------------------------------------------------------------------------------!Draft!-------------------------------------------------------------------------------------------------------------------------
\section{Grupos Livres}
\label{grupos-livres}

Nesta seção, introduzimos o conceito de grupo livre, que desempenha um papel fundamental em álgebra abstrata e na topologia algébrica.

\subsection{Fecho normal} %afirmação aqui significa teorema/proposição/colorário/lema
\label{fecho-normal-def}

\begin{prop}
    Seja $G$ um grupo, e $K$ um subconjunto de $G$. Então, a interseção 
    \[\overline{K}:=\bigcap_{\substack{N\triangleleft G\\K\subseteq N}}N\]
    de todos os subgrupos normais de $G$ que contém $K$ é um subgrupo normal de $G$. Além disso, se $N$ é um subgrupo normal de $G$ contendo $K$, então $\overline{K}\subseteq N$.
\end{prop}

\begin{dem}
    Seja $k\in \overline{K}$, $g\in G$. Pela definição de $\overline{K}$, $k$ pertence a todos os subgrupos normais que contêm $K$. Pela definição de subgrupo normal, $gkg^{-1}$ também pertence a todos os subgrupos normais que contêm $K$, ou seja $gkg^{-1}\in \overline{K}$. Como $k$ e $g$ são arbitrários, concluímos que $\overline{K}$ é um subgrupo normal de $G$. Além disso, pela definição de $\overline{K}$, se $N$ é um subgrupo normal de $G$ contendo $K$, então $\overline{K}\subseteq N$.
\end{dem}

Por essa razão o grupo $\overline{K}$ é chamado \textbf{fecho normal de} $K$ ou \textbf{subgrupo normal gerado por} $K$.
%---------------------------------------------------------------------------------------------------------------------!Draft!-----------------------------------------------------------------------------------------------------------------
\subsection{Geradores e Relações}
\label{geradores-relacoes-def}
\begin{titlemize}{Lista de dependências}
	\item \hyperref[fecho-normal-def]{Fecho normal}.%'dependencia1' é o label onde o conceito Dependência 1 aparece (--à arrumar um padrão para referencias e labels--) 
	%\item \hyperref[]{};\\
% quantas dependências forem necessárias.
\end{titlemize}

\newcommand{\Ast}{\mathop{\scalebox{1.5}{\raisebox{-0.2ex}{$\ast$}}}}

\begin{defi}[Produto Direto]
    Dada uma coleção de grupos $\{G_j~|~j\in J\}$ qualquer, o \textbf{produto direto de $\{G_j~|~j\in J\}$} é o grupo $\prod_{j\in J} G_j$, onde o conjunto de elementos é o produto cartesiano e a operação é dada coordenada a coordenada:
    \[(g_j)_{j\in J} * (h_j)_{j\in J} = (g_j h_j)_{j\in J}\]

    O elemento neutro é $(e_j)_{j\in J}$, onde $e_j$ é o elemento neutro de $G_j$ para cada $j\in J$, e o elemento inverso de cada $(g_j)_{j\in J} \in \prod_{j\in J} G_j$ é $(g_j^{-1})_{j\in J}$.
\end{defi}

\begin{defi}[Produto Livre]
	Seja $\{G_j~|~j\in J\}$ uma coleção de grupos qualquer, e seja $e_j$ o elemento neutro de $G_j$ para cada $j\in J$. Considere o conjunto $S$ de sequências finitas
    $(a_1, \ldots, a_m)$, onde $m\geq 0$ e $a_1,\ldots, a_m \in \bigsqcup_{j\in J} G_j$, e defina $\sim$ como a menor relação de equivalência tal que 
    \[(a_1,\ldots, a_i, a_{i+1},\ldots, a_m) \sim 
    (a_1,\ldots, a_i *_j a_{i+1},\ldots, a_m),\] se $j\in J$ e $a_i, a_{i+1} \in G_j$, e também
    \[(a_1,\ldots, a_i, e_j, a_{i+2},\ldots, a_m) \sim (a_1,\ldots, a_i, a_{i+2},\ldots, a_m)\]
    para todo $j\in J$. Definimos o \textbf{produto livre de $\{G_j~|~j\in J\}$} como $\Ast_{j\in J} G_j = S/\sim$, em que a classe de equivalência de uma sequência $(a_1,\ldots, a_m)$ é denotada por $a_1 \ldots a_m$. A operação $*$ em $\Ast_{j\in J} G_j$ é dada pela concatenação:
    \[a_1 \ldots a_m * b_1 \ldots b_n = a_1 \ldots a_m \, b_1 \ldots b_n.\]
    O elemento neutro é dado pela classe de equivalência da sequência nula, que denotamos por $e$. Esta coincide com a classe de equivalência de cada $e_j$. O elemento inverso pode ser calculado como
    \[(a_1 \ldots a_m)^{-1} = a_m^{-1} \ldots a_1^{-1}.\]
    
    É simples ver que $\Ast_{j\in J} G_j$ é um grupo. Em algumas situações, usamos a notação $(a_1)...(a_m)$ para denotar os elementos $a_1...a_m\in \Ast_{j\in J} G_j $.
    %e que as inclusões naturais de cada $G_j$ em $\Ast_{j\in J} G_j$ são homomorfismos
    % eu provei isso na parte de pushout, então eu vou tirar nisso.
\end{defi}

\begin{defi}
    Seja $\mathcal{G}$ um conjunto não vazio. Definimos o \textbf{grupo livre gerado (ou grupo cíclico infinito gerado) por $a \in \mathcal{G}$} como $F(a) = \langle a\rangle = \{a^n~|~n \in \mathbb{Z}\}$, onde a operação é dada somando-se os expoentes:
    \[a^m * a^n = a^{m+n},\quad \forall m,n \in \mathbb{Z}.\]
    É claro que $\langle a\rangle$ é isomorfo a $\mathbb{Z}$.
    
    O \textbf{grupo livre gerado por $\mathcal{G}$} é definido como $\Ast_{a\in \mathcal{G}} \langle a\rangle$, enquanto o \textbf{grupo abeliano livre gerado por $\mathcal{G}$} é definido como $\prod_{a\in \mathcal{G}} \langle a\rangle$. Estes são denotados, respectivamente, como $F(\mathcal{G}) = \langle \mathcal{G}\rangle$ e $\mathbb{Z}(\mathcal{G})$, respectivamente. No caso em que $\mathcal{G}$ é finito, digamos, 
    $\mathcal{G} = \{a_1,\ldots, a_m\}$, escrevemos $F(\mathcal{G}) = F(a_1,\ldots, a_m)$ e $\mathbb{Z}(\mathcal{G}) = \mathbb{Z}(a_1, \ldots, a_m)$.
    
    Seja agora $R$ um conjunto de elementos de $\langle\mathcal{G}\rangle$. Definimos o grupo $\langle \mathcal{G}~|~R\rangle$ como o quociente de $\langle\mathcal{G}\rangle$ pelo subgrupo normal gerado por $R$. Nesse caso, dizemos que cada elemento $a \in \mathcal{G}$ é um \textbf{gerador}, cada igualdade da forma $a_1 \ldots a_m = b_1 \ldots b_n$, onde $(a_1 \ldots a_m) * (b_1 \ldots b_n)^{-1} \in R$, é uma \textbf{relação}, e o grupo quociente é \textbf{dado por geradores e relações}. Caso $R$ seja finito, também escrevemos cada relação explicitamente. Por exemplo, o subgrupo de $\mathbb{C}^{\times}$ gerado por $i$ é isomorfo a $\langle i^n ~|~ i^4 = 1\rangle$.
\end{defi}

\begin{ex}
    Note que
    \begin{align*}
        D_4 &\cong \langle a,b ~|~a^4 = b^2 = e, ab = ba^{-1} \rangle\\
        &\cong  \langle r,s~|~ r^2 = s^2 = e, (rs)^4=e \rangle,
    \end{align*}
    em especial, a apresentação do grupo por meio de geradores e relações não é única.
\end{ex} 

Todo gurpo pode ser apresentado em termos de geradores e relações 
\[F(G)/\langle R\rangle\cong G\]
onde $R=\{(e),(g)(g)^{-1},(g)(h)(gh)^{-1}\}.$
%---------------------------------------------------------------------------------------------------------------------!Draft!-----------------------------------------------------------------------------------------------------------------
\subsection{\emph{Pushout} de grupos} %afirmação aqui significa teorema/proposição/colorário/lema
\label{pushout-de-grupos-prop}
\begin{titlemize}{Lista de dependências}
	\item \hyperref[fecho-normal-def]{Fecho normal};\\
    \item \hyperref[geradores-relacoes-def]{Geradores e Relações}.%'dependencia1' é o label onde o conceito Dependência 1 aparece (--à arrumar um padrão para referencias e labels--) 
	%\item \hyperref[]{};\\
% quantas dependências forem necessárias.
\end{titlemize}

\begin{lemma}
    Sejam $G_1$ e $G_2$ grupos. Definimos $j_1:G_1\rightarrow G_1*G_2$ pela função $g_1\mapsto (g_1)$ e $j_2:G_2\rightarrow G_1*G_2$ pela função $g_2\mapsto (g_2)$. Então, $j_1$ e $j_2$ são homomorfismos de grupos.
\end{lemma}

\begin{dem}
    Sejam $g,g'\in G$. Pela definição de produto livre de grupos, temos 
    \[j_1(gg')=(gg')=(g)(g')=j_1(g)j_1(g').\]
    Isso mostra que $j_1$ é um homomorfismo de grupos. De maneira análoga, pode-se provar que $j_2$ também é um homomorfismo de grupos.
\end{dem}

\begin{prop}
    Sejam $G_1,G_2$ grupos. Então, o seguinte diagrama 
    % https://q.uiver.app/#q=WzAsNCxbMCwwLCJcXHtlXFx9Il0sWzEsMCwiR18xIl0sWzAsMSwiR18yIl0sWzEsMSwiR18xKkdfMiJdLFsxLDMsImpfMSJdLFsyLDMsImpfMiIsMl0sWzAsMSwiaV8xIl0sWzAsMiwiaV8yIiwyXV0=
\[\begin{tikzcd}
	{\{e\}} & {G_1} \\
	{G_2} & {G_1*G_2}
	\arrow["{i_1}", from=1-1, to=1-2]
	\arrow["{i_2}"', from=1-1, to=2-1]
	\arrow["{j_1}", from=1-2, to=2-2]
	\arrow["{j_2}"', from=2-1, to=2-2]
\end{tikzcd}\]
é um diagrama de \emph{pushout}.
\end{prop}

\begin{dem}
    Como um homomorfismo de grupos mapeia a identidade na identidade, temos que $j_1\circ i_1=j_2\circ i_2$. Isso implica que o diagrama do enunciado é comutativo.
    
    Sejam $\phi_1:G_1\rightarrow H$ e $\phi_2: G_2\rightarrow H$ dois homomorfismos de grupos tais que $\phi_1\circ i_1=\phi_2\circ i_2$ (essa propriedade é satisfeita por quaisquer dois homomorfismos). Definimos $\psi: G_1*G_2\rightarrow H$ pela função $\psi((g_1)...(g_k))=\phi_i(g_1)...\phi_i(g_k)$, onde $\phi_i(g_j)=\phi_1(g_j)$ se $g_j\in G_1$, e $\phi_i(g_j)=\phi_2(g_j)$ se $g_j\in G_2$. 
    
    A função $\psi$ é um homomorfismo de grupos, pois   
    \begin{align*}
    \psi((g_1)...(g_k)(g'_1)...(g'_l))&=\phi_i(g_1)...\phi_i(g'_l)=(\phi_i(g_1)...\phi_i(g_k))(\phi_i(g'_1)...\phi_i(g'_l))\\
    &=\psi((g_1)...(g_k))\psi((g'_1)...(g'_l)).
    \end{align*}
    Além disso $\psi\circ j_1=\phi_1$ e $\psi\circ j_2=\phi_2$, pois essas composições são iguais ponto a ponto. 
    
    Falta mostrar a unicidade: Supõe que $\psi':G_1*G_2\rightarrow H$ é um outro homomorfismo de grupos tal que $\psi'\circ j_1=\phi_1$ e $\psi'\circ j_2=\phi_2$. Então, para qualquer $w=(g_1)...(g_k)\in G_1* G_2$, temos 
    \[\psi'(w)=\psi'((g_1))...\psi'((g_k))=\phi_i(g_1)...\phi_i(g_k)=\psi(w),\]
    o que mostra a unicidade.
    
    Portanto, o diagrama no enunciado é um diagrama de \emph{pushout}.
\end{dem}

Agora, abordaremos o caso geral de \emph{pushout} de grupos:

\begin{thm}
    Sejam $N,G_1,G_2$ grupos. Então, o seguinte diagrama 
    % https://q.uiver.app/#q=WzAsNCxbMCwwLCJOIl0sWzEsMCwiR18xIl0sWzAsMSwiR18yIl0sWzEsMSwiR18xKkdfMi9JIl0sWzEsMywiXFxvdmVybGluZXtqXzF9Il0sWzIsMywiXFxvdmVybGluZXtqXzJ9IiwyXSxbMCwxLCJpXzEiXSxbMCwyLCJpXzIiLDJdXQ==
\[\begin{tikzcd}
	N & {G_1} \\
	{G_2} & {G_1*G_2/I}
	\arrow["{i_1}", from=1-1, to=1-2]
	\arrow["{i_2}"', from=1-1, to=2-1]
	\arrow["{\overline{j_1}}", from=1-2, to=2-2]
	\arrow["{\overline{j_2}}"', from=2-1, to=2-2]
\end{tikzcd}\]
é um diagrama de \emph{pushout}, onde $I=\overline{\{(i_1(n))(i_2(n))^{-1}:n\in N\}}$, e $\overline{j_1}=\pi\circ j_1$ e $\overline{j_2}=\pi\circ j_2$, sendo $\pi:G_1*G_2\rightarrow G_1* G_2/I$ é a projeção associada ao quociente.
\end{thm}

\begin{dem}
    Pela definição de $G_1*G_2/I$, temos que $\overline{j_1}\circ i_1(n)=\overline{j_2}\circ i_2(n)$ para todo $n\in N$. Isso implica que o diagrama do enunciado é comutativo.
    
    Sejam $\phi_1:G_1\rightarrow H$ e $\phi_2: G_2\rightarrow H$ dois homomorfismos de grupos tais que $\phi_1\circ i_1=\phi_2\circ i_2$. Seja $\psi$ o homomorfismo definido na proposição anterior. Como $\text{Ker}(\psi)$ é normal e 
    \begin{align*}
        \psi((i_1(n))(i_2(n))^{-1})&=\phi_1(i_1(n))\phi_2((i_2(n))^{-1})=\phi_1(i_1(n))(\phi_2(i_2(n)))^{-1}\\
        &=\phi_1(i_1(n))(\phi_1(i_1(n)))^{-1}=e,
    \end{align*}
    pela proposição \ref{fecho-normal-def}, temos que $I\subseteq\text{Ker}(\psi)$. Assim, pelo teorema do homomorfismo, existe um único homomorfismo $\overline{\psi}:G_1*G_2/I\rightarrow H$, tal que $\overline{\psi}\circ \pi=\psi$. Além disso, temos 
    \[\overline{\psi}\circ \overline{j_1}=\overline{\psi}\circ \pi\circ j_1=\psi\circ j_1=\phi_1.\]
    De maneira análoga, obtemos $\overline{\psi}\circ\overline{j_2}=\phi_2$

    Resta mostrar a unicidade. Suponhamos que $\psi':G_1*G_2\rightarrow H$ seja outro homomorfismo de grupos tal que $\psi'\circ j_1=\phi_1$ e $\psi'\circ j_2=\phi_2$. Então, para qualquer $w=(g_1)...(g_k)\in G_1* G_2$, temos 
    \begin{align*}
        \psi'(\pi(w))&=\psi'(\pi((g_1)))...\psi'(\pi((g_k)))=\psi'\circ\overline{j}_i(g_1)...\psi'\circ\overline{j}_i(g_k)\\
        &=\phi_i(g_1)...\phi_i(g_k)=\psi(w)=\overline{\psi}(\pi(w)),
    \end{align*}
    onde $\overline{j}_i(g_j)=\overline{j}_1(g_j)$ se $g_j\in G_1$, e $\overline{j}_i(g_l)=\phi_2(g_l)$ se $g_l\in G_2$. Como $\pi$ é sobrejetor, concluímos que $\overline{\psi}=\psi'$, o que mostra a unidade.

    Portanto, o diagrama no enunciado é um diagrama de \emph{pushout}.
\end{dem}

\begin{corol}
Sejam $N,G_1$ grupos. Então, o seguinte diagrama 
    % https://q.uiver.app/#q=WzAsNCxbMCwwLCJOIl0sWzEsMCwiR18xIl0sWzAsMSwiXFx7ZVxcfSJdLFsxLDEsIkdfMS9cXG92ZXJsaW5le1xcdGV4dHtJbX0oaV8xKX0iXSxbMSwzLCJcXG92ZXJsaW5le2pfMX0iXSxbMiwzLCJcXG92ZXJsaW5le2pfMn0iLDJdLFswLDEsImlfMSJdLFswLDIsImlfMiIsMl1d
\[\begin{tikzcd}
	N & {G_1} \\
	{\{e\}} & {G_1/\overline{\text{Im}(i_1)}}
	\arrow["{i_1}", from=1-1, to=1-2]
	\arrow["{i_2}"', from=1-1, to=2-1]
	\arrow["{\overline{j_1}}", from=1-2, to=2-2]
	\arrow["{\overline{j_2}}"', from=2-1, to=2-2]
\end{tikzcd}\]
é um diagrama de \emph{pushout}.
\end{corol}

% \begin{titlemize}{Lista de consequências}
% 	\item \hyperref[consequencia1]{Consequência 1};\\ %'consequencia1' é o label onde o conceito Consequência 1 aparece
% 	\item \hyperref[]{}
% \end{titlemize}
%-------------------------------------------------------------------------------------------------------------!Draft!-------------------------------------------------------------------------------------------------------------------------
\section{Topologia quociente}
\label{topologia-quociente}
Um assunto que aparece de forma recorrente na topologia algébrica é o conceito de topologia quociente, que exploraremos a seguir. 

\subsection{Topologia Quociente}
\label{topologia-quociente-def}
% \begin{titlemize}{Lista de dependências}
% 	\item \hyperref[topologia-final]{Topologia Final}; 
% \end{titlemize}
\begin{defi}[Topologia Quociente]
	Seja \(X\) um espaço topológico e \(\sim\) uma relação de equivalência em \(X\).
	Podemos conferir ao espaço \(X/\sim\) uma estrutura de espaço topológico da seguinte maneira. Considere a função projeção
	\begin{align*}
		\pi:X&\to X/\sim;\\
		x&\mapsto [x].
	\end{align*}
	Podemos fazer com que \(\pi\) seja uma função contínua munindo \(X/\sim\) com a \emph{topologia final} com relação à \(\pi\). Isto é, um subconjunto de \(X/\sim\) é aberto se, e somente se, sua pré-imagem por $\pi$ é aberto de \(X\).
\end{defi}

Varios exemplos importantes de espaços topológicos com os quais trabalharemos no estudo de topologia algébrica podem ser construídos como espaços quocientes. Em particular, uma construção muito útil é a de tomar o quociente de um espaço por um subespaço, como explicado na seguinte definição.
\begin{defi}[Quociente por um subespaço]
	Seja \(X\) um espaço topológico e \(A \subseteq X\) um subespaço. Definimos a seguinte relação binária, \(\sim_A\):\\
    \centerline{
	\(a\sim_A b\) se e somente se \(a=b\) ou \(a,b\in A\).}\\ Essa relação é de equivalência, e assim definimos \(X/A = X/\sim_A\). 
\end{defi}

Vejamos alguns exemplos simples.

\begin{ex}
    \begin{itemize}
        \item O círculo \(\mathbb{S}^1 = \mathbb{T}^1\) pode ser construído como \(I/\{0,1\}\), onde \(I=[0,1]\).
        \item Mais geralmente, o $n$-toro $\mathbb{T}^n$ pode ser construído como $[0,1]^n/\sim$, onde $\sim$ é a relação de equivalência que identifica $x = (x_1,\ldots,x_n), y = (y_1,\ldots,y_n) \in [0,1]^n$ se existe $1\leq i\leq n$ tal que $x_j = y_j$ para todo $j \neq i$ e $\{x_i,x_j\} = \{0,1\}$, ou então se $x=y$.
    \end{itemize}
\end{ex}

\begin{titlemize}{Lista de consequências}
    \item \hyperref[funcao-continua-em-topologia-quociente-prop]{Função contínua em topologia quociente}
	\item \hyperref[topologia-quociente-hausdorff-thm]{Espaços quocientes Hausdorff}
\end{titlemize}


% onde conteudos.tex é o nome do arquivo tex que voce quer incluir nessa secção.
\subsection{Função contínua em topologia quociente}
\label{funcao-continua-em-topologia-quociente-prop}
\begin{titlemize}{Lista de dependências}
	\item \hyperref[topologia-quociente-def]{Topologia quociente}; 
\end{titlemize}

\begin{prop}
    Sejam $X,Y$ espaços topológicos. E seja $\sim$ uma relação de equivalência em $X$. Uma função $f:(X/\sim) \longrightarrow Y$ é contínua se e somente se $f\circ \pi$ é contínua, onde $\pi$ é a função projeção associada ao quociente.
\end{prop}
\begin{dem}
    Por um lado, suponhamos que $f$ seja contínua. Como a composição de funções contínuas é contínua, a função $f\circ \pi$ também seré contínua.

    Por outro lado, suponhamos que $f\circ \pi$ seja contínua. Seja $V\subseteq Y$ um aberto, pela hipóteses, $\pi^{-1}(f^{-1}(V))$ é um aberto. Agora, pela definição de topologia quociente, $f^{-1}(V)$ é um aberto em $X/\sim$, o que implica que $f$ é contínua. 
\end{dem}

Alguns exemplos importantes de espaços quociente são os seguintes.
\subsection{Espaço Projetivo}
\label{espaco-projetivo-def}
\begin{defi}
     Sejam $n\geq 0$ e $V$ um espaço vetorial sobre o corpo $\mathbb{K}$. Definimos o \textbf{espaço projetivo sobre $V$} como o espaço topológico quociente $\mathbb{P}(V) = V/\sim$, onde $x\sim y$ se, e somente, existe $\alpha \in \mathbb{K} \setminus\{0\}$ tal que $x = \alpha y$.

     O \textbf{espaço projetivo $n$-dimensional sobre $\mathbb{K}$} é definido como $\mathbb{KP}^n = \mathbb{P}(\mathbb{K}^n)$.
\end{defi}

\input{conteudo/cone-suspensao}
%---------------------------------------------------------------------------------------------------------------------!Draft!-----------------------------------------------------------------------------------------------------------------
\subsection{Espaços Quociente e a propriedade Hausdorff} %afirmação aqui significa teorema/proposição/colorário/lema
\label{topologia-quociente-hausdorff-thm}
\begin{titlemize}{Lista de dependências}
	\item \hyperref[topologia-quociente-def]{Espaços Quociente};\\ %'dependencia1' é o label onde o conceito Dependência 1 aparece (--à arrumar um padrão para referencias e labels--) 
% quantas dependências forem necessárias.
\end{titlemize}
%Comentário sobre os objetos envolvidos na afirmação.
\begin{thm}[Espaços quocientes Hausdorff]% ou af(afirmação)/prop(proposição)/corol(corolário)/lemma(lema)/outros ambientes devem ser definidos no preambulo de Alg.Top-Wiki.tex 
Sejam $X$ um espaço Hausdorff e $\sim$ uma relação de equivalência em $X$ para a qual a projeção $\pi: X \rightarrow X/\sim$ é uma aplicação aberta. Defina o conjunto $R=\{(x,x')\in X\times X| x\sim x'\}$.

Então $X/\sim$ é Hausdorff se, e somente se, $R\subset X\times X$ é fechado.

\end{thm}
\begin{dem}
    $(\Longrightarrow)$ Se $X/\sim$ é de Hausdorff, gostaríamos de mostrar que $X\times X\backslash R$ é aberto. Para qualquer ponto $(x,x')\in (X\times X)\backslash R$, $x$ e $x'$ são tais que $\pi(x)\neq \pi(x')$. Como $X/\sim$ é de Hausdorff, existem abertos $U_x$ e $U_{x'}$ disjuntos em $X/\sim$ que são vizinhanças abertas de $\pi(x)$ e de $\pi(x')$, respectivamente.% e tais que $U_x\cap U_x' = \emptyset$.

    Temos ainda que $\pi^{-1}(U_x)$ e $\pi^{-1}(U_{x'})$ são abertos, pois a topologia de $X/\sim$ é a topologia quociente, e o produto $U=\pi^{-1}(U_x)\times \pi^{-1}(U_{x'})$ é aberto de $X\times X$ na topologia produto. Além disso, $(x,x')\in U$. Afirmamos que $U\subset X\times X\backslash R$. De fato, se $U\cap R\neq \emptyset$, teríamos $(v_1,v_2)\in U\cap R$ tal que $\pi(v_1)=\pi(v_2)$, mas $v_1 \in \pi^{-1}(U_x)$ e $v_2\in \pi^{-1}(U_{x'})$, e desse modo $\pi(v_1) = \pi(v_2) \in U_x \cap U_{x'} = \varnothing$, absurdo. Portanto, para todo $(x,x')\in X\times X\backslash R$, é possível encontrar uma vizinhança aberta $U$ de $(x,x')$ contida em $X\times X\backslash R$; $R$ é fechado, como queríamos.\newline

    $(\Longleftarrow)$ Dado que $R$ é fechado, gostaríamos de encontrar vizinhanças disjuntas de $a,~b\in X/\sim$ quaisquer para concluir que $X/\sim$ é Hausdorff. Sabemos que existem $x,~y\in X$ tais que $\pi(x)=a$ e $\pi(y)=b$ pois a projeção $\pi$ é uma aplicação sobrejetora. Como $X$ é de Hausdorff, existem abertos disjuntos $U_x$ e $U_y$, vizinhanças de $x$ e de $y$, respectivamente. Além disso, uma vez que $R$ é fechado, $X\times X\backslash R$ é aberto e, portanto, $(U_x\times U_y)\cap((X\times X)\backslash R)$ é aberto na topologia produto.

    Sejam $p_1:X\times X\rightarrow X$ e $p_2:X\times X\rightarrow X$ definidos por $$p_1(x_1,x_2)=x_1,\qquad p_2(x_1,x_2)=x_2 \qquad\forall (x_1,x_2)\in X\times X.$$ Como a topologia produto em $X\times Y$ é gerada pela base dada pelos produtos de abertos $X$ e de $Y$, é possível concluir que $p_1$ e $p_2$ são aplicações abertas. Desse modo, $U_1=p_1((U_x\times U_y)\cap((X\times X)\backslash R))$ e $U_2=p_2((U_x\times U_y)\cap((X\times X)\backslash R))$ são abertos em $X$. Por fim, basta observar que os abertos $\pi(U_1)$ e $\pi(U_2)$ são tais que $\pi(U_1)\cap \pi(U_2)=\emptyset$ uma vez que se $v\in \pi(U_1)\cap\pi(U_2)$, teríamos $v=\pi(v_1)$ para algum $v_1\in U_1$ e $v=\pi(v_2)$ para algum $v_2\in U_2$, o que implicaria $v_1\sim v_2$, um absurdo pois, pela construção de $U_1$ e $U_2$, $(v_1,v_2)\not\in R$.  Também temos $a\in U_1$ e $b\in U_2$ pois, como $a\neq b$, $x\not\sim y$. Encontramos assim os dois abertos que separam $a$ e $b$, mostrando que $X/\sim$ é Hausdorff.
\end{dem}

% Comentários sobre a afirmação.
% \begin{titlemize}{Lista de consequências}
% 	\item \hyperref[consequencia1]{Consequência 1};\\ %'consequencia1' é o label onde o conceito Consequência 1 aparece
% \end{titlemize}
O espaço quociente também é essencial para realizar a colagem de espaços topológicos.
\subsection{\emph{Pushout} de espaços topológicos} %afirmação aqui significa teorema/proposição/colorário/lema
\label{pushout-de-espacos-topologicos-def}
\begin{titlemize}{Lista de dependências}
	\item \hyperref[topologia-quociente-def]{Espaços Quociente};\\ %'dependencia1' é o label onde o conceito Dependência 1 aparece (--à arrumar um padrão para referencias e labels--) 
    \item \hyperref[funcao-continua-em-topologia-quociente-prop]{Função contínua em topologia quociente}.
% quantas dependências forem necessárias.
\end{titlemize}

\begin{defi}
    Sejam $X,Y,Z$ espaços topológicos, e sejam $f:Z\rightarrow X$ e $g:Z\rightarrow Y$ funções contínuas. O \textbf{\emph{pushout} de $f$ e $g$} é o espaço quociente $X\sqcup_Z Y=X\sqcup Y/\sim$, onde $\sim$ é a menor relação de equivalência que contém $\{(f(z),g(z))\in X\times Y:z\in Z\}$. 
\end{defi}

\begin{prop}
    Sejam $X,Y,Z$ espaços topológicos. Além disso, sejam $f:Z\rightarrow X$ e $g:Z\rightarrow Y$ funções contínuas. Seja também $\pi:X\sqcup Y\rightarrow X\sqcup_Z Y$ a função projeção associada ao quociente. Então, o espaço $X\sqcup_Z Y$, juntamente com as funções contínuas definidas por:
    \begin{align*}
        i_X:X &\longrightarrow X\sqcup Y/\sim & i_Y:Y&\longrightarrow X\sqcup Y/\sim\\
        x&\longmapsto \pi(x) & y &\longmapsto \pi(y)
    \end{align*}
    forma um diagrama de \emph{pushout}. Ou seja, dadas quaisquer duas funções contínuas $h_X:X\rightarrow W$ e $h_Y:Y\rightarrow W$ que satisfaçam $h_X\circ f=h_Y\circ g$, existe uma única função contínua 
    $\phi:X\sqcup_Z Y\rightarrow W$ tal que 
    $$h_X=\phi\circ i_X \;\;\;\text{ e }\;\;\; h_Y=\phi\circ i_Y.$$ 
    Isso é ilustrado no diagrama seguinte:
% https://q.uiver.app/#q=WzAsNSxbMCwwLCJaIl0sWzAsMiwiWCJdLFsyLDAsIlkiXSxbMiwyLCJYXFxzcWN1cF9aIFkiXSxbMywzLCJXIl0sWzAsMSwiZiIsMl0sWzAsMiwiZyJdLFsxLDMsImlfWCJdLFsyLDMsImlfWSIsMl0sWzEsNCwiaF9YIiwyXSxbMiw0LCJoX1kiXSxbMyw0LCJcXGV4aXN0cyEgXFxwaGkiLDEseyJzdHlsZSI6eyJib2R5Ijp7Im5hbWUiOiJkYXNoZWQifX19XV0=
\[\begin{tikzcd}
	Z && Y \\
	\\
	X && {X\sqcup_Z Y} \\
	&&& W.
	\arrow["g", from=1-1, to=1-3]
	\arrow["f"', from=1-1, to=3-1]
	\arrow["{i_Y}"', from=1-3, to=3-3]
	\arrow["{h_Y}", from=1-3, to=4-4]
	\arrow["{i_X}", from=3-1, to=3-3]
	\arrow["{h_X}"', from=3-1, to=4-4]
	\arrow["{\exists! \phi}"{description}, dashed, from=3-3, to=4-4]
\end{tikzcd}\]
\end{prop}

\begin{dem}
    De acordo com a construção da topologia quociente e da topologia de união disjunta, as funções $i_X,i_Y$ são contínuas. Além disso, a função dada por 
    \begin{align*}
        h_X\sqcup h_Y:X\sqcup Y&\longrightarrow W\\
        a&\longmapsto h_X\sqcup h_Y(a)=\begin{cases}
         h_X(a) & \text{ if }a\in X\\
         h_Y(a) & \text{ if }a\in Y.
        \end{cases}
    \end{align*}
    também é contínua. Como $h_X\circ f=h_Y\circ g$, pela definição de \emph{pushout} de $f$ e $g$, a função $\phi:=(h_X\sqcup h_Y)\circ \pi^{-1}$ é bem-definida. Além disso, a função $\phi$ é contínua, pois a função $\phi\circ\pi=h_X\sqcup h_Y$ é contínua (pela proposição \ref{funcao-continua-em-topologia-quociente-prop}). Pela construção de $\phi$, temos $h_X=\phi\circ i_X$ e $h_Y=\phi\circ i_Y$, o que prova a existência de tal função.

    Finalmente, provamos que esta função é única: suponha que $\phi'$ seja outra função contínua que satisfaça $h_X=\phi'\circ i_X$ e $h_Y=\phi'\circ i_Y$. Então, temos $\phi'|_{i_X(X)}=\phi|_{i_X(X)}$ e $\phi'|_{i_Y(Y)}=\phi|_{i_Y(Y)}$. Como $i_X(X)\cup i_Y(Y)=X\sqcup_ZY$, concluímos que $\phi=\phi'$.
\end{dem}

%\begin{titlemize}{Lista de consequências}
	%\item %\hyperref[consequencia1]{Consequência 1};\\ %'consequencia1' é o label onde o conceito Consequência 1 aparece
%\end{titlemize}

\subsection{Colagem de n-célula} %afirmação aqui significa teorema/proposição/colorário/lema
\label{colagem-de-n-celula-def}
\begin{titlemize}{Lista de dependências}
	\item \hyperref[topologia-quociente-def]{Espaços Quociente};\\
    \item \hyperref[pushout-de-espacos-topologicos-def]{\emph{Pushout} de espaços topológicos}.%'dependencia1' é o label onde o conceito Dependência 1 aparece (--à arrumar um padrão para referencias e labels--) 
% quantas dependências forem necessárias.
\end{titlemize}

\begin{defi}
    Seja $X$ um espaço topológico, e sejam $f:\mathbb{S}^{n-1}\rightarrow X$ uma função contínua e $i:\mathbb{S}^{n-1}\hookrightarrow D^n$ uma inclusão, onde $n\ge 2$. O \textbf{espaço obtido de} $X$ \textbf{pela colagem de uma $n$-célula por meio da função} $f$ é o \emph{pushout} de $f$ e $i$, denotado por $X_f$ ou $D^n\cup_f X$.
\end{defi}

%\begin{titlemize}{Lista de consequências}
	%\item %\hyperref[consequencia1]{Consequência 1};\\ %'consequencia1' é o label onde o conceito Consequência 1 aparece
%\end{titlemize}

\subsection{Colagem de um disco com um ponto} %afirmação aqui significa teorema/proposição/colorário/lema
\label{colagem-de-um-disco-com-um-ponto-ex}
\begin{titlemize}{Lista de dependências}
	\item \hyperref[topologia-quociente-def]{Espaços Quociente};\\
    \item \hyperref[pushout-de-espacos-topologicos-def]{Pushout de espaços topológicos};\\
    \item \hyperref[colagem-de-n-celula-def]{Colagem de n-célula}%'dependencia1' é o label onde o conceito Dependência 1 aparece (--à arrumar um padrão para referencias e labels--) 
% quantas dependências forem necessárias.
\end{titlemize}

\begin{ex}
    Dado $n\ge 2$, sejam $N = (0,\ldots,0,1) \in \mathbb{S}^n$ e $S = -N$. A colagem $\{x\}_f=D^n\cup_f \{x\}$, em que $f:\mathbb{S}^{n-1}\rightarrow \{x\}$ é a função constante, é homeomorfa à esfera $\mathbb{S}^n$. 
\end{ex}

\begin{dem}
    Note que $\text{int}(D^n)\cong \mathbb{R}^n\cong \mathbb{S}^n\setminus\{N\}$. Seja $g_0:\text{int}(D^n)\rightarrow \mathbb{R}^n$ um homeomorfismo. Podemos estender $g_0$ na seguinte forma 
    \begin{align*}
        g:\{x\}_f&\longrightarrow \mathbb{S}^n\\
        p&\longmapsto g(p)=\begin{cases}
            g_0(p) &\text{ se }p\ne [x]\\
            N &\text{ se }p=[x].
        \end{cases}
    \end{align*}
    Essa função é bem-definida e bijetora. Agora, para mostrar que $g$ é um homeomorfismo, basta mostrar que $g$ é contínua e aberta.\\
    A função $g$ é contínua: A função $g$ é contínua se, e somente se, $g\circ \pi$ é contínua, onde $\pi:D^n\bigsqcup \{x\}\rightarrow \{x\}_f$ é a projeção associada ao quociente. A função $g\circ \pi$ é contínua, pois dado um aberto $U$ de $\mathbb{S}^n$, temos:
    \begin{itemize}
        \item se $N\notin U$, então $(g\circ\pi)^{-1}(U)=g_0^{-1}(U)$, que é aberto;
        \item se $N\in U$, então $(g\circ \pi)^{-1}(U)=g_0^{-1}(U\setminus\{N\})\cup \{x\}$, que é um aberto, pois $x$ é um ponto isolado.
    \end{itemize}
    Portanto, $g$ é contínua.\\
    A função $g$ é aberta: Seja $U$ um aberto de $\{x\}_f$. Teremos dois casos: 
    \begin{itemize}
        \item Se $x\notin U$, então $g(U)=g_0(U)$, que é um aberto em  $\mathbb{S}^n\setminus\{N\}$. Assim, ou $g_0(U)$ é aberto em $\mathbb{S}^n$, ou $g_0(U)\cup\{N\}$ é aberto em $\mathbb{S}^n$. Se $g_0(U)$ for aberto, então $g(U)$ será um aberto em $\mathbb{S}^n$. Se $g_0(U)\cup \{N\}$ for aberto, então $g_0(U)=(g_0(U)\cup\{N\})\setminus\{N\}$ é um aberto em $\mathbb{S}^n$. Em ambos os casos, $g(U)$ será um aberto em $\mathbb{S}^n$;
        \item Se $x\in U$, então $\pi^{-1} (U)$ é um aberto em $D^n\sqcup \{x\}$. Pela construção do quociente, temos $\partial D^n\subseteq\pi^{-1}(U)$, logo, para todo ponto $y\in \partial D^n$, existe um $r_y>0$ tal que 
        $$B_y:=\{z\in D^n: ||z-y||<r_y\}\subseteq \pi_1^{-1}(U).$$
        A coleção $\{B_y\}_{y\in \partial D^n}$ é uma cobertura aberta de $\partial D^n$. Como o bordo $\partial D^n$ é compacto, existem $y_1,...,y_k$ tal que 
        \[\partial D^n\subseteq B_{y_1}\cup...\cup B_{y_k}.\]
        Considere $r=\text{inf}\{r_{y_1},...,r_{y_k}\}$. Assim, o conjunto 
        \[B=\pi(\{y\in D^n: ||y||>(1-r)\}\cup\{x\})\subseteq U\]
        é um aberto em $\{x\}_f$, e $g(B)\subseteq g(U)$ corresponde a uma bola centrada em $N$ em $\mathbb{S}^n$. Note que $U\setminus \{x\}$ é um aberto, pois $x$ é um ponto fechado. Pelo item anterior, temos que o aberto
        \[g(U)=g(B)\cup g(U\setminus\{x\})\]
        é uma união de abertos, o que implica que $g(U)$ é um aberto em $\mathbb{S}^n$.
    \end{itemize} 
    Portanto, $g$ é aberta.
\end{dem}

%\begin{titlemize}{Lista de consequências}
	%\item %\hyperref[consequencia1]{Consequência 1};\\ %'consequencia1' é o label onde o conceito Consequência 1 aparece
%\end{titlemize}

\subsection{Produto \emph{wedge} de espaços topológicos} %afirmação aqui significa teorema/proposição/colorário/lema
\label{produto-wedge-def}
\begin{titlemize}{Lista de dependências}
	\item \hyperref[topologia-quociente-def]{Espaços Quociente};\\ %'dependencia1' é o label onde o conceito Dependência 1 aparece (--à arrumar um padrão para referencias e labels--) 
\end{titlemize}

\begin{defi}
    Sejam $(X,x_0),(Y,y_0)$ espaços topológicos pontuados. O \textbf{produto \emph{wedge}} de $(X,x_0)$ e $(Y,y_0)$, denotado por $(X,x_0)\vee (Y,y_0)$, é o espaço quociente obtido da união disjunta $X\sqcup Y$ por meio da identificação de $x_0$ e $y_0$ a um único ponto. 
\end{defi}



%\begin{titlemize}{Lista de consequências}
	%\item %\hyperref[consequencia1]{Consequência 1};\\ %'consequencia1' é o label onde o conceito Consequência 1 aparece
%\end{titlemize}
%%% Local Variables:
%%% mode: LaTeX
%%% TeX-master: "../Alg.Top-Wiki"
%%% End:


\section{Homotopia}
\label{homotopia}
Um assunto que aparece na definição de objetos importantes na topologia algébrica, como os grupos de homotopia e, em particular, o grupo fundamental.

\input{conteudo/homotopia-def}% onde conteudos.tex é o nome do arquivo tex que voce quer incluir nessa secção.
\input{conteudo/homotopia-relativa-def}
\input{conteudo/homotopia-relaçao-de-equivalencia-prop}
%---------------------------------------------------------------------------------------------------------------------!Draft!-----------------------------------------------------------------------------------------------------------------
\subsection{Equivalência de Homotopia}
\label{equiv-homotopia}
\begin{titlemize}{Lista de dependências}
	\item \hyperref[homotopia-def]{Homotopia};\\
\end{titlemize}

\begin{defi}[Equivalência de Homotopia]
	Sejam $X$ e $Y$ espaços topológicos. Uma função contínua $f:X\to Y$ é dita uma \textbf{equivalência de homotopia} se existe outra função contínua $g:Y\to X$ tal que $f\circ g \sim \text{id}_Y$ e $g\circ f \sim \text{id}_X$. Nesse caso, dizemos que $X$ e $Y$ são \textbf{homotopicamente equivalentes}, e $g$ é \textbf{inversa a menos de homotopia} de $f$.
\end{defi}

É claro que todo homeomorfismo é uma equivalência de homotopia.

\begin{titlemize}{Lista de consequências}
    \item \hyperref[espaco-contratil-def]{Espaço contrátil}
	\item \hyperref[equiv-homotopia-induz-iso]{Equivalência de homotopia e grupo fundamental}
\end{titlemize}

%[Bianca]: é mais fácil criar a lista de dependências do que a de consequências.

\subsection{Espaço contrátil}
\label{espaco-contratil-def}
\begin{titlemize}{Lista de dependências}
	\item \hyperref[homotopia-def]{Homotopia};\\
        \item \hyperref[equiv-homotopia]{Equivalência de Homotopia}.
\end{titlemize}

\begin{defi}
	Seja $X$ um espaço topológico. Diremos que o espaço $X$ é \textbf{contrátil} se $X$ é homotopicamente equivalente a um ponto.
\end{defi}

%%% Local Variables:
%%% mode: LaTeX
%%% TeX-master: "../Alg.Top-Wiki"
%%% End:

\section{Grupo Fundamental}
\label{grupo-fundamental}

\begin{titlemize}{Lista de Dependências}
	\item \hyperref[homotopia]{Homotopia}\\ %homotopia
\end{titlemize}

Considere um espaço topológico com um ponto base fixado. O seu grupo fundamental é o grupo das classes de equivalência (sob homotopia relativa aos extremos) dos laços no espaço saindo do ponto base. Tal grupo armazena certas informações sobre buracos do espaço topologico, e é invariante sobre a equivalência homotópica. Esta é uma ferramenta poderosa para verificar se dois espaços topológicos são homeomorfos. % (homotópicos). %retirei, não entendi o que querem dizer
Veremos como sua construção se dá com mais detalhes.

%---------------------------------------------------------------------------------------------------------------------!Draft!-----------------------------------------------------------------------------------------------------------------
\subsection{Espaço de Laços}
\label{espaco-lacos-def}
\begin{titlemize}{Lista de dependências}
	%\item \hyperref[dependecia1]{Dependência 1};\\ %'dependencia1' é o label onde o conceito Dependência 1 aparece (--à arrumar um padrão para referencias e labels--)
    \item \hyperref[homotopia-relativa-def]{Homotopia Relativa}
	\item \hyperref[homotopia-relaçao-de-equivalencia-prop]{Homotopia é relação de equivalência};\\
% quantas dependências forem necessárias.
\end{titlemize}
\begin{defi}[Espaço de Laços]
	Seja $X$ um espaço topológico e seja $x_0\in X$ um ponto base. O \textbf{espaço de laços} em $X$ que saem de $x_0$ é definido como
\[\Omega(X,x_0) = \left\{\gamma: I \to X ~|~ \gamma\text{ é contínua e }\gamma(0)=\gamma(1)=x_0\right\}.\]
\end{defi}

Investigaremos a fundo o conjunto $\pi_1(X,x_0) = \Omega(X,x_0)/\sim$, onde $\alpha \sim \beta$ se, e somente se, $\alpha$ e $\beta$ são homotópicas relativo a $\partial I = \{0,1\}$.


%[Bianca]: é mais fácil criar a lista de dependências do que a de consequências.
\input{conteudo/produto-concatenacao-def}
\input{conteudo/produto-bem-definido-gr-fundamental-prop}
\subsection{Grupo fundamental}
\label{grupo-fundamental-def}
\begin{titlemize}{Lista de dependências}
	\item \hyperref[espaco-lacos-def]{O espaço de laços}
	\item \hyperref[produto-bem-definido-prop]{O produto do grupo fundamental};\\ %'dependencia1' é o label onde o conceito Dependência 1 aparece (--à arrumar um padrão para referencias e labels--) 
% quantas dependências forem necessárias.
\end{titlemize}
\begin{defi}[Grupo fundamental]
    Seja $X$ um espaço topológico e seja $x_0$ um ponto de $X.$ O grupo fundamental de $X$ em $x_0$ é $(\pi_1(X,x_0),\cdot)$, onde $\pi_1(X,x_0) = \Omega(X,x_0)/\sim$, onde $\alpha \sim \beta$ se, e somente se, $\alpha$ e $\beta$ são homotópicas relativo aos extremos, e o produto $\cdot$ é dado por $[\alpha]\cdot[\beta] = [\alpha \ast \beta]$, em que $\alpha \ast \beta$ é a concatenação de $\alpha$ e $\beta$.
\end{defi}

No geral, o grupo fundamental depende da escolha do ponto base $x_0$. A seguir, apresentamos um exemplo elementar de grupo fundamental.
\begin{ex}
    Seja $X=\{x\}$ é um espaço topológico contendo apenas um ponto. Nesse caso, o único laço em $X$ é a função constante $c_x:I\rightarrow \{x\}$. Assim, a única classe de homotopia é $[c_x]$, o que implica que $\pi_1(\{x\},x)=0$.
\end{ex}

\begin{titlemize}{Lista de consequências}
	\item \hyperref[hom-grupo-fundamental]{Homomorfismo de grupos fundamentais};%'consequencia1' é o label onde o conceito Consequência 1 aparece
	%\item \hyperref[]{}
\end{titlemize}

\input{conteudo/homomorfismo-de-grupo-fundamental-prop}
%---------------------------------------------------------------------------------------------------------------------!Draft!-----------------------------------------------------------------------------------------------------------------
\subsection{Conjugação por uma Curva} %[conjugacao-por-curva-prop]{Conjugação por uma Curva}
\label{conjugacao-por-curva-prop}
\begin{titlemize}{Lista de dependências}
    \item \hyperref[espaco-lacos-def]{Espaço de Laços};\\
    \item \hyperref[produto-bem-definido-prop]{O produto do grupo fundamental};\\
	\item \hyperref[grupo-fundamental-def]{O Grupo Fundamental};
% quantas dependências forem necessárias.
\end{titlemize}
%Comentário sobre os objetos envolvidos na afirmação.
\begin{defi}[Conjugação de laços por uma curva] 
	Sejam $x_0$ e $x_1$ pontos em um espaço topológico $X$ e seja $\gamma: I \to X$ uma curva contínua ligando $x_0$ a $x_1$; isto é, $\gamma(0)=x_0$ e $\gamma(1) = x_1$. Seja também $\eta \in \Omega(X,x_0)$ um laço saindo de $x_0$. Definimos a conjugação de $\eta$ por $\gamma$ como $\overline{\gamma} * \eta * \gamma \in \Omega(X,x_1)$, laço saindo de $x_1$. Isto define uma função $A_{\gamma}: \Omega(X,x_0) \to \Omega(X,x_1)$.
\end{defi}

\begin{prop}[Isomorfismo de grupos induzido por $A_{\gamma}$]
    Sejam $x_0$ e $x_1$ pontos em um espaço topológico $X$ e seja $\gamma: I \to X$ uma curva ligando $x_0$ a $x_1$.
    
    Então $A_{\gamma}$ induz um isomorfismo de grupos \begin{align*}
        \hat{A}_{\gamma}: \pi_1(X,x_0)&\to \pi_1(X,x_1)\\
        \hat{A}_{\gamma}([\eta]) &= [A_{\gamma}(\eta)] = [\overline{\gamma} * \eta * \gamma].
    \end{align*}

    \begin{dem}
        Provemos primeiramente que $\hat{A}_{\gamma}$ está bem definida. Considere $c_{\gamma}: \gamma \Rightarrow \gamma$ e $c_{\overline{\gamma}}: \overline{\gamma} \Rightarrow \overline{\gamma}$ as homotopias constantes. Assim, se $\eta, \nu \in \Omega(X,x_0)$ e $H: \eta \Rightarrow \nu$ é uma homotopia relativa a $\partial I$ então é claro que $c_{\overline{\gamma}}*H*c_{\gamma}: A_{\gamma}(\eta) \Rightarrow A_{\gamma}(\nu)$ também é homotopia relativa a $\partial I$.

        $\hat{A}_{\gamma}$ é um homomorfismo de grupos, já que dadas $\eta, \nu \in \Omega(X,x_0)$,
        \begin{align*}
            \hat{A}_{\gamma}([\eta]\cdot[\nu]^{-1})
            &= \hat{A}_{\gamma}([\eta * \overline{\nu}])\\
            &= [\overline{\gamma} * (\eta * \overline{\nu}) * \gamma]\\
            &= [(\overline{\gamma} * \eta * \gamma)*(\overline{\gamma} * \overline{\nu} * \gamma)]\\
            &= [\overline{\gamma} * \eta * \gamma]\cdot[\overline{\overline{\gamma} * \nu * \gamma}]\\
            &= \hat{A}_{\gamma}([\eta]) \cdot \hat{A}_{\gamma}([\nu])^{-1}.
        \end{align*}
        
        Por fim, note que $\hat{A}_{\gamma}$ e $\hat{A}_{\overline{\gamma}}$ são inversas, pois \[A_{\gamma} \circ A_{\overline{\gamma}}(\eta) = (\overline{\gamma} * \gamma) * \eta * (\overline{\gamma} * \gamma) \sim \eta\text{ relativa a }\partial I\]
        para toda curva $\gamma: I \to X$. Desse modo $\hat{A}_{\gamma}$ é um isomorfismo de grupos.
    \end{dem}
\end{prop}

Um fato importante decorrente de tal proposição é o seguinte.

\begin{corol}
    Se $X$ é um espaço topológico então $\pi_1(X,x_0)$ é isomorfo a $\pi_1(X,x_1)$, para quaisquer $x_0, x_1 \in X$ na mesma componente conexa por caminhos de $X$. Em especial, o grupo fundamental independe do ponto base caso $X$ seja conexo por caminhos.
\end{corol}

Dessa forma, se $X$ é um espaço conexo por caminhos, podemos denotar o grupo fundamental de $X$ por $\pi_1(X)$, omitindo o ponto base.

\begin{nota}
    Sejam $X$ e $Y$ espaços topológicos, $x_0, x_1\in X$ e $\gamma:I\to X$ uma curva ligando $x_0$ a $x_1$. Seja também $f: X\to Y$ uma função contínua e denotemos $y_0 = f(x_0)$ e $y_1 = f(x_1)$. Então $f(\gamma):I \to Y$ liga $y_0$ a $y_1$, e vale que
    \[f_{*,x_1} \circ \hat{A}_{\gamma} = \hat{A}_{f(\gamma)} \circ f_{*,x_0}.\]
    \begin{dem}
        Para cada $\eta \in \Omega(X,x_0)$,
        \begin{align*}
            \hat{A}_{f(\gamma)} \circ f_{*,x_0}([\eta])
            &= \hat{A}_{f(\gamma)} ([f(\eta]))\\
            &= [\overline{f(\gamma)} * f(\eta) * f(\gamma)]\\
            &= [f(\overline{\gamma}) * f(\eta) * f(\gamma)]\\
            &= [f(\overline{\gamma} * \eta * \gamma)]\\
            &= [f(A_{\gamma}(\eta)]
            = f_{*,x_1}\circ \hat{A}_{\gamma}([\eta]).
        \end{align*}
    \end{dem}
\end{nota}

\begin{titlemize}{Lista de consequências}
	\item \hyperref[equiv-homotopia-induz-iso]{Equivalência de homotopia e o grupo fundamental};\\ %'consequencia1' é o label onde o conceito Consequência 1 aparece
	%\item \hyperref[]{}
\end{titlemize}

%[Bianca]: Um arquivo tex pode ter mais de uma afirmação (ou definição, ou exemplo), mas nesse caso cada afirmação deve ter seu próprio label. Dar preferência para agrupar afirmações que dependam entre sí de maneira próxima (um teorema e seu corolário, por exemplo)

\input{conteudo/equivalencia-de-homotopia-induz-iso-thm}
\subsection{Grupo fundamental de espaço contrátil}
\label{grupo-fundamental-de-contratil-prop}
\begin{titlemize}{Lista de dependências}
    \item \hyperref[equiv-homotopia]{Equivalência de Homotopia};\\
	\item \hyperref[hom-grupo-fundamental]{Homomorfismo de grupos fundamentais};\\
    \item \hyperref[equiv-homotopia-induz-iso]{Equivalência de homotopia e o grupo fundamental}.
\end{titlemize}

\begin{prop}
    Se $X$ é um espaço contrátil com $x_0\in X$, então o grupo fundamental $\pi_1(X,x_0)$ é trivial.
\end{prop}

\begin{dem}
    Sem perda de generalidade, $X$ é homotopicamente equivalente ao ponto $x_0$. Logo, existe uma equivalência de homotopia $f:X\rightarrow \{x_0\}$. Pelo Teorema \ref{equiv-homotopia-induz-iso}, o homomorfismo induzido $f_*:\pi_1(X,x_0)\rightarrow \pi_1(\{x_0\},x_0)$ é um isomorfismo. Isso implica que $\pi_1(X,x_0)=0$.
\end{dem}

%\begin{titlemize}{Lista de consequências}
	%\item \hyperref[consequencia1]{Consequência 1};\\ %'consequencia1' é o label onde o conceito Consequência 1 aparece
	%\item \hyperref[]{}
%\end{titlemize}
\subsection{O Grupo Fundamental de um Espaço Convexo}
\label{grupo-fundamental-convexo}
\begin{titlemize}{Lista de dependências}
	\item \hyperref[grupo-fundamental-def]{Grupo Fundamental};\\ %'dependencia1' é o label onde o conceito Dependência 1 aparece (--à arrumar um padrão para referencias e labels--) 
% quantas dependências forem necessárias.
\end{titlemize}

\begin{ex}[Grupo fundamental de um espaço convexo]
O grupo fundamental de um espaço convexo X é sempre trivial, i.e, $\pi_1(X, x_0) = \{ 1\}$.
\end{ex}

De fato, isso se verifica pois, se $\alpha$ é um laço em um espaço topológico $X$ começando em um ponto $x_0$, tomando a homotopia $F:I \times I \longrightarrow X$, onde $F(s,t) = (1 - t)\alpha(s) + tx_0$, tem-se que $\alpha \sim c_{x_0}$. Assim, $\pi_1(X, x_0) = \{1\}$.

%\begin{figure}[]
%	\centering
%	\includegraphics[width=0.8\textwidth]{}
%	\caption{}
%	\label{fig:}
%\end{figure}


\subsection{Grupo fundamental de espaço de produtos}
\label{grupo-fundamental-de-espaco-de-produtos-prop}
\begin{titlemize}{Lista de dependências}
    \item \hyperref[homotopia-def]{Homotopia};\\
    \item \hyperref[grupo-fundamental]{Grupo fundamental};\\
    \item \hyperref[hom-grupo-fundamental]{Homomorfismo de grupos fundamentais}.
\end{titlemize}

\begin{prop}
    Sejam $X,Y$ espaços topológicos, com $x_0\in X$ e $y_0\in Y$, e sejam $p:X\times Y\rightarrow X$ e $q:X\times Y\rightarrow Y$ projeções canônicas. Então, o homomorfismo
    \begin{align*}
        ((p_*,q_*):\pi_1(X\times Y,(x_0,y_0))&\longrightarrow \pi_1 (X,x_0)\times \pi_1(Y,y_0)\\
        [\alpha]&\longmapsto ([p\circ \alpha],[q\circ \alpha]) 
    \end{align*}
    é um isomorfismo.
\end{prop}

\begin{dem}
    Como $p_*$ e $q_*$ são homomorfismos de grupos, $(p_*,q_*)$ também é um homomorfismo de grupos. Vamos agora verificar que $(p_*,q_*)$ é bijetivo.\\
    Injetividade: Note que $([c_{x_0}],[c_{y_0}])$ é a unidade de $\pi_1 (X,x_0)\times \pi_1(Y,y_0)$. Assim, se $[\alpha]\in \text{Ker}(p_*,q_*)$, então $[p\circ \alpha]=[c_{x_0}]$ e $[q\circ \alpha]=[c_{y_0}]$. Ou seja, existem homotopias relativa a $\partial I$, $H_1:p\circ\alpha \Rightarrow c_{x_0}$ e $H_2:q\circ \alpha \Rightarrow c_{y_0}$, o que implica que a função $H:(X\times Y)\times I\rightarrow X\times Y$ definida por
    \begin{align*}
        H((x,y),t):=(H_1(x,t),H_2(y,t))
    \end{align*}
    é uma homotopia relativa a $\partial I$ entre $(p\circ \alpha,q\circ\alpha)$ e $c_{(x_0,y_0)}$. Assim, concluímos que $[\alpha]=[c_{(x_0,y_0)}]$, o que implica que $(p_*,q_*)$ é injetivo.\\
    Sobrejetividade: Sejam $(\alpha,\beta)\in \Omega(X,x_0)\times \Omega(Y,y_0)$. Basta mostrar que existe um $\gamma\in \Omega(X\times Y,(x_0,y_0))$ tal que $p\circ\gamma=\alpha$ e $q\circ \gamma=\beta$. Porém, o laço $(\alpha,\beta)$ satisfaz exatamente essa condição. Portanto, concluímos que $(p_*,q_*)$ é sobrejetivo.
\end{dem}



%%% Local Variables:
%%% mode: LaTeX
%%% TeX-master: "../Alg.Top-Wiki"
%%% End:

\section{Categorias}
\label{categorias}

% \begin{titlemize}{Lista de Dependências}
% 	\item \hyperref[]{};\\
% 	\item \hyperref[]{};
% \end{titlemize}

A teoria das categorias pode ser vista como uma ferramenta usada para os estudos das conexões das diversas áreas da matemática. Nessa Wiki, usaremos a linguagem de teoria das categorias de modo a expressar as relações entre a topologia e a álgebra que a topologia algébrica está interessada.

\subsection{Categorias}
\label{categorias-def}
\begin{defi}[Categorias]
	    Uma categoria $\mathcal{C}$ é formada pelas seguintes coisas:


\begin{itemize}
    \item Uma coleção de objetos $\text{Obj}(\mathcal{C})$, que geralmente serão denotados por letras maiúsculas $A$, $B$, $C$...
    \item Uma coleção de morfismos $\text{Mor}(\mathcal{C})$, que usualmente serão denotadas por letras minúsculas $f$, $g$, $h$...
\end{itemize}

Onde valem os seguintes axiomas:

\begin{enumerate}
    \item A cada morfismo $f$ de $\text{Mor}(\mathcal{C})$ são associados dois objetos $\text{Dom}(f)$ e $\text{Codom}(f)$ de $\text{Obj}(\mathcal{C})$. \\
    Escrevemos % https://tikzcd.yichuanshen.de/#N4Igdg9gJgpgziAXAbVABwnAlgFyxMJZABgBpiBdUkANwEMAbAVxiRAEEQBfU9TXfIRQBGclVqMWbAELdxMKAHN4RUADMAThAC2SMiBwQkokAzoAjGAwAK-PATYMYanCGr1mrRCDVyuQA
\begin{tikzcd}
A \arrow[r, "f"] & B
\end{tikzcd}
 para abreviar $f \in \text{Mor}(\mathcal{C})$, $\text{Dom}(f) = A$ e $\text{Codom}(f) = B$.
 \item A cada objeto $A$ de $\mathcal{C}$ está associado um morfismo $1_A \in \text{Mor}(\mathcal{C})$ tal que $\text{Codom}(1_A) = \text{Dom}(1_A) = A$.
 \item Para quaisquer dois morfismos $f$ e $g$, tais que $\text{Dom}(f)=\text{Codom}(g)$, há um morfismo associado $f \circ g$, onde $\text{Dom}(f \circ g) = \text{Dom}(g)$ e $\text{Codom}(f \circ g) = \text{Codom}(f)$.
 \\ Isso pode ser representado dizendo que o seguinte diagrama comuta:
% https://q.uiver.app/#q=WzAsMyxbMCwwLCJBIl0sWzEsMCwiQiJdLFsxLDEsIkMiXSxbMCwxLCJmIl0sWzEsMiwiZyJdLFswLDIsImYgXFxjaXJjIGciLDJdXQ==
\[\begin{tikzcd}[column sep=large]
	A & B \\
	& C
	\arrow["f", from=1-1, to=1-2]
	\arrow["{f \circ g}"', from=1-1, to=2-2]
	\arrow["g", from=1-2, to=2-2]
\end{tikzcd}\]

\item Para todo morfismo $f$ de $\text{Mor}(\mathcal{C})$ com $\text{Dom}(f) = A$ e $\text{Codom}(f) = B$, vale que $f \circ 1_A = f$ e $1_B \circ f = f$. Ou seja, o seguinte diagrama comuta:

% https://q.uiver.app/#q=WzAsNCxbMCwwLCJBIl0sWzEsMCwiQSJdLFsxLDEsIkIiXSxbMiwxLCJCIl0sWzAsMSwiMV9BIl0sWzEsMiwiZiJdLFswLDIsImYiLDJdLFsxLDMsImYiXSxbMiwzLCIxX0IiLDJdXQ==
\[\begin{tikzcd}[sep=large]
	A & A \\
	& B & B
	\arrow["{1_A}", from=1-1, to=1-2]
	\arrow["f"', from=1-1, to=2-2]
	\arrow["f", from=1-2, to=2-2]
	\arrow["f", from=1-2, to=2-3]
	\arrow["{1_B}"', from=2-2, to=2-3]
\end{tikzcd}\]
\item Dados os morfismos $f$, $g$, $h$ de $\text{Mor}(\mathcal{C})$, vale que 
$(f \circ g) \circ h = f \circ (g \circ h)$. Ou seja, o seguinte diagrama comuta:
% https://q.uiver.app/#q=WzAsNCxbMCwwLCJBIl0sWzEsMCwiQiJdLFsxLDEsIkMiXSxbMiwxLCJEIl0sWzAsMSwiZiJdLFsxLDIsImciXSxbMCwyLCJnIFxcY2lyYyBmIl0sWzIsMywiaCJdLFsxLDMsImggXFxjaXJjIGciXV0=
\[\begin{tikzcd}[sep=large]
	A & B \\
	& C & D
	\arrow["f", from=1-1, to=1-2]
	\arrow["{g \circ f}", from=1-1, to=2-2]
	\arrow["g", from=1-2, to=2-2]
	\arrow["{h \circ g}", from=1-2, to=2-3]
	\arrow["h", from=2-2, to=2-3]
\end{tikzcd}\]
\end{enumerate}
\end{defi}



%[Bianca]: é mais fácil criar a lista de dependências do que a de consequências.
 
%---------------------------------------------------------------------------------------------------------------------!Draft!-----------------------------------------------------------------------------------------------------------------
\subsection{Categorias}
\label{categorias-ex}
\begin{titlemize}{Lista de dependências}
	\item \hyperref[categorias-def]{Definição de Categoria};\\ %'dependencia1' é o label onde o conceito Dependência 1 aparece (--à arrumar um padrão para referencias e labels--) 

\end{titlemize}

\begin{ex}[Exemplos de Categorias]
	Alguns dos seguintes exemplos não serão tratados com detalhes. No entanto, pode-se consultá-los em quaisquer livros de teoria das categorias.
\begin{itemize}
\item \textbf{Mon} é uma categorial em que os objetos são monóides e os morfismos são homomorfismos de monóides.
\item \textbf{Grp} é a categoria dos grupos e homomorfismo de grupos (A categoria \textbf{Ab} é a categoria dos grupos abelianos.
\item A categoria \textbf{TOP} tem como objetos os espaços topológicos e como morfismos as funções contínuas (há também a categoria $\mathbf{TOP_*}$ dos espaços topológicos com um ponto selecionado, onde os morfismos $f:(X,x) \longrightarrow (Y,y)$ são funções contínuas tais que $f(x) = y$).
\item $\mathbf{Vec(\mathbb{K})}$ é a categoria dos espaços vetoriais sobre o corpo $\mathbb{K}$ e as transformações lineares dos espaços
\item A categoria \textbf{SET} tem como objetos os conjuntos e os morfismos são as funções entre os conjuntos. Ainda, pode-se definir $\mathbf{SET}_\omega$, a categoria dos conjuntos finitos e as funções entre eles.
\item A categoria \textbf{Ord} dos ordinais e das funções entre eles (funções entre esses conjuntos transitivos). Da mesma forma, $\mathbf{Ord}_\omega$ é a categoria dos ordinais finitos e as funções entre eles.

\item Uma relação $\leq$ é dita relação de ordem parcial se satisfaz:
\begin{itemize}
    \item $a \leq a$ para todo $a$.
    \item Se $a \leq b$ e $b \leq a$, então $a = b$ para todos $a$ e $b$.
    \item Se $a \leq b$ e $b \leq c$, então $a \leq c$ para todos $a$, $b$ e $c$.
\end{itemize}
A categoria $\mathbf{PO}$ (partial-order) é definida com morfismos estabelecendo a ordem entre os objetos, isto é, $A \leq B$ se, e somente se, existe $f$ em $Mor(\mathbf{PO})$, tal que % https://q.uiver.app/#q=WzAsMixbMCwwLCJBIl0sWzEsMCwiQiJdLFswLDEsImYiXV0=
\begin{tikzcd}[cramped]
	A & B
	\arrow["f", from=1-1, to=1-2]
\end{tikzcd}
\item Já a categoria \textbf{POS} tem como objetos os conjuntos parcialmente ordenados e os morfismos são funções que preservam a ordem, isto é, se $f:$ % https://q.uiver.app/#q=WzAsMixbMCwwLCJBIl0sWzEsMCwiQiJdLFswLDFdXQ==
\begin{tikzcd}[cramped]
	A & B
	\arrow[from=1-1, to=1-2]
\end{tikzcd}
, e $m \leq n$ em $A$, então $f(m) \leq f(n)$ em $B$.




\end{itemize}

\end{ex}


\begin{titlemize}{Lista de consequências}
	\item \hyperref[homotopia]{homotopia};\\ %'consequencia1' é o label onde o conceito Consequência 1 aparece
\end{titlemize}

%---------------------------------------------------------------------------------------------------------------------!Draft!-----------------------------------------------------------------------------------------------------------------
\subsection{Isomorfismo}
\label{isomorfismo-em-categorias-def}
\begin{titlemize}{Lista de dependências}
	\item \hyperref[categorias-def]{Definição de Categoria};\\ %'dependencia1' é o label onde o conceito Dependência 1 aparece (--à arrumar um padrão para referencias e labels--) 
\end{titlemize}
\begin{defi}[Isomorfismo]
	Um morfismo $f:A \longrightarrow B$ de uma categoria $\mathcal{C}$ é um isomorfismo se, e somente se, existe um morfismo $g:B \longrightarrow A$, tal que $f \circ g = 1_B$ e $g \circ f = 1_A$. Nesse caso, dizemos que $A$ e $B$ são isomorfos e escrevemos $A \cong B$.
\end{defi}


%[Bianca]: é mais fácil criar a lista de dependências do que a de consequências.

%---------------------------------------------------------------------------------------------------------------------!Draft!-----------------------------------------------------------------------------------------------------------------
\subsection{Funtor}
\label{funtor-categorias-def}
\begin{titlemize}{Lista de dependências}
	\item \hyperref[categorias-def]{Definição de Categoria};\\ %'dependencia1' é o label onde o conceito Dependência 1 aparece (--à arrumar um padrão para referencias e labels--) 
\end{titlemize}
\begin{defi}[Funtor Covariante]
	Um funtor é uma função entre categorias $F: \mathcal{C} \longrightarrow \mathcal{D}$, que associa para cada $A \in Obj(\mathcal{C})$ um único objeto $F(A) \in Obj(\mathcal{D})$ e associa cada morfismo $f \in Mor(\mathcal{C})$ um morfismo $F(f): (A) \longrightarrow F(B)$ , tal que $F(f \circ g) = F(f) \circ F(g) $ e $F(1_A) = 1_{F(A)}$.
\end{defi}

O conceito de funtor é extremamente importante, pois é ele que estabelece uma "ponte" para as diversas áreas da mátematica. Desse modo, podemos ver o grupo fundamental como um funtor da categoria dos espaços topológicos pontuados para a categoria de grupos e homomorfismo de grupos.

\begin{titlemize}{Lista de consequências}
	\item \hyperref[homotopia]{Homotopia};\\ %'consequencia1' é o label onde o conceito Consequência 1 aparece
	\item \hyperref[grupo-fundamental]{Grupo fundamental}
\end{titlemize}

%[Bianca]: é mais fácil criar a lista de dependências do que a de consequências.

%---------------------------------------------------------------------------------------------------------------------!Draft!-----------------------------------------------------------------------------------------------------------------
\subsection{Funtores-Exemplos}
\label{funtor-categorias-ex}
\begin{titlemize}{Lista de dependências}
	\item \hyperref[categorias-ex]{Categorias-Exemplos};\\
	\item \hyperref[funtor-categorias-def]{Funtor};\\ %'dependencia1' é o label onde o conceito Dependência 1 aparece (--à arrumar um padrão para referencias e labels--) 
\end{titlemize}

\begin{ex}[Funtores Covariantes]
	Os detalhes dos exemplos a seguir são deixados para os leitores.
\begin{itemize}

    \item O funtor identidade $\mathbf{1}_{\mathcal{C}}: \mathcal{C} \longrightarrow \mathcal{C}$ leva todo objeto nele mesmo e todo morfismo nele mesmo, isto é, $\mathbf{1}_{\mathcal{C}}(A) = A$ e $\mathbf{1}_{\mathcal{C}}(f) = f$
    
    \item O funtor potência (power-set functor) $\mathcal{P}:\mathbf{SET} \longrightarrow \mathbf{SET}$, que leva um conjunto $A$ no conjuntos das partes $\mathcal{P}(A)$ e uma função $f:A \longrightarrow B$ para a função $\mathcal{P}(f): \mathcal{P}(A) \longrightarrow \mathcal{P}(B)$, tal que $\mathcal{P}(f)(S) = f(S)$ para todo $S \subseteq A$.

    \item Na topologia algébrica podemos obter de cada espaço topológico pontuado um grupo, chamado de n-ésimo grupo de homotopia. Além disso, para cada função contínua $f: A \longrightarrow B$, podemos obter um homomorfismo de grupos $\pi_n(f):\pi_n(A) \longrightarrow \pi_n(B)$, onde $\pi_n(A)$ e $\pi_n(B)$ são os n-ésimos grupos de homotopia. Dessa forma, construimos um funtor $\pi_n: \mathbf{TOP}_* \longrightarrow \mathbf{Grp}$. 

\end{itemize}
\end{ex}

 
%\begin{figure}[]
%	\centering
%	\includegraphics[width=0.8\textwidth]{}
%	\caption{}
%	\label{fig:}
%\end{figure}

\begin{titlemize}{Lista de consequências}
	\item \hyperref[grupo-fundamental]{Grupo fundamental};\\ %'consequencia1' é o label onde o conceito Consequência 1 aparece
\end{titlemize}

%---------------------------------------------------------------------------------------------------------------------!Draft!-----------------------------------------------------------------------------------------------------------------
\subsection{Transformação Natural}
\label{transformação-natural-categorias-def}
\begin{titlemize}{Lista de dependências}
	\item \hyperref[funtor-categorias-def]{Funtor};\\ %'dependencia1' é o label onde o conceito Dependência 1 aparece (--à arrumar um padrão para referencias e labels--) 
% quantas dependências forem necessárias.
\end{titlemize}
\begin{defi}[Transformação Natural]
	Dados dois funtores % https://q.uiver.app/#q=WzAsMixbMCwwLCJcXG1hdGhjYWx7Q30iXSxbMSwwLCJcXG1hdGhjYWx7RH0iXSxbMCwxLCJGIiwwLHsib2Zmc2V0IjotMX1dLFswLDEsIkciLDIseyJvZmZzZXQiOjF9XV0=
\begin{tikzcd}[cramped,sep=small]
	{\mathcal{C}} & {\mathcal{D}}
	\arrow["F", shift left, from=1-1, to=1-2]
	\arrow["G"', shift right, from=1-1, to=1-2]
\end{tikzcd}, definimos a transformação natural $\eta:F \Longrightarrow G$ da seguinte forma: \\
$\eta$ é uma família de flechas $(\eta_A: F(A) \longrightarrow G(A))_{A \in Obj(\mathcal{C})}$, tal que $\eta_B \circ F(f) = G(f) \circ \eta_A$. Isso equivale a dizer que o seguinte diagrama comuta.
% https://q.uiver.app/#q=WzAsNixbMiwwLCJGKEEpIl0sWzQsMCwiRyhBKSJdLFsyLDIsIkYoQikiXSxbNCwyLCJHKEIpIl0sWzAsMCwiQSJdLFswLDIsIkIiXSxbNCw1LCJmIiwyXSxbMCwyLCJGKGYpIiwyXSxbMSwzLCJHKGYpIiwyXSxbMCwxLCJcXGV0YV9BIiwxXSxbMiwzLCJcXGV0YV9CIiwxXV0=
\[\begin{tikzcd}[sep=large]
	A && {F(A)} && {G(A)} \\
	\\
	B && {F(B)} && {G(B)}
	\arrow["f"', from=1-1, to=3-1]
	\arrow["{\eta_A}"{description}, from=1-3, to=1-5]
	\arrow["{F(f)}"', from=1-3, to=3-3]
	\arrow["{G(f)}"', from=1-5, to=3-5]
	\arrow["{\eta_B}"{description}, from=3-3, to=3-5]
\end{tikzcd}\]

    
\end{defi}

A transformação natural identidade é a transformação $1_F:F \Longrightarrow F$, tal que $(1_F)_A: F(A) \longrightarrow F(A)$ é a identidade de $F(A)$. \\
Ainda, dadas as trasformações naturais $\eta:F \Longrightarrow G$ e $\mu: G \Longrightarrow H$, onde $F, G$ e $H$ são funtores de uma categoria $\mathcal{C}$ para uma categoria $\mathcal{D}$, podemos definir a transformação $(\mu \circ \eta): F \Longrightarrow H$ como sendo a família de flechas $(\mu_A \circ \eta_A: F(A) \longrightarrow G(A))_{A \in Obj(\mathcal{C})}$.

Dessa forma, podemos definir o que é a categoria de funtores: 

$\mathbf{Fun(\mathcal{C}, \mathcal{D})}$ é a categoria em que os objetos são funtores de $\mathcal{C}$ para $\mathcal{D}$ e os morfismos são transformações naturais dos funtores de $Obj(\mathbf{Fun(\mathcal{C}, \mathcal{D})})$.

\begin{titlemize}{Lista de consequências}
	\item \hyperref[grupo-fundamental]{Grupo fundamental};\\ %'consequencia1' é o label onde o conceito Consequência 1 aparece
	\item \hyperref[homotopia]{Homotopia}
\end{titlemize}

%[Bianca]: é mais fácil criar a lista de dependências do que a de consequências.

%---------------------------------------------------------------------------------------------------------------------!Draft!-----------------------------------------------------------------------------------------------------------------
\subsection{Transformação Natural}
\label{transformação-natural-categorias-ex}
\begin{titlemize}{Lista de dependências}
	\item \hyperref[transformação-natural-categorias-def]{Transformação natural};\\ %'dependencia1' é o label onde o conceito Dependência 1 aparece (--à arrumar um padrão para referencias e labels--) 
	\item \hyperref[categorias-ex]{Categorias-Exemplos};\\
% quantas dependências forem necessárias.
\end{titlemize}

\begin{ex}[Transformações Naturais]
	Alguns exemplos de transformações naturais.
 
    \begin{itemize}
        \item $J:\mathbf{1_{\mathbf{Vec(\mathbb{K})}}} \Longrightarrow ()^{**} $ é uma transformação natural do funtor identidade no funtor bidual $()^{**}$, de tal forma que $J_X(x)(z^*) = z^*(x)$. Ainda, $f^{**}: A^{**} \longrightarrow B^{**}$ para algum morfismo $f: A \longrightarrow B$ é a transformação linear, tal que $f^{**}(z^{**})(x^*) = z^{**}(x^{*} \circ f)$. Então o seguinte diagrama comuta:
       % https://q.uiver.app/#q=WzAsNixbMiwwLCJWIl0sWzQsMCwiVl57Kip9Il0sWzIsMiwiVyJdLFs0LDIsIldeeyoqfSJdLFswLDAsIlYiXSxbMCwyLCJXIl0sWzQsNSwiTCIsMl0sWzAsMiwiTCIsMl0sWzEsMywiTF57Kip9IiwyXSxbMCwxLCJKX1YiLDEseyJzdHlsZSI6eyJ0YWlsIjp7Im5hbWUiOiJob29rIiwic2lkZSI6InRvcCJ9fX1dLFsyLDMsIkpfVyIsMSx7InN0eWxlIjp7InRhaWwiOnsibmFtZSI6Imhvb2siLCJzaWRlIjoidG9wIn19fV1d
\[\begin{tikzcd}
	V && V && {V^{**}} \\
	\\
	W && W && {W^{**}}
	\arrow["L"', from=1-1, to=3-1]
	\arrow["{J_V}"{description}, hook, from=1-3, to=1-5]
	\arrow["L"', from=1-3, to=3-3]
	\arrow["{L^{**}}"', from=1-5, to=3-5]
	\arrow["{J_W}"{description}, hook, from=3-3, to=3-5]
\end{tikzcd}\].

\item Temos o funtor $\#: \mathbf{SET_\omega} \longrightarrow \mathbf{Ord}_\omega$ que leva um conjuto finito em seu respectivo ordinal. Dessa forma, uma classe de bijeções $(\alpha_A: A \hookrightarrow \#(A))_{A \in Obj(\mathbf{SET_\omega})}$ define uma transformação natural.

% https://q.uiver.app/#q=WzAsNixbMiwwLCJBIl0sWzQsMCwiXFwjQSJdLFsyLDIsIkIiXSxbNCwyLCJcXCNCIl0sWzAsMCwiQSJdLFswLDIsIkIiXSxbNCw1LCJmIiwyXSxbMCwyLCJmIiwyXSxbMSwzLCJcXCNmIiwyXSxbMCwxLCJcXGFscGhhX0EiLDFdLFsyLDMsIlxcYWxwaGFfQiIsMV1d
\[\begin{tikzcd}
	A && A && {\#A} \\
	\\
	B && B && {\#B}
	\arrow["f"', from=1-1, to=3-1]
	\arrow["{\alpha_A}"{description}, from=1-3, to=1-5]
	\arrow["f"', from=1-3, to=3-3]
	\arrow["{\#f}"', from=1-5, to=3-5]
	\arrow["{\alpha_B}"{description}, from=3-3, to=3-5]
\end{tikzcd}\]


        
    \end{itemize}
\end{ex}


\begin{titlemize}{Lista de consequências}
	\item \hyperref[hom-grupo-fundamental]{homomorfismo-de-grupo-fundamental};\\ %'consequencia1' é o label onde o conceito Consequência 1 aparece
\end{titlemize}

%---------------------------------------------------------------------------------------------------------------------!Draft!-----------------------------------------------------------------------------------------------------------------
\subsection{Equivalência de Categorias}
\label{equivalência-de-categorias-def}
\begin{titlemize}{Lista de dependências}
	\item \hyperref[funtor-categorias-def]{Funtor};\\ %'dependencia1' é o label onde o conceito Dependência 1 aparece (--à arrumar um padrão para referencias e labels--) 
	\item \hyperref[transformação-natural-categorias-def]{Transformação natural};\\
  \item \hyperref[isomorfismo-em-categorias-def]{Isomorfismo};\\
% quantas dependências forem necessárias.
\end{titlemize}
\begin{defi}[Equivalência de Categorias]
	Uma categoria $\mathcal{C}$ é equivalente a uma categoria $\mathcal{D}$ se, e somente se, existem funtores $F: \mathcal{C} \longrightarrow \mathcal{D}$ e $G: \mathcal{D} \longrightarrow \mathcal{C}$, onde se cumpre $F \circ G \cong \mathbf{1}_\mathcal{D}$ e $G \circ F \cong \mathbf{1}_\mathcal{C}$.
 Onde $\cong$ é o isomorfismo entre os objetos da categoria dos funtores $\mathbf{Fun(\mathcal{C}, \mathcal{D})}$, visto na seção \hyperref[transformação-natural-categorias-def]{Transformação Natural}.
\end{defi}

Note a semelhança dessa definição com a noção de espaços homotopicamente equivalentes.

\begin{titlemize}{Lista de consequências}
	\item \hyperref[grupo-fundamental]{Grupo fundamental};\\ %'consequencia1' é o label onde o conceito Consequência 1 aparece
\end{titlemize}

%[Bianca]: é mais fácil criar a lista de dependências do que a de consequências.



%%% Local Variables:
%%% mode: LaTeX
%%% TeX-master: "../Alg.Top-Wiki"
%%% End:

\section{Espaço de recobrimento}
\label{espaco-de-recobrimento}

\begin{titlemize}{Lista de Dependências}
	\item \hyperref[homotopia]{Homotopia};\\ %homotopia
	\item \hyperref[grupo-fundamental]{Grupo fundamental};\\
    \item \hyperref[topologia-quociente]{Espaço quciente};
\end{titlemize}

Na topologia algébrica, espaços de recobrimento estão intimamente relacionados ao grupo fundamental: Todos os recobrimentos têm a propriedade de levantamento de curva e homotopia, portanto ao invés de acha uma homotopia em espaço original podemos verificar se existir uma homotopia num recobrimento que têm melhores propriedades topológicas. Por isso, os espaços de recobrimento são uma ferramenta importante no cálculo de grupos fundamentais.
\subsection{Espaço de recobrimento}
\label{espaco-de-recobrimento-def}
\begin{titlemize}{Lista de dependências}
	\item \hyperref[topologia-quociente]{Espaço quciente};\\ %'dependencia1' é o label onde o conceito Dependência 1 aparece (--à arrumar um padrão para referencias e labels--) 
% quantas dependências forem necessárias.
\end{titlemize}
\begin{defi}[Espaço de recobrimento]
Uma função contínua $p:E\rightarrow X$ é um \textbf{recobrimento} se para todo $x\in X,$ existe uma vizinhança aberta $U\subseteq X$ de $x$ e um conjunto de índices $\Lambda\ne \varnothing$ tal que 
$$p^{-1}(U)=\amalg_{\lambda\in \Lambda} V_\lambda,$$
onde $V_\lambda\subseteq E$ é um subconjunto aberto e $p|_{V_\lambda}:V_\lambda\rightarrow U$ é homeomorfismo.
\end{defi}

\begin{nota}
Introduzimos algumas terminologias: 
    \begin{itemize}
        \item $E$ é um espaço (total) de recobrimento.
        \item $U$ é um aberto uniformemente recoberto de $X.$
        \item $V_\lambda$ é uma placa de $U$ do recobrimento.
        \item A cardinalidade de $\Lambda$ é o número de folhas do recobrimento (veremos que $\# \Lambda$ não depende de $x$). 
    \end{itemize}
\end{nota}

\begin{ex}
A função $p:\mathbb{R}\rightarrow \mathbb{S}^1$ dada por $p(x)=e^{2\pi ix}$ é um recobrimento: dado $y_0=e^{2\pi i x_0}\in\mathbb{S}^1,$ e $U=\mathbb{S}^1\setminus \{-y_0\}$ nós temos 
$$p^{-1}(U)=\amalg_{k\in \mathbb{Z}} (x_0+\frac{2k-1}{2},x_0+\frac{2k+1}{2}),$$
denotamos intervalo aberto $(x_0+\frac{2k-1}{2},x_0+\frac{2k+1}{2})$ por $V_k.$ Logo, a função $p|_{V_k}:V_k\rightarrow \mathbb{S}^1\setminus\{-y\}$ é um homeomorfismo.
\end{ex}

\begin{ex}
    A função $p:\mathbb{S}^n\rightarrow \mathbb{RP}^n=\mathbb{S}^n/\mathbb{Z}_2$ dada por $p(x)=[x]$ é um recobrimento.
\end{ex}

\begin{ex}
    Dado um conjunto $\Lambda\ne \varnothing$ qualquer munido com a topologia discreta, a função projeção $pr_2:E=\Lambda\times X\rightarrow X$ é um recobrimento. Esse recobrimento é dito \textbf{recobrimento trivial}
\end{ex}

\begin{prop}
    Suponha que $X$ é um espaço topológico conexo e $p:E\rightarrow X$ um recobrimento, então toda fibra tem a mesma cardinalidade, i.e. $\# p^{-1}(x_0)=\# p^{-1}(x_1)$ para todo $x_0,\;x_1\in X.$ Isso mostra que $\Lambda$ não depende de $x$.
\end{prop}

\begin{dem}
    Seja $x_0\in X$ e seja $A=\{x_1\in X: \#p^{-1}(x_1)=\# p^{-1}(x_0)\}.$ O conjunto $A$ não é vazio, pois $x_0\in A.$ Agora vamos provar que $A$ é aberto. Suponha que $x\in A$ e seja $U$ uma vizinhança aberta de $x$ tal que $p^{-1}(U)=\amalg_{\lambda\in \Lambda} V_\lambda$ com $p|_{V_\lambda}:V_\lambda\rightarrow U$ hemeomorfismo. Então, $U\subseteq A,$ pois se $x'\in U,$ então 
    $$\# p^{-1}(x')=\# \Lambda=\# p^{-1}(x)=\# p^{-1}(x_0).$$
    O conjunto $A$ é fechado, pois $X\setminus A$ é aberto pelo mesmo argumento acima. Como $X$ é conexo, $X=A$ como queríamos. 
\end{dem}

\begin{nota}
    Localmente todo recobrimento $p:E\rightarrow U$ é isomorfo ao recobrimento trivial, i.e. para todo $x\in X,$ existem uma vizinhança aberta $U$ de $x$, um espaço topológico discreto $\Lambda,$ e um homeomorfismo $h: E|_U\rightarrow U\times \Lambda$ tal que $pr_1\circ h= p.$
\end{nota}

\begin{titlemize}{Lista de consequências}
	\item \hyperref[levantamento-de-caminhos-prop]{Levantamento de caminhos};\\ %'consequencia1' é o label onde o conceito Consequência 1 aparece
	\item \hyperref[levantamento-de-homotopia-prop]{Levantamento de homotopia}
\end{titlemize}

\input{conteudo/levantamento-de-caminhos-prop}
\input{conteudo/levantamento-de-homotopia-prop}
\input{conteudo/grupo-fundamental-de-espaco-projetivo-ex}
\subsection{Grupo fundamental de 1-esfera}
\label{grupo-fundamental-de-S1-prop}
\begin{titlemize}{Lista de dependências}
	\item \hyperref[levantamento-de-homotopia-prop]{Levantamento de homotopia};\\ %'dependencia1' é o label onde o conceito Dependência 1 aparece (--à arrumar um padrão para referencias e labels--) 
	\item \hyperref[espaco-de-recobrimento-def]{Espaço de recobrimento};\\
    \item \hyperref[grupo-fundamental]{Grupo fundamental}
% quantas dependências forem necessárias.
\end{titlemize}

\begin{thm}
    O grupo fundamental $\pi_1(\mathbb{S}^1,1)$ é isomorfo a $\mathbb{Z}.$ 
\end{thm}

\begin{dem}
Vimos que $p:\mathbb{R}\rightarrow \mathbb{S}^1$ com $x\mapsto e^{2\pi i x}$ é um recobrimento. Seja $deg:\pi_1(\mathbb{S}^1,1)\rightarrow \mathbb{Z}=p^{-1}(1)\subseteq \mathbb{R}$ uma função dada por $deg([\alpha])=\Tilde{\alpha}_0(1),$ O corolário \ref{cor:bijedeggene} garante que $deg$ é uma bijeção. Vamos mostrar que $deg$ é um homomorfismo: Note que se $\alpha,\;\beta\in \Omega(X,x),$ então
\begin{itemize}
    \item $\Tilde{\alpha}_e*\Tilde{\beta}_{\Tilde{\alpha}_e (1)}(0)=\Tilde{\alpha}_e(0)=e,$
    \item $p(\Tilde{\alpha}_e*\Tilde{\beta}_{\Tilde{\alpha}_e (1)})=\alpha *\beta,$
\end{itemize}
logo, pelo unicidade de levantamento, temos 
\begin{align*}
\widetilde{(\alpha*\beta)}_e=\begin{cases}
    \Tilde{\alpha}_e (2s)\qquad& 0\le s\le \frac{1}{2}\\
    \Tilde{\beta}_{\Tilde{\alpha}_e (1)}(2s-1)&\frac{1}{2}\le s\le 1
    \end{cases}=\Tilde{\alpha}_e*\Tilde{\beta}_{\Tilde{\alpha}_e (1)}.
\end{align*}
Logo $deg([\alpha*\beta])=\widetilde{(\alpha*\beta)}_0 (1)=\Tilde{\alpha}_0*\Tilde{\beta}_{\Tilde{\alpha}_0 (1)}(1)=\Tilde{\beta}_{deg(\alpha)}(1).$

Por unicidade de levantamento de novo, obtemos $\Tilde{\beta}_n=n+\Tilde{\beta}_0.$ Logo,
\[deg([\alpha*\beta])=\Tilde{\beta}_{deg(\alpha)}(1)=deg(\alpha)+\Tilde{\beta}_0 (1)=deg(\alpha)+deg(\beta).\]
Portanto $deg$ é um isomorfismo de grupo.
\end{dem}

\begin{titlemize}{Lista de consequências}
	\item \hyperref[teo-ponto-fixo-brower]{Teorema Ponto Fixo de Brower};
\end{titlemize}

\subsection{Grupo fundamental de toros}
\label{grupo-fundamental-de-toro-ex}
\begin{titlemize}{Lista de dependências}
    \item \hyperref[homotopia-def]{Homotopia};\\
    \item \hyperref[grupo-fundamental]{Grupo fundamental};\\
    \item \hyperref[hom-grupo-fundamental]{Homomorfismo de grupos fundamentais};\\
    \item \hyperref[grupo-fundamental-de-espaco-de-produtos-prop]{Grupo fundamental de espaço de produtos};\\
    \item \hyperref[grupo-fundamental-de-S1-prop]{Grupo fundamental de 1-esfera}.
    
\end{titlemize}

\begin{ex}
    Como o toro $\mathbb{T}^n$ é homeomorfo ao $\mathbb{S}^1\times \ldots \times \mathbb{S}^1$ ($n$ fatores). Pelas proposições \ref{grupo-fundamental-de-espaco-de-produtos-prop}, \ref{hom-grupo-fundamental} e \ref{grupo-fundamental-de-S1-prop}, obtemos 
    \[\pi_1(\mathbb{T}^n)\cong \pi_1(\mathbb{S}^1)\times\ldots\times\pi_1(\mathbb{S}^1)\cong\mathbb{Z}^n.\]
\end{ex}

%%% Local Variables:
%%% mode: LaTeX
%%% TeX-master: "../Alg.Top-Wiki"
%%% End:

\section{Retração}
\label{retração}
Retração é uma relação de um subespaço com o espaço todo. Podemos pensar essa relação como retraindo todo o espaço para aquele subespaço (por isso o nome). Uma consequência que podemos tirar da existência ou não existência de uma retração é o famoso teorema do ponto fixo de Brower. 

\subsection{Retração}
\label{retração-def}
\begin{defi}[Retração]
Seja $A \subseteq X$ um subespaço de $X$. Uma retração de $A$ em $X$ é uma função contínua $r:X \to A$, tal que $r\restriction_A = id_A$.	 
\end{defi}

% onde conteudos.tex é o nome do arquivo tex que voce quer incluir nessa secção.
%---------------------------------------------------------------------------------------------------------------------!Draft!-----------------------------------------------------------------------------------------------------------------
\subsection{Retrato por Deformação}
\label{retrato-por-deformação-def}
\begin{titlemize}{Lista de dependências}
	\item \hyperref[retração-def]{Retração};\\ %'dependencia1' é o label onde o conceito Dependência 1 aparece (--à arrumar um padrão para referencias e labels--) 
	\item \hyperref[homotopia]{Homotopia};\\
% quantas dependências forem necessárias.
\end{titlemize}
\begin{defi}[Retrato por deformação]
	Uma retração $r:X \rightarrow Y$ é um retrato por deformação se $(i\circ r) \sim id_X$, onde $i:Y \rightarrow X$ é a inclusão de $Y$ em $X$.
\end{defi}
    \begin{ex}
    Denotamos a esfera de raio $1/2$ por $\mathbb{S}^n_{1/2}$, e a função inclusão de $\mathbb{S}^n_{1/2}$ em $\text{int}(D^n)\setminus\{0\}$ por $i$. A função $r:\text{int}(D^n)\setminus\{0\}\longrightarrow \mathbb{S}^{n-1}_{1/2} $ dada por $x\longmapsto \frac{x}{2||x||}$ é um retrato por deformação, para todo $n\ge 2$, pois a função
    \begin{align*}
        H:\text{int}(D^n)\setminus\{0\} \times I &\longrightarrow \text{int}(D^n)\setminus\{0\}\\
        (x,t)&\longmapsto (1-t)x+t\frac{x}{2||x||}
    \end{align*}
    é uma homotopia entre $id_{\text{int}(D^n)\setminus\{0\}}$ e $i\circ r$.
\end{ex}

%---------------------------------------------------------------------------------------------------------------------!Draft!-----------------------------------------------------------------------------------------------------------------
\subsection{Lema da Retração} %afirmação aqui significa teorema/proposição/colorário/lema
\label{lema-retração}
\begin{titlemize}{Lista de dependências}
	\item \hyperref[homotopia]{Homotopia};\\ %'dependencia1' é o label onde o conceito Dependência 1 aparece (--à arrumar um padrão para referencias e labels--) 
	\item \hyperref[retração-def]{Retração};\\
% quantas dependências forem necessárias.
\end{titlemize}
O lema a seguir será importante na demonstração do Teorema do Ponto Fixo de Brower.
\begin{lemma}[Lema da Retração]% ou af(afirmação)/prop(proposição)/corol(corolário)/lemma(lema)/outros ambientes devem ser definidos no preambulo de Alg.Top-Wiki.tex 
	Não existe uma retração $r:D^2 \longrightarrow \partial D^2 = S^1$.
\end{lemma}

\begin{dem}
Suponha que $r:D^2 \longrightarrow S^1$ seja uma retração. Sendo $D^2$ um espaço contrátil, pois ele é convexo, temos que para todo laço $\alpha: I \Longrightarrow D^2$ existe uma homotopia que leva esse laço no ponto $\alpha(0) = \alpha(1) = x_0$ de $D^2$. Em particular, para um laço $\beta: I \longrightarrow S^1 = \partial D^2$ em $S^1$ existe uma homotopia relativa a $\partial I$, $H: I\times I \longrightarrow D^2$ tal que $H(t, 0) = \beta(t)$ e $H(t, 1) = \beta(0) = x \in S^1$. Se a retração $r$ existe, então $r\circ H: I\times I: \longrightarrow S^1$ é uma homotopia relativa a $\partial I$. De fato, $(r\circ H)(0, t) = r(\beta(t)) = \beta(t)$ e $(r\circ H)(s, 0) = r(x) = x$ e $(r\circ H)(1, t) = r(x) = x$. Dessa forma, teríamos que $S^1$ é contrátil, contrariando $\pi_1(S^1) = \mathbb{Z}$.
\end{dem}

\begin{titlemize}{Lista de consequências}
	\item \hyperref[teo-ponto-fixo-brower]{Teorema de ponto fixo de Brouwer};\\ %'consequencia1' é o label onde o conceito Consequência 1 aparece
\end{titlemize}

%[Bianca]: Um arquivo tex pode ter mais de uma afirmação (ou definição, ou exemplo), mas nesse caso cada afirmação deve ter seu próprio label. Dar preferência para agrupar afirmações que dependam entre sí de maneira próxima (um teorema e seu corolário, por exemplo)

\subsection{Teorema de ponto fixo de Brouwer} %afirmação aqui significa teorema/proposição/colorário/lema
\label{teo-ponto-fixo-brower}
\begin{titlemize}{Lista de dependências}
	\item \hyperref[homotopia]{Homotopia};\\ %'dependencia1' é o label onde o conceito Dependência 1 aparece (--à arrumar um padrão para referencias e labels--) 
	\item \hyperref[retração-def]{Retração};\\
    \item \hyperref[lema-retração]{Lema da Retração};\\
% quantas dependências forem necessárias.
\end{titlemize}
O teorema a seguir depende de um lema que será deixado na lista de dependências acima.
\begin{thm}[Teorema do Ponto Fixo de Brower]% ou af(afirmação)/prop(proposição)/corol(corolário)/lemma(lema)/outros ambientes devem ser definidos no preambulo de Alg.Top-Wiki.tex 
	Toda função contínua na bola possui ponto fixo, i.e, se $f:D^2 \longrightarrow D^2$, então existe $x \in D^2$, tal que $f(x) = x$.
\end{thm}

\begin{dem}
    Suponha por absurdo que exista uma função contínua $f:D^2 \longrightarrow D^2$ sem pontos fixos. Defino a função $\alpha: D^2 \longrightarrow S^1$ onde $\alpha(x)$ é o único ponto de intersecção da semirreta $\overrightarrow{f(x)x}$ com $S^1$. Essa função está bem definida, pois $t_x(\lambda) = \|f(x) + \lambda(x - f(x))\|$ é uma função real contínua tal que $t_x(0) \leq 1$ e $t_x \to \infty$ quando $x \to \infty$. Assim, pelo teorema do valor intermediário, existe $\lambda_x$ tal que $t_x(\lambda_x) = 1$. $\alpha$ é contínua, pois $\alpha(x) = f(x) + \lambda_x(x - f(x))$, onde $\lambda_x$ é expresso da seguinte forma:
    $\|f(x) + \lambda_x(x - f(x))\| = \|(x - f(x))\|^2\lambda_x^2 + 2\langle f(x), (x - f(x)) \rangle\lambda_x + \|f(x)\|^2 = 1$. Isso nos dá uma equação quadrática com 2 soluções reais, sendo a maior delas $$\lambda_x = \frac{-2\langle f(x), x - f(x)\rangle + \sqrt{(2\langle f(x), x - f(x)\rangle)^2 - 4(\|x - f(x)\|^2)(\|f(x)\|^2 - 1)}}{2(\|x - f(x)\|)}.$$ Note que não há problema com o quociente desde que assumimos por hipótese que $f(x) \ne x$ para todo $x$. Ainda, o termo dentro da raiz quadrada é sempre maior ou igual a 0, pela desigualdade de Schwarz. Além disso, se $x \in S^1$, então $\alpha(x) = x$. Portanto, $\alpha$ é uma retração, contrariando o lema mencionado anteriormente.

\end{dem}

%[Bianca]: Um arquivo tex pode ter mais de uma afirmação (ou definição, ou exemplo), mas nesse caso cada afirmação deve ter seu próprio label. Dar preferência para agrupar afirmações que dependam entre sí de maneira próxima (um teorema e seu corolário, por exemplo)

\subsection{Lema da Retração (versão geral)} %afirmação aqui significa teorema/proposição/colorário/lema
\label{lema-de-retracao-geral-prop}
\begin{titlemize}{Lista de dependências}
    \item \hyperref[homologia-singular-def]{Homologia singular};\\
    \item \hyperref[homomorfismo-de-homologias-singulares-induzido-prop]{Homomorfismo de homologias singulares induzido};\\
    \item \hyperref[homologia-singular-de-S1-prop]{Homologia singular da circunferência};\\
    \item \hyperref[grupo-de-homologia-singular-de-n-esfera-prop]{Grupo de homologia singular de n-esfera}.
\end{titlemize}

Apresentamos aqui uma versão mais geral do lema de retração.

\begin{lemma}
	Para todo $n\ge 2$, não existe uma retração $r:D^n \longrightarrow \partial D^n = \mathbb{S}^{n-1}$.
\end{lemma}

\begin{dem}
Suponha que $r:D^{n} \longrightarrow \mathbb{S}^{n-1}$ seja uma retração. Sendo $D^n$ um espaço contrátil, pois ele é convexo, temos que $H_{n-1}(D^n)=0$ e, por conseguinte, $r_*$ é uma função nula. Sejam $i:\mathbb{S}^{n-1}\hookrightarrow D^n$ a inclusão. Pela definição de retração, $r\circ i=id_{\mathbb{S}^{n-1}}$. Assim, obtemos 
\[0=r_*\circ i_*=(r\circ i)_*=id_{H_{n-1}(\mathbb{S}^{n-1})}.\]
Dessa forma, teríamos que $H_{n-1}(\mathbb{S}^{n-1})=0$, contrariando $H_{n-1}(\mathbb{S}^{n-1})\cong \mathbb{Z}$.
\end{dem}

\begin{titlemize}{Lista de consequências}
    \item \hyperref[teorema-de-ponto-fixo-de-brouwer-geral-prop]{Teorema de ponto fixo de Brouwer (versão geral)}.
	%\item \hyperref[]{}
\end{titlemize}

\subsection{Teorema de ponto fixo de Brouwer (versão geral)} %afirmação aqui significa teorema/proposição/colorário/lema
\label{teorema-de-ponto-fixo-de-brouwer-geral-prop}
\begin{titlemize}{Lista de dependências}
    \item \hyperref[teo-ponto-fixo-brower]{Teorema de ponto fixo de Brouwer};\\
    \item \hyperref[lema-de-retracao-geral-prop]{Lema de retração (versão geral)}.
\end{titlemize}

Apresentamos aqui uma versão mais geral do Teorema de ponto fixo de Brouwer. A prova 

\begin{thm}[Teorema do Ponto Fixo de Brower]% ou af(afirmação)/prop(proposição)/corol(corolário)/lemma(lema)/outros ambientes devem ser definidos no preambulo de Alg.Top-Wiki.tex 
	Se $f:D^n \longrightarrow D^n$ é contínua, então existe $x \in D^n$, tal que $f(x) = x$, ou seja, existe um ponto fixo.
\end{thm}

\begin{dem}
    Suponhamos que exista uma função contínua $f:D^n\longrightarrow D^n$ sem ponto fixo. Usando a mesma construção apresentada no Teorema \ref{teo-ponto-fixo-brower}, podemos obter uma retração $r: D^n\rightarrow\mathbb
    {S}^{n-1}$, o que entra em contradição com o Lema \ref{lema-de-retracao-geral-prop}.

\end{dem}

%\begin{titlemize}{Lista de consequências}
    %\item %\hyperref[homomorfismo-de-homologias-singulares-induzido-prop]{Homomorfismo de homologias singulares induzido}.\\
	%\item \hyperref[]{}
%\end{titlemize}

%%% Local Variables:
%%% mode: LaTeX
%%% TeX-master: "../Alg.Top-Wiki"
%%% End:

\section{Teorema de Seifert-Van Kampen}
\label{teorema-de-seifert-van-kampen}

\begin{titlemize}{Lista de Dependências}
	\item \hyperref[grupo-fundamental]{Grupo fundamental};\\ %assunto1 é o label onde o Assunto 1 aparece
\end{titlemize}

%O teorema de Seifert-Van Kampen diz que 
\begin{thm}[Teorema de Seifert-Van Kampen (S-VK)]
    Seja $X=U\cup V$, onde $U,V,U\cap V$ são conjuntos abertos e conexos por caminhos, com $x\in U\cap V$. Considere as inclusões 
    \[i_U:U\cap V\hookrightarrow U,\;i_V:U\cap V\hookrightarrow V,\;j_U:U\hookrightarrow X,\;j_V:V\hookrightarrow X.\] 
    Então o seguinte diagrama 
    % https://q.uiver.app/#q=WzAsNCxbMCwxLCJcXHBpXzEoVVxcY2FwIFYseCkiXSxbMSwwLCJcXHBpXzEoVSx4KSJdLFsxLDIsIlxccGlfMShWLHgpIl0sWzIsMSwiXFxwaV8xKFgseCkiXSxbMCwxLCJpX3tVKn0iXSxbMCwyLCJpX3tWKn0iLDJdLFsxLDMsImpfe1UqfSJdLFsyLDMsImpfe1UqfSIsMl1d
\[\begin{tikzcd}
	& {\pi_1(U,x)} \\
	{\pi_1(U\cap V,x)} && {\pi_1(X,x)} \\
	& {\pi_1(V,x)}
	\arrow["{j_{U_*}}", from=1-2, to=2-3]
	\arrow["{i_{U_*}}", from=2-1, to=1-2]
	\arrow["{i_{V_*}}"', from=2-1, to=3-2]
	\arrow["{j_{V_*}}"', from=3-2, to=2-3]
\end{tikzcd}\]
é um \emph{pushout}.
\end{thm}
Como o \emph{pushout} é único a menos de isomorfismo, podemos afirmar que $\pi_1(X,x)$ é totalmente determinado pelos homomorfismos $i_{U_*}$ e $i_{V_*}$.

Nesta seção, discutimos alguns casos particulares e aplicações do teorema de S-VK. Manteremos as notações acima em todas as subseções.
\subsection{Caso A de Teorema de Seifert-Van Kampen} %afirmação aqui significa teorema/proposição/colorário/lema
\label{teorema-s-vk-caso-a-prop}
\begin{titlemize}{Lista de dependências}
	\item \hyperref[grupo-fundamental]{Grupo fundamental};\\
% quantas dependências forem necessárias.
\end{titlemize}
\begin{prop}
    Se $\pi_1(U,x)=\pi_(V,x)=\{e\}$, então $\pi_1(X,x)=\{e\}$
\end{prop}
\begin{dem}
    É fácil verificar que $(\{e\},id_{\{e\}},id_{\{e\}})$ é o \emph{pushout} de $(\pi_1(U\cap V,x),i_{U_*},i_{V_*})$. Pela unicidade do \emph{pushout}, obtemos $\pi_1(X,x)=\{e\}$.
\end{dem}
\begin{titlemize}{Lista de consequências}
	\item \hyperref[grupo-fundamental-de-esferas-prop]{Grupo fundamental de esferas};\\ %'consequencia1' é o label onde o conceito Consequência 1 aparece
	%\item \hyperref[]{}
\end{titlemize}
\subsection{Grupo fundamental de esferas} %afirmação aqui significa teorema/proposição/colorário/lema
\label{grupo-fundamental-de-esferas-prop}
\begin{titlemize}{Lista de dependências}
	\item \hyperref[grupo-fundamental]{Grupo fundamental};\\
    \item \hyperref[teorema-s-vk-caso-a-prop]{Caso A de Teorema de Seifert-Van Kampen}.
% quantas dependências forem necessárias.
\end{titlemize}

\begin{corol}
    O grupo fundamental $\pi_1(\mathbb{S}^n,p)=\{e\}$ para todo $n\ge 2$.
\end{corol}
\begin{dem}
    Considere $U=\mathbb{S}^n\setminus\{(0,...,0,1)\}$ e $V=\mathbb{S}^n\setminus\{(0,...,0,-1)\}$. Nesse caso, $U\cap V$ é um aberto conexo por caminhos para todo $n\ge 2$. Como $U$ e $V$ são homeomorfos a $\mathbb{R}^n$, que é contrátil por ser convexo, temos que $\pi_1(U)=\pi_1(V)=\{e\}$. Pela proposição \ref{teorema-s-vk-caso-a-prop}, obtemos $\pi_1(\mathbb{S}^2)=\{e\}$ para todo $n\ge 2$. Esse argumento não se aplica para $n=1$, pois $U\cap V$ não é conexo por caminhos. 
\end{dem}
\subsection{Caso B de Teorema de Seifert-Van Kampen} %afirmação aqui significa teorema/proposição/colorário/lema
\label{teorema-s-vk-caso-b-prop}
\begin{titlemize}{Lista de dependências}
    \item \hyperref[colagem-de-n-celula-def]{Colagem de n-célula};\\
    \item \hyperref[grupo-fundamental]{Grupo fundamental};\\
    \item \hyperref[variedade-def]{Variedade topológica};\\
    \item \hyperref[grupo-fundamental-de-esferas-prop]{Grupo fundamental de esferas}
% quantas dependências forem necessárias.
\end{titlemize}
\begin{prop}
    Se $\pi_1(U\cap V,x)=\pi_1(V,x)=\{e\}$, então $j_{U_*}:\pi_1(U,x)\rightarrow \pi_1(X,x)$ é um isomorfismo.
\end{prop}
\begin{dem}
    É fácil verificar que $(\pi_1(U,x),\{e\}\hookrightarrow \pi_1(U,x), id_{\pi_1(U,x)})$ é o \emph{pushout} de $(\pi_1(U\cap V,x),i_{U_*},i_{V_*})$. Pela unicidade do \emph{pushout}, $\pi_1(U,x)$ é único a menos de isomorfismo, o que implica que $j_{U_*}$ é um isomorfismo.
\end{dem}

\begin{corol}
    Se $M$ uma variedade conexa de dimensão maior ou igual $3$ com $x\in M$, então $\pi_1(M-\{x\},p)\cong\pi_1(M,p)$ para todo $p\in M\setminus\{x\}$.
\end{corol}
\begin{dem}
Pela definição de variedade topológico, existe uma vizinhança aberta de $x$ em $M$, tal que $U$ é homeomorfo a $\text{int} (D^n)$. Considere $V=M\setminus\{x\}$, assim $U\cap V$ é homeomorfo a $\text{int}(D^n)\setminus \{0\}$ que é homotopicamente equivalente a $\mathbb{S}^{n-1}$. Como $U$, $V$ e $U\cap V$ são abertos conexo por caminhos e $\pi_1(\mathbb{S}^{n-1},p)=\pi_1(\text{int}(D^n),0)=\{e\}$ (o grupo fundamental da esfera pode ser encontrado em \ref{grupo-fundamental-de-esferas-prop} e \ref{grupo-fundamental-de-S1-prop}) para todo $n\ge 3$, pela proposição anterior, temos $\pi_1(M-\{x\},p)\cong\pi_1(M,p)$.
\end{dem}

\begin{corol}
    Seja $X$ um espaço Hausdorff conexo por caminhos. Seja $f:\mathbb{S}^{n-1}\rightarrow X$ uma função contínua e $i:\mathbb{S}^{n-1}\hookrightarrow D^n$ uma inclusão, onde $n\ge 3$. Denotamos o espaço obtido de $X$ pela colagem de uma $n$-célula por meio da função $f$ por $X_f$. Então, temos que $\pi_1(X,h^{-1}(p))\cong \pi_1(X_f, p)$ para todo ponto $p\in h(\text{int}(D^n))\cap X_f\setminus\{h(0)\}$, onde $\pi:X\rightarrow X_f$, $h:D^n\rightarrow X_f$ são as funções associadas ao \emph{pushout}.
\end{corol}
\begin{dem}
     Consideramos $V=h(\text{int}(D^n))$ e $U=X_f\setminus \{h(0)\}$. Como discutido em \ref{sequencia-exata-da-colagem-prop}, temos que $V$ é homeomorfo a $\text{int}(D^n)$, $U$ é homotopicamente equivalente a $X$ e $U\cap V$ é homotopicamente equivalente a $\mathbb{S}^{n-1}$. Dessa forma, temos que $\pi_1(U,p)=\pi_1(U\cap V,p)=\{e\}$ para todo $n\ge 3$ e $p\in U\cap V$. Pela proposição anterior, temos $\pi_1(X,h^{-1}(p))\cong\pi_1(U,p)\cong \pi_1(X_f, p)$, para todo $p\in U\cap V$
\end{dem}
Aqui, o ponto base $h^{-1}(p)$ não é relevante, pois $X$ é um espaço conexo por caminhos. Usamos $h^{-1}(p)$, pois $h$ é um homeomorfismo entre $V$ e $\text{int}(D^n)$ o que garante que $h^{-1}(p)$ é um ponto em $X$.
\subsection{Caso C de Teorema de Seifert-Van Kampen} %afirmação aqui significa teorema/proposição/colorário/lema
\label{teorema-s-vk-caso-c-prop}
\begin{titlemize}{Lista de dependências}
    \item \hyperref[pushout-de-grupos-prop]{\emph{Pushout} de grupos};\\
    \item \hyperref[grupo-fundamental]{Grupo fundamental};\\
    \item \hyperref[teorema-s-vk-caso-b-prop]{Caso B de Teorema de Seifert-Van Kampen}.
% quantas dependências forem necessárias.
\end{titlemize}
Note que, se $\pi_1(V,x)=\{x\}$, então a condição $j_{U_*}\circ i_{U_*}=j_{V_*}\circ i_{V_*}= e$ é equivalente a $\text{Im}(i_{U_*})\subseteq \text{Ker}(j_{U_*})$. A propriedade universal do \emph{pushout} então reduz a condição: para todo homomorfismo de grupos $\phi:\pi_1(U,x)\rightarrow H$ satisfazendo $\text{Im}(i_{U_*})\subseteq \text{Ker}(\phi)$, existe um único homomorfismo $\psi:\pi_1(X,x)\rightarrow H$ tal que $\psi\circ j_{U_*}=\phi$.

Essa condição decorre do teorema do homomorfismo de grupos, quando $\text{Im}(i_{U_*})$ é um subgrupo normal de $\pi_1(U,x)$. No entanto, em geral, $\text{Im}(i_{U_*})$ não é um subgrupo normal. Para podermos aplicar o teorema do homomorfismo de grupos, podemos "normalizar" $\text{Im}(i_{U_*})$. 

\begin{prop}
    Se $\pi_1(V,x)=\{e\}$, então $\pi_1(X,x)\cong \pi_1(U,x)/\ \overline{\textnormal{Im}(i_{U_*})}$.
\end{prop}

\begin{dem}
    Essa proposição é uma consequência direta de um corolário apresentado na subseção \ref{pushout-de-grupos-prop}.
    %Basta provar que $\pi_1(U,x)/\overline{\text{Im}(i_{U_*})},\pi,\{e\}\hookrightarrow \pi_1(U,x)/\overline{\text{Im}(i_{U_*})} )$ é o pushout de $(\pi_1(U\cap V,x),i_{U_*},i_{V_*})$, onde $\pi:\pi_1(U,x)\rightarrow \pi_1(U,x)/\overline{\text{Im}(i_{U_*})}$ é a projeção canônica do quociente. Note que, todo homomorfismo de grupos $\phi:\pi_1(U,x)\rightarrow H$ satisfazendo $\text{Im}(i_{U_*})\subseteq \text{Ker}(\phi)$ também satisfaz $\overline{\text{Im}(i_{U_*})}\subseteq \text{Ker}(\phi)$, pela proposição \ref{fecho normal-def} e pela normalidade de $\text{Ker}(\phi)$. Assim, pelo teorema do homomorfismo, existe um único homomorfismo $\psi:\pi_1(U,x)/\overline{\text{Im}(i_{U_*})}\rightarrow H$ tal que $\psi\circ j_{U_*}=\phi$.
\end{dem}

Agora, podemos analisar o grupo fundamental do espaço obtido pela colagem de uma 2-célula.

\begin{corol}
    Seja $X$ um espaço Hausdorff conexo por caminhos com $x\in X$. Seja $f:\mathbb{S}^{1}\rightarrow X$ uma função contínua e $i:\mathbb{S}^{1}\hookrightarrow D^2$ uma inclusão. Denotamos o espaço obtido de $X$ pela colagem de uma $n$-célula por meio da função $f$ por $X_f$. Então, temos que $\pi_1(X_f, p)\cong \pi_1(X,h^{-1}(p))/\overline{\text{Im}(f_*)}$ para todo ponto $p\in h(\text{int}(D^2))\cap X_f\setminus\{h(0)\}$, onde $\pi:X\rightarrow X_f$, $h:D^2\rightarrow X_f$ são as funções associadas ao \emph{pushout}.
\end{corol}
\begin{dem}
     Consideramos $V=h(\text{int}(D^2))$ e $U=X_f\setminus \{h(0)\}$. Como discutido em \ref{sequencia-exata-da-colagem-prop}, temos que $V$ é homeomorfo a $\text{int}(D^2)$, $U$ é homotopicamente equivalente a $X$ e $U\cap V$ é homotopicamente equivalente a $\mathbb{S}^{1}$. Dessa forma, temos $\pi_1(U,p)=\{e\}$ para todo $p\in U\cap V$. Pela proposição anterior, temos que $\pi_1(X_f,p)\cong \pi_1(U, p)/\overline{\text{Im}(i_{U_*})}$, para todo $p\in U\cap V$. Pelo diagrama comutativo seguinte 
     % https://q.uiver.app/#q=WzAsNixbMCwwLCJcXHBpXzEoVVxcY2FwIFYscCkiXSxbMCwxLCJcXHBpXzEoXFxtYXRoYmJ7U31eMSkiXSxbMSwwLCJcXHBpXzEoVSxwKSJdLFsxLDEsIlxccGlfMShYLGheey0xfShwKSkiXSxbMiwwLCJcXHBpXzEoWF9mLHApIl0sWzIsMSwiXFxwaV8xKFhfZixwKSJdLFswLDIsImlfe1VfKn0iXSxbMiw0LCJqX3tVXyp9Il0sWzEsMywiZl8qIiwyXSxbMyw1LCJcXHBpXyoiLDJdLFs0LDUsIj0iLDFdLFswLDEsIlxcY29uZyIsMV0sWzIsMywiXFxjb25nIiwxXV0=
\[\begin{tikzcd}
	{\pi_1(U\cap V,p)} & {\pi_1(U,p)} & {\pi_1(X_f,p)} \\
	{\pi_1(\mathbb{S}^1)} & {\pi_1(X,h^{-1}(p))} & {\pi_1(X_f,p)},
	\arrow["{i_{U_*}}", from=1-1, to=1-2]
	\arrow["\cong"{description}, from=1-1, to=2-1]
	\arrow["{j_{U_*}}", from=1-2, to=1-3]
	\arrow["\cong"{description}, from=1-2, to=2-2]
	\arrow["{=}"{description}, from=1-3, to=2-3]
	\arrow["{f_*}"', from=2-1, to=2-2]
	\arrow["{\pi_*}"', from=2-2, to=2-3]
\end{tikzcd}\]
     temos que $\pi_1(X_f, p)\cong \pi_1(X,h^{-1}(p))/\overline{\text{Im}(f_*)}$, para todo $p\in U\cap V$.
\end{dem}

Como mencionado no final de \ref{teorema-s-vk-caso-b-prop}, o ponto base $h^{-1}(p)$ não é relevante.
\subsection{Caso D de Teorema de Seifert-Van Kampen} %afirmação aqui significa teorema/proposição/colorário/lema
\label{teorema-s-vk-caso-d-prop}
\begin{titlemize}{Lista de dependências}
	\item \hyperref[grupo-fundamental]{Grupo fundamental};\\
    \item \hyperref[pushout-de-grupos-prop]{\emph{Pushout} de grupos}.\\
% quantas dependências forem necessárias.
\end{titlemize}
\begin{prop}
    Se $\pi_1(U\cap V,x)=\{e\}$, então $\pi_1(X,x)\cong\pi_1(U,x)*\pi_1(V,x).$
\end{prop}
\begin{dem}
    A prova segue da proposição apresentada em subseção \ref{pushout-de-grupos-prop}.
\end{dem}


\subsection{Caso geral de Teorema de Seifert-Van Kampen} %afirmação aqui significa teorema/proposição/colorário/lema
\label{teorema-s-vk-caso-geral-prop}
\begin{titlemize}{Lista de dependências}
    \item \hyperref[geradores-relacoes-def]{Geradores e Relações};\\
	\item \hyperref[grupo-fundamental]{Grupo fundamental};\\
    \item \hyperref[pushout-de-grupos-prop]{\emph{Pushout} de grupos}.\\
% quantas dependências forem necessárias.
\end{titlemize}

Como todo grupo pode ser apresentado em termos de geradores e relações, pelo teorema apresentado na subseção \ref{pushout-de-grupos-prop}, o teorema de Seifert-Van Kampen pode ser formulado como 
\begin{thm}
    Seja $X=U\cup V$, onde $U,V,U\cap V$ são conjuntos abertos e conexos por caminhos, com $x\in U\cap V$. Considere as inclusões 
    \[i_U:U\cap V\hookrightarrow U,\;i_V:U\cap V\hookrightarrow V,\;j_U:U\hookrightarrow X,\;j_V:V\hookrightarrow X.\]
    Sejam também
    \begin{itemize}
        \item $\pi_1(U,x)=F(S_1)/\langle R_1\rangle,$
        \item $\pi_1(V,x)=F(S_2)/\langle R_2\rangle,$
        \item $\pi_1(U\cap V,x)=F(S)/\langle R\rangle,$
    \end{itemize}
    onde $F(S_1)$ e $F(S_2)$ são os grupos livres gerados por $S_1$ e $S_2$, e $\langle R_1\rangle$, $\langle R_2\rangle$, $\langle R\rangle$ são os subgrupos normais correspondentes.

    Para cada $s\in S$, tomamos $f_s\in F(S_1)$ e $g_s\in F(S_2)$, de modo que
    \[i_{U_*}(s\langle R\rangle)=f_s\langle R_1\rangle\;\;\text{ e }\;\;i_{V_*}(s\langle R\rangle)=g_s\langle R_2\rangle.\]
    Defina o conjunto
    \[R'=\{f_sg^{-1}_s:s\in S\}\subset F(S_1)*F(S_2)=F(S_1\cup S_2).\]
    Então, Então, o grupo fundamental de $X$ em $x$ é dado por
    \[\pi_1(X,x)=\frac{F(S_1\cup S_2)}{\langle R_1\cup R_2\cup R' \rangle}.\]
\end{thm}
%-------------------------------------------------------------------------------------------------------------!Draft!-------------------------------------------------------------------------------------------------------------------------
\section{Superfícies}
\label{superficies}

% \begin{titlemize}{Lista de Dependências}
% 	\item \hyperref[assunto1]{Assunto 1};\\ %assunto1 é o label onde o Assunto 1 aparece
% 	\item \hyperref[]{};
% \end{titlemize}

Nesta seção, introduziremos variedades, superfícies, simplexos, complexos simpliciais, triangulação, complexos celulares e provaremos o teorema de classificação de superfícies.

%---------------------------------------------------------------------------------------------------------------------!Draft!-----------------------------------------------------------------------------------------------------------------
\subsection{Variedades Topológicas}
\label{variedade-def}
%\begin{titlemize}{Lista de dependências}
	%\item \hyperref[dependecia1]{Dependência 1};\\ %'dependencia1' é o label onde o conceito Dependência 1 aparece (--à arrumar um padrão para referencias e labels--) 
	%\item \hyperref[]{};\\
% quantas dependências forem necessárias.
%\end{titlemize}
\begin{defi}[Variedade Topológica]
	Fixemos $m\geq 0$. Uma $m$-\textbf{variedade topológica} é um espaço topológico $M$ Hausdorff, $2^o$ enumerável munido de um \textbf{atlas} $\{(\phi_i, U_i) : i \in I\}$. Isto é, $\{U_i : i \in I\}$ é uma cobertura aberta de $M$ e $\phi_i: U_i \to \phi_i(U_i) \subset \mathbb{R}^m$ é um homeomorfismo, para todo $i \in I$.
    
    Denotemos por $\mathbb{H}^m$ o semiespaço $\mathbb{R}^{m-1}\times \left[0,\infty\right[$. Uma $m$-\textbf{variedade topológica com bordo} é um espaço topológico $M$ Hausdorff, $2^o$ enumerável munido de $\{(\phi_i, U_i) : i \in I\}$ (ao que também nos referimos como atlas), em que $\{U_i : i \in I\}$ é uma cobertura aberta de $M$ e $\phi_i: U_i \to \phi_i(U_i) \subset \mathbb{H}^m$ é um homeomorfismo, para todo $i \in I$.

Uma \textbf{superfície (com bordo)} é uma 2-variedade topológica (com bordo).
\end{defi}

\begin{titlemize}{Lista de consequências}
	\item \hyperref[triangulacao-def]{Triangulação};\\
    \item \hyperref[soma-conexa-def]{Soma conexa}
\end{titlemize}

%[Bianca]: é mais fácil criar a lista de dependências do que a de consequências.
%---------------------------------------------------------------------------------------------------------------------!Draft!-----------------------------------------------------------------------------------------------------------------
\subsection{Combinações afins e convexas}
\label{comb-afim-convexa-def}
% \begin{titlemize}{Lista de dependências}
% 	\item \hyperref[dependecia1]{Dependência 1};\\ %'dependencia1' é o label onde o conceito Dependência 1 aparece (--à arrumar um padrão para referencias e labels--) 
% 	\item \hyperref[]{};\\
% % quantas dependências forem necessárias.
% \end{titlemize}
\begin{defi}[Combinações lineares, afins e convexas]
	Seja $V$ um espaço vetorial sobre $\mathbb{R}$ e seja $S \subset V$ um subconjunto. Recordamos que uma \textbf{combinação linear} de elementos de $S$ é um vetor da forma $y = \sum_{j=0}^n \lambda_j x_j$, onde $\lambda_j \in \mathbb{R}$ e $x_j \in S$ para todo $0 \leq j \leq n$. Se $\sum_{j=0}^n \lambda_j = 1$, dizemos que $y$ é \textbf{combinação afim} de elementos de $S$. Por fim, se $\sum_{j=0}^n \lambda_j = 1$ e $\lambda_j \geq 0$ para todo $0 \leq j \leq n$, dizemos que $y$ é \textbf{combinação convexa} de elementos de $S$.

    O conjunto de combinações lineares de elementos de $S$ é o \textbf{espaço gerado} por $S$ e é denotado como $\text{span}(S)$. Do mesmo modo, o conjunto de combinações afins de elementos de $S$ é chamado de \textbf{espaço afim gerado} por $S$, e denotado por $\text{aff}(S)$. Por fim, $\text{conv}(S)$ é o conjunto de combinações convexas de $S$ e é chamado de \textbf{envoltória convexa} de $S$.
\end{defi}

É simples provar que se $x_0 \in S$, então $\text{aff}(S) = \text{span}(S-x_0) + x_0$. De fato, suponha que $y = \sum_{j=0}^n \lambda_j x_j$, onde $\sum_{j=0}^n \lambda_j = 1$ e $x_j \in S$ para todo $1 \leq j \leq n$ (podemos supor que $x_0$ é usado na representação de $y$, tomando $\lambda_0 = 0$ se necessário). Então
\begin{align*}
    y &= \sum_{j=0}^n \lambda_j x_j - x_0 + x_0
    = \sum_{j=0}^n \lambda_j x_j - \sum_{j=0}^n \lambda_j x_0 + x_0\\
    &= \sum_{j=1}^n \lambda_j(x_j - x_0) + x_0
    \in \text{span}(S-x_0) + x_0
\end{align*}

\begin{defi}[Independência afim]
    Dizemos que um subconjunto $S \neq \varnothing$ de um espaço vetorial $V$ é \textbf{``affine independent'' (a.i.)} se $S - x_0$ é linearmente independente, onde $x_0 \in S$. Ou seja, se não existe $S' \subsetneq S$ tal que $\text{aff}(S') = \text{aff}(S)$.
\end{defi}

A equivalência das duas definições segue do raciocínio anterior, pois caso exista $S' \subsetneq S$ tal que $\text{aff}(S') = \text{aff}(S)$ e $x_0 \in S$, então $\text{span}(S'-x_0) = \text{span}(S-x_0)$. Assim, $S - x_0$ não é linearmente independente. E reciprocamente, caso $S - x_0$ não seja linearmente independente, então existe $S' \subsetneq S$ tal que $\text{span}(S' - x_0) = \text{span}(S - x_0)$, logo $\text{aff}(S') = \text{aff}(S)$.

\begin{titlemize}{Lista de consequências}
	\item \hyperref[simplexo-def]{Simplexos};\\ %'consequencia1' é o label onde o conceito Consequência 1 aparece
	%\item \hyperref[]{}
\end{titlemize}
%---------------------------------------------------------------------------------------------------------------------!Draft!-----------------------------------------------------------------------------------------------------------------
\subsection{Simplexos}
\label{simplexo-def}
\begin{titlemize}{Lista de dependências}
	\item \hyperref[comb-afim-convexa-def]{Combinações afins e convexas};\\ %'dependencia1' é o label onde o conceito Dependência 1 aparece (--à arrumar um padrão para referencias e labels--) 
	%\item \hyperref[]{};\\
% quantas dependências forem necessárias.
\end{titlemize}

A seguir, introduzimos os conceitos de simplexos e de faces.

\begin{defi}[Simplexos]
    Se $\{x_0,\ldots,x_n\}$ são a.i. ($n \geq 0$), dizemos que $\sigma = \text{conv}\{x_0,\ldots,x_n\}$ é um $n$-\textbf{simplexo}, e o denotamos por $[x_0,\ldots, x_n]$. Dizemos que $n$ é a dimensão do simplexo $\sigma$, e escrevemos $n = \text{dim}(\sigma)$. O $n$-\textbf{simplexo padrão} é $\Delta^n = [0, e_1, \ldots, e_n]$ onde $\{e_1,\ldots, e_n\}$ é a base canônica de $\mathbb{R}^n$.

    Um $k$-simplexo $\tau$ é dito uma $k$-\textbf{face} de $\sigma$ caso existam $0 \leq i_0 < \ldots < i_k \leq n$ tais que $\tau = [x_{i_0}, \ldots, x_{i_k}]$. $\tau$ é uma face \textbf{própria} se $\tau \neq \sigma$. Notação: $\tau \leq \sigma$ se $\tau$ é face de $\sigma$, e $\tau < \sigma$ se $\tau$ for face própria de $\sigma$.

    As $0$-faces, $1$-faces e $2$-faces de $\sigma$ também são chamadas, respectivamente, de \textbf{vértices}, \textbf{arestas} e \textbf{triângulos} de $\sigma$.
\end{defi}

\begin{titlemize}{Lista de consequências}
	\item \hyperref[complexo-simplicial-def]{Complexos Simpliciais};\\ %'consequencia1' é o label onde o conceito Consequência 1 aparece
	%\item \hyperref[]{}
\end{titlemize}

%[Bianca]: é mais fácil criar a lista de dependências do que a de consequências.
%---------------------------------------------------------------------------------------------------------------------!Draft!-----------------------------------------------------------------------------------------------------------------
\subsection{Complexos Simpliciais}
\label{complexo-simplicial-def}
\begin{titlemize}{Lista de dependências}
	\item \hyperref[simplexo-def]{Simplexos};\\ %'dependencia1' é o label onde o conceito Dependência 1 aparece (--à arrumar um padrão para referencias e labels--) 
	%\item \hyperref[]{};\\
% quantas dependências forem necessárias.
\end{titlemize}

\begin{defi}[Complexos Simpliciais]
    Um \textbf{complexo simplicial} é um conjunto $K$ de simplexos em $\mathbb{R}^m$ tais que:
    \begin{enumerate}
        \item $\tau \leq \sigma, \sigma \in K \Rightarrow \tau \in K$;
        \item $\tau,\sigma \in K, \tau \cap \sigma \neq \varnothing \Rightarrow \tau \cap \sigma \leq \sigma, \tau \cap \sigma \leq \tau$.
    \end{enumerate}

    A \textbf{dimensão} de $K$ é $\text{dim}(K) = \sup\{\text{dim}(\tau): \tau \in K\}$.

    A \textbf{realização geométrica} de $K$ é
    \[|K| = \bigcup_{\sigma \in K} \sigma.\]
\end{defi}

\begin{defi}
    O \textbf{bordo} e o \textbf{interior} de um simplexo $\sigma$ são definidos, respectivamente, como:
    \[\partial \sigma = \bigcup_{\tau < \sigma}\tau \qquad\text{e}\qquad \text{int}(\sigma) = \sigma \setminus \partial \sigma.\]
    Note que $\partial \sigma$ é a realização geométrica do complexo simplicial $K = \{\tau : \tau < \sigma\}$.
\end{defi}

%\begin{titlemize}{Lista de consequências}
	%\item \hyperref[complexo-simplicial-def]{Complexos Simpliciais};\\ %'consequencia1' é o label onde o conceito Consequência 1 aparece
	%\item \hyperref[]{}
%\end{titlemize}

%[Bianca]: é mais fácil criar a lista de dependências do que a de consequências.
%---------------------------------------------------------------------------------------------------------------------!Draft!-----------------------------------------------------------------------------------------------------------------
\subsection{Triangulação}
\label{triangulacao-def}
\begin{titlemize}{Lista de dependências}
	\item \hyperref[complexo-simplicial-def]{Complexos simpliciais};\\ %'dependencia1' é o label onde o conceito Dependência 1 aparece (--à arrumar um padrão para referencias e labels--) 
	\item \hyperref[variedade-def]{Variedades Topológicas};\\
% quantas dependências forem necessárias.
\end{titlemize}

\begin{defi}[Triangulação de uma Variedade]
    Uma \textbf{triangulação} de uma superfície $M$ é um par ordenado $(K,\gamma)$, onde $K$ é um complexo simplicial e $\gamma:K\to 2^M$ é uma função, onde:
    \begin{enumerate}
        \item $\gamma(\sigma \cap \tau) = \gamma(\sigma) \cap \gamma(\tau)$, para todos $\sigma, \tau \in K$;
        \item existe um homeomorfismo $\varphi_{\sigma}:|\sigma| \to \gamma(\sigma)$, para todo $\sigma \in K$;
        \item se $\sigma \in K$ e $\tau \leq \sigma$, então $\varphi_{\sigma}|_{\tau}: |\tau| \to \gamma(\tau)$ é homeomorfismo;
        \item $\{\gamma(\sigma): \sigma \in K\}$ é uma cobertura fechada localmente finita de $M$.
    \end{enumerate}
\end{defi}
Neste caso, em especial, vale que $\text{dim}(K) = 2$.

O conceito de triangulação está intimamente relacionado ao de complexos celulares.

\begin{prop}
    Se $(K,\gamma)$ é uma triangulação de $M$, então existe um homeomorfismo $\varphi: |K| \to M$.
\end{prop}

Além disso, se $M$ e $N$ são duas superfícies, $\phi:M\to N$ é um homeomorfismo e $(K,\gamma)$ é uma triangulação de $M$, então $(K,\phi_* \circ \gamma)$ é triangulação de $N$, onde $\phi_*: 2^M \to 2^N$ é a função imagem direta de $\phi$. Assim, é natural se perguntar em que casos $(K,|.|)$ é uma triangulação de $|K|$. A proposição a seguir fornece uma caracterização desta propriedade. Para a demonstração ou mais informações, consulte a Proposição 3.5 de \textit{Jean Gallier, Dianna Xu, A Guide to the Classification Theorem for Compact Surfaces, Springer Berlin, Heidelberg, 2013.} 

\begin{prop}
    Seja $K$ um $2$-complexo simplicial. Então $(K,|.|)$ é uma triangulação de $|K|$, onde $|.|$ mapeia $\sigma \in K$ em $|\sigma|$, se, e somente se, valem as condições:
    \begin{enumerate}
        \item Toda aresta pertence a exatamente 2 triângulos;
        \item para todo vértice $v$, existe um inteiro $k\geq 0$ tal que $v$ pertence a exatamente $k$ triângulos e $k$ arestas; além disso, é possível ordenar as arestas e os triângulos contendo $v$ em sequências, $(a_1,\ldots,a_k)$ e $(A_1,\ldots,A_k)$, respectivamente, de como que $a_i$ e $a_{i+1}$ são arestas de $A_i$ para todo $1\leq i\leq k-1$, bem como $a_k$ e $a_1$ são arestas de $A_k$;
        \item $|K|$ é conexo.
    \end{enumerate}
\end{prop}

Para a demonstração ou mais informações, consulte a Proposição 3.6 de \textit{Jean Gallier, Dianna Xu, A Guide to the Classification Theorem for Compact Surfaces, Springer Berlin, Heidelberg, 2013.} 

% \begin{titlemize}{Lista de consequências}
% 	\item \hyperref[triangulacao-def]{Triangulação};
% \end{titlemize}

%[Bianca]: é mais fácil criar a lista de dependências do que a de consequências.
%---------------------------------------------------------------------------------------------------------------------!Draft!-----------------------------------------------------------------------------------------------------------------
\subsection{Regiões Poligonais}
\label{regiao-poligonal-def}
\begin{titlemize}{Lista de dependências}
	\item \hyperref[simplexo-def]{Simplexos}%;\\ %'dependencia1' é o label onde o conceito Dependência 1 aparece (--à arrumar um padrão para referencias e labels--) 
	%\item \hyperref[variedade-def]{Variedades Topológicas};\\
% quantas dependências forem necessárias.
\end{titlemize}
%%%%%%%%%%%% Versão antiga %%%%%%%%%%%%%%%%%%
% Antes de definirmos complexos celulares, introduzimos alguns conceitos auxiliares.
% \begin{defi}[Orientação formal]
%     Dado um conjunto $X$, definimos sua \textbf{orientação formal} como $\{-1,1\} \times X$, em que denotamos $(1,x)$ simplesmente como $x$ e $(-1,x)$ como $x^{-1}$, para todo $x\in X$. Também escrevemos $(x^{-1})^{-1} = x$. Dessa forma, a orientação formal de $X$ é denotada como $X \cup X^{-1}$. Definimos também o operador inversão, que mapeia $x \in X\cup X^{-1}$ em $x^{-1}$.
%      Seja $Y$ um conjunto. Denotamos por $Y^\#$ o quociente do conjunto de sequências finitas sobre $Y$, pela relação de equivalência dada por permutações cíclicas dos elementos da sequência. Denotamos a classe de equivalência de $(y_1,\ldots,y_n)$ por $y_1 \ldots y_n$. Desse modo, $y_1 ~y_2 \ldots y_{n-1}~y_n = y_n~y_1 \ldots y_{n-2}~y_{n-1}$.
% \end{defi}
% \begin{defi}[Complexo Celular]
%     Um \textbf{complexo celular} é uma tripla $K = (F,E,\mathcal{B})$, onde os elementos de $F$ são chamados de \textbf{faces} de $K$, os elementos de $E$ são chamados \textbf{arestas (``edges'')} de $K$, e $\mathcal{B}: F\cup F^{-1} \to (E \cup E^{-1})^\#$ é chamada de função bordo, satisfazendo:
%     \begin{enumerate}
%         \item Se $\mathcal{B}(A)= a_1 \ldots a_n$ com $a_1,\ldots, a_n \in E\cup E^{-1}$, então $\mathcal{B}(A^{-1})= a_n^{-1} \ldots a_1^{-1}$, para todo $A \in F$;
%         \item todo $a \in E\cup E^{-1}$ é elemento do bordo de no máximo duas faces.
%     \end{enumerate}
% As vezes também nos referimos aos elementos de $E^{-1}$ de arestas, e aos elementos de $F^{-1}$ de faces, respectivamente.
% Dizemos que um complexo celular $K$ é \textbf{conexo} se para todo par de arestas $a, b \in E$, existem arestas $a = a_1,a_2,\ldots,a_{k-1}, a_k=b\in E$ e faces $A_1,\ldots, A_{k-1} \in F$ de modo que $a$ é elemento do bordo de $A_1$ ou $A_1^{-1}$, $a_i$ é elemento do bordo de $A_{i-1}$ ou $A_{i-1}^{-1}$ bem como de $A_{i}$ ou $A_i^{-1}$, e $b$ é elemento do bordo de $A_{k-1}$ ou $A_{k-1}^{-1}$.
% \end{defi}
% O conceito de complexos celulares está intimamente relacionado ao de \hyperref[triangulacao-def]{triangulação} e também \hyperref[complexo-simplicial-def]{complexos simpliciais}.
% Diversos complexos celulares representam a mesma intuição geométrica. Isso motiva considerarmos a seguinte definição.
% \begin{defi}
%     Dado um complexo celular $K=(F,E,\mathcal{B})$, considere as seguintes construções:
%     \begin{enumerate}
%         \item fixe uma aresta $a \in E$, tome $b,c \notin E$ e defina $K'=(F,E\cup\{b,c\}\setminus\{a\},\mathcal{B}')$; a função $\mathcal{B}'$ é definida substituindo cada ocorrência de $a$ em um bordo por $b~c$, cada ocorrência de $a^{-1}$ por $c^{-1}~b^{-1}$ e mantendo o restante inalterado;
%         \item  fixe uma face $A\in F$ cujo bordo contenha ao menos 4 arestas, ou seja, $\mathcal{B}(A) = a_1\ldots a_k$ para $a_1,\ldots,a_k \in E\cap E^{-1}$ e $k\geq 4$; tome $x\notin E$, $X,Y \notin F$ e defina $K''=(F\cup\{X,Y\}\setminus\{A\},E\cup\{x\},\mathcal{B}'')$, em que $\mathcal{B}''(X) = a_1~x^{-1}~a_4$, $\mathcal{B}''(Y)= a_2~a_3~x$ e $\mathcal{B}''$ coincide com $\mathcal{B}$ nas demais faces.
%     \end{enumerate}
% Assim, definimos uma relação de equivalência de complexos celulares gerada pelas relações $K\sim K'$ e $K\sim K''$ para $K$ um complexo celular qualquer e $K'$ e $K''$ obtidos como nas construções acima.
% \end{defi}
% Todo complexo celular conexa $K=(F,E,\mathcal{B})$ define uma 2-variedade com bordo da seguinte forma: escreva $F= \bigcup_{n \geq 2} F_n$, onde os elementos de $F_n$ são faces com $n$ arestas no bordo. Considere a união disjunta de $|E|$ intervalos $[0,1]$ e $|F_n|$ polígonos de $n$ lados, para todo $n\geq 3$ (polígonos regulares em $\mathbb{R}^2$ com centro na origem e apótema 1, por exemplo). Então é simples ver que $\mathcal{B}$ descreve uma relação de equivalência $R$ que relaciona cada aresta de um polígono com arestas de $K$. Tomemos o espaço quociente $X=Y/R$. A condição de que cada aresta de $K$ pertença ao bordo de no máximo 2 faces garante que $X$ seja de Hausdorff e localmente homeomorfo a um aberto do semiplano $\mathbb{H}^2$. Se cada aresta pertencer a exatamente 2 faces, garantimos ainda que $X$ seja localmente homeomorfo a um aberto do plano.
%%%%%%%%%%%%%%%%%%%%%%%%%%%%%%%%%%%%%%%%%%%%

\begin{defi}[Região poligonal, orientação e etiquetagem]
    Uma \textbf{região poligonal} com $n$ lados é um simplexo $P = [v_1,\ldots,v_n]$, onde $v_1,\ldots, v_n \in \mathbb{S}^2$. Por convenção, ordenamos os pontos $v_1,\ldots, v_n$ em ordem anti-horária.

    Sejam $P_1,\ldots, P_k$ regiões poligonais dadas. Denotemos por $\partial_j P_i$ o conjunto de $j$-faces de $P_i$, $1\leq i\leq k$. Então, uma \textbf{orientação} nas arestas da união de regiões poligonais $\bigsqcup_{i=1}^k P_i$ é uma função $\mathcal{O}_i: \bigsqcup_{i=1}^k\partial_1 P_i\to \bigsqcup_{i=1}^k\partial_0 P_i$ tal que $\mathcal{O}(a) \in \partial a$ para toda aresta $a$ de $P_i$, $1\leq i\leq k$. Ou seja, é a escolha de um ``ponto inicial'' para cada aresta de cada região poligonal $P_i$.

    Já uma \textbf{etiquetagem} na união de regiões poligonais $\bigsqcup_{i=1}^k P_i$ é uma função $L: \bigsqcup_{i=1}^k\partial_1 P_i\to \Lambda$, onde $\Lambda \neq \varnothing$ é dito o conjunto de \textbf{etiquetas}.
\end{defi}

\begin{defi}[Transformação linear positiva e espaço obtido por colagem de arestas]
    Dadas duas arestas $A = [v_i, v_{i+1}]$ e $B = [v_j, v_{j+1}]$ com orientação $\mathcal{O}$ fixada, seja $\overline{\mathcal{O}}$ a orientação inversa. Isto é, $\mathcal{O}(A) \neq \overline{\mathcal{O}}(A)$ e $\mathcal{O}(B) \neq \overline{\mathcal{O}}(B)$. Definimos a \textbf{transformação linear positiva} de $A$ sobre $B$ como a função $h: A\to B$ que mapeia $(1-t) \mathcal{O}(A) + t \overline{\mathcal{O}}(A)$ em $(1-t) \mathcal{O}(B) + t \overline{\mathcal{O}}(B)$ para todo $t \in [0,1]$.
    
    Considere regiões poligonais $P_1,\ldots, P_k$, uma orientação $\mathcal{O}$ e uma etiquetagem $L$ de $\bigsqcup_{i=1}^k P_i$. Defina o espaço
    \[X = \bigsqcup_{i=1}^k P_i/\sim\]
    em que $x \sim y$ se, e somente se, $x = y$ ou então $x \in A$, $y \in B$ e $h(x) = y$, onde $A$ e $B$ são arestas com a mesma etiqueta e $h$ é a transformação linear positiva de $A$ sobre $B$. Isto é,
    \begin{align*}
        x \sim y \;\Longleftrightarrow \;
        &x=y\text{ ou }\exists a \in \Lambda, \exists A,B \in L^{-1}(a), \exists t\in [0,1]:\\ 
        &x = (1-t) \mathcal{O}(A) + t \overline{\mathcal{O}}(A), 
        y = (1-t) \mathcal{O}(B) + t \overline{\mathcal{O}}(B)
    \end{align*}
    
    Então, dizemos que o espaço quociente $X$ (bem como qualquer espaço topológico homeomorfo) é \textbf{obtido das regiões poligonais $P_1,\ldots, P_k$ por colagem de arestas} de acordo com a orientação $\mathcal{O}$ e a etiquetagem $L$.
\end{defi}

Note que quaisquer duas regiões poligonais com $n$ lados são homeomorfas. Mais do que isso, o espaço quociente $X$ obtido da região poligonal $P = [v_1,\ldots,v_n]$ por colagem de arestas é totalmente determinado, a menos de homeomorfismo, pelo símbolo
\[w = a_{i_1}^{\varepsilon_1} \ldots a_{i_n}^{\varepsilon_n},\]
onde $a_{i_1}$ é a etiqueta de $[v_1, v_2]$, $a_{i_2}$ é a etiqueta de $[v_2, v_3]$, e assim por diante ($a_{i_n}$ é a etiqueta de $[v_n, v_1]$), e $\varepsilon_i = \pm 1$, a depender se a orientação fixada em $P$ coincide com a orientação na ordenação dos vértices. Por exemplo, se o ponto inicial em $[v_1, v_2]$ é $v_1$, então $\varepsilon_1 = +1$. O símbolo $w$ é dito o \textbf{esquema de etiquetagem} para $P$ (com respeito à orientação e etiquetagem fixadas).

Para um espaço $X$ obtido pela colagem de arestas das regiões poligonais $P_1,\ldots, P_k$, o esquema de etiquetagem é dado por $w_1,\ldots, w_k$, onde $w_i$ é o esquema de etiquetagem de $P_i$ para cada $1\leq i \leq k$.

\begin{titlemize}{Lista de consequências}
    \item \hyperref[construcoes-regiao-poligonal-prop]{Construções com Regiões Poligonais}%;\\ %'consequencia1' é o label onde o conceito Consequência 1 aparece
    %\item \hyperref[]{}
\end{titlemize}
%---------------------------------------------------------------------------------------------------------------------!Draft!-----------------------------------------------------------------------------------------------------------------
\subsection{Construções com Regiões Poligonais}
\label{construcoes-regiao-poligonal-prop}
\begin{titlemize}{Lista de dependências}
	\item \hyperref[regiao-poligonal-def]{Regiões Poligonais}%;\\ %'dependencia1' é o label onde o conceito Dependência 1 aparece (--à arrumar um padrão para referencias e labels--) 
	%\item \hyperref[variedade-def]{Variedades Topológicas};\\
% quantas dependências forem necessárias.
\end{titlemize}

\begin{lemma}\label{varias-etiquetagens-lemma}
    Sejam $P_1,\ldots, P_k$ regiões poligonais, e seja $w_1,\ldots, w_k$ um esquema de etiquetagem. Para cada $1\leq i\leq n$, considere o espaço quociente $X_i = P_i/\sim_i$ (com respeito ao esquema de etiquetagem $w_i$) e, depois, realize a colagem dos espaços resultantes da seguinte forma:
    \[\bigsqcup_{i=1}^k X_i/\approx\]
    onde
    % \begin{align*}
    %     [x]\approx [y] ~\Longleftrightarrow ~ &[x]=[y]\text{ ou }\exists a\in \Lambda, \exists i,j\leq k, \exists A \in \partial_1 P_i, \exists B \in \partial_1 P_j, \exists t\in [0,1]:\\
    %     &L(A) = L(B) = a,\\
    %     &x = (1-t) \mathcal{O}(A) + t \overline{\mathcal{O}}(A), 
    %     y = (1-t) \mathcal{O}(B) + t \overline{\mathcal{O}}(B).
    % \end{align*}
    \[[x]_i \approx [y]_j ~\Longleftrightarrow x \sim y.\]
    Então, o espaço resultante é homeomorfo ao espaço $X = \bigsqcup_{i=1}^k P_i/\sim$ obtido pelo esquema de etiquetagem $w_1,\ldots, w_k$.
    \begin{dem}
        Note que, se $\Tilde{x} \in [x]_i \in X_i$, então o único $t \in [0,1]$ tal que $x = (1-t) \mathcal{O}(A) + t \overline{\mathcal{O}}(A)$ também satisfaz $\Tilde{x} = (1-t) \mathcal{O}(\Tilde{A}) + t \overline{\mathcal{O}}(\Tilde{A})$ para alguma aresta $\Tilde{A}$ com mesma etiqueta que $A$. Aplicando o mesmo raciocínio para $y$, concluímos que $\sim$ está bem definido.
        
        Pelo $1^o$ Teorema do Homomorfismo, o espaço quociente é unicamente determinado por sua propriedade universal. Fixemos um espaço topológico $Y$ e uma função contínua $\phi: \bigsqcup_{i=1}^k X_i \to Y$ constante nas classes de equivalência de $\approx$. Então, $\phi$ é induzida por uma função $\phi_0: \bigsqcup_{i=1}^k P_i \to Y$ constante nas classes de equivalência de $\sim$, e, desse modo, induz uma função $\overline{\phi}: \bigsqcup_{i=1}^k P_i/\sim \to Y$. Sendo $\pi: \bigsqcup_{i=1}^k P_i \to \bigsqcup_{i=1}^k X_i/\approx$ a projeção canônica, vale que $\overline{\phi}\circ \pi = \phi$, e concluímos.
    \end{dem}
\end{lemma}

Vamos definir algumas construções possíveis para alterar o esquema de etiquetagem de modo que o espaço quociente obtido seja homeomorfo (como pode ser facilmente verificado, de maneira análoga à demonstração do lema anterior). Vamos utilizar estas construções para classificar as superfícies a menos de homeomorfismo.

\begin{prop}
    Os seguintes procedimentos sobre os esquemas de etiquetagem resultam em espaços quociente homeomorfos:
    \begin{enumerate}
        \item \textbf{Recorte:} dados um esquema de etiquetagem $w = a_{i_1}^{\varepsilon_1} \ldots a_{i_n}^{\varepsilon_n}$, $1\leq j\leq n$ uma etiqueta $b$ não utilizada anteriormente e $\varepsilon = \pm 1$, substituímos $w$ por $w_1, w_2$, onde $w_1 = a_{i_1}^{\varepsilon_1} \ldots a_{i_j}^{\varepsilon_j} b^{\varepsilon}$ e $w_2 = b^{-\varepsilon} a_{i_{j+1}}^{\varepsilon_{j+1}} \ldots a_{i_n}^{\varepsilon_n}$.
        
        Geometricamente, estamos dividindo uma região poligonal em duas, adicionando uma aresta no interior da região poligonal original. Como utilizamos a mesma etiqueta nas arestas ``novas'' das regiões poligonais resultantes, o espaço quociente não se altera, pois tais arestas serão identificadas.
    
        \item \textbf{Colagem:} o processo contrário ao de recorte. Dado um esquema de etiquetagem $w_1, w_2$, onde $w_1 = a_{i_1}^{\varepsilon_1} \ldots a_{i_j}^{\varepsilon_j} b$ e $w_2 = b^{-1} a_{i_{j+1}}^{\varepsilon_{j+1}} \ldots a_{i_n}^{\varepsilon_n}$, suponha que a etiqueta $b$ só tenha uma ocorrência em $w_1$ e uma ocorrência em $w_2$. Então, podemos substituir $w_1, w_2$ por $w = a_{i_1}^{\varepsilon_1} \ldots a_{i_n}^{\varepsilon_n}$.
    
        \item \textbf{Endireitar de arestas:} dado um esquema de etiquetagem $w$, suponha que exista uma sequência $y = c_1^{\delta_1} \ldots c_k^{\delta_k}$ tal que $c_i \neq c_j$ para todos $i\neq j$, e tal que as únicas ocorrências das etiquetas $c_i$ são em uma sequência $y$ ``contida'' em $w$. Então, podemos substituir todas as ocorrências da sequência $y$ por $b^{\varepsilon}$, onde $b$ é uma etiqueta não utilizada anteriormente e $\varepsilon = \pm 1$. Uma sequência $y$ nessas condições é dita \textbf{removível}.
    
        Geometricamente, estamos substituindo uma sequência de lados da região poligonal (que sempre aparecem juntos no esquema de etiquetagem) por apenas um lado.
    
        \item \textbf{Dobradura de arestas:} o processo contrário ao de endireitar de arestas. Dado um esquema de etiquetagem $w$, uma etiqueta $b$ cujas ocorrências em $w$ sempre possuem mesma orientação $\varepsilon$ e uma sequência $y$ com etiquetas não utilizadas em $w$, substituímos todas as ocorrências de $b^{\varepsilon}$ por $y$.

        \item \textbf{Troca de etiquetas:} podemos substituir todas as ocorrências de uma etiqueta $a$ por outra etiqueta $c$ não utilizada. Disso é imediato que podemos trocar as ocorrências de quaisquer duas etiquetas dadas $a$ e $b$ (substituindo $a$ por $c$, depois $b$ por $a$ e, por fim, $c$ por $b$).
        
        \item \textbf{Troca de orientação:} podemos inverter o sinal da orientação de todas as ocorrências de uma etiqueta $b$ fixada. Isso segue de que o espaço quociente é definido por meio de transformações lineares positivas entre estes lados (que não se alteram, caso a orientação de todos estes lados seja invertida).
        
        \item \textbf{Permutação cíclica:} um esquema de etiquetagem $w$ representa o mesmo espaço quociente, caso comecemos a ordenar os pontos a partir de pontos distintos. Desse modo, podemos substituir $w = a_{i_1}^{\varepsilon_1} \ldots a_{i_n}^{\varepsilon_n}$ por\break $w' = a_{i_n}^{\varepsilon_n} a_{i_1}^{\varepsilon_1} \ldots a_{i_{n-1}}^{\varepsilon_{n-1}}$ (bem como qualquer permutação cíclica da sequência).
    
        \item \textbf{Inversão formal:} dado um esquema de etiquetagem $w$, caso usássemos a ordenação dos pontos em sentido horário, o espaço resultante seria o mesmo, a menos de uma reflexão (em especial, seriam homeomorfos). Assim, podemos substituir $w = a_{i_1}^{\varepsilon_1} \ldots a_{i_n}^{\varepsilon_n}$ por $w = a_{i_n}^{-\varepsilon_n} \ldots a_{i_1}^{-\varepsilon_1}$.

        \item \textbf{Cancelamento:} dado um esquema de etiquetagem $w$, suponha que existam duas sequências $y_0, y_1$ com comprimento maior ou igual a 2 e uma etiqueta $c$ sem ocorrências em $y_0$ e em $y_1$ tais que $w = [y_0] cc^{-1} [y_1]$. Então, podemos substituir $w$ por $w' = [y_0 y_1]$.

        \item \textbf{Adjunção:} dado um esquema de etiquetagem $w$, suponha que existam duas sequências $y_0, y_1$ com comprimento maior ou igual a 2 tal que $w = [y_0 y_1]$, e seja $c$ uma etiqueta sem ocorrências em $y_0$ e em $y_1$. Então, podemos substituir $w$ por $w' = [y_0] cc^{-1} [y_1]$.
    \end{enumerate}
\end{prop}

\begin{defi}
    Um esquema de etiquetagem é \textbf{próprio} se cada etiqueta possui exatamente 2 ocorrências. Já este é dito \textbf{irredutível} se não há ocorrência de uma sequência da forma $c c^{-1}$ ou $c^{-1} c$ para alguma etiqueta $c$.
    
    Dizemos que dois esquemas de etiquetagem próprios $w_1,\ldots,w_k$ e $\Tilde{w}_1,\ldots,\Tilde{w}_l$ são \textbf{equivalentes} se é possível obter um a partir do outro por meio das construções da proposição anterior.
\end{defi}

Como tais construções são reversíveis, isto define uma relação de equivalência entre os esquemas de etiquetagem próprios. Além disso, por conta da proposição anterior, dois esquemas de etiquetagem equivalentes definem espaços topológicos homeomorfos.

É interessante nos restringirmos a analisar esquemas de etiquetagem próprios pois, se realizamos a colagem de $k\geq 1$ arestas, o espaço quociente não é uma superfície para $k\neq 2$. Para ver isso, note que, nesse caso, qualquer ponto em tal aresta possui uma vizinhança homeomorfa à colagem de $k$ hemisférios de um disco $D^2$ (identificando o equador de todos os hemisférios), o que não é localmente euclidiano para $k\neq 2$ (no caso em que $k=1$, teríamos uma variedade com bordo).

Também vamos nos restringirmos a analisar esquemas de etiquetagem de comprimento $4$ ou maior.

\begin{defi}
    Seja $w$ um esquema de etiquetagem próprio (de uma única região poligonal). Se cada etiqueta aparece uma vez com a orientação $+1$ e uma vez com a orientação $-1$, dizemos que $w$ é do \textbf{tipo toro}. Caso contrário, $w$ é dito do \textbf{tipo projetivo}.
\end{defi}

\begin{titlemize}{Lista de consequências}
	\item \hyperref[forma-normal-thm]{Teorema de Forma Normal}
\end{titlemize}
%---------------------------------------------------------------------------------------------------------------------!Draft!-----------------------------------------------------------------------------------------------------------------
\subsection{Caso A do Teorema de Forma Normal}
\label{forma-normal-caso-a-thm}
\begin{titlemize}{Lista de dependências}
	\item \hyperref[regiao-poligonal-def]{Regiões Poligonais};\\
	\item \hyperref[construcoes-regiao-poligonal-prop]{Construções com Regiões Poligonais};\\
% quantas dependências forem necessárias.
\end{titlemize}

Seja $w_1,\ldots,w_k$ um esquema de etiquetagem próprio, e suponha que toda etiqueta aparece uma vez com a orientação $+1$ e uma vez com a orientação $-1$.

% \begin{titlemize}{Lista de consequências}
% 	\item \hyperref[forma-normal-caso-a-thm]{Caso A do Teorema de Forma Normal};\\
% 	\item \hyperref[forma-normal-caso-b-thm]{Caso B do Teorema de Forma Normal}
% \end{titlemize}
\input{conteudo/forma-normal-caso-b-thm}

% Cada novo assunto deve ser adcionado no corpo do texto, como explicado no arquivo Alg.Top-Wiki.tex.

\section{Homologia}
\label{homologia}

\begin{titlemize}{Lista de Dependências}
    \item \hyperref[homotopia]{Homotopia};\\ %assunto1 é o label onde o Assunto 1 aparece
    \item \hyperref[grupos-livres]{Grupos Livres};\\
    \item \hyperref[simplexo-def]{Simplexos};\\
\end{titlemize}

Em topologia algébrica, a homologia é a sequência de grupos de homologia associada a um espaço topológico. Esses grupos capturam de forma algébrica a ideia dos "buracos" em diferentes dimensões no espaço. Assim, a homologia é uma ferramenta fundamental para distinguir e classificar espaços topológicos.

\subsection{Complexo de cadeias}
\label{complexo-de-cadeias-def}
%\begin{titlemize}{Lista de dependências}
	%\item %\hyperref[homologia-simplicial-def]{Homologia Simplicial};\\ %'dependencia1' é o label onde o conceito Dependência 1 aparece (--à arrumar um padrão para referencias e labels--) 
% quantas dependências forem necessárias.
%\end{titlemize}
\begin{defi}[Complexo de cadeias]
	Um \textbf{Complexo de cadeias} é uma sequência $C_{-1}=0,C_0,C_1, C_2,...$ de grupos abelianos acompanhada de homomorfismos $d_n:C_n\rightarrow C_{n-1}$ para cada $n\ge 0$, tais que $d_{n}\circ d_{n+1}=0$. Denotamos esse complexo de cadeias por $C_{\bullet}$, e os homomorfismos $d_n$ são chamados de \textbf{diferenciais} de $C_\bullet$. A n-ésima \textbf{homologia} de $C_\bullet$ é definida por
    \[H_n(C_\bullet):=\frac{\text{Ker}(d_n)}{\text{Im}(d_{n+1})}\]
    Usualmente, $\text{Ker}(d_n)$ é chamado de grupo abeliano de $n$-ciclo em $C_\bullet$ e é denotado por $Z_n(C_\bullet)$. Por outro lado, $\text{Im}(d_{n+1})$ é chamada de grupo abeliano de $n$-bordos em $C_\bullet$ e é denotada por $B_n(C_\bullet)$.
\end{defi}

A homologia é a medida não numérica de quão diferentes $Z_n(C_\bullet)$ e $B_n(C_\bullet)$ são.

\begin{titlemize}{Lista de consequências}
    \item \hyperref[aplicacao-de-cadeias-def]{Aplicação de cadeias};\\
    \item \hyperref[homotopia-de-cadeias-def]{Homotopia de cadeia};\\
    \item \hyperref[homomorfismo-induzido-de-cadeias-prop]{Homomorfismo induzido de cadeias};\\
    \item \hyperref[equivalencia-de-homotopia-de-cadeias-def]{Equivalência de homotopia de cadeias};\\
    \item \hyperlink{homologia-simplicial-def}{Homologia simplicial};\\
    \item \hyperref[homologia-singular-def]{Homologia singular};\\
    \item \hyperref[homomorfismo-de-homologias-singulares-induzido-prop]{Homomorfismo de homologias singulares induzido}.
\end{titlemize}

\subsection{Aplicação de cadeias}
\label{aplicacao-de-cadeias-def}
\begin{titlemize}{Lista de dependências}
	\item \hyperref[complexo-de-cadeias-def]{Complexo de cadeias}.\\ %'dependencia1' é o label onde o conceito Dependência 1 aparece (--à arrumar um padrão para referencias e labels--) 
% quantas dependências forem necessárias.
\end{titlemize}

\begin{defi}
    Uma \textbf{aplicação de cadeias} ou homomorfismo de cadeias $f_\bullet:C_\bullet\rightarrow D_\bullet$ é uma sequência de homomorfismos $f_n:C_n\rightarrow D_n$ tal que $f_n\circ d_{n+1}^C=d_{n+1}^D\circ f_{n+1}$.
\end{defi}

\begin{titlemize}{Lista de consequências}
    \item \hyperref[homotopia-de-cadeias-def]{Homotopia de cadeia;}\\
    \item \hyperref[homomorfismo-induzido-de-cadeias-prop]{Homomorfismo induzido de cadeias};\\
    \item \hyperref[equivalencia-de-homotopia-de-cadeias-def]{Equivalência de homotopia de cadeias};\\
    \item \hyperref[homomorfismo-de-homologias-singulares-induzido-prop]{Homomorfismo de homologias singulares induzido}.
\end{titlemize}

\subsection{Homotopia de cadeias}
\label{homotopia-de-cadeias-def}
\begin{titlemize}{Lista de dependências}
	\item \hyperref[complexo-de-cadeias-def]{Complexo de cadeias};\\ %'dependencia1' é o label onde o conceito Dependência 1 aparece (--à arrumar um padrão para referencias e labels--) 
% quantas dependências forem necessárias.
    \item \hyperref[aplicacao-de-cadeias-def]{Aplicação de cadeias}.
\end{titlemize}

\begin{defi}
    Sejam $f_\bullet, g_\bullet:C_\bullet\rightarrow D_\bullet$ duas aplicações de cadeias. Uma \textbf{homotopia de cadeias} $h:f_\bullet\Rightarrow g_\bullet$ é uma sequência de homomorfismos $h_n:C_n\rightarrow D_{n+1}$, indexada por $n\ge -1$, tal que 
    \[g_n-f_n=d_{n+1}^D\circ h_n+ h_{n-1}\circ d_n^C:C_n\rightarrow D_n.\]
    Nessa caso, diremos que $f_\bullet$ e $g_\bullet$ são \textbf{homotópica de cadeias}, e denotaremos por $f_\bullet\simeq g_\bullet$.
\end{defi}

\begin{titlemize}{Lista de consequências}
    \item \hyperref[homomorfismo-induzido-de-cadeias-prop]{Homomorfismo induzido de cadeias};\\
    \item \hyperref[equivalencia-de-homotopia-de-cadeias-def]{Equivalência de homotopia de cadeias};\\
    \item \hyperref[homomorfismo-de-homologias-singulares-induzido-prop]{Homomorfismo de homologias singulares induzido}.\\
\end{titlemize}

\subsection{Homomorfismo induzido de cadeias} %afirmação aqui significa teorema/proposição/colorário/lema
\label{homomorfismo-induzido-de-cadeias-prop}
\begin{titlemize}{Lista de dependências}
	\item \hyperref[complexo-de-cadeias-def]{Complexo de cadeias};\\ 
    \item \hyperref[aplicacao-de-cadeias-def]{Aplicação de cadeias};\\
    \item \hyperref[homotopia-de-cadeias-def]{Homotopia de cadeia}.
\end{titlemize}
Assim como uma função contínua entre espaços topológicos induz um homomorfismo entre os grupos fundamentais associados, uma aplicação de cadeias induz um homomorfismo entre os grupos de homologias correspondentes.
\begin{lemma}%af(afirmação)/prop(proposição)/corol(corolário)/lemma(lema)/outros ambientes devem ser definidos no preambulo de Alg.Top-Wiki.tex 
	Uma aplicação de cadeias $f_\bullet: C_\bullet\rightarrow D_\bullet$ induz um homomorfismo 
    \begin{align*}
        f_*:H_n(C_\bullet)&\longrightarrow H_n(D_\bullet)\\
        [x]&\longmapsto [f_n(x)].
    \end{align*}
    Para todo $n\ge 0$. Além disso, se $f_\bullet$ e $g_\bullet$ são homotópicas de cadeias, então $f_*=g_*$.
\end{lemma}

\begin{proof}
    Vamos checar que $f_*$ é bem definido.
    \begin{itemize}
        \item Primeiramente, mostramos que $[f_n(x)]$ está dentro do codomínio. Seja $[x]\in H_n(C_\bullet)=\frac{Z_n(C_\bullet)}{B_n(C_\bullet)}$ representado por um $x\in Z_n(C_\bullet)$. Consideramos dois casos:\\
        Caso 1: $n=0$. Nesse caso, temos $Z_0(C_\bullet)=C_0$ e $Z_0(D_\bullet)=D_0$. Como $f_0(Z_0(C_\bullet))\subseteq Z_0(D_\bullet)$, temos que $f_0(x)$ é um ciclo, ou seja, $f_0(x)$ representa uma classe de homologia em $H_0(D_\bullet)$.\\
        Caso 2: $n\ge 1$. Como $x$ é um ciclo em $C_n$, temos:
        \[d_n^D\circ f_n(x)=f_{n-1}\circ d_n^C(x)=f_{n-1}(0)=0,\]
        ou seja, o elemento $f_n(x)\in D_n$ é um ciclo. Portanto $f_n(x)$ representa uma classe de homologia em $H_n(D_\bullet)$.
        \item Agora, suponha que $[x]=[y]$, ou seja $x-y\in B_n(C_\bullet)$. Logo, existe um $z\in C_{n+1}$ tal que $x-y=d_{n+1}^D(z)$. Então, temos: 
        \[f_n(x)-f_n(y)=f_n(d_{n+1}^C(z))=d_{n+1}^D (f_{n+1}(z))\]
        é um bordo. Portanto $[f_n(x)]=[f_n(y)]\in H_n(D_\bullet).$
    \end{itemize}
    Como $f_n$ são homomorfismos, o mapa $f_*$ também é um homomorfismo. 

    Agora, suponha que $h_\bullet: f_\bullet \Rightarrow g_\bullet$ é uma homotopia de cadeias. Seja $x\in Z_n(C_\bullet)$. Então, pela definição de homotopia de cadeias, temos 
    \[g_n(x)-f_n(x)=d_{n+1}^D(h_n(x))+h_{n-1}(d_n^C(x)).\]
    Mas $x$ é um ciclo, logo $g_n(x)-f_n(x)=d_{n+1}^D(h_n(x))$. Isso mostra que $g_n(x)-f_n(x)$ é um bordo, portanto, $[g_n(x)]=[f_n(x)]$.
\end{proof}

De acordo com a construção do homomorfismo induzido, é fácil observar as seguintes propriedades.

\begin{corol}
    \begin{enumerate}
        \item Se $f_\bullet: C_\bullet\rightarrow D_\bullet$ e $g_\bullet:D_\bullet\rightarrow E_\bullet$ são aplicações de cadeias, então 
        \[(g_\bullet\circ f_\bullet)_*=g_*\circ f_*.\]
        \item $(id_{C_\bullet})_*=id_{H_n(C_\bullet)}$.
    \end{enumerate}

\end{corol}

\begin{titlemize}{Lista de consequências}
    \item \hyperref[equivalencia-de-homotopia-de-cadeias-def]{Equivalência de homotopia de cadeias};\\
    \item \hyperref[homomorfismo-de-homologias-singulares-induzido-prop]{Homomorfismo de homologias singulares induzido}.
	%\item \hyperref[consequencia1]{Consequência 1};\\ %'consequencia1' é o label onde o conceito Consequência 1 aparece
\end{titlemize}

\subsection{Equivalência de homotopia de cadeias} %afirmação aqui significa teorema/proposição/colorário/lema
\label{equivalencia-de-homotopia-de-cadeias-def}
\begin{titlemize}{Lista de dependências}
	\item \hyperref[complexo-de-cadeias-def]{Complexo de cadeias};\\ 
    \item \hyperref[aplicacao-de-cadeias-def]{Aplicação de cadeias};\\
    \item \hyperref[homotopia-de-cadeias-def]{Homotopia de cadeia};\\
    \item \hyperref[homomorfismo-induzido-de-cadeias-prop]{Homomorfismo induzido de cadeias}.
\end{titlemize}

\begin{defi}
    Uma aplicação de cadeias $f_\bullet:C\bullet\rightarrow D_\bullet$ é uma \textbf{equivalência de homotopia de cadeias} se existem uma aplicação de cadeias $g_\bullet:D_\bullet\rightarrow C_\bullet$ e homotopias de cadeias $f_\bullet\circ g_\bullet\simeq id_{D_\bullet}$ e $g_\bullet\circ f_\bullet \simeq id_{C_\bullet}$.
\end{defi}

Uma consequência imediata de \ref{homomorfismo-induzido-de-cadeias-prop} é: 

\begin{lemma}
    Se $f_\bullet:C_\bullet\rightarrow D_\bullet$ é uma equivalência de homotopia de cadeia, então $f_*:H_n(C_\bullet)\rightarrow H_n(D_\bullet)$ é um isomorfismo para todo $n\ge 0$.
\end{lemma}

\begin{proof}
    Pelo Lema \ref{homomorfismo-induzido-de-cadeias-prop}, temos
    \[f_*\circ g_*=(f_\bullet\circ g_\bullet)_*=(id_{D_\bullet})_*=id_{H_n(D_\bullet)},\]
    e vice-versa. 
\end{proof}

\begin{titlemize}{Lista de consequências}
    \item \hyperref[homomorfismo-de-homologias-singulares-induzido-prop]{Homomorfismo de homologias singulares induzido}.\\
	%\item \hyperref[]{}
\end{titlemize}

\subsection{Sequência Exata} %afirmação aqui significa teorema/proposição/colorário/lema
\label{sequencia-exata-def}
\begin{titlemize}{Lista de dependências}
	\item \hyperref[complexo-de-cadeias-def]{Complexo de cadeias};\\ 
    \item \hyperref[aplicacao-de-cadeias-def]{Aplicação de cadeias}.
\end{titlemize}

\begin{defi}
    Seja 
    \[...\rightarrow A_0\xrightarrow{f_0}A_1\xrightarrow{f_1} A_2\rightarrow ...\]
    uma sequência de homomorfismos de grupos. Diremos que essa sequência é \textbf{exata} se para cada $i$, $\text{Im}(f_i)=\text{Ker}(f_{i+1})$
\end{defi}

\begin{ex}
    A sequência de homomorfismos de grupos da forma 
    \[0\rightarrow A\xrightarrow{f}B\xrightarrow{g}C\rightarrow0\]
    é exata se, e somente se, as seguintes condições são satisfeitas
    \begin{enumerate}
        \item $f$ é um injetor (isto é, $\text{Ker}(f)=\{0\}$),
        \item $g$ é um sobrejetor (isto é, $\text{Im}(g)=\text{Ker}(C\rightarrow 0)$=$C$),
        \item $\text{Ker}(g)=\text{Im}(f)$. 
    \end{enumerate}
    Essa sequência é chamada de \textbf{sequência exata curta}.
\end{ex}

\begin{defi}
    Sejam $\mathcal{A},\mathcal{B},\mathcal{C}$ complexos de cadeias, e sejam $f= (f_n):\mathcal{A}\rightarrow \mathcal{B}$ e $g=(g_n):\mathcal{B}\rightarrow \mathcal{C}$ Aplicações de cadeias. A sequência 
    \[0\rightarrow \mathcal{A}\xrightarrow{f} \mathcal{B}\xrightarrow{g} \mathcal{C}\rightarrow 0\]
    é \textbf{exata} se, e somente se, para cada $n$, a sequência
    \[0\rightarrow A_n\xrightarrow{f_n} B_n\xrightarrow{g_n}C_n\rightarrow 0\]
    é exata.
\end{defi}

%\begin{titlemize}{Lista de consequências}
    %\item %\hyperref[homomorfismo-de-homologias-singulares-induzido-prop]{Homomorfismo de homologias singulares induzido}.\\
	%\item \hyperref[]{}
%\end{titlemize}

\subsection{Homomorfismo Conectante} %afirmação aqui significa teorema/proposição/colorário/lema
\label{homomorfismo-conectante-def}
\begin{titlemize}{Lista de dependências}
	\item \hyperref[complexo-de-cadeias-def]{Complexo de cadeias};\\ 
    \item \hyperref[aplicacao-de-cadeias-def]{Aplicação de cadeias};\\
    \item \hyperref[sequencia-exata-def]{Sequência exata}.
\end{titlemize}

    Sejam $\mathcal{A},\mathcal{B},\mathcal{C}$ complexos de cadeias, e sejam $\phi= (\phi_n):\mathcal{A}\rightarrow \mathcal{B}$ e $\psi=(\psi_n):\mathcal{B}\rightarrow \mathcal{C}$ Aplicações de cadeias. Supõe que a sequência 
    \[0\rightarrow \mathcal{A}\xrightarrow{\phi} \mathcal{B}\xrightarrow{\psi} \mathcal{C}\rightarrow 0\]
    é exata. 

    Construiremos uma função $\tilde{\delta}:Z_n(\mathcal{C})\rightarrow Z_{n-1}(\mathcal{A})$. Seja $z\in Z_n(\mathcal{C})$. Como $\psi_n$ é sobrejetora, existe $b\in B_n$ tal que $\psi_n(b)=z$. Como $\psi$ é uma aplicação de cadeias, temos
    \[\psi_{n-1}(\partial b)=\partial\psi_n(b)=\partial z=0.\]
    Como a sequência é exata, existe $a\in A_{n-2}$ tal que $\phi_{n-1}(a)=\partial b$. Como $\phi$ é uma aplicação de cadeias, temos
    \[\phi_{n-2}(\partial a)=\partial \phi_{n-1}(a)=\partial\partial b=0.\]
    Como $\phi_{n-2}$ é injetor, $\partial a=0$, e portanto $a\in Z_{n-1}(\mathcal{A})$. Dessa forma, podemos definir uma função 
    \begin{align*}
        \tilde{\delta_n}:Z_n(\mathcal{C})&\longrightarrow Z_{n-1}(\mathcal{A})\\
        z&\longmapsto a.
    \end{align*}

    Agora, mostramos que $\tilde{\delta_n}$ induz um homomorfismo $\delta_n:H_n(\mathcal{C})\rightarrow H_{n-1}(\mathcal{A})$, ou seja, provamos que: se $z_1,z_2$ em $Z_n(\mathcal{C})$ são ciclos tais que $z_1-z_2=\partial c$ para algum $c\in C_{n+1}$, então $\tilde{\delta_n}(z_1-z_2)= \partial a$, para algum $a\in A_n$. 

    Denotamos $a_1:=\tilde{\delta_n} (z_1)$ e $a_2:=\tilde{\delta_n}(z_2)$. Por construção $a_1$ e $a_2$ são tais que $\phi_{n-1}(a_1)=\partial b_1$ e $\phi_{n-1}(a_2)=\partial b_2$, onde $b_1$ e $b_2$ são elementos de $B_n$ que verificam $\psi(b_1)=z_1$ e $\psi(b_2)=z_2$. Como $\psi_{n+1}$ é sobrejetor, existe $b\in B_{n+1}$ tal que $\psi_{n+1}(b)=c$. Então, 
    \[\psi_n(\partial b)=\partial \psi_{n+1}(b)=\partial c=z_1-z_2.\]
    Logo, $b_1-b_2-\partial b\in \text{Ker}(\psi_n)=\text{Im}(\phi_n)$. Por conseguinte, existe $a\in A_n$ tal que $\phi_n(a)=b_1-b_2-\partial b$. Pela definição de aplicação de cadeias 
    \begin{align*}
        \phi_{n-1}(\partial a)&=\partial\phi_n(a)=\partial(b_1-b_2-\partial b)=\partial b_1-\partial b_2-\partial\partial b\\
        &=\phi_{n-1}(a_1)-\phi_{n-1}(a_2)=\phi_{n-1}(a_1-a_2).
    \end{align*}
    Como $\phi_{n-1}$ é injetor, $a_1-a_2=\partial a$. como queríamos.

    Portanto, a função $\tilde{\delta_n}:Z_n(\mathcal{C})\rightarrow Z_{n-1}(\mathcal{A})$ induz, por passagem ao quociente, um homomorfismos $\delta_n:H_n(\mathcal{C})\rightarrow H_{n-1}(\mathcal{A})$ para cada $n\ge 0$, dado por 
    \[\delta_n (z+B_n(\mathcal{C}))=\tilde{\delta_n}(z)+B_{n-1}(\mathcal{A}).\]
    \begin{defi}
        O homomorfismo $\delta_n:H_n(\mathcal{C})\rightarrow H_{n-1}(\mathcal{A})$ é chamado \textbf{homomorfismo conectante}. O índice $n$ será omitido quando não houve risco de confusão.
    \end{defi}
\begin{titlemize}{Lista de consequências}
    \item \hyperref[sequencia-exata-longa-induzida-prop]{Sequência exata longa induzida}.\\
	%\item \hyperref[]{}
\end{titlemize}

\subsection{Sequência exata longa induzida} %afirmação aqui significa teorema/proposição/colorário/lema
\label{sequencia-exata-longa-induzida-prop}
\begin{titlemize}{Lista de dependências}
	\item \hyperref[complexo-de-cadeias-def]{Complexo de cadeias};\\ 
    \item \hyperref[aplicacao-de-cadeias-def]{Aplicação de cadeias};\\
    \item \hyperref[homomorfismo-induzido-de-cadeias-prop]{Homomorfismo induzido de cadeias};\\
    \item \hyperref[sequencia-exata-def]{Sequência exata};\\
    \item \hyperref[homomorfismo-conectante-def]{Homomorfismo conectante}.
\end{titlemize}

\begin{thm}
    Seja 
    \[0\rightarrow \mathcal{A}\xrightarrow{\phi} \mathcal{B}\xrightarrow{\psi} \mathcal{C}\rightarrow 0\]
    uma sequência exata de complexos. Então, a sequência 
    \[...\rightarrow H_n(\mathcal{A})\xrightarrow{\phi_*}H_n(\mathcal{B})\xrightarrow{\psi_*} H_n (\mathcal{C})\xrightarrow{\delta}(H_{n-1}(\mathcal{A}))\rightarrow...\]
    é exata.
\end{thm}

\begin{dem}
    Verificamos a exatidão em 3 passos:

    Passo 1: $\text{Im}(\phi_*)=\text{Ker}(\psi_*)$

    Por hipótese, $\psi\circ \phi=0$. Logo $\psi_*\circ \phi_*=0$, ou seja $\text{Im}(\phi_*)\subseteq \text{Ker}(\psi_*).$ Para provar a outra inclusão, seja $b+B_n(\mathcal{B})\in \text{Ker} (\psi_*)\subseteq H_n (\mathcal{B})$. Então, $\psi(b)\in B_n(\mathcal{C})$, ou seja, existe $c\in C_{n+1}$ tal que $\partial c=\psi(b)$. Como $\psi$ é sobrejetora, existe $b^+\in B_{n+1}$ tal que $\psi(b^+)=c$. Então, $b-\partial b^+\in B_n$ e temos: 
    \[\psi(b-\partial b^+)=\psi(b)-\psi(\partial b^+)=\partial c-\partial \psi(b^+)=\partial(c-\psi(b^+))=0.\]
    Isso mostra que $b-\partial b^+ \in \text{Ker}(\psi)=\text{Im}(\phi)$, ou seja, existe $a\in A_n$ tal que $\phi(a)=b-\partial b^+$, e temos: 
    \[\phi(\partial a)=\partial\phi (a)=\partial (b-\partial b^+)=\partial b=0.\]
    Como $\phi$ é injetora, $\partial a=0$, ou seja, $a\in Z_n(\mathcal{C})$. Além disso, 
    \[\phi_*(a+B_n(\mathcal{A}))=\phi(a)+B_n(\mathcal{B})=b-\partial b^++B_n(\mathcal{B})=b+B_n(\mathcal{B}).\]
    Isso prova a inclusão $\text{Ker}(\psi_*)\subseteq \text{Im}(\phi_*)$.

    Passo 2: $\text{Im}(\psi_*)=\text{Ker}(\delta).$

    Seja $\overline{z}=z+B_n(\mathcal{C})\in \text{Im}(\psi_*)$, onde $z=\psi(b)$ para $b\in Z_n(\mathcal{B})$. Pela construção do homomorfismo conectante, $\delta(\overline{z})=a+B_{n-1}(\mathcal{A})$, em que $a\in A_{n-1}$ é tal que $\phi(a)=\partial b$. Como $\partial b=0$ e $\phi$ é injetora, temos que $a=0$, portanto, $\overline{z}\in \text{Ker}(\delta)$. Isso prova que $\text{Im}(\psi_*)\subseteq \text{Ker}(\delta)$.
    
    Por outro lado, seja $\overline{z}=z+B_n(\mathcal{C})\in \text{Ker}(\delta)$. Pela construção de $\delta$, definimos $a:=\tilde{\delta_n}(z)\in A_{n-1}$ e seja $b\in B_n$ tal que $\psi(b)=z$ e, ainda, $\phi(a)=\partial b$. Como $\overline{z}\in \text{Ker}(\delta)$, temos $a\in B_{n-1}(\mathcal{A})$, ou seja, existe $a^+\in A_n$ tal que $a=\partial a^+$. Além disso, temos
    \[\partial(b-\phi(a^+))=\partial b-\partial \phi(a^+)=\partial b- \phi(\partial a^+)=\partial b-\phi(a)=\partial b-\partial b=0.\]
    Isso mostra que $b-\phi(a^+)\in Z_n(\mathcal{B})$ e, portanto, está bem definida a classe $(b-\phi(a^+))+B_n(\mathcal{B})\in H_n(\mathcal{B})$. Além disso, 
    \[\psi(b-\phi(a^+))=\psi (b)-\psi\circ\phi(a^+)=\psi(b)=z.\]
    Portanto, $\psi_*((b-\phi(a^+))+B_n(\mathcal{B}))=\overline{z}$. Isso prova que $\text{Ker}(\delta)\subseteq \text{Im}(\psi_*)$.

    Passo 3: $\text{Im}(\delta)=\text{Ker}(\phi_*)$.

    Por construção, se $a+B_{n-1}(\mathcal{A})\in \text{Im}(\delta)$, então $\phi(a)=\partial b$ para algum $b\in B_n$ e, assim, $\phi_*(a+B_{n-1}(\mathcal{A}))=B_n(\mathcal{B})$, ou seja $a+B_{n-1}(\mathcal{A})\in \text{Ker}(\phi_*)$. Isso prova que $\text{Im}(\delta)\subseteq \text{Ker}(\phi_*)$. 

    Por outro lado, seja $a+B_{n-1}(\mathcal{A})\in \text{Ker}(\phi_*)$. Então, $\phi(a)\in B_{n-1}(\mathcal{B})$, ou seja, existe $b\in B_n$ tal que $\phi(a)=\partial b$. Para o elemento $\psi(b)\in C_n$, temos:
    \[\partial \psi(b)=\psi(\partial b)=\psi\circ\phi(a)=0.\]
    Logo, $\psi(b)\in Z_n(\mathcal{C})$ e, pela construção de $\delta$, temos $\delta
    (\psi(b)+B_n(\mathcal{C}))=a+B_{n-1}(\mathcal{A}).$ isso mostra que $\text{Ker}(\phi_*)\subseteq \text{Im}(\delta)$.
\end{dem}
    
\begin{titlemize}{Lista de consequências}
    \item \hyperref[sequencia-de-mayer-vietoris-prop]{Sequência de Mayer-Vietoris}.
	%\item \hyperref[]{}
\end{titlemize}

\subsection{Homologia Simplicial}
\label{homologia-simplicial-def}
\begin{titlemize}{Lista de dependências}
    \item \hyperref[complexo-simplicial-def]{Complexos simpliciais};\\
	\item \hyperref[complexo-de-cadeias-def]{Complexo de cadeias};\\ 
    \item \hyperref[aplicacao-de-cadeias-def]{Aplicação de cadeias};\\
    \item \hyperref[homotopia-de-cadeias-def]{Homotopia de cadeia}.
% quantas dependências forem necessárias.
\end{titlemize}
\begin{defi}
	Seja $K$ um complexo simplicial, e seja $O_n (K)$ um grupo abeliano livre com base dada por símbolos 
    \[\{[v_0,...,v_n]:v_0,v_1,...,v_n \text{ geram um simplexo em }K\}.\]
    Aqui, $v_i$ são considerado ordenado, e eles podem gerar um simplexo de dimensão menor que $n$ (i.e. a lista pode repetir).

    Seja $T_n(K)\le O_n(K)$ um subgrupo gerado por seguintes elementos 
    \begin{itemize}
        \item a sequência $[v_0,...,v_n]$ tem vertices repetidos,
        \item $[v_0,v_1,...,v_n]-sign(\sigma)\cdot[v_{\sigma(0)},v_{\sigma(1)},...,v_{\sigma(n)}]$, onde $\sigma$ é uma permutação em $\{0,1,...,n\}$. 
    \end{itemize}
    Definoms $C_n(K):=O_n(K)/T_n(K)$ como grupo quociente.
\end{defi}

\begin{defi}
    O \textbf{operador bordo} é um homomorfismo de grupo dado por 
    \begin{align*}
        d_n:C_n(K)&\longrightarrow C_{n-1}(K)\\
        [v_0,v_1,...,v_n]&\longmapsto \sum_{i=0}^n (-1)^i \cdot[v_0,v_1,...,\widehat{v_i},...,v_n],
    \end{align*}
    onde $[v_0,v_1,...,\widehat{v_i},...,v_n]$ denota a sequência obtida pela remoção de $v_i$.
\end{defi}

\begin{lemma}
    O homomorfismo $d_{n-1}\circ d_n:C_n(K)\rightarrow C_{n-2}(K)$ é nulo.
\end{lemma}

\begin{dem}
    Seja $[v_0,...,v_n]\in C_n(K)$, então 
    \begin{align*}
        d_{n-1}\circ d_n ( [v_0,...,v_n])&=d_{n-1}\Bigl(\sum_{i=0}^n (-1)^i \cdot[v_0,v_1,...,\widehat{v_i},...,v_n] \Bigr) \\
        &=\sum_{i=0}^n (-1)^i  \Bigl( \sum_{k=0}^{i-1}(-1)^k[v_0,...,\widehat{v_k},...,\widehat{v_i},...,v_n])\\
        &+\sum_{k=i}^{n-1} (-1)^k[v_0,...,\widehat{v_i},...,\widehat{v_{k+1}},...,v_n]  \Bigr).
    \end{align*}
    O coeficiente de $[v_0,...,\widehat{v_a},...,\widehat{v_b},...,v_n]$ é $(-1)^a(-1)^b$ de $k=a$ e $i=b$ mais $(-1)^a(-1)^{b-1}$ de $i=a$ e $k+1=b$. Assim, cada termo se cancela, o que implica que $d_{n-1}\circ d_n([v_0,...,v_n])=0$. Como $C_n(K)$ é gerado pelos simplexos $[v_0,...,v_n]$, concluímos que $d_{n-1}\circ d_n=0$.
\end{dem}

Como a consequência, esse lema garante que $\text{Im}(d_n)\subseteq \text{Ker}(d_{n-1})$. Ou seja, a sequência 
\[...\rightarrow C_{n+1}(K)\xrightarrow{d_{n+1}}C_n(K)\xrightarrow{d_n} C_{n-1}(K)\rightarrow...\rightarrow 0\]
é um complexo de cadeias.

\begin{defi}
    O n-ésima \textbf{grupo de homologia simplicial} de um complexo simplicial $K$ é 
    \[H_n(K):=\frac{\text{Ker}(d_n)}{\text{Im}(d_{n+1})}.\]
\end{defi}


%\begin{titlemize}{Lista de consequências}
	%\item %\hyperref[consequencia1]{Consequência 1};\\ %'consequencia1' é o label onde o conceito Consequência 1 aparece
	%\item \hyperref[]{}
%\end{titlemize}

\subsection{Homologia singular} %afirmação aqui significa teorema/proposição/colorário/lema
\label{homologia-singular-def}
\begin{titlemize}{Lista de dependências}
    \item \hyperref[simplexo-def]{Simplexos}
	\item \hyperref[complexo-de-cadeias-def]{Complexo de cadeias};\\ 
    \item \hyperref[aplicacao-de-cadeias-def]{Aplicação de cadeias};\\
    \item \hyperref[homotopia-de-cadeias-def]{Homotopia de cadeia}.\\
\end{titlemize}

\begin{defi}
    Seja $X$ um espaço topológico. Um p-\textbf{simplexo singular} em $X$ é uma função contínua 
    \[\phi:\Delta^p\longrightarrow X.\]
\end{defi}

\begin{defi}
    Se $\phi$ é um $p$-simplexo singular em um espaço topológico $X$, e $i$ é um inteiro tal que $0\le i\le p$, definimos $\partial_i (\phi)$, um (p-1)-simplexo singular em $X$, por 
    \[\partial_i \phi(t_0,...,t_{p-1})=\phi(t_0,...,t_{i-1},0,t_{i+1},...,t_{p-1}).\]
    Ou seja, $\partial_i \phi=\phi|_{[v_0,...,\widehat{v_i},...,v_{p}]}$ é a $i$-ésima face de $\phi$, obtida pela substituição do parâmetro $t_i$ por zero, onde $[v_0,...,v_p]=\Delta^p$
\end{defi}

\begin{defi}
    Seja $X$ um espaço topológico, definimos $S_n(X)$ como grupo abeliano livre cujo base é o conjunto de todos $n$-simplexos singulares de $X$. Um elemento de $S_n(X)$ é dito $n$-\textbf{cadeia singular} de $X$ e tem a forma 
    \[\sum_\phi n_\phi \phi\]
    onde $n_\phi$ é um inteiro igual a zero para todos, exceto um número finito de $\phi$.
\end{defi}

Podemos estender o operador de $i$-ésima face para um homomorfismo de $S_n(X)$ em $S_{n-1} (X)$. 

\begin{defi}
    Seja $X$ um espaço topológico, definimos o operador $\partial_i$ como
    \begin{align*}
        \partial_i: S_n(X)&\longrightarrow S_{n-1}(X)\\
        \sum_\phi n_\phi \phi&\longmapsto \sum_\phi n_\phi \partial_i\phi.
    \end{align*}
    O \textbf{operador bordo} é então um homomorfismo definido por
    \begin{align*}
        \partial_{(n)}=\sum_{i=0}^n (-1)^i \partial_i:S_n(X)\longrightarrow S_{n-1}(X).
    \end{align*}
    Para simplificar a notação, omitiremos o índice do operador $\partial_{(n)}$.
\end{defi}

\begin{lemma}
    O homomorfismo $\partial\circ \partial:S_n(X)\rightarrow S_{n-2}(X)$ é nulo.
\end{lemma}

\begin{dem}
    Seja $\phi\in S_n(X)$, então 
    \begin{align*}
        \partial\circ \partial(\phi)&=\partial\Bigl(\sum_{i=0}^n (-1)^i \partial_i\phi \Bigr) \\
        &=\sum_{i=0}^n (-1)^i  \Bigl( \sum_{k=0}^{i-1}(-1)^k \partial_k\circ\partial_i \phi)+\sum_{k=i}^{n-1} (-1)^k\partial_{k}\circ\partial_i \phi  \Bigr).
    \end{align*}
    Note que $\partial_k\circ\partial_i\phi=\partial_{i-1}\circ\partial_k \phi$ se $k<i$. Logo, o coeficiente de $\partial_a\circ\partial_b \phi$ é $(-1)^a(-1)^b$ de $k=a$ e $i=b$ mais $(-1)^a(-1)^{b-1}$ de $i=a$ e $k=b-1$. Assim, cada termo se cancela, o que implica que $\partial\circ\partial\phi=0$. Como $S_n(K)$ é gerado pelos n-simplexos singulares, concluímos que $\partial\circ\partial=0$.
\end{dem}

Como a consequência, esse lema garante que $\text{Im}(\partial_{(n+1)})\subseteq \text{Ker}(\partial_{(n)})$. Ou seja, a sequência 
\[...\rightarrow S_{n+1}(X)\xrightarrow{\partial}S_n(X)\xrightarrow{\partial} S_{n-1}(X)\rightarrow...\rightarrow 0\]
é um complexo de cadeias. Denotamos esse complexo por $S(X)_*$.

Assim como no complexo de cadeias, denotamos $\text{Im}(\partial_{(n+1)})$ por $B_n(X)$ e $\text{Ker}(\partial_{(n)})$ por $Z_n(X)$.

\begin{defi}
    O n-ésima \textbf{grupo de homologia singular} de um espaço topológico $X$ é 
    \[H_n(X):=\frac{Z_n(X)}{B_n(X)}.\]
\end{defi}

\begin{titlemize}{Lista de consequências}
    \item \hyperref[homomorfismo-de-homologias-singulares-induzido-prop]{Homomorfismo de homologias singulares induzido}.\\ %'consequencia1' é o label onde o conceito Consequência 1 aparece
	%\item \hyperref[]{}
\end{titlemize}

\subsection{Homomorfismo de homologias singulares induzido} %afirmação aqui significa teorema/proposição/colorário/lema
\label{homomorfismo-de-homologias-singulares-induzido-prop}
\begin{titlemize}{Lista de dependências}
	\item \hyperref[complexo-de-cadeias-def]{Complexo de cadeias};\\ 
    \item \hyperref[aplicacao-de-cadeias-def]{Aplicação de cadeias};\\
    \item \hyperref[homotopia-de-cadeias-def]{Homotopia de cadeia};\\
    \item \hyperref[homomorfismo-induzido-de-cadeias-prop]{Homomorfismo induzido de cadeias};\\
    \item \hyperref[equivalencia-de-homotopia-de-cadeias-def]{Equivalência de homotopia de cadeias};\\
    \item \hyperref[homologia-singular-def]{Homologia singular}.
\end{titlemize}

\begin{lemma}
    Uma função contínua $f:X\rightarrow Y$ entre espaços topológicos induz um homomorfismo de cadeia 
    \begin{align*}
    f_n:S_n(X)&\longrightarrow S_n(Y)\\
    \sum_{\phi}n_\phi\phi&\longmapsto \sum_\phi n_\phi (f\circ \phi).
    \end{align*}
\end{lemma}

\begin{dem}
    É fácil ver que $f_n$ é um homomorfismo de grupo, basta mostrar que $f_{n-1}\circ\partial=\partial\circ f_n$. Seja $\phi\in S_n(X)$. Como 
    \begin{align*}
        f_{n-1}\partial (\phi)=f_{n}(\sum_{i=0}^n (-1)^i \partial_i \phi)=\sum_{i=0}^n(-1)^i f\circ\partial_i\phi=\sum_{i=0}^n (-1)^i \partial_i(f\circ\phi)=\partial\circ f_n (\phi),
    \end{align*}
    podemos concluir que $f_\bullet:=(f_n)_{n\ge 0}$ é um homomorfismo de cadeias.
\end{dem}

Por lemma \ref{homomorfismo-induzido-de-cadeias-prop}, temos 

\begin{corol}
    Uma função contínua $f:X\rightarrow Y$ entre espaços topológicos induz um homomorfismo 
    \begin{align*}
        f_*: H_n(X)&\longrightarrow H_n(Y)\\
        [\sum_\phi n_\phi \phi]&\longmapsto [\sum_\phi n_\phi (f\circ \phi)]
    \end{align*} 
    entre homologias singulares.

    Além disso, temos 
    \begin{enumerate}
        \item $(f\circ g)_*=f_*\circ g_*$,
        \item $(id_X)_*=id_{H_n(X)}$ para todo $n\ge 0$.
    \end{enumerate}
\end{corol}
Isso mostra que a homologia singular é um invariante topológico.
\begin{corol}
    Se $f:X\rightarrow Y$ é um homeomorfismo, então $f_*:H_n(X)\rightarrow H_n(Y)$ é um isomorfismo para todo $n\ge 0$. 
\end{corol}

\begin{proof}
    Seja $g:Y\rightarrow X$ a função inversa de $f$, pelo Corolário anterior, temos 
    \[f_*\circ g_*=(f\circ g)_*=(id_Y)_*=id_{H_n(Y)},\]
    e vice-versa.
\end{proof}

\begin{thm}
    Sejam $f,g:X\rightarrow Y$ funções contínuas entre espaços topológicos. Se $F:X\times I\rightarrow Y$ é uma homotopia de $f$ em $g$. Então, $f$ e $g$ induzem um mesmo homomorfismo $f_*=g_*:H_n(X)\rightarrow H_n(Y).$
\end{thm}

\begin{dem}
    A demonstração é baseada no Algebraic Topology do Allen Hatcher; o leitor pode encontrar uma interpretação geométrica dessa prova no livro.

    O ponto crucial é um procedimento para subdividir $\Delta^n\times I$ em simplexos. Em $\Delta^n\times I$, seja $\Delta^n\times \{0\}=[v_0,...,v_n]$ e $\Delta^n\times\{1\}=[w_0,...,w_n]$, onde $v_i$ e $w_i$ possuem a mesma imagem sob a projeção $\Delta^n\times I\rightarrow \Delta^n$. Podemos passar de $[v_0,...,v_n]$ para $[w_0,...,w_n]$ interpolando uma sequência de $n$-simplexos, cada um obtido do anterior movendo um vértice $v_i$ até $w_i$, começando com $v_n$ e trabalhando para trás até $v_0$. Portanto, o primeiro passo é mover $[v_0,...,v_n]$ para cima até $[v_0,...,v_{n-1},w_n],$ então o segundo passo é mover isso para $[v_0,...,v_{n-2}, w_{n-1},w_n]$ e assim por diante. Na etapa típica $[v_0,...,v_{i},w_{i+1},...,w_n]$ move-se para cima até $[v_0,...,v_{i-1},w_i,...,w_n]$. A região entre esses dois simplexos é exatamente o (n+1)-simplexo $[v_0,...,v_i,w_i,...,w_n]$ que tem $[v_0,...,v_i,w_{i+1},...,w_n]$ como face inferior e $[v_0,...,v_{i-1},w_i,...,w_n]$ como face superior. Em conjunto, $\Delta^n\times I$ é a união de (n+1)-simplexos $[v_0,...,v_i,w_i,...,w_n]$, cada um intersectando o próximo em uma face.

    Agora, definimos \textbf{operador prisma} $P:S_n(X)\rightarrow S_{n+1}(Y)$ pela seguinte fórmula 
    \[P(\phi)=\sum_{i=0}^n (-1)^i F\circ (\phi\times id_I)|_{[v_0,...,v_i,w_i,...,w_n]},\]
    onde $\phi$ é um $n$-simplexo singular. Vamos mostrar que esses operadores prisma satisfazem a seguinte relação 
    \[\partial P=g_\bullet-f_\bullet-P\partial.\]
    Geometricamente, o lado esquerdo da equação representa o bordo da prisma, e os três termos do lado direito representam a base superior $\Delta^n\times \{1\}$, a base inferior $\Delta^n\times\{0\}$, e os lados $\partial \Delta^n\times I$ da prisma. Para provar a relação, calculamos 
    \begin{align*}
        \partial P(\phi)=&\sum_{i=0}^{n} \Bigl(\sum_{j=0}^{i} (-1)^i(-1)^j F\circ (\phi\times id_I)|_{[v_0,...,\widehat{v_j},...,v_i, w_i,...,w_n]} \\
        &+ \sum_{j=i}^{n} (-1)^i(-1)^{j+1} F\circ (\phi\times id_I)|_{[v_0,...,v_i,w_i,...,\widehat{w_j},...,w_n]}\Bigr)
    \end{align*}
    Ou seja, 
    \begin{align*}
        \partial P(\phi)=&\sum_{j\le i\le n} (-1)^i(-1)^j F\circ (\phi\times id_I)|_{[v_0,...,\widehat{v_j},...,v_i, w_i,...,w_n]} \\
        &+ \sum_{i\le j\le n} (-1)^i(-1)^{j+1} F\circ (\phi\times id_I)|_{[v_0,...,v_i,w_i,...,\widehat{w_j},...,w_n]}
    \end{align*}
    Os termos com $i=j$ nas duas somas se cancelam exceto para $F\circ(\phi\times id_I)|_{[\widehat{v_0},w_0,...,w_n]}$, que é $g\circ\phi=g_\bullet (\phi)$, e $-F\circ (\phi\times id_I)|_{[v_0,...,v_n,\widehat{w_n}]},$ que é $-f\circ\phi=-f_\bullet(\phi)$. Os termos com $i\ne j$ são exatamente $-P\partial (\phi)$, pois 
    \begin{align*}
        P\partial(\phi)=&\sum_{j=0}^{n} \Bigl( \sum_{i=j+1}^{n} (-1)^{i-1}(-1)^j F\circ (\phi\times id_I)|_{[v_0,...,\widehat{v_j},...,v_i,w_i,...,w_n]}\\
        &+\sum_{i=0}^{j-1} (-1)^i(-1)^j F\circ(\phi\times id_I)|_{[v_0,...,v_i,w_i,...,\widehat{w_j},...,w_n]}\Bigr)
    \end{align*}
    ou seja,
    \begin{align*}
        P\partial(\phi)=&\sum_{j<i\le n} (-1)^{i-1}(-1)^j F\circ (\phi\times id_I)|_{[v_0,...,\widehat{v_j},...,v_i,w_i,...,w_n]}\\
        &+\sum_{i<j\le n} (-1)^i(-1)^j F\circ(\phi\times id_I)|_{[v_0,...,v_i,w_i,...,\widehat{w_j},...,w_n]}\Bigr).
    \end{align*}
    Portanto, $P$ é uma homotopia de cadeias de $f$ e $g$. Pelo Lema \ref{homomorfismo-induzido-de-cadeias-prop}, temos que $f_*=g_*$.
\end{dem}

\begin{corol}
    Se dois espaços topológicos $X,Y$ são equivalentes homotópicos, então $H_n(X)\cong H_n (Y)$ para todo $n\ge 0$. 
\end{corol}

%\begin{titlemize}{Lista de consequências}
	%\item \hyperref[consequencia1]{Consequência 1};\\ %'consequencia1' é o label onde o conceito Consequência 1 aparece
%\end{titlemize}

\subsection{Homologia singular de um ponto} %afirmação aqui significa teorema/proposição/colorário/lema
\label{homologia-singular-de-um-ponto-prop}
\begin{titlemize}{Lista de dependências}
	\item \hyperref[complexo-de-cadeias-def]{Complexo de cadeias};\\ 
    \item \hyperref[homologia-singular-def]{Homologia singular}.
\end{titlemize}

\begin{prop}
    O n-ésimo grupo de homologia de um ponto é igual a
    \begin{align*}
        H_n(\{x\})\cong\begin{cases}
            \mathbb{Z}&\text{se }n=0\\
            0&\text{se }n>0.
        \end{cases}
    \end{align*}
\end{prop}

\begin{dem}
    Note que só existe um único n-simplexo singular $\sigma_n:\Delta^n\rightarrow \{x\}$ que é uma função constante. Logo, $S_n(\{x\})$ é o grupo cíclico infinito gerado por $\sigma_n$. Além disso, para $n\ge 1$, $\partial_n \sigma_n=\sigma_{n-1}$. Dessa forma, obtemos 
    \begin{align*}
        \partial\sigma_n=\sum_{i=0}^n (-1)^i \partial_n\sigma_n=\begin{cases}
            \sigma_{n-1}&\text{se n é par}\\
            0&\text{se n é ímpar}.
        \end{cases}
    \end{align*}
    Portanto, o complexo de cadeias $S(\{x\})_*$ é a sequência 
    \[...\xrightarrow{0} S_{4}(\{x\})\xrightarrow{\partial} S_3(\{x\})\xrightarrow{0} S_2(\{x\})\xrightarrow{\partial}...\rightarrow 0,\]
    em que $\partial_{(2k)}$ é o isomorfismo dado por $\sigma_{2k}\rightarrow \sigma_{2k-1}$ para todo $k\ge 1$. Logo $Z_n(\{x\})=B_n(\{x\})=\{0\}$ para todo $n\ge 1$, enquanto que $Z_0(\{x\})=S_0 (\{x\})\cong \mathbb{Z}$ e $B_0(\{x\})=0$.
\end{dem}

Como $\Delta^n$ é conexo, qualquer função contínua de $\Delta^n$ para um conjunto finito $\{x_1,...,x_m\}$, onde o conjunto é equipado com a topologia discreta, deve ser constante. Consequentemente, o mesmo argumento utilizado anteriormente implica que o n-ésimo grupo de homologia de $\{x_1,...,x_n\}$ é igual a
\begin{align*}
        H_n(\{x_1,...,x_m\})\cong\begin{cases}
            \mathbb{Z}^m&\text{se }n=0\\
            0&\text{se }n>0.
        \end{cases}
    \end{align*}
\begin{titlemize}{Lista de consequências}
    \item \hyperref[homologia-singular-de-um-espaco-contratil-prop]{Homologia singular de um espaço contrátil}.
	%\item \hyperref[]{}
\end{titlemize}

\subsection{0-ésimo grupo de homologia singular de um espaço 0-conexo} %afirmação aqui significa teorema/proposição/colorário/lema
\label{0-esimo-grupo-de-homologia-de-espaco-zero-conexo-prop}
\begin{titlemize}{Lista de dependências}
	\item \hyperref[complexo-de-cadeias-def]{Complexo de cadeias};\\ 
    \item \hyperref[homologia-singular-def]{Homologia singular}.
\end{titlemize}

\begin{prop}
    Se um espaço topológico não vazio $X$ é 0-conexo ou conexo por caminho, então $H_0(X)\cong \mathbb{Z}$.
\end{prop}
\begin{dem}
    Note que $Z_0(X)=S_0(X)$. Logo, cada elemento de $Z_0(X)$ tem a forma $z=a_1x_1+...+a_kx_k$, onde $a_i\in \mathbb{Z}$ e $x_i\in X$.

    Consideramos o homomorfismo $\alpha:Z_0(X)\rightarrow \mathbb{Z}$ definido por 
    \[\alpha (a_1x_1+...+a_kx_k)=a_1+...+a_k.\]
    Como $X$ não é vazio, $\alpha$ é sobrejetor. Vamos provar que $B_0(X)=\text{Ker}(\alpha)$.

    Para cada 1-simplexo singular $\sigma\in S_1(X)$ temos 
    \[\alpha(\partial \sigma)=\alpha(\sigma(0,1)-\sigma(1,0))=1-1=0,\]
    o que implica que $B_0(X)\subseteq \text{Ker}(\alpha)$. 
    
    Por outro lado, seja dada uma 0-cadeia $c_0=a_1x_1+...+a_kx_k$ em $\text{Ker}(\alpha)$. Fixemos um ponto $x_0\in X$. Pela definição de 0-conexo, para cada índice $i$ existe um 1-simplexo singular (uma curva) $\sigma_i:\Delta^1\rightarrow X$ tal que $\sigma_i (1,0)=x_0$ e $\sigma_i(0,1)=x_i$. Assim, segue que a 1-cadeia singular $c_1=a_1\sigma_1+...+a_k\sigma_k$ tem bordo 
    \begin{align*}
        \partial(c_1)=&a_1\partial \sigma_1+...+a_k\partial\sigma_i=a_1 (x_1-x_0)+...+a_k (x_k-x_0)\\
        &=a_1x_1+...+a_kx_k -(\sum_{i=0}^k a_i)x_0.
    \end{align*}
    Como $c_0\in \text{Ker}(\alpha)$, $\sum_{i=0}^k a_i=0$. Logo, segue que 
    \[\partial c_1=c_0.\]
    Isso mostra que $c_0\in B_0(X)$, ou seja, $\text{Ker}(\alpha)\subseteq B_0(X).$

    Com isso provamos que $B_0(X)=\text{Ker}(\alpha)$. Finalmente, pelo Teorema do Isomorfismo, obtemos
    \[H_0(X)=Z_0(X)/B_0(X)=Z_0(X)/\text{Ker}(\alpha)\cong \mathbb{Z}.\]
\end{dem}

Se $X$ não é 0-conexo, assumamos que seja $X=\bigsqcup_\lambda X_\lambda$ a separação de $X$ em componentes por caminho. Por continuidade, a imagem de cada n-simplexo singular $\sigma:\Delta^n\rightarrow X$ está contida em um e somente um componente por caminho $X_\lambda$. Isso resulta que $S_n(X)=\bigoplus_\lambda S_n(X_\lambda)$. Nesse contexto, o operador bordo opera componente a componente. Logo, 
\[Z_n(X)=\bigoplus_\lambda Z_n(X_\lambda)\;\;\;\text{ e }\;\;\;B_n(X)=\bigoplus_\lambda B_n(X_\lambda).\] 
Portanto, para cada $n\ge 0,$
\[H_n(X)=\frac{\bigoplus_\lambda Z_n(X_\lambda)}{\bigoplus_\lambda B_n(X_\lambda)}\cong\bigoplus\frac{Z_n(X_\lambda)}{B_n(X_\lambda)}=\bigoplus_\lambda H_n(X_\lambda
).\]
Em particular, 
\begin{corol}
    O grupo $H_0(X)$ é abeliano livre com tantos geradores quanto as componentes por caminho do espaço $X$.
\end{corol}

\begin{titlemize}{Lista de consequências}
    \item \hyperref[0-conexo-e-homomorfismo-de-homologia-induzido-prop]{0-conexo e homomorfismo de homologia induzido}.\\
	%\item \hyperref[]{}
\end{titlemize}

\subsection{0-conexo e homomorfismo de homologias induzido} %afirmação aqui significa teorema/proposição/colorário/lema
\label{0-conexo-e-homomorfismo-de-homologia-induzido-prop}
\begin{titlemize}{Lista de dependências}
    \item \hyperref[homologia-singular-def]{Homologia singular};\\
    \item \hyperref[homomorfismo-de-homologias-singulares-induzido-prop]{Homomorfismo de homologias singulares induzido};\\
    \item \hyperref[0-esimo-grupo-de-homologia-de-espaco-zero-conexo-prop]{0-ésimo grupo de homologia singular de um espaço 0-conexo}.
\end{titlemize}

\begin{prop}
    Sejam $X,Y$ espaços topológicos não vazios. Seja $f:X\rightarrow Y$ uma função contínua e considere o homomorfismo induzido $f_*:H_0(X)\rightarrow H_0(Y)$. Então:
    \begin{enumerate}
        \item Se $X$ é 0-conexo, então $f_*$ é um monomorfismo.
        \item Se $Y$ é 0-conexo, então $f_*$ é um epimorfismo.
        \item Se $X$ e $Y$ são 0-conexos, então $f_*$ é um isomorfismo.
    \end{enumerate}
\end{prop}

\begin{dem}
    Consideramos os epimorfismos $\overline{\alpha}:H_0(X)\rightarrow\mathbb{Z}$ e $\overline{\beta}:H_0(Y)\rightarrow\mathbb{Z}$ definidos por 
    \[\overline{\alpha}(a_1x_1+...+a_kx_k+B_0(X))=a_1+...+a_k;\]
    \[\overline{\beta}(b_1y_1+...+b_ly_l+B_0(Y))=b_1+...+b_l.\]
    Temos que $\overline{\alpha}$ (resp. $\overline{\beta}$) é um isomorfismo se $X$ (resp. $Y$) é 0-conexo. Além disso, para uma classe de homologia $[z]:=a_1x_1+...+a_kx_k+B_0(X)$ em $H_0(X)$, temos: 
    \begin{align*}
        \overline{\beta}(f_*([z]))&=\overline{\beta}(f\circ (a_1x_1+...+a_kx_k)+B_0(Y))\\
        &=\overline{\beta}(a_1(f\circ x_1)+...+a_k (f\circ x_k)+B_0(Y))\\
        &=a_1+...+a_k=\overline{\alpha}([z])
    \end{align*}
    Portanto, $\overline{\beta}\circ f_*=\overline{\alpha}$. Assim:
    \begin{enumerate}
        \item Se $X$ é 0-conexo, então $\overline{\alpha}$ é um isomorfismo e, consequentemente, $f_*$ é um monomorfismo.
        \item Se $Y$ é 0-conexo, então $\overline{\beta}$ é um isomorfismo e, consequentemente, $f_*$ é um epimorfismo.
        \item Se $X$ e $Y$ são 0-conexos, então, com base nos dois itens anteriores, $f_*$ é um isomorfismo.
    \end{enumerate}
\end{dem}

%\begin{titlemize}{Lista de consequências}
    %\item %\hyperref[homomorfismo-de-homologias-singulares-induzido-prop]{Homomorfismo de homologias singulares induzido}.\\
	%\item \hyperref[]{}
%\end{titlemize}

\subsection{Homologia singular de um espaço contrátil} %afirmação aqui significa teorema/proposição/colorário/lema
\label{homologia-singular-de-um-espaco-contratil-prop}
\begin{titlemize}{Lista de dependências}
	\item \hyperref[complexo-de-cadeias-def]{Complexo de cadeias};\\ 
    \item \hyperref[homologia-singular-def]{Homologia singular};\\
    \item \hyperref[homomorfismo-de-homologias-singulares-induzido-prop]{Homomorfismo de homologias singulares induzido};\\
    \item \hyperref[homologia-singular-de-um-ponto-prop]{Homologia singular de um ponto}.
\end{titlemize}

\begin{prop}
    Se $X$ é um espaço contrátil, então o n-ésimo grupo de homologia de $X$ é igual a
    \begin{align*}
        H_n(X)\cong\begin{cases}
            \mathbb{Z}&\text{se }n=0\\
            0&\text{se }n>0.
        \end{cases}
    \end{align*}
\end{prop}

\begin{dem}
    Como grupo de homologia é uma invariante homotópica, para um ponto $x\in X$, a inclusão $\{x\}\hookrightarrow X$ induz isomorfismo em homologia, o que mostra que $X$ tem o mesmo grupo de homologia de um ponto.
\end{dem}
    
%\begin{titlemize}{Lista de consequências}
    %\item %\hyperref[homomorfismo-de-homologias-singulares-induzido-prop]{Homomorfismo de homologias singulares induzido}.\\
	%\item \hyperref[]{}
%\end{titlemize}

\subsection{Simplexos singulares subordinados a uma cobertura} %afirmação aqui significa teorema/proposição/colorário/lema
\label{simplexos-singulares-subordinados-a-uma-cobertura-def}
\begin{titlemize}{Lista de dependências}
	\item \hyperref[complexo-de-cadeias-def]{Complexo de cadeias};\\ 
    \item \hyperref[homologia-singular-def]{Homologia singular};\\
    \item \hyperref[homomorfismo-de-homologias-singulares-induzido-prop]{Homomorfismo de homologias singulares induzido};\\
    \item \hyperref[simplexos-singulares-subordinados-a-uma-cobertura-def]{Simplexos singulares subordinados a uma cobertura};\\
    
\end{titlemize}

\begin{defi}
    Seja $X$ um espaço topológico e seja $\mathcal{U}$ uma cobertura de $X$, não necessariamente aberto. Denotamos por $S^{\mathcal{U}}_n (X)$ o subgrupo (abeliano livre) de $S_n (X)$ gerado pelos n-simplexos singulares $\sigma:\Delta^n\rightarrow X$ cuja imagem $\sigma(\Delta^n)$ está contido em algum elemento de $\mathcal{U}$, os quais chamamos \textbf{n-simplexos singulares subordinados} à $\mathcal{U}$.
\end{defi}
    Como para cada $0\le i\le n$, $\text{Im}(\partial_i \sigma )\subseteq \text{Im}(\sigma)$, para cada $\sigma\in S_n^{\mathcal{U}}(X)$, temos que $\partial\sigma\in S^{\mathcal{U}}_{n-1}(X)$. Desse modo, o operador bordo $\partial$ do complexo de cadeias singulares de $X$ induz, por restirção de domínio e contradomínio um operador bordo $\partial^{\mathcal{U}}$ tal que 
    \[...\rightarrow S^{\mathcal{U}}_{n+1}(X)\xrightarrow{\partial^{\mathcal{U}}}S_n^{\mathcal{U}}(X)\xrightarrow{\partial^{\mathcal{U}}}S_{n-1}^{\mathcal{U}} (X)\rightarrow...\]
    é um complexo de cadeias, denotado por $S^{\mathcal{U}}(X)_*$.

\begin{defi}
O n-ésimo grupo de homologia $H_n^{\mathcal{U}}(X)$ do complexo de cadeias $S^{\mathcal{U}}(X)_*$ é chamado o \textbf{n-ésimo grupo de homologia de} $X$ \textbf{subordinada} à $\mathcal{U}$.
\end{defi}

A função identidade $id:X\rightarrow X$ induz uma aplicação de cadeias $i_n:S_n^{\mathcal{U}}(X)\rightarrow S_n (X)$ e, por conseguinte, um homomorfismo em homologia.

%Suponhamos que a coleção $int(\mathcal{U}):=\{int(U):U\in \mathcal{U}\}$ seja uma cobertura de $X$. Então, dado um n-simplexo singular $\sigma:\Delta^n\rightarrow X$, a coleção $\mathcal{V}=\{\sigma^{-1}(int(U)):U\in \mathcal{U}\}$ é uma cobertura aberta do compacto $\Delta^n$. Tomando um número de Lebesgue $\delta>0$ para tal cobertura para cada $K\subseteq \Delta^n$ com diâmetro menor que $\delta$, existe um $U\in\mathcal{U}$ tal que $\sigma(K)\subseteq int (U)$. Por meio de iterações do processo de subdivisões baricêntricas de $\Delta^n$, o n-simplexo $\sigma$ pode ser expressado como uma n-cadeia $\sigma=a_1\sigma_1+...+a_k\sigma_k$ em que $\sigma_i \in C^{\mathcal{U}}_n(X)$ para cada $i$

\begin{prop}
    Suponhamos que a coleção $int(\mathcal{U}):=\{int(U):U\in \mathcal{U}\}$ seja uma cobertura de $X$. Então, $i_*:H^{\mathcal{U}}_n (X)\rightarrow H_n (X)$ é um isomorfismo para todo $n\ge 0$
\end{prop}

\begin{dem}
    A prova é feita por iterações do processo de subdivisões baricêntricas. Esta prova é bastante longa e trabalhosa, por isso omitimos a demonstração. O leitor consegue achar a demonstração em Proposição 2.21 em 
    \textit{Hatcher, Allen. Algebraic Topology. Cambridge, Cambridge University Press, 2001.}
\end{dem}

\begin{titlemize}{Lista de consequências}
    \item \hyperref[sequencia-de-mayer-vietoris-prop]{Sequência de Mayer-Vietoris}.\\
	%\item \hyperref[]{}
\end{titlemize}

\subsection{Sequência de Mayer-Vietoris} %afirmação aqui significa teorema/proposição/colorário/lema
\label{sequencia-de-mayer-vietoris-prop}
\begin{titlemize}{Lista de dependências}
	\item \hyperref[complexo-de-cadeias-def]{Complexo de cadeias};\\ 
    \item \hyperref[aplicacao-de-cadeias-def]{Aplicação de cadeias};\\
    \item \hyperref[homomorfismo-induzido-de-cadeias-prop]{Homomorfismo induzido de cadeias};\\
    \item \hyperref[sequencia-exata-def]{Sequência exata};\\
    \item \hyperref[homomorfismo-conectante-def]{Homomorfismo conectante};\\
    \item \hyperref[sequencia-exata-longa-induzida-prop]{Sequência exata longa induzida};\\
    \item \hyperref[homologia-singular-def]{Homologia singular};\\
    \item \hyperref[homomorfismo-de-homologias-singulares-induzido-prop]{Homomorfismo de homologias singulares induzido};\\
    \item \hyperref[simplexos-singulares-subordinados-a-uma-cobertura-def]{Simplexos singulares subordinados a uma cobertura};\\
    
\end{titlemize}

A sequência de Mayer-Vietoris é uma das mais poderosas ferramentas para o cálculo da homologia de um espaço topológico.
\begin{thm}
    Sejam $X$ um espaço topológico e $U$ e $V$ subconjuntos de $X$ tais que $int(U)\cup int(V)=X$. Considere as inclusões 
    \[i:U\cap V\hookrightarrow U,\;j:U\cap V\hookrightarrow V,\;k:U\hookrightarrow X,\;l:V\hookrightarrow X.\]
    Então, a sequência 
    \[...H_{n+1}(X)\xrightarrow{\delta} H_n(U\cap V)\xrightarrow{\Phi}H_n(U)\oplus H_n(V)\xrightarrow{\Psi} H_n(X)\rightarrow ...\]
    é exata, onde $\Phi:=i_*\oplus-j_*$ e $\Psi:=k_*+l_*$ 
\end{thm}

\begin{dem}
    Tomamos a cobertura $\mathcal{U}=\{U,V\}$ e consideramos, para cada $n\ge 0$, os homomorfismos $\phi:S_n(U\cap V)\rightarrow S_n(U)\oplus S_n(V)$ e $\psi:S_n(U)\oplus S_n(V)\rightarrow S_n^{\mathcal{U}}(X)$ dados por 
    \[\phi_n(c)=i_n(c)\oplus -j_n(c)\;\;\text{ e }\;\;\psi_n(c_1\oplus c_2)=k_n(c_1)+l_n(c_2).\]
    Para uma n-cadeia $c=a_1\sigma_1+...+a_r\sigma_r\in S_n(U\cap V)$, temos que 
    \[i_n(c)=a_1 (i\circ \sigma_1)+...+a_r(i\circ \sigma_r).\]
    Como $i\circ\sigma_1,...,i\circ \sigma_r$ são elementos da base do grupo abeliano livre $S_n(U)$, temos que $i_n(c)=0$ se, e somente se, $c=0$, o que mostra que $i_n$ é injetor. Analogamente, $j_n$ também é injetor. Portanto, $\phi_n$ é injetor.

    Pela definição, cada n-cadeia $c\in S_n^{\mathcal{U}}(X)$ é, da forma $c=c_1+c_2$ onde $c_1\in S_n (U)$ e $c_2\in S_n(V)$. Como $c=\psi_n(c_1\oplus c_2)$, provamos que $\psi_n$ é sobrejetor.

    Dada uma n-cadeia $c=a_1\sigma_1+...+a_r\sigma_r\in S_n(U\cap V)$, temos 
    \begin{align*}
        \psi_n\circ \phi_n(c)=& a_1(k\circ i\circ \sigma_1)+...+a_r(k\circ i\circ \sigma_r)\\
        &- (a_1(l\circ j\circ \sigma_1)+...+a_r (l\circ j\circ \sigma_r)).
    \end{align*}
    Como para cada n-simplexo singular $\sigma_i$, as composições $k\circ i\circ \sigma_i$ e $l\circ j \circ \sigma_i$ são iguais em $X$, obtemos $\psi_n\circ\phi_n(c)=0$. Portanto $\text{Im}(\phi_n)\subseteq \text{Ker}(\psi_n)$.

    Por outro lado, se as n-cadeias $c_1=a_1\sigma_1+...+ a_r\sigma_r \in S_n(U)$ e $c_2=a_1'\sigma_1'+...+a_s' \sigma_s'\in S_n(V)$ são não nulas tais que $c_1\oplus c_2\in \text{Ker}(\psi_n)$, então 
    \[a_1(k\circ \sigma_1)+...+a_r (k\circ \sigma_r)=-(a_1'(l\circ \sigma_1')+...+a_s'(l\circ \sigma_s')).\]
    Como só há uma única expressão de cada elemento do grupo abeliano livre $S_n(X)$ como combinação linear dos n-simplexos singulares, $r=s$ e, a menos de reordenação dos índices, $\sigma_t=\sigma_{t}'$ e $a_t'=-a_t$ para cada $1\le t\le r=s$. Dessa forma, $c_1,c_2\in S_n(U\cap V)$ e $c_2=-c_1$. Isso mostra que $c_1\oplus c_2=\phi_n (c_1)$, o que implica que $\text{Ker}(\psi_n)\subseteq \text{Im}(\phi_n)$.

    Pelos resultados obtidos acima, temos a sequência 
    \[0\rightarrow S_n (U\cap V)\xrightarrow{\phi_n} S_n (U)\oplus S_n(V)\xrightarrow{\psi_n} S_n^{\mathcal{U}}(X)\rightarrow 0\]
    é exata para cada $n\ge 0$. Logo a
    sequência dos complexos de cadeias 
    \[0\rightarrow S (U\cap V)_*\xrightarrow{\phi} S (U)_*\oplus S(V)_*\xrightarrow{\psi} S^{\mathcal{U}}(X)_*\rightarrow 0\]
    é exata, onde $\phi=(\phi_n)_{n\ge 0}$ e $\psi=(\psi_n)_{n\ge 0}$.

    Pelo Teorema \ref{sequencia-exata-longa-induzida-prop}, essa sequência induz a sequência 
    \[...\rightarrow H_{n+1}^\mathcal{U}(X)\xrightarrow{\delta} H_n(U\cap V)\xrightarrow{\phi_*}H_n(U)\oplus H_n(V)\xrightarrow{\psi_*} H^\mathcal{U}_n(X)\rightarrow ...\;.\]
    Como $\Phi=\phi_*$ e $\Psi=\psi_*$, a proposição \ref{simplexos-singulares-subordinados-a-uma-cobertura-def} garante que a sequência do enunciado do teorema 
    \[...H_{n+1}(X)\xrightarrow{\delta} H_n(U\cap V)\xrightarrow{\Phi}H_n(U)\oplus H_n(V)\xrightarrow{\Psi} H_n(X)\rightarrow ...\]
    é exata.
\end{dem}
Pela definição do homomorfismo conectante e pelo teorema \ref{sequencia-exata-longa-induzida-prop}, podemos caracterizar o homomorfismo $\delta$ na seguinte forma: uma classe de homologia $\overline{z}\in H_n (X)$ é da forma $\overline{z}=z+B_n(X)\in H_n(X)$, com o n-ciclo $z$ escrito como a soma $z=z_1+z_2$ de uma n-cadeia $z_1$ em $U$ e uma n-cadeia $z_2$ em $V$ (pelo proposição \ref{simplexos-singulares-subordinados-a-uma-cobertura-def}). Como $z$ é um n-ciclo, temos que $\partial z_1=-\partial z_2$ são (n-1)-ciclos em $U\cap V$. Então, 
\[\delta(\overline{z})=\partial z_1+B_{n-1}(U\cap V).\]

Sob certas condições, uma função contínua induz uma aplicação natural entre sequências de Mayer-Vietoris.

\begin{prop}
    Sejam $X, X'$ espaços topológicos decompostos como $X=int(U)\cup int(V)$ e $X'=int(U')\cup int(V')$. Seja $f:X\rightarrow X'$ uma função contínua tal que $f(U)\subseteq U'$ e $f(V)\subseteq V'$. Então, é comutativo o diagrama seguinte
    % https://q.uiver.app/#q=WzAsOCxbMCwwLCJIX3tuKzF9KFgpIl0sWzEsMCwiSF9uKFVcXGNhcCBWKSJdLFsyLDAsIkhfbihVKVxcb3BsdXMgSF9uKFYpIl0sWzMsMCwiSF9uKFgpIl0sWzAsMSwiSF97bisxfShYJykiXSxbMSwxLCJIX24oVSdcXGNhcCBWJykiXSxbMiwxLCJIX24oVScpXFxvcGx1cyBIX24oVicpIl0sWzMsMSwiSF9uKFgnKSJdLFswLDEsIlxcZGVsdGEiXSxbMSwyLCJcXFBoaSJdLFsyLDMsIlxcUHNpIl0sWzAsNCwiZl8qIiwyXSxbMSw1LCJmfF8qIiwyXSxbNCw1LCJcXGRlbHRhJyIsMl0sWzUsNiwiXFxQaGknIiwyXSxbNiw3LCJcXFBzaSciLDJdLFszLDcsImZfKiJdLFsyLDYsImZ8XypcXG9wbHVzIGZ8XyoiXV0=
\[\begin{tikzcd}
	{H_{n+1}(X)} & {H_n(U\cap V)} & {H_n(U)\oplus H_n(V)} & {H_n(X)} \\
	{H_{n+1}(X')} & {H_n(U'\cap V')} & {H_n(U')\oplus H_n(V')} & {H_n(X')}
	\arrow["\delta", from=1-1, to=1-2]
	\arrow["{f_*}"', from=1-1, to=2-1]
	\arrow["\Phi", from=1-2, to=1-3]
	\arrow["{f|_*}"', from=1-2, to=2-2]
	\arrow["\Psi", from=1-3, to=1-4]
	\arrow["{f|_*\oplus f|_*}", from=1-3, to=2-3]
	\arrow["{f_*}", from=1-4, to=2-4]
	\arrow["{\delta'}"', from=2-1, to=2-2]
	\arrow["{\Phi'}"', from=2-2, to=2-3]
	\arrow["{\Psi'}"', from=2-3, to=2-4]
\end{tikzcd}\]
    onde cada linha é um trecho da sequência de Mayer-Vietoris correspondente, e $f|$ denota a restrição de $f$ em domínio associado.
\end{prop}

\begin{dem}
    Como $i'\circ f|=f|\circ i$, $j'\circ f|=f|\circ j$, $f\circ k=k'\circ f|$ e $f\circ l=l'\circ f|$, temos que $i'_*\circ f|_*=f|_*\circ i_*$, $j'_*\circ f|_*=f|_*\circ j_*$, $f_*\circ k_*=k'_*\circ f|_*$ e $f_*\circ l_*=l'_*\circ f|_*$. Isso implica que os quadrados do centro e da direita são comutativos.

    Agora, vamos provar que o quadrado à esquerda também é comutativo. Seja $\overline{z}=z+B_{n+1}(X)\in H_{n+1}(X)$, pela observação acima, o $z$ pode ser escolhido como a soma $z=z_1+z_2$ de uma (n+1)-cadeia $z_1$ em $U$ e uma (n+1)-cadeia $z_2$ em V. Como $f:X\rightarrow X'$ induz uma aplicação de cadeias, temos 
    \begin{align*}
        f|_*\circ \delta(\overline{z}) & =f|_*(\partial z_1+B_{n}(U\cap V))=f(\partial z_1)+B_n(U'\cap V')\\
        &=\partial f(z_1)+B_n(U'\cap V')=\delta'(f(z_1)+f(z_2)+B_{n+1}(X'))\\
        &=\delta'(f(z)+B_{n+1}(X'))=\delta'(f_*(\overline{z})).
    \end{align*}
    Isso mostra que o quadrado à esquerda é comutativo.
\end{dem}
\begin{titlemize}{Lista de consequências}
    \item \hyperref[homologia-singular-de-S1-prop]{Homologia singular da circunferência};\\
    \item \hyperref[sequencia-exata-da-colagem-prop]{Sequência exata da colagem};\\
    \item \hyperref[grau-da-reflexao-prop]{Grau da reflexão}
	%\item \hyperref[]{}
\end{titlemize}

\subsection{Homologia singular da circunferência} %afirmação aqui significa teorema/proposição/colorário/lema
\label{homologia-singular-de-S1-prop}
\begin{titlemize}{Lista de dependências}
    \item \hyperref[sequencia-exata-def]{Sequência exata};\\
    \item \hyperref[homomorfismo-conectante-def]{Homomorfismo conectante};\\
    \item \hyperref[homologia-singular-def]{Homologia singular};\\
    \item \hyperref[homomorfismo-de-homologias-singulares-induzido-prop]{Homomorfismo de homologias singulares induzido};\\
    \item \hyperref[homologia-singular-de-um-ponto-prop]{Homologia singular de um ponto};\\
    \item \hyperref[0-esimo-grupo-de-homologia-de-espaco-zero-conexo-prop]{0-ésimo grupo de homologia singular de um espaço 0-conexo};\\
    \item \hyperref[homologia-singular-de-um-espaco-contratil-prop]{Homologia singular de um espaço contrátil};\\
    \item \hyperref[sequencia-de-mayer-vietoris-prop]{Sequência de Mayer-Vietoris}.

    
    
\end{titlemize}

\begin{prop}
    O n-ésimo grupo de homologia da circunferência é igual a 
    \begin{align*}
        H_n(\mathbb{S}^1)\cong\begin{cases}
            \mathbb{Z}&\text{se }n=0,1\\
            0&\text{se }n>1.
        \end{cases}
    \end{align*}
\end{prop}

\begin{proof}
    Denotamos os polos norte e sul de $\mathbb{S}^1\subseteq \mathbb{R}^2$ por $pn=(0,1)$ e $ps=(0,-1)$ respectivamente. Tomamos os abertos $U=\mathbb{S}^1\setminus \{ps\}$ e $V=\mathbb{S}^1\setminus \{pn\}$, cuja união $U\cup V=\mathbb{S}^1$. Pelo Teorema \ref{sequencia-de-mayer-vietoris-prop}, a sequência de Mayer-Vietoris
    \[...H_{n+1}(\mathbb{S}^1)\xrightarrow{\delta} H_n(U\cap V)\xrightarrow{\Phi}H_n(U)\oplus H_n(V)\xrightarrow{\Psi} H_n(\mathbb{S}^1)\rightarrow ...\]
    é exata.

    Os abertos $U$ e $V$ são ambos contráteis e, além disso, existe uma equivalência de homotopia sobrejetora $r:U\cap V\rightarrow\{q_1,q_2\}$, onde os pontos $q_1=(-1,0)$ e $q_2=(1,0)$. Dessa forma, os grupos de homologias $H_n(U),H_n(V)$ e $H_n(U\cup V)$ são todos triviais para $n\ge 1$, enquanto que $H_0(U)\cong \mathbb{Z}\cong H_0 (V)$ e $H_0 (U\cap V)\cong \mathbb{Z}\oplus \mathbb{Z}$.

    Como $\mathbb{S}^1$ é 0-conexo, segue que $H_0(\mathbb{S}^1)\cong \mathbb{Z}$.

    Para $n\ge 2$, pela sequência de Mayer-Vietoris, o trecho  
    \[0\rightarrow H_n (\mathbb{S}^1)\rightarrow 0\]
    é exata, consequentemente, $H_n(\mathbb{S}^1)=0.$

    O grupo $H_1(\mathbb{S}^1)$ aparece no trecho 
    \[0\rightarrow H_1(\mathbb{S}^1)\xrightarrow{\delta} H_0 (U\cap V)\xrightarrow{\Phi} H_0(U)\oplus H_0 (V),\]
    onde $\Phi=i_*\oplus -j_*$, sendo $i:U\cap V\hookrightarrow U$ e $j:U\cap V\hookrightarrow V$ as inclusões. Os pontos $q_1$ e $q_2$ podem vistos como 0-ciclos, representam geradores do grupo $H_0(U\cap V)\cong \mathbb{Z}\oplus \mathbb{Z}$. Por outro lado, esses elementos também são geradores tanto de $H_0 (U)$ quanto de $H_0(V)$, pois, em ambos os grupos, eles representam a mesma classe. Isso ocorre porque $q_1-q_2$ é o bordo de um arco hemisféricos da circunferência de $q_2$ para $q_1$ passando por cima (ou orientados no sentido anti-horário) em $U$ e de um arco hemisféricos passando por baixo (orientados no sentido horário) em $V$. Logo, existe um $q\in U\cap V$ tal que $q_1,q_2\in [q]_U=q+B_0 (U)$ e $q_1,q_2\in [q]_V=q+B_0 (V)$. Portanto, $\Phi$ é dado por 
    \begin{align*}
        \Phi(q_1+B_0(U\cap V))=[q]_U\oplus-[q]_V;\\
        \Phi(q_2+B_0(U\cap V))=[q]_U\oplus -[q]_V.
    \end{align*}
    Resulta que $\text{Ker}(\Phi)\cong \mathbb{Z}$, correspondendo ao subgrupo $\langle (1,-1) \rangle\subseteq \mathbb{Z}\oplus \mathbb{Z}\cong H_0(U\cap V)$. Por exatidão da última sequência, obtemos 
    \[H_1(\mathbb{S}^1)\cong \text{Im}(\delta)=\text{Ker}(\Phi)\cong \mathbb{Z}.\]
    Portanto, $H_n(\mathbb{S}^1)\cong \mathbb{Z}$ para $n=0$ ou $n=1$, e $H_n(\mathbb{S}^1)=0$ para todo $n\ge 2$.
\end{proof}
Para identificar um 1-ciclo $z_1\in Z_1(\mathbb{S}^1)$ cuja classe de homologia $\overline{z_1}=z_1+B_1 (\mathbb{S}^1)$ seja um gerador de $H_1(\mathbb{S}^1)$, podemos escolher $z_1$ como a soma de 1-cadeias, ou seja $z_1=c_1+c_2$, com $c_1\in S_1(U)$ e $c_2\in S_1(V)$. Como $\delta(\overline{z_1})=\partial c_1+B_0(U\cap V)$ (a observação no final de \ref{sequencia-de-mayer-vietoris-prop}), pelos isomorfismos $H_1(\mathbb{S}^1)\cong\text{Im}(\delta))\cong \text{Ker}(\Phi)$, temos que $\partial c_1=q_1-q_2=-\partial c_2$. Nessas condições, $c_1$ e $c_2$ são os 1-simplexos singulares correspondentes aos arcos hemisféricos da circunferência, orientados no sentido anti-horário.

\begin{titlemize}{Lista de consequências}
    \item \hyperref[grupo-de-homologia-singular-de-n-esfera-prop]{Grupo de homologia singular de n-esfera}.\\
	%\item \hyperref[]{}
\end{titlemize}

\subsection{Sequência exata da colagem} %afirmação aqui significa teorema/proposição/colorário/lema
\label{sequencia-exata-da-colagem-prop}
\begin{titlemize}{Lista de dependências}
    \item \hyperref[sequencia-exata-def]{Sequência exata};\\
    \item \hyperref[homomorfismo-conectante-def]{Homomorfismo conectante};\\
    \item \hyperref[homologia-singular-def]{Homologia singular};\\
    \item \hyperref[homomorfismo-de-homologias-singulares-induzido-prop]{Homomorfismo de homologias singulares induzido};\\
    \item \hyperref[homologia-singular-de-um-ponto-prop]{Homologia singular de um ponto};\\
    \item \hyperref[0-esimo-grupo-de-homologia-de-espaco-zero-conexo-prop]{0-ésimo grupo de homologia singular de um espaço 0-conexo};\\
    \item \hyperref[0-conexo-e-homomorfismo-de-homologia-induzido-prop]{0-conexo e homomorfismo de homologia induzido};\\
    \item \hyperref[homologia-singular-de-um-espaco-contratil-prop]{Homologia singular de um espaço contrátil};\\
    \item \hyperref[sequencia-de-mayer-vietoris-prop]{Sequência de Mayer-Vietoris};\\
    \item \hyperref[colagem-de-n-celula-def]{Colagem de n-célula}

    
    
\end{titlemize}

Seja $X$ um espaço Hausdorff. E Seja $f:\mathbb{S}^{n-1}\rightarrow X$ uma função contínua, onde $n\ge 2$. Nesse caso, o espaço $X_f$ também é Hausdorff. Sejam $i:X\hookrightarrow D^n\sqcup X$ e $j:D^n\hookrightarrow D^n$ as inclusões naturais e seja $\pi:D^n\sqcup X\rightarrow X_f$ a função quociente. Temos:
\begin{itemize}
    \item Como os pontos de $Y$ não se relacionam entre si, exceto consigo mesmos, a composição $l=\pi\circ i:X\rightarrow X_f$ induz um homeomorfismo entre $X$ e $l(X)\subset X_f$.
    \item Como a esfera $\mathbb{S}^{n-1}$ é conexa por caminho , a imagem de $f$ está contida em uma componente por caminho de $X$. Isso garante que as componentes por caminhos de $X$ e de $X_f$ estão em correspondência bijetiva. Pelo Corolário final do \ref{0-esimo-grupo-de-homologia-de-espaco-zero-conexo-prop}, $l_*:H_0 (X)\rightarrow H_0(X_f)$ é um isomorfismo.
    \item Como os pontos da bola aberta $B=int(D^n)$ não se relacionam entre si, exceto consigo mesmo, a composição $k=\pi\circ j|_B: B\rightarrow X_f$ induz um homeomorfismo entre $B$ e $k(B)\subseteq X_f$.
    \item $k(B)$ é aberto e $l(X)$ é fechado em $X_f$, pois $l(X)$ é compacto e $X_f$ é Hausdorff. Além disso, ambos são disjuntos, a fronteira de $k(B)$ é $l(f(\mathbb{S}^{n-1}))$, e o espaço de colagem $X_f=k(B)\sqcup l(X)$.
    \item Seja $\pi(0)\in X_f$ a imagem por $\pi$ do origem $0\in D^n$ e seja $\rho: D^n\setminus 0\rightarrow \mathbb{S}^{n-1}$ o retrato por deformação radial dado por $\rho(x)=x/||x||$. Então, está bem definida e é um retrato por deformação a função $\mathbf{r}:X_f\setminus \pi(0)\rightarrow l(X)$ dada por 
    \begin{align*}
        \textbf{r}(x)=\begin{cases}
            x&\text{ se }x\in l(X);\\
            \pi(\rho(k^{-1}(x))) &\text{ se }x\in k(B).
        \end{cases}
    \end{align*}
\end{itemize}

Agora, definimos $U=X_f\setminus \pi(0)$ e $V=X_f\setminus l(X)$. Como $\pi(0)$ e $l(Y)$ são fechados em $X_f$, $U$ e $V$ são abertos em $X_f$. Pelas definições, a união $U\sqcup V=X_f$. Além disso:
\begin{itemize}
    \item $U$ se deforma sobre $l(X)$ que é homeomorfo a $X$.
    \item O aberto $V=k(B)$ se deforma sobre o ponto $\pi(0)$, ou seja, $V$ é contrátil.
    \item $U\cap V$ é homeomorfo a $B\setminus \{0\}$ e, portanto, deforma sobre $k(S^{n-1}_{1/2})$, onde $S^{n-1}_{1/2}\subseteq B$ é a $(n-1)$-esfera de centro 0 e raio $1/2$.
\end{itemize}
Por Teorema \ref{sequencia-de-mayer-vietoris-prop}, a sequência de Mayer-Vietoris da decomposição $X_f=U\sqcup V$ 
\[...H_{m+1}(X_f)\xrightarrow{\delta} H_m(U\cap V)\xrightarrow{\Phi}H_m(U)\oplus H_m(V)\xrightarrow{\Psi} H_m(X_f)\rightarrow ...\]
é exata. Pelas observações acima, temos $H_m(U)\cong H_m(X)$ e $H_m (V)\cong H_m(\{0\})$ para todo $m\ge 0$. Por causa disso, para $m\ge 1$, o grupo $H_m(U)\oplus H_m(V)$ pode ser substituído por $H_m(X)$, e o homomorfismo $\Psi$ pode ser substituído por $l_*:H_m(X)\rightarrow H_m (X_f)$. Por outro lado, $H_0(U)\oplus H_0 (V)\cong H_0(X)\oplus \mathbb{Z}$ e $l_*:H_0(X)\rightarrow H_0 (X_f)$ é um isomorfismo. Logo, $\Psi: H_0(U)\oplus H_0 (V)\rightarrow H_0 (X_f)$ é sobrejetor e pode ser reescrito como $\Psi:H_0(X)\oplus \mathbb{Z}\rightarrow H_0(X_f)$.

Para $\Phi$, o contradomínio dele é isomorfo a $H_m(X)\oplus H_m (\pi(0))$, enquanto que o domínio dele $H_m(U\cap V)$ é isomorfo a $H_m(S^{n-1}_{1/2})$. Visto que $S^{n-1}_{1/2}$ é uma $(n-1)$-esfera e $V$ é contrátil, para todo $m>0$, o homomorfismo $\Phi:H_m(U\cap V)\rightarrow H_m(U)\oplus H_m(V)$ pode ser substituído na sequência de Mayer-Vietoris, a menos de isomorfismo, por $\Phi:H_m(\mathbb{S}^{n-1})\rightarrow H_m(X)$. Uma vez que $\mathbb{S}^{n-1}$ é 0-conexo, temos $\Phi:H_0(\mathbb{S}^{n-1})\rightarrow H_0(X)\oplus\mathbb{Z}$ é injetor (\ref{0-conexo-e-homomorfismo-de-homologia-induzido-prop}).
    
Chegamos à conclusão de que: 
\begin{prop}
    A sequência 
    \begin{align*}
        ...H_{m+1}(X_f)&\xrightarrow{\delta} H_m(\mathbb{S}^{n-1})\xrightarrow{\Phi}H_m(X)\xrightarrow{l_*} H_m(X_f)\rightarrow ...\\
        ...& \rightarrow H_0 (\mathbb{S}^{n-1})\xrightarrow{\Phi} H_0(X)\oplus \mathbb{Z}\xrightarrow{\Psi} H_0(X_f)\rightarrow 0
    \end{align*}
    é exata.
\end{prop}

Como $H_m(\mathbb{S}^{n-1})=0$ para todo $m\in\mathbb{N}-{0,n-1}$ (veremos daqui a pouco em \ref{grupo-de-homologia-singular-de-n-esfera-prop}), temos que $\Phi:H_m(\mathbb{S}^{n-1})\rightarrow H_m(X)$ é nulo para todo $m$ positivo tal que $m\ne n-1$.

\begin{titlemize}{Lista de consequências}
    \item \hyperref[grupo-de-homologia-singular-de-n-esfera-prop]{Grupo de homologia singular de n-esfera}.\\
	%\item \hyperref[]{}
\end{titlemize}

\subsection{Grupo de homologia singular de n-esfera} %afirmação aqui significa teorema/proposição/colorário/lema
\label{grupo-de-homologia-singular-de-n-esfera-prop}
\begin{titlemize}{Lista de dependências}
    \item \hyperref[sequencia-exata-def]{Sequência exata};\\
    \item \hyperref[homomorfismo-conectante-def]{Homomorfismo conectante};\\
    \item \hyperref[homologia-singular-def]{Homologia singular};\\
    \item \hyperref[homomorfismo-de-homologias-singulares-induzido-prop]{Homomorfismo de homologias singulares induzido};\\
    \item \hyperref[homologia-singular-de-um-ponto-prop]{Homologia singular de um ponto};\\
    \item \hyperref[0-esimo-grupo-de-homologia-de-espaco-zero-conexo-prop]{0-ésimo grupo de homologia singular de um espaço 0-conexo};\\
    \item \hyperref[0-conexo-e-homomorfismo-de-homologia-induzido-prop]{0-conexo e homomorfismo de homologia induzido};\\
    \item \hyperref[homologia-singular-de-um-espaco-contratil-prop]{Homologia singular de um espaço contrátil};\\
    \item \hyperref[homologia-singular-de-S1-prop]{Homologia singular da circunferência};\\
    \item \hyperref[sequencia-de-mayer-vietoris-prop]{Sequência de Mayer-Vietoris};\\
    \item \hyperref[colagem-de-n-celula-def]{Colagem de n-célula};\\
    \item \hyperref[colagem-de-um-disco-com-um-ponto-ex]{Colagem de um disco com um ponto};\\
    \item \hyperref[sequencia-exata-da-colagem-prop]{Sequência exata de colagem}
\end{titlemize}

\begin{prop}
    Seja $n$ um inteiro estritamente maior que $0$. O m-ésimo grupo de homologia singular de $\mathbb{S}^n$ é igual a  
    \begin{align*}
        H_m(\mathbb{S}^n)\cong\begin{cases}
            \mathbb{Z}&\text{ se }m=0\text{ ou }m=n\\
            0&\text{ caso contrário.}
        \end{cases}
    \end{align*}
\end{prop}

\begin{dem}
    Já vimos o caso $n=1$ em \ref{homologia-singular-de-S1-prop}. Para $n\ge 2$, consideramos a esfera $\mathbb{S}^n$ como o espaço de colagem $\mathbb{S}^n=D^n\cup_f\{x\}$, onde $f:\mathbb{S}^{n-1}\rightarrow \{x\}$ é a função constante (veja detalhe em \ref{colagem-de-um-disco-com-um-ponto-ex}). Como $\mathbb{S}^n$ é 0-conexo, $H_0(\mathbb{S}^n)\cong \mathbb{Z}$.
    
    Para $m\ge 2$, visto que $H_m(\{x\})$ e $H_{m-1}(\{x\})$ são ambos nulos, pela sequência exata da colagem, é exata a sequência 
    \[0\rightarrow H_m(\mathbb{S}^n)\xrightarrow{\delta} H_{m-1}(\mathbb{S}^{n-1})\rightarrow 0.\]
    Isso mostra que, para $m\ge 2$, $H_m(\mathbb{S}^n)\cong H_{m-1}(\mathbb{S}^{n-1}).$

    Novamente, pela sequência exata de colagem, é exata a sequência 
    \[0\rightarrow H_1(\mathbb{S}^n)\xrightarrow{\delta_1} H_0(\mathbb{S}^{n-1})\xrightarrow{\Phi} H_0(\{x\})\oplus \mathbb{Z}.\]
    Como a esfera $\mathbb{S}^{n-1}$ é 0-conexo, temos que $\Phi$ é um monomorfismo (\ref{0-conexo-e-homomorfismo-de-homologia-induzido-prop}) e, consequentemente, $\delta_1$ é a função nula, mas $\delta_1$ é um injetor. Isso mostra que $H_1(\mathbb{S}^n)=0$ para todo $n\ge 2$.

    Assim, como $H_1(\mathbb{S}^2)=0$ e $H_m(\mathbb{S}^2)=H_{m-1}(\mathbb{S}^1)$ para todo $m\ge 2$, obtemos 
    \begin{align*}
        H_m(\mathbb{S}^2)\cong\begin{cases}
            \mathbb{Z}&\text{ se }m=0,2\\
            0&\text{ caso contrário.}
        \end{cases}
    \end{align*}
    Seguindo indutivamente, obtemos 
    \begin{align*}
        H_m(\mathbb{S}^n)\cong\begin{cases}
            \mathbb{Z}&\text{ se }m=0,n\\
            0&\text{ caso contrário.}
        \end{cases}
    \end{align*}
\end{dem}

\begin{titlemize}{Lista de consequências}
    \item \hyperref[teorema-de-invariancia-de-dimensao-de-esfera-prop]{Teorema de invariância de dimensão de esfera};\\
	\item \hyperref[teorema-de-ponto-fixo-de-brouwer-geral-prop]{Teorema de ponto fixo de Brouwer (versão geral)};\\
    \item \hyperref[grau-de-funcoes-em-esferas-def]{Grau de funções}.
    
\end{titlemize}

\subsection{Teorema de invariância de dimensão de esfera} %afirmação aqui significa teorema/proposição/colorário/lema
\label{teorema-de-invariancia-de-dimensao-de-esfera-prop}
\begin{titlemize}{Lista de dependências}
    \item \hyperref[homologia-singular-def]{Homologia singular};\\
    \item \hyperref[homomorfismo-de-homologias-singulares-induzido-prop]{Homomorfismo de homologias singulares induzido};\\
    \item \hyperref[homologia-singular-de-S1-prop]{Homologia singular da circunferência};\\
    \item \hyperref[grupo-de-homologia-singular-de-n-esfera-prop]{Grupo de homologia singular de n-esfera}.
\end{titlemize}

\begin{prop}
    A n-esfera $\mathbb{S}^n$ é homeomorfa à m-esfera $\mathbb{S}^m$ se, e somente se, $n=m$.
\end{prop}

\begin{dem}
    Se $n=m$, então $\mathbb{S}^n=\mathbb{S}^m$.

    Para a outra direção, provamos por contra-positiva. Suponha que $0<n\ne m$. Como $H_n(\mathbb{S}^n)\cong \mathbb{Z}$ e $H_n(\mathbb{S}^m)=0$, concluímos que $\mathbb{S}^n$ não é homeomorfa a $\mathbb{S}^m$, como queríamos.
\end{dem}

%\begin{titlemize}{Lista de consequências}
    %\item %\hyperref[homomorfismo-de-homologias-singulares-induzido-prop]{Homomorfismo de homologias singulares induzido}.\\
	%\item \hyperref[]{}
%\end{titlemize}

\subsection{Grau de funções em esferas} %afirmação aqui significa teorema/proposição/colorário/lema
\label{grau-de-funcoes-em-esferas-def}
\begin{titlemize}{Lista de dependências}
    \item \hyperref[homologia-singular-def]{Homologia singular};\\
    \item \hyperref[homomorfismo-de-homologias-singulares-induzido-prop]{Homomorfismo de homologias singulares induzido};\\
    \item \hyperref[homologia-singular-de-S1-prop]{Homologia singular da circunferência};\\
    \item \hyperref[grupo-de-homologia-singular-de-n-esfera-prop]{Grupo de homologia singular de n-esfera}.
\end{titlemize}

O conceito de grau de funções contínuas de esferas, devido a Brouwer, é uma ferramenta poderosa, que produz resultados importantes.

\begin{defi}
    Seja $f:\mathbb{S}^n\rightarrow \mathbb{S}^n$ uma função contínua, com $n>0$. Consideramos o homomorfismo de homologias singulares induzido
    \[f_*:H_n(\mathbb{S}^n)\rightarrow H_n(\mathbb{S}^n).\]
    Como $H_n(\mathbb{S}^n)\cong \mathbb{Z}$, o homomorfismo $f_*$ é necessariamente da forma. $f_*(\alpha)=k\alpha$. O inteiro $k$ é chamado \textbf{grau da função} $f$ e é denotado por $deg(f)$.
\end{defi}

%\begin{titlemize}{Lista de consequências}
    %\item %\hyperref[homomorfismo-de-homologias-singulares-induzido-prop]{Homomorfismo de homologias singulares induzido}.\\
	%\item \hyperref[]{}
%\end{titlemize}

\subsection{Propriedades de grau de funções de esferas} %afirmação aqui significa teorema/proposição/colorário/lema
\label{propriedades-de-grau-de-funções-prop}
\begin{titlemize}{Lista de dependências}
    \item \hyperref[homologia-singular-def]{Homologia singular};\\
    \item \hyperref[homomorfismo-de-homologias-singulares-induzido-prop]{Homomorfismo de homologias singulares induzido};\\
    \item \hyperref[homologia-singular-de-um-espaco-contratil-prop]{Homologia singular de um espaço contrátil};\\
    \item \hyperref[homologia-singular-de-S1-prop]{Homologia singular da circunferência};\\
    \item \hyperref[grupo-de-homologia-singular-de-n-esfera-prop]{Grupo de homologia singular de n-esfera};\\
    \item \hyperref[grau-de-funcoes-em-esferas-def]{Grau de funções em esferas}.
\end{titlemize}

\begin{prop}
    O grau $deg(-)$ satisfaz as seguintes propriedades:
    \begin{enumerate}
        \item A função identidade $id:\mathbb{S}^n\rightarrow \mathbb{S}^n$ tem grau 1.
        \item Se $f,g:\mathbb{S}^n\rightarrow\mathbb{S}^n$ são contínuas, então $deg(f\circ g)=deg(f)deg(g)$.
        \item Se $f,g:\mathbb{S}^n\rightarrow\mathbb{S}^n$ são homotópicas, então $deg(f)=deg(g)$.
        \item Se $f:\mathbb{S}^n\rightarrow\mathbb{S}^n$ é uma equivalência de homotopia, então $deg(f)=+1\;\text{ ou }-1$.
        \item Se $f:\mathbb{S}^n\rightarrow\mathbb{S}^n$ é contínua e não sobrejetora, então $deg(f)=0$.
    \end{enumerate}
\end{prop}

\begin{dem}
    Os itens 1), 2) e 3) seguem dos resultados em \ref{homomorfismo-de-homologias-singulares-induzido-prop}. O item 4) segue dos itens 1), 2) e 3). 

    Agora, provamos o item 5): Como $f$ não é sobrejetor, podemos escolher um ponto $x\in \mathbb{S}^n\setminus f(\mathbb{S}^n)$. Então, $f$ fatora-se como uma composição \[\mathbb{S}^n\xrightarrow{g} \mathbb{S}^n\setminus\{x\}\hookrightarrow \mathbb{S}^n,\]
    onde denotamos a inclusão $\mathbb{S}^n\setminus\{x\}\hookrightarrow \mathbb{S}^n$ por $i$.
    Como $\mathbb{S}^n\setminus\{x\}$ é contrátil, obtemos $H_n(\mathbb{S}^n\setminus \{x\})=0$. Isso implica que $f_*=i_*\circ g_*=0$.
\end{dem}

%\begin{titlemize}{Lista de consequências}
    %\item %\hyperref[homomorfismo-de-homologias-singulares-induzido-prop]{Homomorfismo de homologias singulares induzido}.\\
	%\item \hyperref[]{}
%\end{titlemize}

\subsection{Grau da reflexão} %afirmação aqui significa teorema/proposição/colorário/lema
\label{grau-da-reflexao-prop}
\begin{titlemize}{Lista de dependências}
    \item \hyperref[homologia-singular-def]{Homologia singular};\\
    \item \hyperref[homomorfismo-de-homologias-singulares-induzido-prop]{Homomorfismo de homologias singulares induzido};\\
    \item \hyperref[homologia-singular-de-um-espaco-contratil-prop]{Homologia singular de um espaço contrátil};\\
    \item \hyperref[sequencia-de-mayer-vietoris-prop]{Sequência de Mayer-Vietoris};\\
    \item \hyperref[homologia-singular-de-S1-prop]{Homologia singular da circunferência};\\
    \item \hyperref[grupo-de-homologia-singular-de-n-esfera-prop]{Grupo de homologia singular de n-esfera};\\
    \item \hyperref[grau-de-funcoes-em-esferas-def]{Grau de funções em esferas};\\
    \item \hyperref[propriedades-de-grau-de-funções-prop]{Propriedade de grau de funções em esferas}
\end{titlemize}

\begin{defi}
    Uma \textbf{reflexão} é uma função de $r:\mathbb{S}^n\rightarrow \mathbb{S}^n$ que troca o sinal de uma coordenada. Quando a troca de sinal ocorrer na i-ésima coordenada, diremos que $r$ é uma \textbf{reflexão da i-ésima coordenada}. 
\end{defi}

\begin{lemma}
    A reflexão da primeira coordenada na esfera $\mathbb{S}^n$ tem grau $-1$.
\end{lemma}

\begin{dem}
    Provamos o lema por indução sobre a dimensão n.

    Base de indução: consideramos a reflexão $r:\mathbb{S}^1\rightarrow \mathbb{S}^1$ dada por $r(x,y)=(-x,y)$. Pelo exemplo \ref{homologia-singular-de-S1-prop}, $H_1(\mathbb{S}^1)$ é gerado pela classe de homologia do 1-ciclo $z_1=c_1+c_2$, onde $c_1$ e $c_2$ são os 1-simplexos singulares correspondentes aos arcos hemisféricos da circunferência, orientados no sentido anti-horário. A função $r$ mapeia $c_1$ a $-c_1$ e $c_2$ a $-c_2$. Isso mostra que $r\circ (z_1)=-z_1$ e, por conseguinte, o homomorfismo $r_*:H_1(\mathbb{S}^1)\rightarrow H_1(\mathbb{S}^1)$ tem grau $-1$.

    Passo de indução: Suponhamos que o resultado valha para a dimensão $n-1\ge 1$ e seja $r:\mathbb{S}^n\rightarrow \mathbb{S}^n$ a reflexão da primeira coordenada. Considerando a inclusão $e:\mathbb{S}^{n-1}\hookrightarrow \mathbb{S}^n$ dada por $(x_1,...,x_n)\mapsto (x_1,...,x_n,0)$ e considerando os abertos $U=\mathbb{S}^n\setminus\{(0,...,0,-1)\}$ e $V=\mathbb{S}^n\setminus \{(0,...,0,1)\}$. Note que a inclusão $i:\mathbb{S}^{n-1}\hookrightarrow U\cap V$ é uma equivalência de homotopia. Como o equador $\mathbb{S}^{n-1}$ é invariante por $r$ (i.e., $r(\mathbb{S}^{n-1})\subseteq\mathbb{S}^{n-1}$), a função restrição $r|:\mathbb{S}^{n-1}\rightarrow \mathbb{S}^{n-1}$ é uma reflexão da primeira coordenada bem definida. Pela hipótese de indução, $r|$ tem grau $-1$. Como $n\ge 2$ e $U$ e $V$ são contráteis, o seguinte trecho da sequência de Mayer-Vietoris 
    \[0=H_{n}(U)\oplus H_n(V)\rightarrow H_n(\mathbb{S}^n)\xrightarrow{\delta}H_{n-1}(U\cap V)\rightarrow H_{n-1}(U)\oplus H_{n-1}(V)=0\]
    é exata, o que mostra que o homomorfismo $\delta$ é um isomorfismo. Com isso, obtemos o diagrama comutativo seguinte 
    % https://q.uiver.app/#q=WzAsNixbMCwwLCJIX3tufShcXG1hdGhiYntTfV5uKSJdLFsxLDAsIkhfe24tMX0oVVxcY2FwIFYpIl0sWzIsMCwiSF9uKFxcbWF0aGJie1N9XntuLTF9KSJdLFswLDEsIkhfe259KFxcbWF0aGJie1N9Xm4pIl0sWzEsMSwiSF97bi0xfShVXFxjYXAgVikiXSxbMiwxLCJIX24oXFxtYXRoYmJ7U31ee24tMX0pIl0sWzAsMSwiXFxkZWx0YSIsMCx7InN0eWxlIjp7InRhaWwiOnsibmFtZSI6ImFycm93aGVhZCJ9fX1dLFswLDMsInJfKiIsMl0sWzEsNCwicnxfKiIsMl0sWzMsNCwiXFxkZWx0YSIsMix7InN0eWxlIjp7InRhaWwiOnsibmFtZSI6ImFycm93aGVhZCJ9fX1dLFsyLDUsInJ8XyoiXSxbMiwxLCJpXyoiLDIseyJzdHlsZSI6eyJ0YWlsIjp7Im5hbWUiOiJhcnJvd2hlYWQifX19XSxbNSw0LCJpXyoiLDAseyJzdHlsZSI6eyJ0YWlsIjp7Im5hbWUiOiJhcnJvd2hlYWQifX19XV0=
\[\begin{tikzcd}
	{H_{n}(\mathbb{S}^n)} & {H_{n-1}(U\cap V)} & {H_n(\mathbb{S}^{n-1})} \\
	{H_{n}(\mathbb{S}^n)} & {H_{n-1}(U\cap V)} & {H_n(\mathbb{S}^{n-1})}
	\arrow["\delta", tail reversed, from=1-1, to=1-2]
	\arrow["{r_*}"', from=1-1, to=2-1]
	\arrow["{r|_*}"', from=1-2, to=2-2]
	\arrow["{i_*}"', tail reversed, from=1-3, to=1-2]
	\arrow["{r|_*}", from=1-3, to=2-3]
	\arrow["\delta"', tail reversed, from=2-1, to=2-2]
	\arrow["{i_*}", tail reversed, from=2-3, to=2-2]
\end{tikzcd}\]
    onde flechas horizontais são isomorfismo.
    Seja $\overline{z}$ um gerador de $H_n(\mathbb{S}^n)$. Então, 
    \begin{align*}
        r_*(\overline{z})=&\delta^{-1}\circ r|_*\circ \delta(\overline{z})=\delta^{-1}\circ i_*\circ r|_*\circ i_*^{-1}\circ \delta(\overline{z})\\
        =& -\delta^{-1}\circ i_*\circ i_*^{-1}\circ \delta(\overline{z})=-\overline{z}.
    \end{align*}
    Isso demonstra que $deg(r)=-1$. Assim concluímos o passo de indução e a prova do lema.
\end{dem}

\begin{corol}
    Qualquer reflexão na esfera $\mathbb{S}^n$ tem grau -1.
\end{corol}

\begin{dem}
    Para cada $1\le i\le n+1$, seja $r_i: \mathbb{S}^n\rightarrow \mathbb{S}^n$ a reflexão da i-ésima coordenada. Pelo lema anterior, $deg(r_1)=-1$. Para $i\ge 2$, temos $r_i=h_i\circ r_1\circ h_i$, onde $h_i:\mathbb{S}^n\rightarrow \mathbb{S}^n$ é o homeomorfismo que permuta a primeira e a i-ésima coordenadas. Pela propriedades de grau de funções, segue que 
    \[deg(r_i)=deg(h_i)deg(r_1)deg(h_i)=deg(h_i)^2deg(r_1)=deg(r_1)=-1\]
\end{dem}

\begin{titlemize}{Lista de consequências}
    \item \hyperref[grau-de-antipoda-prop]{Grau de antípoda}.\\
	%\item \hyperref[]{}
\end{titlemize}

\subsection{Grau de antípoda} %afirmação aqui significa teorema/proposição/colorário/lema
\label{grau-de-antipoda-prop}
\begin{titlemize}{Lista de dependências}
    \item \hyperref[homologia-singular-def]{Homologia singular};\\
    \item \hyperref[homomorfismo-de-homologias-singulares-induzido-prop]{Homomorfismo de homologias singulares induzido};\\
    \item \hyperref[homologia-singular-de-S1-prop]{Homologia singular da circunferência};\\
    \item \hyperref[grupo-de-homologia-singular-de-n-esfera-prop]{Grupo de homologia singular de n-esfera};\\
    \item \hyperref[grau-de-funcoes-em-esferas-def]{Grau de funções em esferas};\\
    \item \hyperref[propriedades-de-grau-de-funções-prop]{Propriedade de grau de funções em esferas};\\
    \item \hyperref[grau-da-reflexao-prop]{Grau da reflexão}
\end{titlemize}

\begin{defi}
    Uma \textbf{antípoda} é uma função de $a:\mathbb{S}^n\rightarrow \mathbb{S}^n$ dada por $a(x)=-x$.
\end{defi}

\begin{lemma}
    A antípoda da esfera $\mathbb{S}^n$ tem grau $(-1)^{n+1}$
\end{lemma}

\begin{dem}
    Para cada $1\le i\le n+1$, seja $r_i: \mathbb{S}^n\rightarrow \mathbb{S}^n$ a reflexão da i-ésima coordenada. A antípoda fatora-se como a composição $a=r_1\circ ...\circ r_{n+1}$. Segue das propriedades de grau de funções em esferas que 
    \[deg(a)=deg(r_1)\cdot...\cdot deg(r_{n+1})=(-1)^{n+1}\]
\end{dem}

\begin{titlemize}{Lista de consequências}
    \item \hyperref[]{Teorema da Bola Cabeluda}.\\
	%\item \hyperref[]{}
\end{titlemize}


\end{document}


%novos assuntos/secções devem ser adicionados através do comando \import{conteudo/}{assunto} para adicionar o arquivo conteudo/assunto.tex

%%% Local Variables:
%%% mode: LaTeX
%%% TeX-master: t
%%% End:
